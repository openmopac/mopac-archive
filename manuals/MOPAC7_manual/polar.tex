

                                     POLAR (C)
     
          The polarizability and first and second hyperpolarizabilities are to
     be  calculated.   At present this calculation does not work for polymers,
     but should work for all other systems.  Two different options are
     implemented: the older finite field method and a new time-dependent
     Hartree-Fock method.
     
     Time-Dependent Hartree-Fock:

          This procedure is based on the detailed description given by
     M. Dupuis and S. Karna (J. Comp. Chem. 12, 487 (1991)).  The
     program is capable of calculating:
 
        Frequency Dependent Polarizability         alpha(-w;w)
        Second Harmonic Generation                 beta(-2w;w,w)
        Electrooptic Pockels Effect                beta(-w;0,w)
        Optical Rectification                      beta(0;-w,w)
        Third Harmonic Generation                  gamma(-3w;w,w,w)
        DC-EFISH                                   gamma(-2w;0,w,w)
        Optical Kerr Effect                        gamma(-w;0,0,w)
        Intensity Dependent Index of Refraction    gamma(-w;w,-w,w)

          The input is given at the end of the MOPAC deck and 
     consists of two lines of free-field input followed by a list
     energies.  
          The variables on the first line are:
             Nfreq  =  How many energies will be used to calculate 
                       the desired quantities.
             Iwflb  =  Type of beta calculation to be performed.
                       This valiable is only important if iterative
                       beta calculations are chosen.
                        0 - static
                        1 - SHG
                        2 - EOPE
                        3 - OR 
             Ibet   =  Type of beta calculation:
                        0 - beta(0;0)  static
                        1 - iterative calculation with type of
                            beta chosen by Iwflb.
                       -1 - Noniterative calculation of SHG
                       -2 - Noniterative calculation of EOPE
                       -3 - Noniterative calculation of OR
             Igam   =  Type of gamma calculation:
                        0 - No gamma calculation
                        1 - THG
                        2 - DC-EFISH
                        3 - IDRI
                        4 - OKE

          The vaiables on the second line are:
             Atol   =  Cutoff tolerance for alpha calculations
                       (1.0e-4 seems reasonable)
             Maxitu =  Maximum number of iteractions for beta
                       calculations
             Maxita =  Maximum number of iterations for alpha
                       calculations
             Btol   =  Cutoff tolerance for beta calculations

          Nfreq lines follow, each with an energy value in eV's at
     which the hyperpolarizabilites are to be calculated.




