\section{Reaction paths}\index{Reaction paths}\index{Reaction path calculation}
\label{rpaths}
\index{Path calculations}

\begin{figure}
\begin{makeimage}
\end{makeimage}
\begin{verbatim}
 CHARGE=-1
 SN2 reaction, Cl(-) + CH3F = CH3Cl + F(-)
   C
   F   1.4  1
   H   1.1  1   109.5 1     0    0    1 2
   H   1.1  1   109.5 1   120.0  0    1 2 3
   H   1.1  1   109.5 1   120.0  0    1 2 3
   Cl 20.0 -1   127.3 1   180.0  0    1 2 3
   0   0.00 0     0.0 0     0.0  0    0 0 0
  10.0 5.0 4.0 3.0 2.9 2.8 2.7 2.6 2.5
   2.4 2.3 2.2 2.1 2.0 1.9 1.8 1.7 1.6
\end{verbatim}
\index{S$_{N^2}$!reaction}
\caption{\label{sn2} Example of an S$_{N^2}$ reaction path calculation}
\end{figure}

\begin{figure}
\begin{makeimage}
\end{makeimage}
\begin{verbatim}
step=5 points=13  SYMMETRY
Ethane, Barrier to Rotation
C
C  1.5 1    0 0    0  0   1 0 0
H  1.0 1  111 1    0  0   2 1 0
H  1.0 0  111 0  120  0   2 1 3
H  1.0 0  111 0 -120  0   2 1 3
H  1.0 0  111 0   60 -1   1 2 3
H  1.0 0  111 0  180  0   1 2 3
H  1.0 0  111 0  -60  0   1 2 3
0  0.0 0    0 0    0  0   0 0 0
3 1 4 5 6 7 8
3 2 4 5 6 7 8
6 7 7
6 11 8
\end{verbatim}
\index{Rotation barrier calculation}
\caption{\label{c2h6p} Example of a rotation barrier calculation}
\end{figure}

\begin{figure}
\begin{makeimage}
\end{makeimage}
\begin{verbatim}
step=0.05 points=20
Trans-polyparaphenylene benzobisthiazole
Stretching the polymer
  C    0.0  0      0  0      0  0    0  0  0
  N    1.3  1      0  0      0  0    1  0  0
  S    1.7  1    115  1      0  0    1  2  0
  C    1.6  1     92  1      0  1    3  1  2
  C    1.4  1    109  1     -0  1    2  1  3
  C    1.4  1    124  1   -180  1    5  2  1
  C    1.4  1    116  1    180  1    6  5  2
  C    1.4  1    121  1      0  1    7  6  5
  C    1.4  1    129  1    180  1    4  3  1
  S    1.6  1    129  1    180  1    7  6  5
  C    1.7  1     92  1    180  1   10  7  6
  N    1.4  1    113  1   -180  1    8  7  6
  C    1.4  1    121  1   -180  1   11 10  7
  C    1.4  1    120  1    -90  1   13 11 10
  C    1.4  1    120  1    180  1   14 13 11
  C    1.4  1    120  1      0  1   15 14 13
  C    1.4  1    118  1     -0  1   16 15 14
  C    1.4  1    120  1      0  1   17 16 15
  H    1.0  1    121  1     -0  1    6  5  2
  H    1.0  1    121  1      0  1    9  4  3
  H    1.0  1    120  1     -0  1   14 13 11
  H    1.0  1    119  1   -180  1   15 14 13
  H    1.0  1    120  1   -180  1   17 16 15
  H    1.0  1    119  1    180  1   18 17 16
 xx    1.4  1    120  1    180  1   16 15 14
 Tv   12.6 -1      0  0      0  0    1 25 24
\end{verbatim}
\index{Polymers!stretching}
\index{PBT}
\caption{\label{hook} Data set to stretch a polymer}\index{Hook's law calculation}
\end{figure}

MOPAC has the capability to model the effects of changing an internal
coordinate. In the data-set, the relevant internal coordinate is flagged with a
`-1' rather than a `1' or `0'.  Two options then exist to allow the values of
the changing coordinate to be defined. 

First, the various values of the coordinate can be supplied after the geometry
and any symmetry data have been entered.  An example for the S$_{N^2}$ reaction
Cl$^-$ + CH$_3$F $\rightarrow$ CH$_3$Cl + F$^-$  is given in Figure~\ref{sn2}.

\index{Symmetry!use in reaction paths}
\index{Rotation barriers|ff}
\index{Ethane!barrier in}
Second, if the step-size is a constant, then the step-size and number of steps
can be defined on the keyword line.  An example of such a ``reaction'' would be
the rotation of a methyl group in, e.g., ethane, Figure~\ref{c2h6p}. Here,
symmetry is used to maintain D$_3$ symmetry as the rotation takes place. Note
that \comp{SYMMETRY} can be used to relate coordinates to the reaction
coordinate. The path calculations work by optimizing the geometry while the
reaction coordinate is fixed at the starting value.  Once the geometry is
optimized, the reaction coordinate is changed, and the geometry re-optimized. 
This is done for all points on the reaction path.  

Reaction paths can be used for calculating mechanical properties.  For example,
to calculate Hook's force constant for stretching polyethylene, the translation
vector could be steadily increased, Figure~\ref{hook}.

\section{Grid Calculation}
\label{grid}
\index{Grid Calculation|ff}
The GRID calculation is the two-dimensional analog of the PATH calculation.  In
a PATH calculation, one coordinate is flagged with a `-1'.  In a GRID
calculation, two coordinates are flagged by `-1's.  An example of a  GRID
calculation is shown in Figure~\ref{exgrid}. Note that the keywords
\comp{STEP1=$n.nn$} and \comp{STEP2=$m.mm$} are essential.

\begin{figure}
\begin{makeimage}
\end{makeimage}
\begin{verbatim}
SYMMETRY STEP1=0.01 STEP2=1
Water, potential energy surface for

 H
 O   0.92 -1     0  0    0 0   1
 H   0.92  0   104 -1    0 0   2 1

 2 1 3
\end{verbatim} 
\caption{\label{exgrid} Example of a GRID Calculation}
\end{figure}

In this example, the potential energy surface for water is generated.  For one
axis of the 2-D plot, the O--H bond length is varied from 0.92\AA\ to 1.02\AA ,  in 11 steps of 0.01\AA , and in the other axis, the H--O--H angle is varied
from 104  to 114$^{\circ}$ in 11 steps of 1.0 degree.  Because of the use of
symmetry, there are no variables to be optimized.  If symmetry were not used,
then the second O--H bond length could either be optimized, by setting its flag
to 1, or held constant, by setting its flag to 0.  

Other keywords which can be used with the GRID option are: \comp{MAX}: to use
23 points in each direction, rather than the default 11; \comp{POINT1=n} and
\comp{POINT2=m}: to use $n$ and $m$ points in directions 1 and 2; and keywords
to specify how the geometry should be optimized.
