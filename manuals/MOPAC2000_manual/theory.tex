\chapter{Theory}\label{theory}
\section{Introduction}
Most of the theory used in MOPAC is in the literature, so that in principle one
could read and understand the algorithm.  However, it is convenient to have the
theory gathered together in one document.  The theory given here is intended
for users who (a) want to modify MOPAC, or (b) want to understand how MOPAC
does what it does.  All derivations can be followed with a little patience,
and, as far as possible, ``It can be shown that'' jumps in logic have been
avoided.

\input{t_semi}              %  Conventional semiempirical theory
\section{Miscellaneous Topics in Semiempirical Theory}
\subsection{Koopmans' Theorem.}\index{Koopmans' theorem}
\index{Ionization potential}
Koopmans' theorem~\cite{koopmans} can be understood as follows:  for
closed-shell systems, the negative of the HOMO energy is the ionization
potential. That is, the energy required to form the cation {\em provided that
the ionization process is adequately represented by the removal of an electron
from an orbital without change in the wave-functions of the other electrons.}

MOZYME does not use  eigenvectors (LMOs are not eigenvectors), therefore,
Koopmans' theorem cannot be used unless eigenvectors are generated.  To use
Koopmans' theorem, add \comp{VECTORS} and \comp{EIGEN} to the keyword line.
This will cause the eigenvectors to be generated, and from the eigenvalues the
ionization potential can be calculated.

The only alternative way to calculate the I.P.\ is to calculate the $\Delta
H_f$ of the parent species, then, without allowing the geometry to relax,
calculate the $\Delta H_f$ of the ionized system.  The difference in  $\Delta
H_f$, in kcal.mol$^{-1}$, divided by 23.06, is the predicted I.P., in eV.

\subsection{Dipole moments.}\index{Dipole moment!definition}
\index{Dipole moment!for ions}
For neutral systems, the dipole moment is calculated from the atomic charges
and the lone-pairs as
\begin{equation}
\mu_x  = cC\sum_AQ_Ax_A + cCa_o2\sum_A P(s-p_x)_AD_1(A)
\end  {equation}
\begin{equation}
\mu_y  = cC\sum_AQ_Ay_A + cCa_o2\sum_A P(s-p_y)_AD_1(A)
\end  {equation}
\begin{equation}
\mu_z  = cC\sum_AQ_Az_A + cCa_o2\sum_A P(s-p_z)_AD_1(A)
\end  {equation}
\begin{equation}
\mu = \mu_x+\mu_y+\mu_z
\end  {equation}
Where $c$ = speed of light, $C$ = charge on the electron, and $a_o$ = Bohr
radius, or $cC = 2.99792458 \times 1.60217733 = 4.8032066$, and $cCa_o2 = 
2.99792458 \times 1.60217733 \times 0.529177249 \times 2.0  = 5.0834948$.  $D_1(A)$ is defined  in
Table~\ref{aa}
\begin{latexonly}
(see p.~\pageref{aa})
\end{latexonly}.

Formally, the dipole moment for an ion is undefined; however, it is convenient
to set up a `working definition.'  Consider a heteronuclear diatomic ion in a
uniform electric field.  The ion will accelerate.  To compensate for this, it
is convenient to consider the ion in an accelerating frame of reference.  The
ion will  experience a torque which acts about the center of mass, in a manner
similar to that of a polar molecule.  This allows us to define the dipole of an
ion as the dipole the system would exhibit while accelerating in a uniform
electric field.  To formalize this definition:
\begin{equation}
\mu_x  = cC\sum_AQ_A(x_A-x_{cog}) + cCa_o2\sum_A P(s-p_x)_AD(A)
\end  {equation}
\begin{equation}
\mu_y  = cC\sum_AQ_A(y_A-y_{cog}) + cCa_o2\sum_A P(s-p_y)_AD(A)
\end  {equation}
\begin{equation}
\mu_z  = cC\sum_AQ_A(z_A-z_{cog}) + cCa_o2\sum_A P(s-p_z)_AD(A)
\end  {equation}
\begin{equation}
\mu = \mu_x+\mu_y+\mu_z,
\end  {equation}
where $x_{cog}$ is the $x$-coordinate of the center of gravity of the system
\begin{equation}
x_{cog} = \sum_AM_Ax_A,
\end  {equation}
and $y_{cog}$ and $z_{cog}$ have similar definitions.  This general expression
will work for all discrete species, charged and uncharged, and is rotation and
position invariant.


                %  Koopmans' theorem
%
%  THIS FILE IS COMMON
%
\subsection{Bond Orders}\label{bonds}
\index{Valency|ff}\index{Lone pairs|(}\index{Bond order}\index{Wiberg indices}
Three quantities can be derived~\cite{bonds} from the density matrix for use in
discussing bonding.  These are: atomic bond index, anisotropy, and  bond
order. 

The density matrix, $P$, can be decomposed into sub-matrices representing atoms
or interactions between atoms.  The three  quantities just mentioned can then
be defined in terms of these sub-matrices.

\subsubsection{Atomic bond index}
A measure of the valency of an atom.
\begin{equation}
V_A=\sum_{\lambda\in A}2P_{\lambda\lambda}-\sum_{\lambda\in A}\sum_{\sigma\in A}P_{\lambda\sigma}^2.
\end{equation}
Typical valencies are: 1.0 for hydrogen, 2.0--2.4 for oxygen to 3.8--4.0 for
carbon. The maximum valency of an atom is equal to the number of atomic
orbitals, e.g.\  1 or 4 (for a $sp^3$ system), or 9 (in MNDO-$d$). This maximum
is only achieved when the orbital population is 1.00, and all off-diagonal
terms on the atom are zero.

\subsubsection{Anisotropy}
A measure of the number of lone-pairs on an atom.
\begin{equation}
L_A=\sum_{\lambda\in A}\sum_{\sigma\in A}P_{\lambda\sigma}^2 - \sum_{l=0}^{k}
\frac{1}{2l+1}(\sum_{\lambda=l^2+1}^{(l+1)^2}P_{\lambda\lambda})^2.
\end{equation}
Typical numbers of lone-pairs are: 0 in, for example, hydrogen and carbon, 1 for
 nitrogen in amines, and 2 in oxygen.

To see how this expression is derived, consider an atom having valence orbitals
defined by angular quantum numbers $l=0,1,\ldots,k$.  For H, $k=0$, for all
other elements, $k=1$. In order to have spherical symmetry, all orbitals in any
shell must be equally occupied. In addition, since the product of any two
different atomic orbitals is non-spherical, all off-diagonal density matrix
terms on any one atom must be zero.

\index{Lone pairs|)}
For a spherical atom having the atomic populations $s^pp^qd^r$, the valency
would be:
\begin{equation}
V_A=p^2+\frac{q^2}{3}+\frac{r^2}{5},
\end{equation}
or, in general:
\begin{equation}
V_A = \sum_{l=0}^{k} \label{eq:V_A}
\frac{1}{2l+1}(\sum_{\lambda=l^2+1}^{(l+1)^2}P_{\lambda\lambda})^2.
\end{equation}
Since, in general, atoms are non-spherical, then:
\begin{equation} 
V_A=\sum_{\lambda\in A}\sum_{\sigma\in A}P_{\lambda\sigma}^2 \label{eq:V_A2}.
\end{equation}
The difference between equations \ref{eq:V_A} and \ref{eq:V_A2} is a measure of how
unspherical the atom is.

\subsubsection{Bond order}
A measure of the number of bonds between atoms in a compound.
\begin{equation}
B_{AB}=\sum_{\lambda\in A}\sum_{\sigma\in B}P_{\lambda\sigma}^2
\end{equation}
Typical bond-orders are: 1.0, e.g.,  C-C in ethane; 2.0, e.g.,  C=C  in
ethylene; 3.0, e.g.,  C$\equiv$C in acetylene.  Bond orders of less than about
0.1--0.2 are indicative of ``no bond''.

The ideas here are an extension of Wiberg's indices~\cite{wiberg}.

Given the normal semiempirical density matrix, it is easy to show that
\begin{equation}
P^2=2P,
\end{equation}
from which it follows that
\begin{equation}
P_{\lambda\lambda}=1/2\sum_{\sigma}P_{\lambda\sigma}^2.
\end{equation}
This is the starting point for the derivation of
\begin{equation}
V_A=\sum_{B\neq A}B_{AB},
\end{equation}
from which the above definitions follow.



\subsection{Mulliken populations}
\index{Mulliken populations}
By default, the density matrix printed is the Coulson matrix, which
assumes that the atomic orbitals are orthogonalized.

If the assumption of orthogonality is not made, then the Mulliken density
matrix can be constructed. To construct the Mulliken density matrix (also known
as the Mulliken population analysis), the M.O.s must first be re-normalized,
using the overlap matrix, $S$:
$$
\psi_i^{'} = \psi_i\times S^{-\frac{1}{2}}. 
$$
From these M.O.s, a Coulson population is carried out. The off diagonal terms
are simply the Coulson terms multiplied by the overlap:
$$
P_{\lambda\sigma\neq\lambda}'=S_{\lambda\sigma}2\sum_{i=1}^{occ}c_{\lambda i}
c_{\sigma i},
$$
while the on-diagonal terms are given by the Coulson terms, plus half the sum
of the off-diagonal elements:
$$
P_{\lambda \lambda}' =S_{\lambda\sigma}2\sum_{i=1}^{occ}c_{\lambda i}c_{\lambda i}
 + \frac{1}{2}\sum_{\sigma\neq\lambda}P_{\lambda \sigma}'.
$$
A check of the correctness of the Mulliken populations is to add the diagonal
terms: these should equal the number of electrons in the system.

\subsubsection*{Theory of Mulliken Populations}
The NDDO methods (MNDO, AM1, PM3, and MNDO-$d$) all use Slater orbitals,
but an implication of one of the approximations made, that $\sum(F_{\mu \nu}-E_i\delta_{\mu\nu})
C_{\nu i} =0$, is that the conventional molecular orbitals are normalized
to unity:
$$
\psi_i=\sum_{\lambda}c_{\lambda i}\phi_{\lambda}
$$
with
$$
<\psi_i^2> = 1 = \sum_{\lambda}c_{\lambda i}^2
$$
For example, for H$_2$, the occupied M.O.\ is:
$$
\psi_1 = \sqrt{\frac{1}{2}}(\phi_{H_1}+\phi_{H_2}),
$$
and the unoccupied M.O.\ is:
$$ 
\psi_2 = \sqrt{\frac{1}{2}}(\phi_{H_1}-\phi_{H_2}). 
$$ 
The diagonal of the density matrix is then constructed using the Coulson
formula:
$$
P_{1,1}=P_{2,2}=2.0\times\left(\sqrt{\frac{1}{2}}\right)^2 =1.0.
$$
The off-diagonal terms are constructed in the same way:
$$
P_{1,2}=P_{1,2}=2.0\times\left(\sqrt{\frac{1}{2}}\right)^2 =1.0.
$$

If, instead of using $\sum(F_{\mu \nu}-E_i\delta_{\mu\nu}) C_{\nu i} =0$,
$\sum(F_{\mu \nu}-E_i) C_{\nu i} =0$ is used, then the occupied and unoccupied
M.O.s become:
$$ 
\psi_1 = \sqrt{\frac{1}{2(1+S)}}(\phi_{H_1}+\phi_{H_2}),
$$
and the unoccupied M.O.\ is:
$$
\psi_1 = \sqrt{\frac{1}{2(1-S)}}(\phi_{H_1}-\phi_{H_2}).
$$       
where $S$ is the overlap integral: $\int\phi_{H_1}\phi_{H_2}{\rm d}v$.

In this case, the Coulson population would give 
$$
\begin{array}{cc|cc|}
   &   & \frac{1}{1+S} & \frac{1}{1+S} \\
P  & = &               &               \\
   &   & \frac{1}{1+S} & \frac{1}{1+S} \\
\end{array}
$$
From this we see that the Coulson representation is unsuitable for  two
reasons: first, the number of electrons in the system, represented by the
diagonal terms, does not add to 2.0. Second, the off-diagonal terms, which
should represent the  number of electrons resulting from the overlap of the two
atomic orbitals, becomes unity as the overlap {\em decreases}.  

To correct for this, it is physically meaningful to multiply the matrix 
elements by the overlap.  This gives:
$$
\begin{array}{cc|cc|}
   &   & \frac{1}{1+S} & \frac{S}{1+S} \\
P  & = &               &               \\
   &   & \frac{S}{1+S} & \frac{1}{1+S} \\
\end{array}
$$
Now the off-diagonal terms accurately represent the number of electrons which are
associated with the overlap electron density.  The total number of electrons
in the system is now correct: $ \frac{1}{1+S} $ on atom 1, $ \frac{1}{1+S} $
on atom 2, and $ \frac{2S}{1+S} $ in the overlap region, giving a total of 2.0.

Although this representation is correct, it is potentially misleading, in that
the diagonal terms do not add to the number of electrons.  Mulliken reasoned
that the electron density resulting from the overlaps should be divided into
two equal parts and added to the diagonal terms.  When that is done, we get:
$$
\begin{array}{cc|ll|}
   &   & \frac{1}{1+S}+\frac{S}{1+S} & \frac{S}{1+S} \\ 
P  & = &               &               \\ 
   &   & \frac{S}{1+S} & \frac{1}{1+S} +\frac{S}{1+S}\\
\end{array} 
$$
or
$$
\begin{array}{cc|ll|}
   &   & 1.0  & \frac{S}{1+S} \\ 
P  & = &               &               \\ 
   &   &  \frac{S}{1+S} & 1.0\\ 
\end{array} 
$$
This simple example can be extended to systems involving heteroatoms and to
polyatomics, and is fully general.

The Mulliken analysis can be applied to semiempirical methods.  To do this, it
is necessary to first convert the M.O.s from solutions of  $\sum(F_{\mu
\nu}-E_i\delta_{\mu\nu}) C_{\nu i} =0$ to solutions of $\sum(F_{\mu \nu}-E_i)
C_{\nu i} =0$.  The simplest way to do this is to take the conventional M.O.s
and multiply them by $S^{-\frac{1}{2}}$.  In the case of H$_2$, the resulting
M.O.s are exactly correct; in general, a small error is introduced.  This error
arises from the incomplete annihilation of the secular matrix elements, and is
quite unimportant.


\subsection{Localized orbitals}\label{local}
\index{Localized orbitals}
The molecular orbitals generated by diagonalization are normally delocalized
over the system.  By using a unitary transform of the occupied M.O.s, it is
possible to generate a set of molecular orbitals which are localized on from
one up to three centers.

These localized M.O.s are not eigenvectors of the Hamiltonian, nor are their
energies eigenvalues.  However, localized orbitals can be equated with the
single, double, triple, and delocalized $\pi$-bonds of classical organic
chemistry.

\subsubsection{Eigenvectors}
\index{Schr\"{o}dinger}
Conventional M.O.s are one-electron solutions to Schr\"{o}dinger's equation,
and can be studied by photoelectron and other methods.  Such methods work on a
timescale of about $10^{-17}$s, roughly the time it takes light to cross a
molecule.  Normal chemical methods (study of reactions, etc.)  yield little
information about conventional M.O.s.  Some information is obtained by
inference, such as aromatic substitution directing considerations.

\subsubsection{Localized M.O.s}
In addition to being equivalent to the classical bonds, localized M.O.s are
useful in understanding chemical reactions, and other phenomena which take
place slowly relative to the speed of light: i.e., phenomena which take place
in $10^{-12}$s or slower.

\index{S$_{N^2}$}
Consider a S$_{N^2}$ reaction.  All the bonds forming and breaking can be seen
in the localized M.O.s.  Consider the reactive sites in a molecule (double bonds,
lone pairs, etc.).  These have exact equivalents in the localized M.O.s.  The
energies of localized M.O.s are indicative of the reactivity of the associated
electron pair.  Consider an excited, insulating polymer, such as excited
polyethylene.  The excited state is usually written with an asterisk ($*$), and
is generated simply by localizing the M.O.s of an excited polymer cluster.
\subsubsection{Localization Theory}
Various methods of localizing M.O.s have been 
proposed~\cite{boys,edmrue,von-niessen}.  The method 
described here is a modification of  Von Niessen's
\index{Von Niessen}
technique, and is ideally suited for semiempirical methods.

  For a set of LMOs, $$ \sum_i<\psi_i^4> $$
is a maximum.  Since $$ \sum_i\sum_j<\psi_i^2><\psi_j^2> $$ is a constant,
$$ \sum_i\sum_{j<i}<\psi_i^2><\psi_j^2> $$ must be a minimum.


The operation to localize M.O.\ consists of a series of binary unitary 
transforms of the type:
$$
|\psi_i> =a|\psi_k> +b|\psi_l> $$$$ 
|\psi_j> =-b|\psi_k> +a|\psi_l>  
$$
where $|\psi_k>$ and $|\psi_l>$ are normal M.O.s, and $|\psi_i>$ and $|\psi_j>$ 
are the LMOs.

The ratio $a/b$ is given by
$$
a/b = \frac{1}{4}\arctan\left(\frac{4(<\psi_k\psi_l^3>-<\psi_k^3\psi_l>)}
{<\psi_k^4>+<\psi_l^4>-6<\psi_k^2\psi_l^2>}\right)
$$
Note that in normal semiempirical work: $ <\phi_{\lambda}|\phi_{\sigma}>
=\delta(\lambda,\sigma)$.

From this it follows that, given 
$\psi_k = \sum_{\lambda}C_{\lambda k}\phi_{\lambda}$,
$$
<\psi_k\psi_l^3> = \sum_{\lambda}C_{\lambda k}C_{\lambda l}^3
$$
\index{Rotational invariance}
In order to preserve rotational invariance, all contributions on each atom must be
added together.  This gives:
$$
<\psi_k^4> = \sum_A(\sum_{\lambda\in A}C_{\lambda k}^2)^2 $$$$
<\psi_k^3\psi_l> = \sum_A(\sum_{\lambda\in A}C_{\lambda k}^2)
\sum_{\lambda\in A}C_{\lambda k}C_{\lambda l}$$$$
<\psi_k^2\psi_l^2> = \sum_A(\sum_{\lambda\in A}C_{\lambda k}^2)(\sum_{\lambda\in A}C_{\lambda l}^2)
$$

%\subsubsection{Localized Bond Dipoles}
%The total dipole for a system can be expressed as the sum of bond-dipoles.
%When {\bf LOCAL} is used with RHF systems, the bond-dipoles are also printed.
%
%The definition of bond-dipole, $\mu_x(j)$, for localized bond $\psi_j$  is:
%$$
%\mu_x(j) = \sum_A(Q_A(j) - Z_A(j))x_A +\sum_A(c_AP_{s,x}^A(j)) 
%$$
%where $Q_A(j)$ is the total electron density on atom $A$ arising from localized 
%M.O.\ $\psi_j$, $c_A$ is a constant for each atom $A$, and $P_{s,x}^A(j)$ is the
%one-center $s$-$p_x$ electron density matrix element.
%
%Specification of $Z_A(j)$ is not straightforward. This term can be interpreted
%as `the nuclear contribution to the bond dipole for localized M.O.\ $\psi_j$.'
%
%There are three constraints on the form of $Z_A(j)$:
%\begin{enumerate}
%\item  A localized bond
%contains exactly two electrons, therefore the nuclear term must represent
%exactly two positive charges.
%\item  The sum of the nuclear terms for each atom over all LMOs must equal the
%total nuclear charge on that atom. 
%\item The contribution of the nuclear charge from each atom
%must be a function of the electron density arising from the localized molecular
%orbital.
%\end{enumerate}
%These conditions allow us to write $Z_A(j)$ as follows:
%$$
%Z_A(j)=Q_A(j)*D(A)*E(j)
%$$
%where the vectors $D(A)$ and $E(j)$ are defined as follows:
%$$
%\sum_AQ_A(j)*D(A)*E(j) = 2
%$$$$
%\sum_jQ_A(j)*D(A)*E(j)=  Z_A
%$$
%$D(A)$ and $E(j)$ are defined by the constraints for $Z_A(j)$.  In practise, an
%iterative scheme is used to find the vectors $D$ and $E$.

\subsection{Outer Valence Green's Function}\label{gf}
This section is based on materials supplied by
\begin{center}Dr David Danovich\\The Fritz Haber Research Center for 
Molecular Dynamics\\ The Hebrew University of Jerusalem\\ 91904 Jerusalem\\
Israel \end{center}

The OVGF technique was used with the self-energy part extended to include
third order perturbation corrections,~\cite{gf1}.  The higher order contributions
were estimated by the renormalization procedure.  The actual expression used to
calculate the self-energy part, $\sum_{pp}(w)$, chosen in the diagonal form,
is given in equation~(\ref{gfeq1}), where $\sum_{pp}^{(2)}(w)$ and $\sum_{pp}^{(3)}(w)$ are
the second- and third-order corrections, and $A$ is the screening factor accounting
for all the contributions of higher orders.
\begin{equation}\label{gfeq1}
\sum_{pp}(w) = \sum_{pp}^{(2)}(w)+(1-A)^{-1}\sum_{pp}^{(3)}(w).
\end{equation}
The particular expression which was used for the second-order corrections is given in
equation~(\ref{gfeq2}).
\begin{equation}\label{gfeq2}
\sum_{pp}^{(2)}(w) = \sum_a\sum_{i,j}\frac{(2V_{paij}-V_{paji})V_{paij}}{w+e_a-e_i-e_j}
+\sum_{a,b}\sum_i\frac{(2V_{piab}-V_{piba})V_{piab}}{w+e_i-e_a-e_b},
\end{equation}
where
$$
V_{pqrs} = \int\int\psi_p^*(1)\psi_q^*(2)(1/r_{12})\psi_r^*(1)\psi_s^*(2){\rm d}\tau_1{\rm d}\tau_2.
$$

In equation~(\ref{gfeq2}), $i$ and $j$ denote occupied orbitals, $a$ and $b$
denote virtual orbitals, $p$ denotes orbitals of unspecified occupancy, and
$e$  denotes an orbital energy. The equations are solved by an iterative
procedure, represented in  equation~(\ref{gfeq3}).
\begin{equation}\label{gfeq3}
w_p^{i+1}=e_p+\sum_{pp}(w^i).
\end{equation}

The SCF energies and the corresponding integrals, which were calculated by one
of the semiempirical methods (MNDO, AM1, or PM3), were taken as the zero'th
approximation and all M.O.s may be included in the active space for the OVGF
calculations.

The expressions used for $\sum_{pp}^{(3)}$ and $A$ are given in \cite{gf2}.

The OVGF method itself, is described in detail in \cite{gf1}.

%\subsubsection{Example of OVGF calculation}
%  The data-set {\bf test\_greenf.dat} will calculate the first 8 I.P.s
%for dimethoxy-$s$-tetrazine.  This calculation is discussed in detail in \cite{gf6}.
%The experimental and calculated I.P.s are shown in Table~\ref{gftab}.
%\begin{table}
%\caption{\label{gftab}OVGF Calculation, Comparison with Experiment}
%\begin{center}
%\begin{tabular}{lccccc}\\
%M.O.    &  Expt*  &   PM3   & Error   &  OVGF(PM3)  &  Error \\
%$n_1$    &  9.05   &   10.15 &  1.10   &   9.46      &   0.41 \\
%$\pi_1$  &  9.6    &   10.01 &  0.41   &   9.65      &   0.05 \\
%$n_2$    &  11.2   &   11.96 &  0.76   &  11.13      &  -0.07 \\
%$\pi_2$  &  11.8   &   12.27 &  0.47   &  11.43      &  -0.37 \\
%\end{tabular}
%
%*: R. Gleiter,  V. Schehlmann, J. Spanget-Larsen, H. Fischer and F. A. Neugebauer,
%{\em J. Org. Chem.}, {\bf 53}, 5756 (1988).
%\end{center}
%\end{table}
%
%
%From this, we see that for PM3 the average error is 0.69eV, but after OVGF 
%correction, the error drops to 0.22eV.  This is typical of nitrogen heterocycle 
%%calculations.
             %  Green's Function
\subsection{Using  {\em ab initio} derivatives}
\index{ab initio@{{\em ab initio}}!derivatives}
\index{Derivatives!{\em ab initio}}
\comp{AIDER} will allow gradients to be defined for a system.  MOPAC  will
calculate  gradients, as usual, and will then use the supplied gradients to
form an error function.  This error function is  (supplied gradients $-$ 
initial  calculated  gradients),  which is then added to the computed
gradients, so that for the initial  SCF,  the  apparent  gradients  will equal
the supplied gradients.

A typical data-set using \comp{AIDER} is shown in Figure~\ref{aider}.

\begin{figure}
\begin{makeimage}
\end{makeimage}
\index{Cyclohexane}
\index{ab initio@{{\em ab initio}}!geometry}
\index{Data!for cyclohexane}
\compresstable
\begin{verbatim}
 PM3 AIDER AIGOUT GNORM=0.01
 Cyclohexane

   X
   X   1 1.0
   C   1 CX     2 CXX
   C   1 CX     2 CXX     3 120.000000
   C   1 CX     2 CXX     3 120.000000
   X   1 1.0    2 90.0    3   0.000000
   X   1 1.0    6 90.0    2 180.000000
   C   1 CX     7 CXX     3 180.000000
   C   1 CX     7 CXX     3  60.000000
   C   1 CX     7 CXX     3 -60.000000
   H   3 H1C    1 H1CX    2   0.000000
   H   4 H1C    1 H1CX    2   0.000000
   H   5 H1C    1 H1CX    2   0.000000
   H   8 H1C    1 H1CX    2 180.000000
   H   9 H1C    1 H1CX    2 180.000000
   H  10 H1C    1 H1CX    2 180.000000
   H   3 H2C    1 H2CX    2 180.000000
   H   4 H2C    1 H2CX    2 180.000000
   H   5 H2C    1 H2CX    2 180.000000
   H   8 H2C    1 H2CX    2   0.000000
   H   9 H2C    1 H2CX    2   0.000000
   H  10 H2C    1 H2CX    2   0.000000

   CX     1.46613
   H1C    1.10826
   H2C    1.10684
   CXX   80.83255
   H1CX 103.17316
   H2CX 150.96100

   AIDER
     0.0000
    13.7589 -1.7383
    13.7589 -1.7383  0.0000
    13.7589 -1.7383  0.0000
     0.0000  0.0000  0.0000
     0.0000  0.0000  0.0000
    13.7589 -1.7383  0.0000
    13.7589 -1.7383  0.0000
    13.7589 -1.7383  0.0000
   -17.8599 -2.1083  0.0000
   -17.8599 -2.1083  0.0000
   -17.8599 -2.1083  0.0000
   -17.8599 -2.1083  0.0000
   -17.8599 -2.1083  0.0000
   -17.8599 -2.1083  0.0000
   -17.5612 -0.6001  0.0000
   -17.5612 -0.6001  0.0000
   -17.5612 -0.6001  0.0000
   -17.5612 -0.6001  0.0000
   -17.5612 -0.6001  0.0000
\end{verbatim}
\caption{\label{aider} Example of the use of \comp{AIDER}}
\end{figure}
 
Each  supplied  gradient  goes  with  the  corresponding   internal
coordinate.   In  the  example  given,  the  gradients came from a 3-21G
calculation on the geometry shown.  Symmetry will be taken into  account
automatically. Gaussian  prints  out  gradients in atomic units; these need to
be converted into kcal/mol/\AA ngstrom or kcal/mol/radian for MOPAC to use. 
The resulting geometry from the MOPAC run will be nearer to the optimized 3-21G
geometry than  if  the  normal  geometry  optimizers  in Gaussian had been
used.\index{Coordinates!Gaussian!example}

\subsection{Correction to the Peptide Linkage}
\index{Peptide linkage!correction to barrier}
The residues in peptides are joined together by  peptide  linkages, --HNCO--.  
These  linkages  are  almost  flat, and normally adopt a trans configuration,
the hydrogen and oxygen atoms being on opposite sides  of the  C--N  bond.  
Experimentally,  the  barrier  to  interconversion  in \index{N-methyl
acetamide|ff} N-methyl acetamide is about 14 kcal/mol, but all  four  methods 
within \index{PM3!fault in peptide linkage} MOPAC  predict  a  significantly 
lower  barrier,  PM3 giving the lowest value.

The low barrier can be traced  to  the  tendency  of  semiempirical methods  
to   give   pyramidal   nitrogens.    The   degree   to  which pyramidalization
of the nitrogen atom is preferred can be  seen  in  the series of compounds
given in Table~\ref{py}.

To correct this, a molecular-mechanics correction has been applied.
\index{MMOK}   This  consists  of  identifying  the --R--HNCO-- unit, and
adding a torsion potential of form: 
\[
k \sin \! ^{2} \theta
\]
where $\theta$ is the X--N--C--O angle, X=R or H, and $k$ varies from method to 
method.   This  has  two effects:  there is a force constraining the nitrogen
to be planar, and the HNCO barrier in N--methyl acetamide is  raised to  14.00 
kcal/mol.   When the MM correction is in place, the nitrogen atom for all
methods for the last three compounds shown above is planar. The correction
should be user-transparent.

\subsubsection{Cautions}\label{mmok}
\begin{enumerate}
\item This correction will lead to errors of 0.5--1.5  kcal/mol  if
the   peptide   linkage   is  made  or  broken  in  a  reaction
calculation.
\item If the correction is applied to formamide the nitrogen will  be
            flat, contrary to experiment.
\item When calculating rotation barriers, take into account the rapid
\index{Rotation barriers}
            rehybridization  which  occurs.   When the dihedral is 0 or 180
            degrees the nitrogen will be planar (sp$^2$), but  at  90  degrees
            the nitrogen should be pyramidal, as the partial double bond is
            broken.  At that geometry the true  transition  state  involves
            motion  of the nitrogen substituent so that the nitrogen in the
            transition state is more nearly sp$^2$.  In other words, a  simple
            rotation  of  the  HNCO  dihedral will not yield the activation
            barrier, however it will be within 2 kcal/mol of  the  correct
            answer.   The  14  kcal barrier mentioned earlier refers to the
            true transition state.
\end{enumerate}     
% 16 lines, including this line
\begin{table}
\caption{\label{py}Comparison of Observed and Calculated Pyramidalization of 
Nitrogen}
\begin{center}
\begin{tabular}{llllll} \hline
        Compound    & MINDO/3  & MNDO  & AM1   &PM3  &  Exp  \\ \hline
  Ammonia           & Py       &  Py   & Py    &Py   &  Py  \\
  Aniline           & Py       &  Py   & Py    &Py   &  Py  \\
  Formamide         & Py       &  Py   & Flat  &Py   &  Py  \\
  Acetamide         & Flat     &  Py   & Flat  &Py   &  Flat  \\
  N-methyl formamide& Flat     &  Py   & Flat  &Py   &  Flat  \\
  N-methyl acetamide& Flat     &  Flat & Flat  &Py   &  Flat \\ \hline 
\end{tabular}
\end{center}
\end{table}


\subsection{Convergence in SCF Calculation}
\index{SCF!convergence techniques}
A  brief  description  of  the  convergence  techniques   used   in subroutine
ITER follows.

ITER, the  SCF  calculation,  employs  six  methods  to  achieve  a
self-consistent field.  In order of usage, these are:

\begin{enumerate}
\index{Fock matrix}
\item Intrinsic convergence by virtue of the way the  calculation  is carried 
out.   Thus  a trial Fock gives rise to a trial density matrix, which in turn
is used to generate a better Fock matrix.

This is normally convergent, but many exceptions  are  known.   The main
situations when the intrinsic convergence does not work are:
\begin{enumerate}
\item A bad starting density  matrix.   This  normally  occurs when the default
starting density matrix is used.  This is a very crude approximation, and is
only  used  to  get  the  calculation started.   A  large  charge  is generated
on an atom in the first iteration,  the  second   iteration  
overcompensates,   and   an oscillation is generated.

\item The equations are only very slowly convergent.  This can be  due  to  a 
long-lived  oscillation  or to a slow transfer of charge.
\end{enumerate}

\index{Oscillations!damping|ff}
\item Oscillation damping.  If, on any two consecutive iterations,  a density 
matrix  element  changes  by  more  than 0.05, then the density matrix element
is set equal to the old element shifted by  0.05  in  the direction  of  the
calculated element.  Thus, if on iterations 3 and 4 a certain density matrix
element was 0.55 and 0.78, respectively, then the element  would  be set to
0.60 (= 0.55 + 0.05) on iteration 4.  The density matrix from iteration 4 would
then be used in the  construction  of  the next  Fock  matrix.   The arrays
which hold the old density matrices are not filled until after iteration 2. 
For this reason they are  not  used in the damping before iteration 3.

\item Three-point interpolation of the  density  matrix.   Subroutine CNVG
monitors the number of iterations, and if this is exactly divisible by three,
and certain other conditions relating to the density  matrices are  satisfied, 
a  three-point interpolation is performed.  This is the default converger, 
and  is  very  effective  with  normally  convergent calculations.    It 
fails  in  certain  systems,  usually  those  where significant charge build-up
is present.

\item Energy-level shift technique (the \comp{SHIFT} technique).  The  virtual
M.O.\  energy  levels are  normally  shifted  to more positive energy.  This
has the effect of damping oscillations, and intrinsically divergent equations
can often be changed   to  intrinsically  convergent  form.   With 
slowly-convergent systems the virtual M.O.\  energy levels can be moved to a
more  negative value.

The precise value of the shift used depends on the behavior of  the iteration
energy.  If it is dropping, then the HOMO-LUMO gap is reduced; if the iteration
energy rises, the gap is increased rapidly.

\index{DIIS}\index{Pulay's DIIS}
\item Pulay's method.  If  requested,  when  the  largest  change  in density 
matrix elements on two consecutive iterations has dropped below \index{CNVG|ff}
0.1, then routine CNVG is abandoned in  favor  of  a  multi-Fock  matrix
interpolation.   This  relies  on  the fact that the eigenvectors of the
density and Fock matrices are identical at self-consistency, so  [P.F] =  (P.F
$-$ F.P) = 0 at  SCF.  The extent to which this condition does not occur is a
measure of  the  deviance  from  self-consistency.   Pulay's  Direct Inversion
of the Iterative Subspace (DIIS) method  uses  this relationship to calculate
that linear combination of Fock matrices which minimize  [P.F].   This  new 
Fock  matrix  is  then  used  in  the  SCF calculation.

Under certain circumstances, Pulay's method  can  cause  very  slow
convergence,   but   sometimes   it   is  the  only  way  to  achieve  a
self-consistent field.  At other times the procedure  gives  a  ten-fold
increase  in  speed,  so care must be exercised in its use.  (started by the
keyword \comp{PULAY})

\item The Camp-King converger.  If  all  else  fails,  the  Camp-King
\index{Camp-King converger} converger  is  just about guaranteed to work every
time.  However, it is time-consuming, and therefore should only be started as a
last resort.

It  evaluates  that  linear  combination   of   old   and   current
eigenvectors  which  minimize the total energy.  One of its strengths is that
systems which  otherwise  oscillate  due  to  charge  surges,  e.g.\ CHO--H, 
the C--H distance being very large, will converge using this very sophisticated
converger.
\end{enumerate}

\subsubsection{Causes of failure to achieve an SCF}
\index{SCF!failure to achieve}
In a system where a biradical can form, such as ethane  decomposing into  
two   CH$_3$  units,  the  normal  RHF  procedure  can  fail  to  go
\index{Biradicaloid character}\index{BIRADICAL}\index{UHF}\index{TRIPLET}
self-consistent.  If the system has marked biradicaloid character,  then
\comp{BIRADICAL}  or \comp{UHF} and \comp{TRIPLET} can often  prove
successful.  These options rely on the assumption that two unpaired  electrons 
can  represent  the open shell part of the wave-function.

\subsection{Use of C.I.\ in Reaction Path Calculations}
\begin{figure}
\begin{makeimage}
\end{makeimage}
\begin{center}
\includegraphics{hof}
\end{center}
\caption{\label{cich2o} Effect of C.I.\ on $\Delta H_f$ in a bond-breaking
reaction (for the reaction CH$_2$O $\rightarrow$ CHO + H)}
\end{figure}

\begin{figure}
\begin{makeimage}
\end{makeimage}
\begin{center}
\includegraphics{charges}
\end{center}
\caption{\label{cich2o2} Effect of C.I.\ on the charge  in a bond-breaking
reaction (for the reaction CH$_2$O $\rightarrow$ CHO + H)}
\end{figure}

Although closed-shell methods are suitable for normal systems, when a reaction
occurs such that a bond makes or breaks, then configuration interaction
can help in the description of the system.

Consider CH$_2$O,  with  the  interatomic  distance  between carbon and  one of
the hydrogen atoms being   steadily increased.    At   first  the  covalent 
bond  will  be  strong,  and  a self-consistent field is readily  obtained.  
Gradually  the  bond  will become  more  ionic,  and  if configuration
interaction is not used, a highly strained system will result.  This exotic
system will still have a large C--H bond order, despite the fact that the C--H
distance is very large.

To a degree,  configuration interaction can correct this picture. When
\comp{C.I.=2} is used, Figure~\ref{cich2o} and \ref{cich2o2}, a more realistic
description of the dissociation is obtained.  Now the  leaving hydrogen atom
becomes neutral as the distance increases, and the energy becomes constant at
large distances.   A C--H bond in formaldehyde is being stretched.  The effect
of C.I.\ is to make the dissociated state more realistic.  Without C.I., the
energy rises continuously, and the charge on the departing hydrogen atom
becomes unrealistic.

\subsection{Sparkles}\label{sparkles}
\index{Sparkles!description}
Four extra ``elements'' have been put into  MOPAC.   These  represent pure 
ionic  charges,  roughly  equivalent  to  the  following  chemical entities:

\index{+!sparkle}\index{++}\index{$-$}\index{$--$}
\begin{center}
\begin{tabular}{cl}
    Chemical Symbol &        Equivalent to  \\
\hline
+    & Tetramethyl ammonium radical \\
     &  Potassium atom or Cesium atom.  \\ 
++   & Barium atom.  \\
$-$  & Borohydride radical, Halogen, or  Nitrate radical  \\
$-\! -$ & Sulfate, oxalate.
\end{tabular} 
\end{center}

For  the  purposes  of  discussion  these   entities   are   called
`sparkles':  the name arises from consideration of their behavior.

\subsection*{Behavior of sparkles in MOPAC}
Sparkles have the following properties:
\begin{enumerate}
\item Their nuclear charge is integer, and is $+1$, $+2$, $-1$,  or  $-2$;
there  are  an  equivalent  number  of  electrons  to  maintain
electroneutrality, $+1$, $+2$, $-1$, and $-2$ respectively.  For example, a 
`+'  sparkle  consists  of  a  unipositive  nucleus  and  an electron.  The
electron is donated  to  the  quantum  mechanics calculation.

\item  They all have an  ionic  radius  of  $0.7$~\AA.   Any  two sparkles  of 
opposite  sign  will  form  an  ion-pair  with  a interatomic separation of
$1.4$~\AA.

\item They have a zero heat  of  atomization,  no  orbitals,  and  no
ionization potential.

\item The associated one-center two-electron integral, G$_{ss}$ is 27.21 for
all sparkles.  Because of this, the monopole-monopole interaction, $AM$, is set
to 1.0.  This is different to that in earlier (before MOPAC 6) MOPACs where
the  value of $AM$ was set to 0.5D0.
\end{enumerate}

They can be regarded as unpolarizable ions of diameter $1.4$\AA.   They do 
not  contribute  to  the  orbital count, and cannot accept or donate electrons.

Since they appear as uncharged species  which  immediately  ionize, attention 
should  be  given  to  the  charge  on the whole system.  For example, if the
alkaline metal salt of formic acid was run, the  formula would be: HCOO+ where
`+' is the unipositive sparkle.   The charge on the system would then be zero.

A water molecule polarized by a positive  sparkle  would  have  the formula
H$_2$O$^+$, and the charge on the system would be +1.

At first sight, a sparkle would appear to be  too  ionic  to  be  a point
charge and would combine with the first charge of opposite sign it encountered.

This representation is faulty, and a better description would be of an  ion, 
of diameter $1.4$\AA, and the charge delocalized over its surface.
Computationally, a sparkle is an integer  charge  at  the  center  of  a
repulsion  sphere  having the exponential form $\exp(-\alpha r)$.   The
hardness of the sphere is such that other atoms or sparkles can approach within
about $2$\AA\ quite easily, but only with great difficulty come closer than
$1.4$\AA.

\subsection*{Uses of Sparkles}
\begin{enumerate}
\item They can be used as counterions, e.g.\  for acid anions  or  for
cations.   Thus,  if  the ionic form of an acid is wanted, then the moieties
H$\cdot$X, H$\cdot -$, and $+\cdot$X could be examined.

\item Two sparkles of equal and opposite sign can form a  dipole  for
\index{Dipole!made from sparkles} mimicking solvation effects.  Thus water
could be surrounded by six dipoles to simulate the solvent cage.  A dipole of
value  D can  be made by using the two sparkles + and $-$, or using ++ and {\bf
$ --$}.  If + and $-$ are used, the inter-sparkle separation would be
$D/4.803$\AA.  If {\bf $ ++$} and {\bf $ --$} are used, the separation would be
$D/9.606$\AA.  If the inter-sparkle separation is  less than $1.0$\AA\  (a
situation that cannot occur naturally) then the energy due to the dipole on its
own is subtracted from the total energy.
\end{enumerate}

\subsection{Capped Bonds}\label{cb}\index{Capped bonds|ff}
\index{Cb}
Sometimes the system being studied is too large to be calculated using MOPAC. 
When only a part of the system is of interest, a mechanism exists to allow only
that part to be calculated, and to ignore or not consider the rest of the
system.  Capped bonds are used to satisfy valency requirements.

The procedure for using capped bonds (Cb) is as follows:
\begin{enumerate}
\item Identify all atoms which are important to the calculation.  For an
enzyme, this would be the residues of the active site, for example.
\item Identify bonds which would be broken in order to isolate the atoms of
interest. Make sure these are single bonds, and try to ensure that there is not
more than one  broken bond on any atom.  Examples of ``good'' broken bonds are:
CH$_2$--CH$_2$, CH$_2$--NH, NH--NH.
\item Attach a Cb to each atom which has a broken bond.  The Cb should be
positioned in the direction of the atom which has deleted, and should have a
bond-length of 1.7\AA , exactly. Do not mark the Cb bond length for
optimization. If two broken bonds exist on an atom, use two Cb, but make the
Cb-atom-Cb angle 109.471221$^\circ$, exactly, and do not let it optimize.
\end{enumerate}

A Cb behaves like a monovalent atom, but always has a zero charge.  In other
words, a Cb has a core charge of +1, and always has an electron population of
1.0.  Cbs  have one orbital, and so can be regarded as being hydrogen-like.

Capped bonds are different from hydrogen atoms, however, in that they have a
large $\beta$-value.  The $\beta$ value is used in the calculation of the
one-electron two-center integral. Because the $\beta$ value is so large, the
difference in electronegativity of the Cb and the atom it is attached to, $A$,
becomes negligible.   Therefore the bonding M.O. consists of $1/\sqrt{2}({\rm
Cb}+A)$.  From this, it follows that the bonding M.O. contributes 1.0 electrons
to the Cb.  

In addition to the bonding M.O., there is an antibonding M.O.  This M.O.\ is of
form $1/\sqrt{2}({\rm Cb}-A)$, and is of very high energy.

To prevent capped bonds from forming bonds to all nearby atoms, overlaps from
capped bonds to all atoms further away than 1.8\AA\ are set to zero.  Because
of this, the coupling between capped bonds and all atoms, other than $A$, is
zero. Of course, $A$ can interact with other atoms, once it has satisfied the
demands of the attached capped bond.

Because of the  huge $\beta$-value, the energy of a Cb-atom bond is very large.
To prevent this interfering with the calculation, when the electronic energy is
calculated, contributions from Cb are ignored.  For this reason it is important
that the bond-length for the Cb should not be optimized.

The M.O.\ energy levels due to Cb-type bonds are also enormous.  Before the
M.O.\ energy levels are printed in the normal output, energy levels arising
from Cb-atom bonds are first set to zero.

The electronic behavior of capped bonds can easily be studied by use of
\comp{1SCF} \comp{DENSITY} \comp{VECTORS}.

\input{t_overlap}           %  Overlap integrals
\subsection{Energies of Isolated Atoms}\index{EISOL}\index{EHEAT}\index{ATHEAT}
The $\Delta H_f$ calculated by semiempirical methods is defined as the  energy
in kcal.mol$^{-1}$ required to form one mole of the system in the gas phase at
298K from its elements in their standard state:
$$
\Delta H_f = E_{elect} + E_{nuc} + \sum_AE_{isol}(A) +\sum_AE_{atom}(A)
$$
In order to calculate $\Delta H_f$, the quantity $E_{isol}$ must be 
determined; this is the energy required to form the isolated atom from its 
valence electrons:
$$
E_{isol}(A)=E_{\rm neutral\ atom}(A) -E_{\rm nucleus}(A)-
E_{\rm valence\ electrons}(A)
$$
In the calculation of $E_{elect}$, the energy of valence electrons is  defined
as zero, likewise in calculating $E_{nuc}$, the energy of the  isolated nucleus
is defined as zero, therefore the calculation of $E_{isol}$ simplifies to the
calculation of $E_{\rm neutral\ atom}$.

The energy of $E_{eisol}$ is the energy released when the valence  electrons
are added to the nucleus.  For example, for the hydrogen atom, this would be 
$U_{ss}$.  For poly-electronic atoms, the electron-electron  interactions must
be included, in addition to the one-electron contributions. Most elements have
open shell ground states, and for these systems, the  nature of the state is
important.

For all main group elements, that is, elements with valence shell
configurations of the form $ns^anp^b$, other than the alkali metals, the value
of $E_{isol}$ is  given by:
$$
E_{isol} =aU_{ss}+bU_{pp}+(a-1)G_{ss}+a.bG_{sp}+(b(b-1))/2G_{p2}-bH_{sp}-
cH_{pp}
$$
in which $c=min(b(b-1)/2,(6-b).(5-b)/2)$.  Except for the $H_{pp}$ term, all
the contributions to $E_{isol}$ are obvious.  Non-zero $H_{pp}$ terms occur
when there are two or more unpaired electrons in the ground state, in  which
case there is an exchange stabilization that is otherwise absent.

Because $H_{pp}$ is usually written as $1/2(G_{pp}-G_{p2})$, the expression 
for systems with 2 to 4 $p$ electrons is recast as:
$$
E_{isol} =aU_{ss}+bU_{pp}+(a-1)G_{ss}+a.bG_{sp}+((b(b-1))/2+c/2)G_{p2}-
(a-1)bH_{sp}-c/2G_{pp},
$$
or
$$
E_{isol} =2U_{ss}+bU_{pp}+G_{ss}+2.bG_{sp}+((b(b-1))/2+c/2)G_{p2}-
(a-1)bH_{sp}-c/2G_{pp}.
$$
For the alkali metals, the equation for $E_{isol}$ is the same as that for 
hydrogen.

For the transition metals, the coefficients for the $d-d$ interactions are more
complicated.

The general form for $E_{isol}$ for a transition metal of configuration
$s^md^n$s, in which there are $m_a$ $\alpha$ $s$-electrons and $m_b$ $\beta$ 
$s$-electrons, and $n_a$ $\alpha$ $d$-electrons and $n_b$ $\beta$ 
$d$-electrons, and the total angular quantum number is $L$, is:
\begin{eqnarray} \nonumber
E_{isol}&=&mU_{ss}+nU_{dd}+
(m(m-1))/2G_{ss}+m.nG_{sd}-(m_an_a+m_bn_b)H_{sd}\\ \nonumber
&& +(n(n-1))/2\frac{G_{dd}^0}{5} +(-4(n_a^2+n_b^2)+13n-3/2(L(L+1)))\frac{G_{dd}^2}{49}\\ \nonumber 
&& +(-(n_a^2+n_b^2)/2-9/2n+5/6(L(L+1)))\frac{G_{dd}^4}{49}. \nonumber
\end{eqnarray}
As might be imagined, derivation of this expression is by no means  obvious,
particularly the terms for $G_{dd}^2$ and $G_{dd}^4$.  Interested readers are
referred to Racah's paper in {\it Phys Rev}, {\bf 61}, 186 (1942). In this,
Racah derived an expression for the $d$ orbital energy of the ground  state in
terms of three quantities, $A$, $B$, and $L$, the total angular momentum:
$$
<\! ^{n+1}L|H|^{n+1}L>=\frac{1}{2}n(n-1)(A-8B)+\frac{3}{2}[6n-L(L+1)]B.
$$

The quantities $A$ and $B$, and a third quantity, $C$, not used here, are
related  to the $G_k$ as follows:
\begin{eqnarray} \nonumber
A&=&G_{dd}^0-49G_{dd}^2\\ \nonumber
B&=&G_{dd}^2-5G_{dd}^4\\ \nonumber
C&=&35G_{dd}^4 \\ \nonumber
\end{eqnarray}
Using Racah's equation, derivation of  $E_{isol}$ is straightforward.  In texts
on transition metal ion theory, the quantities $G_{dd}^0$, $G_{dd}^0$, and 
$G_{dd}^0$ are usually represented by the symbols $F_0$, $F_2$, and $F_4$,
respectively.  However, care should be exercised when reading these texts:
sometimes other quantities, $F^0$, $F^2$, and $F^4$ are used.  The relationship
between these three sets of symbols is as follows:
\begin{eqnarray}\nonumber
G_{dd}^0  =  F_0 &=& F^0 \\ \nonumber
G_{dd}^2  =  F_2 &=& \frac{1}{49}F^2 \\ \nonumber
G_{dd}^4  =  F_4 &=& \frac{1}{441}F^4 \\ \nonumber
\end{eqnarray}


Because the coefficients for the two electron terms are so complicated,  values
for all elements likely to be parameterized for semiempirical  methods are
presented in Table~\ref{confs}.  From this table, the values of some 
coefficients are readily derived.  Thus for the $s$-$d$ coulomb integral,
$G_{sd}$, the coefficient is simply the number of $s$ electrons times the
number of $d$ electrons.  One $s$-$d$ exchange integral, $H_{sd}$, exists for 
each electron in the $s$ shell for which there is an electron of the  same spin
in the $d$ shell.  For elements with two $s$ electrons, this is simply the
number of $d$ electrons, for elements with one $s$ electron, the Aufbau
principle indicates that the $d$ shell with higher occupancy has the same spin
as that of the $s$ electron.  Finally, the coefficients for the  simple $d$-$d$
repulsion integral, $G_{dd}^0$, are given by the number of possible $d$-$d$
interactions.

Note also that there are no elements with both $p$ and $d$ valence electrons,
therefore terms of the type $G_{pd}$ are not necessary.

\begin{table}
\caption{\label{confs}Two Electron Energy Contributions to EISOL for Atoms in 
their Ground States}
\begin{center}
\compresstable
\begin{tabular}{cclcccccccccccc}\hline
\multicolumn{2}{c}{Element}   &  Orbital & State&  &
G$_{ss}$ & G$_{sp}$&H$_{sp}$& G$_{pp}$ &G$_{p2}$ &G$_{sd}$&
H$_{sd}$&G$_{dd}^0$&G$_{dd}^2$& G$_{dd}^4$\\ 
%                       Gss  Gsp Hsp  Gpp   Gp2  Gsd Hsd Gdd0 Gdd2 Gdd4
&   & Config.& & Mult.: & 1  & 1  &-1 & -1/2&1/2 &  1& -1/5 &1 &-1/49&-1/49\\
\hline
1&H  & $1s^1    $ &$^2S$&&    &    &   &     &    &   &   &    &    &     \\
2&He & $1s^2    $ &$^1S$&& 1  &    &   &     &    &   &   &    &    &     \\
3&Li & $2s^1    $ &$^2S$&&    &    &   &     &    &   &   &    &    &     \\
4&Be & $2s^2    $ &$^1S$&& 1  &    &   &     &    &   &   &    &    &     \\
5&B  & $2s^22p^1$ &$^2P$&& 1  & 2  & 1 &     &    &   &   &    &    &     \\
6&C  & $2s^22p^2$ &$^3P$&& 1  & 4  & 2 &  1  &  3 &   &   &    &    &     \\
7&N  & $2s^22p^3$ &$^4S$&& 1  & 6  & 3 &  3  &9   &   &   &    &    &     \\
8&O  & $2s^22p^4$ &$^3P$&& 1  & 8  & 4 &  1  &13  &   &   &    &    &     \\
9&F  & $2s^22p^5$ &$^2P$&& 1  & 10 & 5 &     &20  &   &   &    &    &     \\
10&Ne & $2s^22p^6$ &$^1S$&& 1  & 12 & 6 &     &30  &   &   &    &    &     \\
11&Na & $3s^1    $ &$^2S$&&    &    &   &     &    &   &   &    &    &     \\
12&Mg & $3s^2    $ &$^1S$&& 1  &    &   &     &    &   &   &    &    &     \\
13&Al & $3s^23p^1$ &$^2P$&& 1  & 2  & 1 &     &    &   &   &    &    &     \\
14&Si & $3s^23p^2$ &$^3P$&& 1  & 4  & 2 &  1  &3   &   &   &    &    &     \\
15&P  & $3s^23p^3$ &$^4S$&& 1  & 6  & 3 &  3  &9   &   &   &    &    &     \\
16&S  & $3s^23p^4$ &$^3P$&& 1  & 8  & 4 &  1  &13  &   &   &    &    &     \\
17&Cl & $3s^23p^5$ &$^2P$&& 1  & 10 & 5 &     &20  &   &   &    &    &     \\
18&Ar & $3s^23p^6$ &$^1S$&& 1  & 12 & 6 &     &30  &   &   &    &    &     \\
19&K  & $4s^1    $ &$^2S$&&    &    &   &     &    &   &   &    &    &     \\
20&Ca & $4s^2    $ &$^1S$&& 1  &    &   &     &    &   &   &    &    &     \\
21&Sc & $4s^23d^1$ &$^2D$&& 1  &    &   &     &    & 2 & 1 &    &    &     \\
22&Ti & $4s^23d^2$ &$^3F$&& 1  &    &   &     &    & 4 & 2 & 1  & 8  & 1   \\
23&V  & $4s^23d^3$ &$^4F$&& 1  &    &   &     &    & 6 & 3 & 3  & 15 & 8   \\
24&Cr & $4s^13d^5$ &$^7S$&&    &    &   &     &    & 5 & 5 & 10 & 35 & 35  \\
25&Mn & $4s^23d^5$ &$^6S$&& 1  &    &   &     &    &10 & 5 & 10 & 35 & 35  \\
26&Fe & $4s^23d^6$ &$^5D$&& 1  &    &   &     &    &12 & 6 & 15 & 35 & 35  \\
27&Co & $4s^23d^7$ &$^4F$&& 1  &    &   &     &    &14 & 7 & 21 & 43 & 36  \\
28&Ni & $4s^23d^8$ &$^3F$&& 1  &    &   &     &    &16 & 8 & 28 & 50 & 43  \\
29&Cu & $4s^13d^{10}$&$^2S$&&  &    &   &     &    &10 & 5 & 45 & 70 & 70  \\
30&Zn & $4s^2     $&$^1S$&& 1  &    &   &     &    &   &   &    &    &     \\
31&Ga & $4s^24p^1$ &$^2P$&& 1  & 2  & 1 &     &    &   &   &    &    &     \\
32&Ge & $4s^24p^2$ &$^3P$&& 1  & 4  & 2 &  1  &3   &   &   &    &    &     \\
33&As & $4s^24p^3$ &$^4S$&& 1  & 6  & 3 &  3  &9   &   &   &    &    &     \\
34&Se & $4s^24p^4$ &$^3P$&& 1  & 8  & 4 &  1  &13  &   &   &    &    &     \\
35&Br & $4s^24p^5$ &$^2P$&& 1  & 10 & 5 &     &20  &   &   &    &    &     \\
36&Kr & $4s^24p^6$ &$^1S$&& 1  & 12 & 6 &     &30  &   &   &    &    &     \\
37&Rb & $5s^1    $ &$^2S$&&    &    &   &     &    &   &   &    &    &     \\
38&Sr & $5s^2    $ &$^1S$&& 1  &    &   &     &    &   &   &    &    &     \\
39&Y  & $5s^24d^1$ &$^2D$&& 1  &    &   &     &    & 2 & 1 &    &    &     \\
40&Zr & $5s^24d^2$ &$^3F$&& 1  &    &   &     &    & 4 & 2 & 1  & 8  & 1   \\
41&Nb & $5s^14d^4$ &$^6D$&&    &    &   &     &    & 4 & 4 & 6  & 21 & 21  \\
42&Mo & $5s^14d^5$ &$^7S$&&    &    &   &     &    & 5 & 5 & 10 & 35 & 35  \\
43&Tc & $5s^24d^5$ &$^6S$&& 1  &    &   &     &    &10 & 5 & 10 & 35 & 35  \\
44&Ru & $5s^14d^7$ &$^5F$&&    &    &   &     &    & 7 & 5 & 21 & 43 & 36  \\
45&Rh & $5s^14d^8$ &$^4F$&&    &    &   &     &    & 8 & 5 & 28 & 50 & 43  \\
46&Pd & $5s^04d^{10}$&$^1S$&&  &    &   &     &    &   &   & 45 & 70 & 70  \\
47&Ag & $5s^14d^{10}$&$^2S$&&  &    &   &     &    &10 & 5 & 45 & 70 & 70  \\
\hline
\end{tabular}
\end{center}
\end{table}

\begin{table}
\caption{Two Electron Energy Contributions to EISOL for Atoms in their Ground 
States}
\begin{center}
\compresstable
\begin{tabular}{cclcccccccccccc} \hline
\multicolumn{2}{c}{Element}  &  Orbital & State& &
G$_{ss}$ & G$_{sp}$&H$_{sp}$& G$_{pp}$ &G$_{p2}$ &G$_{sd}$&
H$_{sd}$&G$_{dd}^0$&G$_{dd}^2$& G$_{dd}^4$\\ 
%                     Gss  Gsp Hsp  Gpp   Gp2  Gsd Hsd Gdd0 Gdd2 Gdd4
&   & Config.& & Mult.: & 1  & 1  &-1 & -1/2&1/2 &  1& -1/5 &1 &-1/49&-1/49\\
\hline
48&Cd & $5s^2     $&$^1S$&& 1  &    &   &     &    &   &   &    &    &     \\ 
49&In & $5s^25p^1$ &$^2P$&& 1  & 2  & 1 &     &    &   &   &    &    &     \\
50&Sn & $5s^25p^2$ &$^3P$&& 1  & 4  & 2 &  1  &3   &   &   &    &    &     \\
51&Sb & $5s^25p^3$ &$^4S$&& 1  & 6  & 3 &  3  &9   &   &   &    &    &     \\
52&Te & $5s^25p^4$ &$^3P$&& 1  & 8  & 4 &  1  &13  &   &   &    &    &     \\
53&I  & $5s^25p^5$ &$^2P$&& 1  & 10 & 5 &     &20  &   &   &    &    &     \\
54&Xe & $5s^25p^6$ &$^1S$&& 1  & 12 & 6 &     &30  &   &   &    &    &     \\
55&Cs & $6s^1    $ &$^2S$&&    &    &   &     &    &   &   &    &    &     \\
56&Ba & $6s^2    $ &$^1S$&& 1  &    &   &     &    &   &   &    &    &     \\
72&Hf & $6s^25d^2$ &$^3F$&& 1  &    &   &     &    & 4 & 2 & 1  & 8  & 1   \\
73&Ta & $6s^25d^3$ &$^4F$&& 1  &    &   &     &    & 6 & 3 & 3  & 15 & 8   \\
74&W  & $6s^25d^4$ &$^5D$&& 1  &    &   &     &    & 8 & 4 & 6  & 21 & 21  \\
75&Re & $6s^25d^5$ &$^6S$&& 1  &    &   &     &    &10 & 5 & 10 & 35 & 35  \\
76&Os & $6s^25d^6$ &$^5D$&& 1  &    &   &     &    &12 & 6 & 15 & 35 & 35  \\
77&Ir & $6s^25d^7$ &$^4F$&& 1  &    &   &     &    &14 & 7 & 21 & 43 & 36  \\
78&Pt & $6s^15d^9$ &$^3D$&&    &    &   &     &    & 9 & 5 & 36 & 56 & 56  \\
79&Au & $6s^15d^{10}$&$^2S$&&  &    &   &     &    &10 & 5 & 45 & 70 & 70  \\
80&Hg & $6s^2     $&$^1S$&& 1  &    &   &     &    &   &   &    &    &     \\
81&Tl & $6s^26p^1$ &$^2P$&& 1  & 2  & 1 &     &    &   &   &    &    &     \\
82&Pb & $6s^26p^2$ &$^3P$&& 1  & 4  & 2 &  1  &3   &   &   &    &    &     \\
83&Bi & $6s^26p^3$ &$^4S$&& 1  & 6  & 3 &  3  &9   &   &   &    &    &     \\
84&Po & $6s^26p^4$ &$^3P$&& 1  & 8  & 4 &  1  &13  &   &   &    &    &     \\ 
85&At & $6s^25p^5$ &$^2P$&& 1  & 10 & 5 &     &20  &   &   &    &    &     \\
\hline
\end{tabular}
\end{center}
\end{table}
             %  Energy of isolated atoms
\section{Multi-Electron Configuration Interaction}\label{meci}
\index{MECI|(}\index{Configuration interaction|(}
\index{C.I.!description} For some systems a single determinant is insufficient
to describe the electronic wave function. For example, square
\index{Cyclobutadiene!need for C.I.}\index{Ethylene, twisted!need for C.I.}
cyclobutadiene and twisted ethylene require at least two configurations to
describe their ground states. More than one configuration is also needed if an
excited state is required---the RHF SCF converges on a ground state or, if
\index{Half-electron} half-electron methods are used, on a mixture of states,
while the excited state involves a different configuration. \index{Radicals|ff}
Radicals also present a difficulty at the RHF level in that the SCF
wavefunction corresponds to an equal mixture of the two doublets, with a
corresponding error in the total energy. In order to correct for this error,
MOPAC contains   a    very   large   Multi-Electron   Configuration
Interaction  calculation,  MECI~\cite{meci} (pronounced ``me-sigh'')   which,
in addition to automatically correcting ``half-electron'' energies, allows
almost any configuration interaction calculation to be performed.  Because of
its complexity, two distinct  levels  of  input are supported; the default
values will be of use to the novice while an expert has available  an
exhaustive  set  of keywords from which a specific C.I.\ can be tailored.

MECI is a completely general C.I. The resulting states \index{Space
quantization}\index{Spin!quantization} are space and spin-quantized, there is
no restriction on total spin, the starting wavefunction can be closed or open
shell, and both even and odd electron systems are allowed, although for
simplicity in describing the method,  the starting configuration is assumed to
be closed-shell.

\subsection{Starting electronic configuration}
As MECI requires the space parts of the $\alpha$ and $\beta$ molecular orbitals
to be identical, only RHF wavefunctions are used. However, this is not a severe
restriction in  that  any  starting configuration will be supported.  Examples
of starting configurations are shown in Table~\ref{conf_meci}.

\begin{table}
\caption{\label{conf_meci}Examples of SCF configurations used in MECI}
\index{SCF! configurations}
\begin{center}
\begin{tabular}{lcc} \hline
    System    &          KeyWords used  &    Starting Configuration\\ \hline
   Methane            &     none      &        2.00 2.00 2.00 2.00  \\
   Methyl Radical     &     none      &       2.00 2.00 2.00 1.00  \\
   Twisted Ethylene   &   \comp{OPEN(2,2)}   &       2.00 2.00 1.00 1.00  \\
Twisted Ethylene Cation& \comp{OPEN(1,2)}    &     2.00 2.00 0.50 0.50  \\
   Methane Cation     &    \comp{CHARGE=1 OPEN(5,3)}&2.00 1.67 1.67 1.67  \\ \hline
\end{tabular}
\end{center}
\end{table}

\index{TRIPLET}\index{OPEN} Choice of starting configuration is  important.
For  example,  if twisted  ethylene,  a ground-state triplet, is not defined
using  \comp{OPEN(2,2)}, then  the  closed-shell  ground-state  structure
will  be calculated.   Obviously,  this configuration is a legitimate
microstate, but from the symmetry of the system a better choice would be  to
define \index{Degenerate M.O.s|ff} one electron in each of the two formally
degenerate $\pi$-type M.O.s.

Each configuration which can be generated in a molecule may be represented by
a single Slater determinant; this is called a microstate. The final states will
be linear combinations of these microstates. In general, microstates will not
be eigenfunctions of the total spin operator, but will be mixtures of different
spin states.

The initial configuration used to generate the SCF is arbitrary; for
half-electron systems it will not even correspond to a microstate, each M.O.\
having a fractional \index{Fractional!M.O.\ occupancy} electron occupancy. Even
if the starting wavefunction is a closed shell it would still correspond to
only one of a large number of microstates to be used in the MECI.\ As a result,
before the MECI is started all electronic terms arising from the electrons in
the initial configuration, which will be used by MECI, are removed. The
starting wavefunction will thus consist of a low-lying doubly occupied set of
M.O.s and a high-lying empty set of M.O.s, neither of which will be involved in
the MECI, and in between a small set of M.O.s from which the electrons have
been removed. This set of M.O.s will be involved in the MECI.

\subsection{Microstates} \index{Slater determinant|( }\index{Microstates|ff}\label{sd}
Microstates, which are normally represented by a Slater determinant, are
normally written as an antisymmetrized product of $p\  \alpha$- and $q\
\beta$-electrons:
$$
\hspace*{-0.5in} \Psi_g=[(p+q)!]^{-\frac{1}{2}}\sum_P(-1)^PP[\psi_1(1)\alpha(1)\ \psi_2(2)\alpha(2)\
\ldots \psi_p(p)\alpha(p)\ \psi_1(p+1)\beta(p+1)\ \ldots \psi_q(p+q)\beta(p+q)],
$$
where $[(p+q)!]^{-\frac{1}{2}}$ is the normalization constant, $P$ is an
operator which permutes the electron coordinates, and $(-1)^P$ assumes the
values --1 or +1 for odd and even permutations respectively.  A more compact
and useful notation for representing a  general microstate is:
$$
\Psi_j = \frac{1}{\sqrt{N!}}\sum_{P=1}^{N!}(-1)^PP(\prod_{k=1}^N\psi_k^j)
$$
where $\Psi_j$ is any microstate consisting of $N$ electrons.  Given the full
set of M.O.s, a subset of these is used in the microstate.  This subset is
defined by the M.O.s $\psi_k^j$, $k$=1,$N$.  Each microstate will consist of a
different set of M.O.s from the full set.

Rather than having all the $\alpha$ electrons appearing first in a microstate,
it is more convenient to order the one electron wavefunctions in the order in
which their indices occur in the full set of M.O.s.  If both $\alpha$ and
$\beta$ M.O.s of the same index occur, then $\alpha$ precedes $\beta$, thus:
$$
\hspace*{-0.2in}\Psi_g=[(p+q)!]^{-\frac{1}{2}}\sum_P(-1)^PP[\psi_1(1)\alpha(1)\ \psi_1(2)\beta(2)\
\psi_2(3)\alpha(3)\ \psi_2(4)\beta(4)\ \ldots  \psi_j(p+q)\beta(p+q)]
$$

This numbering scheme follows  the  Aufbau  principle, in that the  order  of
\index{Aufbau principle} filling is in order of energy.  This point is
critically  important  in deciding  the  sign of matrix elements.  For a 5
M.O.\  system, then, the order of filling is:
$$      (1)(\bar{1})(2)(\bar{2})(3)(\bar{3})(4)(\bar{4})(5)(\bar{5}) $$
A triplet state arising from two microstates, each with a component
of spin = 0, will thus be the positive combination:
$$        (\bar{1})(2)   +    (1)(\bar{2}). $$
 This standard sign convention was chosen in order to allow the signs of the
microstate coefficients  to  conform  to those resulting from the spin
step-down operator.

Only those M.O.s involved in the MECI are of interest,
thus from the full set of M.O.s, filled and empty
$$
\left[
\begin{array}{lll}
\psi_1(1)\alpha(1)&\psi_2(3)\alpha(3)&\ldots\\
\psi_1(2)\beta(2)&\psi_2(4)\beta(4)&\ldots\\
\end{array}
\right]
$$
the ground-state configuration (assumed to be closed shell for simplicity) can
be represented by
$$
\left[
\begin{array}{llllllllllllll}
1&1&1&1&\ldots&1&1&1&0&0&0&\ldots&0&0\\
1&1&1&1&\ldots&1&1&1&0&0&0&\ldots&0&0\\
\end{array}
\right]
$$
where a 1 represents a spin molecular orbital occupied by one electron and 0
represents an empty M.O.

\index{Active space!in C.I.|ff} The M.O.s involved in the C.I.\ are called the
``active space''. For convenience, the index of the M.O.\ at the lower bound of
the active space will be called ``B'', and the index of the M.O.\ at the upper
bound of the active space will be called ``A''. All M.O.s below the active
space can be considered as part of the core while those above it are empty and
can likewise be ignored. We can thus focus our attention on the M.O.s in the
active space. Most of the time, MECI calculations will  involve  between  1
and  5 M.O.s,  so  a system such as pyridine, with 15 filled levels and 29
M.O.s, would \index{Pyridine} use M.O.s 13--17 in a large C.I.

For convenience, microstates will be expressed as a sum of molecular orbital
occupancies, so that:
$$
\Psi_p=\sum_{i=B}^A(O_i^{\alpha p} + O_i^{\beta  p} ).
$$
For example, if the ground state configuration $\Psi_g$ is closed shell, then
the occupancy of the M.O.s would be
$$
O^{\alpha g} = O^{\beta  g} = |1,\ldots,1,0,\ldots,0|
$$

Microstates are particular electron configurations.  Examples of
microstates involving  5  electrons  in  5  levels are given in Table~\ref{m55}.

\begin{table}
\begin{center}
\caption{\label{m55}  Microstates for 5 electrons in 5 M.O.s}
\begin{tabular}{cccrcccc}\\ \hline
& \multicolumn{2}{c}{Electron Configuration}& & &  \multicolumn{2}{c}{Electron Configuration}& \\
\cline{2-3} \cline{6-7}
  &     Alpha   &   Beta    & M$_S$  &    &   Alpha    &  Beta  & M$_S$    \\
M.O.  &   1 2 3 4 5 &1 2 3 4 5  &       & M.O. &  1 2 3 4 5 &1 2 3 4 5\\ \hline

1 &   1,1,1,0,0 &1,1,0,0,0  & 1/2   &  4 &  1,1,1,1,1 &0,0,0,0,0 &   5/2\\
2 &   1,1,0,0,0 &1,1,1,0,0  &-1/2   &  5 &  1,1,0,1,0 &1,1,0,0,0 &   1/2\\
3 &   1,1,1,0,0 &0,0,0,1,1  & 1/2   &  6 &  1,1,0,1,0 &1,0,1,0,0 &   1/2 \\ \hline
\end{tabular}
\end{center}
\end{table}
\index{Slater determinant|)}

\subsubsection{Permutations}
For  5  electrons  in  5   M.O.s   there   are   252   microstates
($10!/(5!5!)$),  but as states of different spin do not mix, we can use a
smaller  number.   If  doublet  states  are  needed, then   100   states
($5!/(2!3!)(5!/3!2!$)  are  needed.   If  only  quartet  states  are of
\index{Quartet states} interest, then 25 states ($5!/(1!4!)(5!/4!1!$) are
needed  and  if  the \index{Sextet states} sextet state is required, then only
one state is calculated.

In  the  microstates  listed,   state   1   is   the   ground-state
configuration.   This can be written as (2,2,1,0,0), meaning that M.O.s 1 and 2
are doubly occupied, M.O.\  3 is  singly  occupied  by  an  alpha electron, and
M.O.s 4 and 5 are empty.  Microstate 1 has a component of \index{Kramer's
degeneracy} spin of 1/2, and is a pure doublet.  By Kramer's
degeneracy---sometimes called time-inversion symmetry---microstate 2 is also a
doublet, and has a spin of 1/2 and a component of spin of $-1/2$.

Microstate 3, while it has a component of spin of  1/2,  is  not  a doublet,
but  is  in  fact  a  component  of a doublet, a quartet and a sextet.  The
coefficients of these states can  be  calculated  from  Wigner's symbol, also
called the \index{Clebsch-Gordon 3-J symbol}\index{Wigner's symbol}\index{3-J
symbol} Clebsch-Gordon  3-J  symbol\footnote{\samepage The symbol is of form
\begin{eqnarray}
\hspace*{-0.5in}<j_1j_2m_1m_2|j_1j_2jm>&=&
\left \{\frac{(j+m)!(j-m)!(j_1-m_1)!(j_2-m_2)!(j_1+j_2-j)!(2j+1)}
{(j_1+m_1)!(j_2+m_2)!(j_1-j_2+j)!(j_2-j_1+j)!(j_1+j_2+j+1)!}\right \}^{\frac{1}{2}}
\nonumber \\
&&\delta(m,m_1+m_2)\sum_r(-1)^{j_1+r-m_1}\frac{(j_1+m_1+r)!(j_2+j-r-m_1)!}
{r!(j-m-r)!(j_1-m_1-r)!(j_2-j+m_1+r)!} \nonumber
\end{eqnarray}
where the summation is over all values of r such that all factorials occurring
are of non-negative integers (0!=1). See~\cite{griffith}. To use the symbol,
the coefficient of momentum $(j,m)$ due to two momenta $(j_1,m_1)$ and
$(j_2,m_2)$ is
$<j_1j_2m_1m_2|j_1j_2jm>$
}.
Thus, the coefficient in the doublet is $\sqrt{1/2}$ ($j_1=3/2,\; m_1=3/2,
\;j_2=1, \;m_2=-1, \;j=1/2$), in the quartet is $\sqrt{4/10}$ ($j_1=3/2,\;
m_1=3/2, \;j_2=1, \;m_2=-1, \;j=3/2$), and in the sextet,  $\sqrt{1/10}$
($j_1=3/2,\; m_1=3/2, \;j_2=1, \;m_2=-1, \;j=5/2$).

Microstate 4 is a pure sextet.  If all 100 microstates of component of  spin
=  1/2  were used in a C.I., one of the resulting states would have the same
energy as the state resulting from microstate 4.

Microstate 5 is an excited doublet, and microstate 6 is an  excited state of
the system, but not a pure spin-state.

By default, if $n$ M.O.s are included in the MECI, then all possible
microstates which give rise to a component of spin = 0 for even electron
systems, or 1/2 for odd electron systems, will be used.

\index{Microstates!number used in C.I.}\label{nmeci}
\begin{table}
\caption{\label{setmic}Sets of Microstates for Various MECI Calculations}
\begin{center}
\begin{tabular}{|rccr|ccr|}
\hline
&\multicolumn{2}{c}{Odd Electron Systems} &  & \multicolumn{2}{c}{Even Electron Systems}&\\ \hline
        &    Alpha   Beta &&No. of   &    Alpha   Beta &&No. of\\
        &                 &&Configs. &                 &&Configs.\\\hline
 C.I.=1&(1,1) $\times$ (0,1) &=&  1    &     (1,1) $\times$ (1,1)&=&   1  \\
      2&(1,2) $\times$ (0,2) &=&  2    &     (1,2) $\times$ (1,2)&=&   4  \\
      3&(2,3) $\times$ (1,3) &=&  9    &     (2,3) $\times$ (2,3)&=&   9  \\
      4&(2,4) $\times$ (1,4) &=& 24    &     (2,4) $\times$ (2,4)&=&  36  \\
      5&(3,5) $\times$ (2,5) &=&100    &     (3,5) $\times$ (3,5)&=& 100  \\\hline
\end{tabular}\\
($n$,$m$) means $n$ electrons in $m$ M.O.s.
\end{center}
\end{table}

\index{MECI!Increasing number of states} MOPAC is configured to  allow a
maximum of \comp{MAXCI} states, where \comp{MAXCI} is defined in the file
\comp{meci.h}.  If more states are needed (see Table~\ref{setmic}), then
\comp{MAXCI} in  \comp{meci.h} should be modified. Of course, if \comp{MAXCI}
is changed, MOPAC should be recompiled.

If \comp{CIS}, \comp{CISD}, or \comp{CISDT} are specified, then the number of
microstates is defined by \comp{C.I.=$k$} and the keyword.   The number of
microstates is a function of $k$.  Let $n$ and $m$ be integers, such that:
$$
n=\frac{k}{2}
$$
$$
m=\frac{k+1}{2}
$$
If $k$ is odd, then round down to the next lower integer.  Then the number
of microstates $n_{CIS}$, $n_{CISD}$, and $n_{CISDT}$, for even-electron
systems is:
$$
\begin{array}{lcll}
n_{CIS}&=& &2nm \nonumber \\
n_{CISD}&=& 1 + &2nm + (nm)^2 + \frac{n!m!}{2(n-2)!(m-2)!}\nonumber  \\
n_{CISDT} &=& 1 + &2nm + (nm)^2 + \frac{n!m!}{2(n-2)!(m-2)!} +
\frac{n!m!}{18(n-3)(m-3)}+\frac{nm\times n!m!}{2(n-2)!(m-2)!}\nonumber
\end{array}
$$

Note that when \comp{CIS} is used, the ground state is {\em not} included in
the list of microstates. Values for the more important $k$ are given in
Table~\ref{micsdt}.
\begin{table}
\caption{\label{micsdt}Number of Microstates for CIS, CISD, and CISDT}
\begin{center}
\begin{tabular}{|crcrcrc|}
\hline
C.I.=$k$ & CIS &  & CISD &  & CISDT & \\ \hline
   1     &  0  &    &  1   &    &   1   &    \\
   2     &  2  &    &  4   &    &   4   &    \\
   3     &  4  &    &  9   &    &   9   &    \\
   4     &  8  &    &  27  &    &  35   &    \\
   5     & 12  &    &  55  &    &  91   &    \\
   6     & 18  &    &  118 &    & 282   &    \\
   7     & 24  &    &  205 &    & 635   &    \\
   8     & 32  &    &  361 &    & 1545  &    \\
\hline
\end{tabular}\\
(for even electron systems only)
\end{center}
\end{table}
\subsubsection{Energy of microstates}
The electronic energy, $E_r$, of any microstate $\Psi_r$ is the sum
\index{Electronic energy!of microstates}
on the one and two-electron energies:
$$
E_r = \sum_i^pH_{ii}+\sum_i^qH_{ii}+\frac{1}{2}\sum_{ij}^p(J_{ij}-K_{ij})
+\frac{1}{2}\sum_{ij}^q(J_{ij}-K_{ij}) +\sum_i^p\sum_j^qJ_{ij}
$$
where $H_{ii}$ is the one-electron energy of M.O.\ $\psi_i$,
$J_{ii}$ is the two-electron Coulomb integral\\
$<\psi_i\psi_i|\psi_j\psi_j>$,
and $K_{ii}$ is the two-electron exchange integral  $<\psi_i\psi_j|\psi_i\psi_j>$.

In this section it is more convenient to express it in terms
of molecular orbital occupancies:
$$
E_r = \sum_{i=B}^AH_{ii}(O_i^{\alpha r}+ O_i^{\beta r})
+\sum_{ij=B}^A(\frac{1}{2}(J_{ij}-K_{ij})
(O_i^{\alpha r}O_j^{\alpha r}+ O_i^{\beta r}O_j^{\beta r})
+J_{ij}O_i^{\alpha r}O_j^{\beta r})
$$
Similarly, the orbital energies can be written
\index{Orbital!energies in C.I.}
$$
\epsilon_{ii}^{\alpha r} = H_{ii}+\sum_j^p(J_{ij}-K_{ij})+\sum_j^qJ_{ij}
$$
or, in terms of orbital occupancies
$$
\epsilon_{ii}^{\alpha r} = H_{ii}+\sum_{j=B}^A(J_{ij}-K_{ij})O_j^{\alpha r}
+\sum_{j=B}^AJ_{ij}O_j^{\beta r}.
$$
\subsubsection{Zero of energy used in MECI}
The energy of the system after all the electronic terms arising from the
electrons of the M.O.s involved in the starting configuration are removed
is a useful quantity. Removal of these terms lowers the orbital
energies thus:
$$
\epsilon_{ii}^+ = \epsilon_{ii} -\sum_{j=B}^A(J_{ij}-K_{ij})O_j^g.
$$
The arbitrary zero of energy in a MECI calculation is the  starting ground
state, without any correction for errors introduced by the use of fractional
occupancies.  In order to calculate the energy of the various configurations,
the  energy  of  the  vacuum  state  (i.e.,  the  state resulting from removal
of the electrons used in the C.I.)  needs  to  be evaluated.  This energy is
given by:
$$
GSE=E_g^+ = - \sum_{i=B}^A2\epsilon_{ii}^+O_i^g+J_{ii}(O_i^g)^2+
\sum_{i=B}^A\sum_{j=B}^{i-1}2(2J_{ij}-K_{ij})O_i^gO_j^g
$$
\index{GSE@{$GSE$, used in MECI}}
(Within the MECI routine, $GSE$ refers to $E_g^+$.)

By redefining the system so that those filled M.O.s which are not used in the
MECI are considered part of an \index{Unpolarizable core!used in C.I.}
unpolarizable core, the new energy levels $\epsilon_i^+$ can be identified with
the one-electron energies $H_{ii}$ and the total electronic energy $E_r$ of any
microstate is set equal to the sum of the energy of the electrons considered in
the microstate plus $E_g^+$.

\subsection{Construction of secular determinant}
\index{Secular determinant!in MECI}
Microstates can be generated by permuting available electrons among the
available levels. Elements of the C.I.\ matrix are then defined by
$$
<\Psi_a|H|\Psi_b>
$$
Evaluation of these matrix elements is difficult. Each microstate is a Slater
determinant, and the Hamiltonian operator involves all electrons in the system.
Fortunately, most matrix elements are zero because of the orthogonality of the
M.O.s. Only the non-zero elements need be evaluated; three types of interaction
are possible:
\begin{enumerate}
\item $\Psi_a=\Psi_b$. Since the two wavefunctions are the same,
this corresponds to the energy of a microstate. As the
electronic energy of the closed shell is common to all
configurations considered in the C.I., it is sufficient to
add on to $E_g^+$ the energy terms which are specific to the
microstate, thus
\begin{eqnarray}
<\Psi_a|H|\Psi_b>& =& E_g^+ +\sum_{i=B}^A\left(\epsilon_{ii}+\sum_{j=B}^A(J_{ij}-K_{ij})
O_j^{\alpha p} \right )O_i^{\alpha p} \nonumber  \\&&
+\sum_{i=B}^A\left(\epsilon_{ii}+\sum_{j=B}^A(J_{ij}-K_{ij}) O_j^{\beta p}
\right )O_i^{\beta p} + \sum_{i=B}^A\sum_{j=B}^AJ_{ij}O_i^{\alpha p}
O_j^{\beta p}. \nonumber
\end{eqnarray}
\item \label{b} Except for $\psi_i$ in $\Psi_a$ and  $\psi_j$ in $\Psi_b$;
$\Psi_a = \Psi_b$. Assuming $\psi_i$  and $\psi_j$  to be $\alpha$-spin the
interaction energy is
$$
<\Psi_a|H|\Psi_b> = (-1)^W(\epsilon_{ij}^++\sum_{k=B}^A(<ij|kk>-<ik|jk>)O_k^{\alpha a}
+<ij|kk>O_k^{\beta a}).
$$
This presents a problem. Unlike $\epsilon_{ii}^+$, which has already been
defined, there is no easy way to calculate $\epsilon_{ij}^+$. Rather than
undertake this calculation, use can be made of the fact that, for the starting
configuration:
$$
\epsilon_{ij} = <\psi_i|H|\psi_j> = H_{ij} +
\sum_{k=B}^A(<ij|kk>-<ik|jk>)O_k^{\alpha g}
+<ij|kk>O_k^{\beta g}
$$
or
$$
\epsilon_{ij} = \epsilon_{ij}^+ + \sum_{k=B}^A(<ij|kk>-<ik|jk>)O_k^{\alpha g}
+<ij|kk>O_k^{\beta g} .
$$
$\epsilon_{ij}$ corresponds to an off-diagonal term in the Fock matrix,
which at self-consistency is, by definition, zero.
Therefore:
$$
\epsilon_{ij}^+ = - \sum_{k=B}^A(<ij|kk>-<ik|jk>)O_k^{\alpha g}
+<ij|kk>O_k^{\beta g}  ,
$$
which can be substituted directly into the expression for
$<\Psi_a|H|\Psi_b>$ to give
$$
<\Psi_a|H|\Psi_b> = (-1)^W\sum_{k=B}^A(<ij|kk>-<ik|jk>)(O_k^{\alpha a}-O_k^{\alpha g})
+<ij|kk>(O_k^{\beta a} -O_k^{\beta g} ).
$$
All that remains is to determine the phase factor. One
of the microstates is permuted until the two unmatched M.O.s
occupy the same position. The number of permutations needed
to do this when the two M.O.s are of $\alpha$ spin is
$$
W=\sum_{k=i+1}^{j-1}(O_k^{\alpha p}-O_k^{\beta p}),
$$
assuming $j > i$; otherwise:
$$
W=O_j^{\alpha p} +\sum_{k=i+1}^{j-1}(O_k^{\alpha p}-O_k^{\beta p}).
$$

\item Except for $\psi_i$ and $\psi_j$ in $\Psi_a$ and $\psi_k$ and
$\psi_l$ in $\Psi_b$; $\Psi_a = \Psi_b$.
 Two situations exist: (a) when all four M.O.s are of
the same spin; and (b) when two are of each spin. Thus,
\begin{enumerate}
\item All four M.O.s are of the same spin. The
interaction energy is
$$
<\Psi_a|H|\Psi_b> = (-1)^W[<ik|jl>-<il|jk>],
$$
in which the phase factor is:
$$
W=\sum_{m=i+1}^{j-1}(O_m^{\alpha a}-O_m^{\beta a})
+ \sum_{m=k+1}^{l-1}(O_m^{\alpha a}-O_m^{\beta a})+O_i^{\beta a} + O_k^{\beta a},
$$
if the four M.O.s are of $\alpha$ spin; otherwise,
$$
W=\sum_{m=i+1}^{j-1}(O_m^{\alpha a}-O_m^{\beta a})
+ \sum_{m=k+1}^{l-1}(O_m^{\alpha a}-O_m^{\beta a})+O_j^{\beta a} + O_l^{\beta a}.
$$
\item Two M.O.s are of each spin. In this case there is
no exchange integral, therefore the interaction energy is
$$
<\Psi_a|H|\Psi_b> = (-1)^W<ik|jl>
$$
and the phase factor is:
$$
W=\sum_{m=k}^{i}(O_m^{\alpha a}-O_m^{\beta a})
+ \sum_{m=j}^{l}(O_m^{\alpha a}-O_m^{\beta a}).
$$
If $i>k$, then $W=W+O_k^{\alpha a}+O_i^{\beta a}$,
if $j>l$, then $W=W+O_k^{\alpha b}+O_i^{\beta b}$,
finally, if $i>k$ and $j>l$ or  $i<k$ and $j<l$, then $W=W+1$.

All other matrix elements are zero. The completed secular determinant is then
diagonalized. This yields the \index{State!vectors}\index{State!energies} state
vectors and state energies, relative to the starting configuration. In turn,
the state vectors can be used to generate  spin density (at the RHF level) for
pure spin states. If the density matrix for the state is of interest, such as
in the calculation of transition dipoles for vibrational modes of excited or
open shell systems, or for other use, the perturbed density matrix is automatically  reconstructed.
\index{Density matrix!reconstruction in MECI}
\end{enumerate}
\end{enumerate}
\subsection{Atom Transition Moments}\index{Transition!dipole moments}\index{Polarizability!atomic transition}
\index{Dipole!transition}\index{Exponents!transition dipole}
\index{Polarizability}\label{oscil}
\index{Oscillator}
A system can go from the ground state to an excited  state as the result of the
absorption of a photon.  The probability of this happening, $\kappa$, is
given\footnote{\samepage Wilson Decius and Cross,  ``Molecular Vibrations'', p
163, McGraw-Hill (1955)} in terms of the  oscillator integral:
\begin{eqnarray}
<\!\Psi_{0}|\stackrel{\rightharpoonup}{r}|\Psi_{*}\!> \label{os1},
\end{eqnarray}
by
$$
\kappa = \frac{8\pi^3}{3ch}\nu_{n'n''}(N_{n'}-N_{n''})
<\!\Psi_{0}|\stackrel{\rightharpoonup}{r}|\Psi_{*}\!>^2 .
$$
For electronic photoexcitations, $\Psi_A$ are state functions:
$$
\Psi_A = \sum_ic_i\Psi_i,
$$
and the $\Psi_i$ are microstates; see p.~\pageref{sd} for a
definition of microstates.
\subsubsection*{Some Mathematical tools}
In order to evaluate \ref{os1}, a property of integrals of the type:
$$
<\!\psi_{i}|\stackrel{\rightharpoonup}{r}|\psi_{j}\!>
$$
will be used several times.  This property is:
$$
<\!\psi_{i}|\stackrel{\rightharpoonup}{r}|\psi_{i}\!> = 0.
$$
From this, it follows that, if
$$
<\!\psi_{j}|\stackrel{\rightharpoonup}{r}|\psi_{i}\!> \neq 0,
$$
then
$$
<\!\psi_{i}|\stackrel{\rightharpoonup}{r}|\psi_{j}\!> =
-<\!\psi_{j}|\stackrel{\rightharpoonup}{r}|\psi_{i}\!> .
$$
To prove this relationship, consider the integral
$$
<\!(\psi_i+\psi_j)|\stackrel{\rightharpoonup}{r}|(\psi_i+\psi_j)\!>.
$$
Obviously, this integral has a value of zero, therefore
$$
<\!\psi_i|\stackrel{\rightharpoonup}{r}|\psi_i\!>  +
<\!\psi_j|\stackrel{\rightharpoonup}{r}|\psi_i\!>  +
<\!\psi_i|\stackrel{\rightharpoonup}{r}|\psi_j\!>  +
<\!\psi_j|\stackrel{\rightharpoonup}{r}|\psi_j\!>  =0.
$$
In this expression, the first and fourth terms are obviously zero, therefore
$$
<\!\psi_j|\stackrel{\rightharpoonup}{r}|\psi_i\!>  =
-<\!\psi_i|\stackrel{\rightharpoonup}{r}|\psi_j\!> .
$$

\subsubsection{Evaluation of Transition Dipole}

\begin{table}
\caption{\label{transx} ``$x$" Transition Integrals}
\begin{center}
\begin{tabular}{l|ccccccccc} \hline
& $s$  &  $p_x$  &  $p_y$  &  $p_z$  &  $d_{x^2-y^2}$  & $d_{xz}$  &
$d_{z^2}$  &  $d_{yz}$  &  $d_{xy}$ \\ \hline
$s$ & X$_A$\\
$p_x$ & sp & X$_A$\\
$p_y$  & 0 & 0 & X$_A$ \\
$p_z$  & 0 & 0 & 0 & X$_A$\\
$d_{x^2-y^2}$ & 0 & pd & 0 & 0 & X$_A$\\
$d_{xz}$      & 0 & 0 & 0 & pd & 0 & X$_A$\\
$d_{z^2}$     & 0 & -$\frac{1}{\sqrt{3}}$pd & 0 & 0 & 0 & 0 & X$_A$\\
$d_{yz}$      & 0 & 0 & 0 & 0 & 0 & 0 & 0 & X$_A$\\
$d_{xy}$      & 0 & 0 & pd & 0 & 0 & 0 & 0 & 0 & X$_A$\\  \hline


\end{tabular}\\
\hspace{-0.3in}Note: X$_A$ = $<\! \phi_{\lambda}|\stackrel{\rightharpoonup}{x}|\phi_{\lambda}\! >$;
sp = $<\! ns|\stackrel{\rightharpoonup}{r}|np\! >$; pd = $<\! np|\stackrel{\rightharpoonup}{r}|nd\! >$ (see below).
\end{center}
\end{table}

The probability, $B_{0\rightarrow *}$, that a photon will be absorbed by a system that has a
ground state $\Psi_0$ and an excited state $\Psi_*$ separated by an energy $\epsilon$ when
irradiated by an energy density $\rho_{\epsilon}$
is given by
$$
B_{0\rightarrow *} = \frac{2\pi}{3\hbar^2}|R_{0*}|^2\rho_{\epsilon},
$$
in which
$$
|R_{0*}|^2 = |X_{0*}|^2 + |Y_{0*}|^2 + |Z_{0*}|^2.
$$

$X_{0*}$ is the matrix element for the $x$ component of the dipole moment:
$$
X_{0*} = \int \Psi_0|e\sum_j \stackrel{\rightharpoonup}{x_j} |\Psi_* d\tau.
$$

Evaluation of this integral requires evaluating the effect of the operators  $\stackrel{\rightharpoonup}{x_j}$,
$\stackrel{\rightharpoonup}{y_j}$, and $\stackrel{\rightharpoonup}{z_j}$
acting on an atomic orbital.  Tables \ref{transx}, \ref{transy}, and \ref{transz} show
the integrals of the type $<\! \phi_{\lambda}
 |\stackrel{\rightharpoonup}{r_j}|\phi_{\sigma}\!>$, where $\phi_{\lambda}$ and $\phi_{\sigma}$ are
 pairs of atomic orbitals.


The integral
 $<\! \phi_{\lambda}|\stackrel{\rightharpoonup}{r}|\phi_{\lambda}\! >$ is simply the appropriate Cartesian coordinate,
that is, the $x$, $y$, or $z$ coordinate of the atom that $\phi_{\lambda}$ is on.

\begin{table}
\caption{\label{transy} ``$y$" Transition Integrals}
\begin{center}
\begin{tabular}{l|ccccccccc} \hline
& $s$  &  $p_x$  &  $p_y$  &  $p_z$  &  $d_{x^2-y^2}$  & $d_{xz}$  &
$d_{z^2}$  &  $d_{yz}$  &  $d_{xy}$ \\ \hline
$s$ & Y$_A$\\
$p_x$ & 0 & Y$_A$\\
$p_y$  & sp & 0 & Y$_A$ \\
$p_z$  & 0 & 0 & 0 & Y$_A$\\
$d_{x^2-y^2}$ & 0 & 0 & -pd & 0 & Y$_A$\\
$d_{xz}$      & 0 & 0 & 0 & 0 & 0 & Y$_A$\\
$d_{z^2}$     & 0 & 0 & -$\frac{1}{\sqrt{3}}$pd & 0 & 0 & 0 & Y$_A$\\
$d_{yz}$      & 0 & 0 & 0 & pd & 0 & 0 & 0 & Y$_A$\\
$d_{xy}$      & 0 & pd & 0 & 0 & 0 & 0 & 0 & 0 & Y$_A$\\  \hline


\end{tabular}\\
\end{center}
\end{table}


 $<\! ns|\stackrel{\rightharpoonup}{r}|np\! >$
 and  $<\! np|\stackrel{\rightharpoonup}{r}|nd\! >$
 can be evaluated using the following expressions:

$$
<\! ns|\stackrel{\rightharpoonup}{r}|np\! > = a_0\frac{(2n+1).2^{2n+1}.(\xi_s\xi_p)^{n+1/2}}{\sqrt{3}(\xi_s+\xi_p)^{2n+2}}
$$

$$
<np|\stackrel{\rightharpoonup}{r}|nd>=a_0\frac{(n_p+n_d+1)!.2^{n_p+n_d+1}.\xi_p^{n_p+1/2}.\xi_d^{n_d+1/2}}
{\sqrt{5}(\xi_p+\xi_d)^{n_p+n_d+2}.\sqrt{(2n_p)!(2n_d)!}},
$$
where $ns$, $np$, and $nd$ are $s$, $p$, and $d$ quantum numbers, respectively.
For the $sp$ transition, $n=ns=np$.
The Slater orbital exponents, $\xi_s$, $\xi_p$, and $\xi_d$, are usually
given in atomic units, that is, in
inverse Bohr, therefore they must converted to \AA ngstroms before use, hence the
presence of the $a_0=0.529$ in these expressions.

All integrals of the type used here are in \AA ngstroms, therefore the units of the integral
of the dipole operator on a M.O. is also in \AA ngstroms:
$$
<\! \psi_i|\stackrel{\rightharpoonup}{r}|\psi_j\! > =\sum_{\lambda}\sum_{\sigma}c_{\lambda i}c_{\sigma j}
<\! \phi_{\lambda}\stackrel{\rightharpoonup}{r}\phi_{\sigma} \!>,
$$
 Once the value of the integral is known,
the phase to be used must be determined. The simplest way to achieve this
is to reverse the sign of the oscillator whenever the second M.O.\ has a
higher index than the first.

\begin{table}
\caption{\label{transz} ``$z$" Transition Integrals}
\begin{center}

\begin{tabular}{l|ccccccccc} \hline
& $s$  &  $p_x$  &  $p_y$  &  $p_z$  &  $d_{x^2-y^2}$  & $d_{xz}$  &
$d_{z^2}$  &  $d_{yz}$  &  $d_{xy}$ \\ \hline
$s$ & Z$_A$\\
$p_x$ & 0 & Z$_A$\\
$p_y$  & 0 & 0 & Z$_A$ \\
$p_z$  & sp & 0 & 0 & Z$_A$\\
$d_{x^2-y^2}$ & 0 & 0 & 0 & 0 & Z$_A$\\
$d_{xz}$      & 0 & pd & 0 & 0 & 0 & Z$_A$\\
$d_{z^2}$     & 0 & 0 & 0 & $\frac{2}{\sqrt{3}}$pd & 0 & 0 & Z$_A$\\
$d_{yz}$      & 0 & 0 & pd & 0 & 0 & 0 & 0 & Z$_A$\\
$d_{xy}$      & 0 & 0 & 0 & 0 & 0 & 0 & 0 & 0 & Z$_A$\\  \hline
\end{tabular}\\
\end{center}
\end{table}


  Evaluation of the integrals over microstates is straightforward,
in that all integrals are zero, unless the number of differences between the microstates is exactly two,
in which case the integral is equal to that of the two M.O.s involved, times a phase factor.  That is,
for each pair of microstates that are identical, except for $\psi_i$
in $\Psi_a$ and  $\psi_j$ in $\Psi_b$, the integral is:.

$$
<\! \Psi_a|\stackrel{\rightharpoonup}{r}|\Psi_b\! > =<\! \psi_i|\stackrel{\rightharpoonup}{r}|\psi_j\! >*(-1)^n,
$$
where $n$ is the number of permutations necessary to move $\psi_i$ in microstate $\Psi_a$ to the position
occupied by $\psi_j$ in microstate $\Psi_b$. This is
similar to the `b' option on page~\pageref{b}.  As with
the molecular orbitals, the oscillators for microstates change sign
when the order of the microstates is reversed.  The simplest way to
achieve this is to use the same device that was used with the M.O.s;
that is, to reverse the sign of the oscillator whenever the second
microstate has a higher index than the first.


For completeness, the sign of $R_{0*}$ should be reversed if $k>l$, but
since only the modulus is used, this operation does not need to be done.


Finally, the state transition dipole can be calculated from:
$$
<\! \Psi_A|\stackrel{\rightharpoonup}{r}|\Psi_B\! > =\sum_a\sum_bc_{A a}c_{B b}<\! \Psi_a\stackrel{\rightharpoonup}{r}\Psi_b \!>,
$$

Although the transition dipole is normally regarded as involving the ground and an excited state, it is
possible to calculate the transition between two excited states.  The initial state is, by default, the
ground state, however if \comp{ROOT=n} $n\neq 1$, or any other keyword that specifies
a state other than the ground state, then the initial state will be an excited state.

For degenerate states, the transition dipole is the sum over all states involved.

\subsection{States arising from various calculations}
Each MECI calculation invoked by use of the keyword C.I.=$n$ normally
\index{Spin!quantization|ff} gives  rise to states of quantized spins.  When
C.I.\ is used without any other modifying keywords, the states shown in
Table~\ref{sq} will be obtained. These numbers of spin states will be obtained
irrespective  of  the chemical nature of the system.

\begin{table}
\caption{\label{sq} States arising from \comp{C.I.=$n$}}
\begin{center}
\begin{tabular}{crrrrrr} \hline
No.\ of M.O.s & \multicolumn{3}{c}{States Arising from} &
\multicolumn{3}{c}{States Arising from}\\ in MECI &\multicolumn{3}{c}{Odd
Electron Systems}& \multicolumn{3}{c}{Even Electron Systems}\\ \hline
\index{States!arising from C.I.}
 & Doublets &Quartets&Sextets &Singlets& Triplets&Quintets
\\ \hline
1 & 1 &   & &1\\
2& 2 &   &   &3& 1\\
3& 8 & 1 & &6& 3 \\
4&20 & 4 & & 20&15 &1\\
5&75 &24 & 1& 50&  45&  5 \\ \hline
\end{tabular}
\end{center}
\end{table}

\subsection{Spin angular momentum}
State functions are eigenvalues of the $S_z$ and $S^2$ operators. The
derivation of the expectation value of the $S^2$ operator is given in this
section.

The fundamental spin operators have the following \index{Spin!operators}
effects:
$$
\begin{array}{ll}
S_x\alpha = \frac{1}{2}\beta & S_x\beta =\ \  \frac{1}{2}\alpha \\
S_y\alpha = \frac{i}{2}\beta & S_y\beta = -\frac{i}{2}\alpha \\
S_z\alpha = \frac{1}{2}\alpha & S_z\beta = -\frac{1}{2}\beta \\
\end{array}
$$
Using these expressions, various useful identities can be
established:
\index{Shift operators}
$$
\begin{array}{ll}
S^2 = S_x^2+S_y^2+S_z^2 \\
I^+ = (S_x+iS_y);& I^+\beta = \alpha \\
I^- = (S_x-iS_y);& I^-\alpha = \beta
\end{array}
$$
\begin{eqnarray}
S_x^2+S_y^2&=&(I^+I^-)+i(S_xS_y-S_yS_x)\nonumber \\
           &=&(I^-I^+)+i(S_yS_x-S_xS_y)\nonumber \\
           &=&\frac{1}{2}(I^+I^- + I^-I^+)\nonumber
\end{eqnarray}
and finally $i(S_yS_x - S_x S_y ) = S_z$.

For any microstate $\Psi$, the expectation value of the $S^2$ operator is
given by
$$
<S^2>=<\Psi|S_z^2+S_y^2+S_x^2|\Psi>.
$$
The first part of this expression is obvious, {\em vis}:
$$
<\Psi|S_z^2|\Psi> = \frac{1}{4}(N^{\alpha}+N^{\beta}).
$$
However, the effect of $S_y^2+S_x^2$ is not so simple. By making use of the
fact that the operators involve two electrons, a large number of integrals
resulting from the expansion of the Slater determinants can be readily
eliminated. The only integrals which are not zero due to the orthogonality of
the eigenvectors, i.e., those which may be finite due to the spin operators,
are
$$
<\Psi|S_y^2+S_x^2|\Psi>  = 2\sum_{i<j}[<\psi_i\psi_i|S_y^2+S_x^2|\psi_j\psi_j>-
<\psi_i\psi_j|S_y^2+S_x^2|\psi_i\psi_j>].
$$
Using the relationships already defined, this expression
simplifies~\cite{delaat} as follows:
$$
S_1S_2=S_{1z}S_{2z}+\frac{1}{2}(I_1^+I_2^-+I_1^-I_2^+)
$$
$$
<\Psi|S^2|\Psi> = 2\sum_{i<j}[\frac{1}{4}(2\delta(m_{s_i}m_{s_j}
-1-\frac{1}{2}(1-\delta(m_{s_i}m_{s_j}))<\psi_i\psi_j>^2]
$$
 or,
$$
<\Psi|S^2|\Psi> =\frac{3(p+q)}{4}+\frac{p(p-1)}{2}+\frac{q(q-1)}{2}
-\frac{(p+q)(p+q-1)}{4}-\sum_{ij}^{pq}<\psi_i\psi_j>^2.
$$
Recall that $p$ is the number of $\alpha$ electrons, and $q$, the number of
$\beta$ electrons.  This expression simplifies to yield
$$
<\Psi|S^2|\Psi> =\frac{1}{2}(p+q)+\frac{1}{4}(p-q)^2-
\sum_i^p\sum_j^q<\psi_i\psi_j>^2.
$$
For the general case, in which the state function $\Phi$, is a
\index{State!function|ff}
linear combination of microstates, the expectation value of
S is more complicated:
$$
<\Phi_k|S^2|\Phi_k> = \sum_i\sum_jC_{ik}C_{jk}<\Psi_i|S^2|\Psi_j> .
$$
As with the construction of the C.I.\ matrix, the elements of this expression
can be divided into a small number of different types:
\begin{enumerate}
\item $\Psi_a=\Psi_b$: Since the two wavefunctions are the same, this
corresponds to the expectation value of a microstate, and has already been
derived.

\item Except for $\psi_i$ in $\Psi_a$ and $\psi_j$ in $\Psi_b$; $\Psi_a
=\Psi_b$:  Assuming $\psi_i$ and $\psi_j$ to have alpha-spin the expectation
value is
$$
<\Psi_a|S_y^2+S_x^2|\Psi_b>=
 (-1)^W\sum_{k=B}^A(<ij|kk>-<ik|jk>)O_k^{\alpha a}
+<ij|kk>O_k^{\beta a}.
$$
The effect of the spin operator is to change the spin of the electrons but
leave the space part unchanged. All integrals vanish identically due to one or
more of the following identities:
$$
\begin{array}{rclrcl}
<\psi_i\psi_j>&=&0;&<m_im_j>&=&\delta(i,j)\\
<\psi_i\psi_k>&=&\delta(i,k);&<\psi_j\psi_k>&=&\delta(j,k).
\end{array}
$$
Therefore, $<\Psi_a|S^2|\Psi_b>=0$.

\item Except for $\psi_i$ and $\psi_j$ in $\Psi_a$ and $\psi_k$  and $\psi_l$
in $\Psi_b$; $\Psi_a = \Psi_b$. Two situations exist: (a) when all four M.O.s
are of the same spin; and (b) when two are of each spin.

When all four M.O.s have the same spin, the effect of the spin operator is to
reverse the spin of two M.O.s in the ket half of the integral. By spin
orthogonality this results in an integral value of zero.

In the case where two M.O.s are of $\alpha$ spin and two are of $\beta$ spin,
the matrix elements, after elimination of those terms which are zero due to
space orthogonality, are
$$
<\Psi_a|S^2|\Psi_b> = (-1)^W(<\psi_i\psi_k|S^2|\psi_j\psi_l>-
<\psi_i\psi_l|S^2|\psi_j\psi_k>)
$$
The effect of $S^2$ on $\psi_k$ and $\psi_l$ is to reverse the spin of these
functions; this gives
$$
<\Psi_a|S^2|\Psi_b> = (-1)^W(<\psi_i\psi_k'><\psi_j\psi_l'>-
<\psi_i\psi_l'><\psi_j\psi_k'>) ,
$$
where $\psi'$ has the opposite spin to that of $\psi$.

Thus, only if $\psi_i$ and $\psi_j$ are spatially identical with $\psi_k$ and
$\psi_l$ will $<\Psi_a|S^2|\Psi_b >$ be non-zero. The phase-factor W is such
that if $i=k$ and $j=l$ then W=--1, and if $i=l$ and $j=k$ then W=1; for all
other cases the matrix element is zero, so the phase of W is irrelevant. For
these two cases, the matrix element is $ <\Psi_a|S^2|\Psi_b>=1$ if
$(I^++I^-)(\psi_i+\psi_j)=(\psi_k+\psi_l)$, otherwise $<\Psi_a|S^2|\Psi_b> =
0$.

\item If more than two differences exist, $<\Psi_a|S^2|\Psi_b> = 0$.
\end{enumerate}

\subsubsection{Calculation of spin-states}
In order to calculate the spin-state, the expectation value  of  $S^2$ is
calculated.
\begin{eqnarray*}
<\Phi_k|S^2|\Phi_k> & = & S(S+1) = S_z^2 + 2 I^+I^-   \\
   & = & \frac{1}{2}(p+q) - \\
   &&\sum_i
\left\{C_{ik}C_{ik}
\left((1/4)(N^{\alpha}_i-N^{\beta}_i)^2
+ \sum_l O^{\alpha}_{li} O^{\beta}_{li}\right)
+ \sum_j2
\left[C_{ik}C_{jk} [\delta(\Psi_i,(I^+I^-)\Psi_j) ]\right]
\right\}
\end{eqnarray*}
where  $C_{ik}$  is the coefficient of microstate $\Psi_i$ in State $\Phi_k$,
$N^{\alpha}_i$ is the number of alpha electrons in microstate $\Psi_i$,
$N^{\beta}_i $ is the number of beta electrons in microstate $\Psi_i$,
$O^{\alpha}_{lk}$ is the occupancy of alpha M.O.\ $l$ in microstate $\Psi_k$,
$O^{\beta}_{lk}$ is the occupancy of beta M.O.\ $l$ in microstate $\Psi_k$,
$I^+$ is the spin shift up or step up operator,  and  $I^- $ is the spin shift
down or step down operator.

The spin state is calculated from:
$$
S = (1/2) [\sqrt{(1+4 S^2)} - 1 ]
$$
In practice, $S$  is  calculated  to  be  exactly  integer,  or  half
integer.   That  is,  there is insignificant error due to approximations used.
This does not mean, however, that the method  is  accurate.   The spin
calculation  is  completely precise, in the group theoretic sense, but the
accuracy of the calculation is limited by the Hamiltonian  used, a
space-dependent function.

\subsection{Choice of State to be Optimized}\label{cos}
MECI can calculate a large number of states of various total  spin. Two
schemes are provided to allow a given state to be selected.  First,
\comp{ROOT=$n$} will, when used on its own, select the $n$'th  state,
irrespective of  its  total  spin.  By default, $n$=1.  If \comp{ROOT=$n$} is
used in conjunction with a keyword from the set \comp{SINGLET}, \comp{
DOUBLET}, \comp{ TRIPLET},  \comp{ QUARTET}, \comp{ QUINTET}, \comp{SEXTET},
\comp{SEPTET},  \comp{OCTET}, or \comp{NONET}, then  the  $n$'th  root of that
spin-state  will be used.  For example, \comp{ROOT=4} and \comp{SINGLET} will
select the 4th singlet state.  If there are  two  triplet  states  below the
fourth singlet state then this will mean that the sixth state will be selected.

Sometimes the energy required to form an excited state is wanted.  By this
we mean the energy of the excited state relative to the energy of the ground
state, and not the heat of formation of the excited state.  To calculate this
quantity, the keywords \comp{PRECISE, GNORM=0.01, MECI} and
\comp{C.I.=2} should be used.  For
formaldehyde, these keywords would produce the output shown in Figure~\ref{figch2o}.
\begin{figure}
\begin{makeimage}
\end{makeimage}
\caption{\label{figch2o} Energies of Excited States}
\begin{center}
\begin{tabular}{cccccc}         \hline
  State &  \multicolumn{2}{c}{Energy (eV)}     &  Q.N.&  Spin &  Symmetry  \\
        & Absolute  &  Relative  \\  \hline
    1 & -0.0049   &  0.0000    &  1 & Singlet  &  A1  \\
    2 &  \ 2.7109   &  2.7158    &  1 & Triplet  &  A2  \\
    3 &  \ 3.1029   &  3.1078    &  2 & Singlet  &  A2  \\
    4 &  \ 7.8630   &  7.8679    &  2 & Singlet  &  A1  \\
\hline
\end{tabular}
\end{center}
\end{figure}
This output can be read as follows:  The first state (the one at -0.004891eV)
is the new ground state.  C.I.\ will lower the energy of the ground state,
relative to the SCF ground state, and for formaldehyde this extra stabilization
amounts to 0.0049 eV.  The ground state is a singlet, and has A$_1$ symmetry.
The second state is a triplet, with energy 2.7109eV above the SCF energy, or
2.7158eV above the ground state, and has A$_2$ symmetry.  The third and fourth
states are both singlets.

Using the two keywords given, the system would optimize on the ground singlet
state, and the bond orders and density matrix would reflect this.   If the
first excited singlet state were wanted, then the extra keywords \comp{ROOT=2}
and \comp{SINGLET} would also be used.  Alternatively, the single extra keyword
\comp{ROOT=3} could be used.  If the first triplet state were wanted, then
\comp{TRIPLET} or \comp{ROOT=2} (but not both!) could be used.

\subsubsection{Quantum Numbers}\index{Q.N.}\index{Quantum numbers!of states}
When \comp{MECI} is used, the output contains information on the symmetry of
each state.  States of different symmetries are automatically orthogonal, but
states of the same symmetry do not need to be orthogonal. Of course they are
orthogonal, and, to emphasize this fact, an extra symmetry label is added. This
label is, in fact, a quantum number, and is given under the heading ``Q.N.'' in
the output. The first occurrence of a given irreducible representation is given
the Q.N.\ ``1'', the second, ``2'', etc.  By using the Q.N.\ and the symmetry
label, each state can be assigned a unique label.

\subsubsection{Polarizability}\index{Polarizability}
The expectation value of the polarization operator is given under
``POLARIZABILITY.'' This is an approximation to the transition moment
for the absorption or emission of a photon.  One of the two states
involved is the state defined by the keywords.  By default, this is the
ground state, but might be an excited state, for example
\hyperref[pageref]{if \comp{ROOT=2} is used.}{ For a description of
this calculation, see  p.~}{}{oscil}.

\subsubsection{Franck-Condon considerations}\index{Franck-Condon}\label{FC}
This section was written based on discussions with
\begin{center} Victor I. Danilov\\
Department of Quantum Biophysics\\ Academy of Sciences of the Ukraine\\
Kiev 143\\Ukraine\end{center}
The Frank-Condon principle states that electronic transitions take place in
times that are very short compared to the time required for the nuclei to move
significantly.  Because of this, care must be taken to ensure that the
calculations actually do reflect what is wanted.

Examples of various phenomena which can be studied are:
\begin{description}
\item[Photoexcitation]\index{Photoexcitation energy}
If the purpose of a calculation is to predict the energy of photoexcitation,
then the ground-state should first be optimized.  Once this is done, then a
C.I.\ calculation can be carried out using \comp{1SCF}.  With the appropriate
keywords (\comp{MECI C.I.=$n$ } etc.), the energy of photoexcitation to the
various states can be predicted.

A more expensive, but more rigorous, calculation would be to optimize the
geometry using all the C.I.\ keywords.  This is unlikely to change the results
significantly, however.

\item[Fluorescence]\index{Fluorescence}\index{Red-shift}\index{Photoemission}
If the excited state has a sufficiently long lifetime, so that the geometry
can relax, then if the system returns to the ground state by emission of
a photon, the energy of the emitted photon will be less (it will be red-shifted) than
that of the exciting photon.  To do such a calculation, proceed as follows:
\begin{itemize}
\item Optimize the ground-state geometry using all the keywords for the
later steps, but specify the ground state, e.g.\ \comp{C.I.=3  GNORM=0.01 MECI}.
\item Optimize the excited state, e.g.\ \comp{C.I.=3 ROOT=2  GNORM=0.01 MECI}.
\item Calculate the Franck-Condon excitation energy, using the results of the
ground-state calculation only.
\item Calculate the Franck-Condon emission energy, using the results of the
excited state calculation only.
\item If indirect emission energies are wanted, these can be obtained from
the $\Delta H_f$ of the optimized excited and optimized ground-state calculations.
\end{itemize}
In order for fluorescence to occur, the photoemission probability must be quite
large, so only transitions of the same spin are allowed.  For example, if the
ground state is S$_0$, then the fluorescing state would be S$_1$.
\item[Phosphorescence]\index{Phosphorescence}
If the photoemission probability is very low, then the lifetime of the excited
state can be very long (sometimes minutes).  Such states can become populated
by S$_1 \rightarrow $ T$_1$ intersystem crossing.  Of course, the geometry of
the system will relax before the photoemission occurs.
\item[Indirect emission]
If the system relaxes from the excited electronic, ground vibrational state to
the ground electronic, ground vibrational state, then a more complicated
calculation is called for.  The steps of such a calculation are:
\begin{itemize}
\item Optimize the geometry of the excited state.
\item Using the same keywords, except that the ground state is specified,
optimize the geometry of the ground state.
\item Take the difference in $\Delta H_f$ of the optimized excited and optimized
ground-state calculations.
\item Convert this difference into the appropriate units.
\end{itemize}
\item[Excimers]\index{Excimers}
An excimer is a pair of molecules, one of which is in an electronic excited
state.  Such systems are usually stabilized relative to the isolated systems.
Optimization of the geometries of such systems is difficult.  Suggestions on
how to improve this type of calculation would be appreciated.
\end{description}

\subsection{Definition of some C.I.\ Keywords}\label{chadef}
It has been my policy, ever since the first release of MOPAC in 1983, to resist
changing the definition of keywords.  This policy has allowed users to
confidently use a new MOPAC in the belief that old keywords will have their
old, familiar, meaning.  However, an ambiguity was found in certain keywords,
an ambiguity which has, at times, resulted in severe frustration.

Consider the word \comp{TRIPLET}.  This meant (but no longer means) ``Do an SCF
calculation in which the M.O.\ populations are [\ldots,2,2,2,1,1,0,0,\ldots],
then do a C.I.\ on the two half-occupied M.O.s, and select the triplet state.''
This definition meant that twisted ethylene would have the correct symmetry, as
would triplet oxygen.  However, if a user wanted to examine triplet
formaldehyde, and compare it with the ground state, problems arose.  The
keywords \comp{C.I.=2 ROOT=2} would generate the correct energy, but a user
might expect that \comp{TRIPLET} should achieve the same result.  Because of
the definition of \comp{TRIPLET}, the SCF starting configuration was different,
and as a result, the $\Delta H_f$ was also different. Under earlier MOPACs,
there was no way to set up a calculation using the keyword \comp{TRIPLET} and
go SCF on a closed-shell configuration as the precursor to a C.I.\
calculation.


Because of the limitations of the earlier definitions of spin-states
(\comp{TRIPLET}, \comp{QUARTET}, \comp{QUINTET}, \comp{SEXTET}, etc.), these
words were all redefined in 1993,  in MOPAC~93.  In order to reproduce the
earlier keywords, pairs of keywords, such as \comp{TRIPLET} \comp{OPEN(2,2)} or
\comp{SEXTET} \comp{OPEN(5,5)} must now be used.  Spin-states which result from
SCF calculations on ground-state configurations can now be specified by the
following pairs of keywords: \comp{TRIPLET C.I.=2}; \comp{QUARTET C.I.=3};
\comp{QUINTET C.I.=4}; \comp{SEXTET C.I.=5}.

Using these new definitions, spin-states of a system can now be more easily
related.  Consider the various states of formaldehyde (Table~\ref{cich2o1}),
in which all calculations use the ground-state geometry and \comp{1SCF}.
\begin{table}
\caption{\label{cich2o1}Examples of Use of C.I.\ Keywords}
\begin{center}
\begin{tabular}{lr}
\hline
Keywords Used   &  \multicolumn{1}{c}{$\Delta H_f$}   \\ \hline
(No keywords)          & -32.9040 \\
\comp{C.I.=1} & -32.9040 \\
\comp{C.I.=2} & -33.0166\\
\comp{C.I.=3} & -39.7234\\
\comp{C.I.=4} & -39.9665\\
\comp{C.I.=5} & -40.1743\\
\comp{C.I.=2 TRIPLET} &  29.6348\\
\comp{C.I.=3 ROOT=2} &  28.2840\\
\comp{C.I.=3 TRIPLET} &  28.2840\\
\comp{C.I.=3 TRIPLET MS=0} &  28.2840\\
\comp{OPEN(2,2) TRIPLET} &  27.9318\\
\hline
\end{tabular}
\end{center}
\end{table}

Now we see that \comp{C.I.=3} \comp{ROOT=2} and \comp{C.I.=3} \comp{TRIPLET}
do, in fact, give the same result.  The ``old'' MOPAC (pre-1993) result of
using  \comp{TRIPLET} can still be generated by \comp{OPEN(2,2)}
\comp{TRIPLET}.  Note that \comp{C.I.=1} generates the normal $\Delta H_f$ of
CH$_2$O, and that increasing the C.I.\ lowers the energy steadily.

\subsection{Degenerate States}\label{dest}
\index{Jahn-Teller!theorem}
\index{Point-group!dynamic Jahn-Teller}
By the Jahn-Teller theorem, systems with orbital degeneracy will distort so as
to remove the degeneracy.  However, many dynamic Jahn-Teller systems are known
in which the time-average geometry is of the higher point-group.  These systems
are the kind that will be addressed here. \index{Liotard@{\bf Liotard, Daniel}}

The analytical RHF configuration interaction first derivative calculation
developed by Liotard~\cite{analci} has been modified to allow systems with
degenerate states to be run.

Each of the degenerate states is a linear combination of microstates. Each
microstate can be described by a Slater
determinant~\cite{slater_det1,slater_det2}, which represents a specific pattern
of occupancy of molecular orbitals. Each M.O.\ is a linear combination of
Slater atomic orbitals.

\index{States!mixtures of} The whole state is best described by an equal
mixture of the degenerate states of which it is composed.  Note that this is
NOT a combination of states, rather it is a mixture.  An example of a
combination of  states is a state function, composed of a linear combination of
microstates.  In such a combination the phase-factor between microstates is
significant, thus state(1)= $1/\sqrt{2}$(Microstate(a) + Microstate(b)) is
different from state(2) $1/\sqrt{2}$(Microstate(a) - Microstate(b)).  An
example of a mixture of states is the $^2$T$_{2g}$ state of TiF$_6^{3-}$, a
$d^1$ system, in which the best description of the state is an equal mixture of
the three degenerate space components of T$_{2g}$, and an equal mixture of the
two spin components of the Kramer's doublet.  The overall state is thus
1/6($\alpha$(T$_{2g}$(x)+T$_{2g}$(y)+T$_{2g}$(z))+
$\beta$(T$_{2g}$(x)+T$_{2g}$(y)+T$_{2g}$(z)).

If equimixtures are not used, then the Jahn-Teller theorem applies, and the
system would immediately distort so as to remove the degeneracy.  In the case
of TiF$_6^{3-}$, this would result in distortion from O$_h$ to D$_{4h}$
symmetry.


\subsection{Calculation of  Spin Density}
\index{Unpaired spin density|ff}\index{Spin!density, unpaired|ff}
Starting  with  the  state  functions  as  linear  combinations  of
configurations,  the    spin density, corresponding to the alpha spin density
minus the beta spin density, will  be  calculated  for  the first  few
states.   This  calculation  is straightforward for diagonal terms, and only
those terms are used. \index{MECI|)}\index{Configuration interaction|)}
              %  Configuration Interaction
              %  Miscellaneous topic  
\section{An SCF Calculation}\label{1scf}\index{SCF1@{1SCF!worked example}}
\index{SCF!worked example}
Semiempirical calculations can be run on computers using readily available
programs such as MOPAC. It is possible to use MOPAC for research without having
any knowledge of its workings. This does not imply any failing on the part of
the researcher; after all, it is possible to write extensive computer programs
while having little knowledge of how a computer works. However, in order to
more efficiently use MOPAC, a more than casual knowledge of the theory involved
is desirable. In the following section the details of a very simple calculation
will be described. This calculation can be carried out as a `black box'
calculation, but the following exercise may help to satisfy the intellectual
curiosity of users of semiempirical methods regarding the mechanics of carrying
out an SCF calculation.  \index{MNDO|ff}\index{CNDO/2}

The MNDO method will be used because it is the oldest of the ``NDO''  methods.
The CNDO/2 method is very similar, and the example we will look at will
emphasize the similarity. The system to be examined is a regular hexagon of
hydrogen atoms in which the H--H distance is 0.98316 \AA ngstrom. Of course, a
regular hexagon of hydrogen atoms is not a stable system; the only reason we
are using it here is to demonstrate the working of an SCF calculation. The
optimized geometry was obtained by defining all bond lengths to be equal,
constraining all bond angles to be 120 degrees and defining the system as being
planar. We will need various reference data in order to follow \index{BLOCK
data|ff} the calculation. MOPAC contains a large data-set, BLOCK.F, of atomic
and diatomic parameters for all the elements which have been parameterized. By
reference to this source file we find that, for hydrogen:
$$
\begin{array}{lll}
  G_{ss}    = <\varphi_s\varphi_s|1/r|\varphi_s\varphi_s> &=& \ \ 12.848 {\rm eV} \\
  U_{ss}    = <\varphi_s|H|\varphi_s>               &=& -11.906276 {\rm eV }\\
  \xi_s                                       &=& \ \ \ \ 1.331967 {\rm Bohr} \\
  \beta_s                                     &=& \ \ -6.989064 {\rm eV} \\
  E_{atom}                                    &=& \ \ 52.102 {\rm kcal/mol}
\end{array}
$$
This exercise is designed to allow the reader to
reproduce each step. All that is needed in order to follow
the working is a hand calculator.

\begin{center}
Interatomic Distance Matrix (\AA) \nopagebreak\\
\begin{tabular}{ccccccc} \hline
 Atom &   1    &  2   &   3    &  4    &  5  &    6 \\ \hline
   1  &  0.0000& \\
   2  &  0.9832& 0.0000 \\
   3  &  1.7029& 0.9832& 0.0000 \\
   4  &  1.9663& 1.7029& 0.9832& 0.0000 \\
   5  &  1.7029& 1.9663& 1.7029& 0.9832& 0.0000 \\
   6  &  0.9832& 1.7029& 1.9663& 1.7029& 0.9832& 0.0000 \\ \hline
\end{tabular}
\end{center}

The overlap integral of two Slater orbitals between two
\index{Slater orbitals!overlap of}\index{Overlap!integrals!between two hydrogen atoms}
hydrogen atoms is particularly simple:
$$
<\varphi|\varphi> = (e^{-a})(\frac{a^2}{3}+a+1)
$$
where $a=\xi R/a_o$.

 At the optimum H-H distance of 0.9831571\AA, this yields
an overlap integral of 0.4643. The nearest-neighbor
one-electron integral is thus

$$
H(1,2) = S_{1,2}(\beta_s + \beta_s)/2 = -3.2457 eV.
$$
In general, overlap integrals are more complicated and also
involve angular components, but the principles involved are
the same. You may want to check other off-diagonal terms in
the one-electron matrix, or you may accept the results given
here.

\begin{center}
One-electron matrix (eV) \nopagebreak \\
\begin{tabular}{rrrrrrr} \hline
 Atom & 1 & 2 & 3 & 4 & 5 & 6 \\ \hline
 1 & -51.7124 \\
 2 &  -3.2457& -51.7124 \\
 3 & -1.0970& -3.2457& -51.7124 \\
 4 & -0.6992& -1.0970& -3.2457& -51.7124 \\
 5 & -1.0970& -0.6992& -1.0970& -3.2457& -51.7124 \\
 6& -3.2457& -1.0970& -0.6992& -1.0970& -3.2457& -51.7124 \\ \hline
\end{tabular}
\end{center}

\index{One-electron integrals}
On-diagonal one-electron integrals are more complicated than the off-diagonal
terms. The one-electron energy of an electron in an atomic orbital is the sum
of its kinetic energy and stabilization due to the positive nucleus of its own
atom, U$_{ss}$ or U$_{pp}$, plus the stabilization due to all the other nuclei
in the system. Each electron on a hydrogen atom experiences a stabilization due
to the five other unipositive nuclei in the system. Within semiempirical theory
the electron-nuclear interaction is related to the electron-electron
interaction via
$$
E_{e,n}=-Z_n<\varphi_s\varphi_s|\varphi_s\varphi_s>.
$$
Given the two-electron two-center integral matrix the calculation of the
diagonal terms of the one-electron matrix is straightforward:
$$
 H_{n,n} = -11.9063 - 2(9.6585) - 2(7.0635) -6.3622 = -51.7124.
$$
For interactions between an atomic orbital and a non-hydrogen atom there will
be ten terms; these arise from all permutations of the basis set, $s$, $p_x$,
$p_y$, $p_z$ with the atomic orbital under the neglect of differential overlap
approximation. The ten integrals are $<$ii$|ss>$, $<$ii$|sp_x>$,
$<$ii$|p_xp_x>$, $<$ii$|sp_y>$, $<$ii$|p_xp_y>$,$ <$ii$|p_yp_y>$,
$<$ii$|sp_z>$, $<$ii$|p_xp_z>$, $<$ii$|p_yp_z>$, and $<$ii$|p_zp_z>$.

\begin{center}
Two-Electron Integrals (eV) \nopagebreak \\
\begin{tabular}{rrrrrrr} \hline
 Atom & 1 & 2 & 3 & 4 & 5 & 6 \\ \hline
 1 & 12.8480 \\
 2 & 9.6585& 12.8480 \\
 3 & 7.0635& 9.6585& 12.8480 \\
 4 & 6.3622& 7.0732& 9.6585& 12.8480 \\
 5 & 7.0635& 6.3622& 7.0732& 9.6585& 12.8480 \\
 6 & 9.6585& 7.0635& 6.3622& 7.0732& 9.6585& 12.8480 \\ \hline
\end{tabular}
\end{center}

\subsection{Starting density matrix}
\index{Density matrix!starting}
The density matrix is necessary in order to calculate the Fock matrix, but, in
turn, the Fock matrix is necessary in order to calculate the density matrix. To
break this impasse, a guessed density matrix is used. The guess is very crude:
all off-diagonal matrix elements are set to zero, and all on-diagonal terms on
any atom are set equal to the core charge of that atom divided by the number of
atomic orbitals. Our starting guess for H$_6$ consists of a unit matrix.

Each iteration of the SCF calculation consists of assembling a Fock matrix from
the one-electron matrix, the two-electron integrals, and the density matrix,
diagonalizing it to obtain the eigenvectors, and finally reassembling the
density matrix. At some point the change in density matrix drops below a preset
limit. When this happens we say that the field is self-consistent. We will now
carry out these steps for the H$_6$ system.

\subsection{Assembly of the starting Fock matrix}
\index{Fock matrix!starting}
In the first iteration this is particularly simple, as there are no
off-diagonal terms in the density matrix. Only the on-diagonal terms are
affected. Each on-diagonal term in the Fock matrix F$_{aa}$ is modified by the
electrostatic field of all the electrons in the system except the electron or
fraction of an electron in the atomic orbital $\varphi_a$. Consider F(1,1). The
total initial population of $\varphi_1$ is 1.0, composed of equal amounts of
$\alpha$ and $\beta$ electron density. An electron in $\varphi_1$ would
therefore experience the electrostatic repulsion of half an electron. An
electron cannot repel itself; however, it will be repelled by its partner
electron of opposite spin.

In addition, each electron will be affected, normally repelled, by the
electrostatic field of all the electrons on all the other atoms. Each atom has
one electron, so the total energy of an electron, i.e., the diagonal Fock
matrix element, is given by:
$$
 F(1,1) = -51.7124 + \frac{1}{2}(12.848) + 2(9.6585 + 7.0635) + 6.3622.
$$

The Fock matrix is obtained by adding the two-electron
terms to the one electron matrix. The elements of the Fock
matrix represent the sum of the one and two electron
interactions. For the system of six hydrogen atoms, this
has the following form:
\begin{center}
Initial Fock Matrix (eV) \nopagebreak \\
\begin{tabular}{ccccccc} \hline
 Atom &   1    &  2   &   3    &  4    &  5  &    6 \\ \hline
 1& -5.4823  \\
 2& -3.2457& -5.4823  \\
 3& -1.0970& -3.2457& -5.4823  \\
 4& -0.6992& -1.0970& -3.2457& -5.4823  \\
 5& -1.0970& -0.6992& -1.0970& -3.2457& -5.4823  \\
 6& -3.2457& -1.0970& -0.6992& -1.0970& -3.2457& -5.4823  \\ \hline
\end{tabular}
\end{center}

\subsection{Diagonalization of the Fock matrix}\label{mo}
The Fock matrix is then diagonalized to yield the
\index{Eigenvalues}\index{Eigenvectors|ff}\index{Molecular orbitals|ff}
following set of eigenvalues, or one-electron energies,  and eigenvectors, or
molecular orbitals:
\begin{center}
\begin{tabular}{rrrrrrrr} \hline
Energy Level & \multicolumn{6}{c}{Molecular Orbital Coefficients} \\ \cline{2-7}
            & 1 & 2 & 3 & 4 & 5 & 6 \\ \hline
6 -0.4857  &0.4082& -0.4082&  0.4082& -0.4082&  0.4082& -0.4082  \\
5 -1.8388  &0.5774& -0.2887& -0.2887&  0.5774& -0.2887& -0.2887  \\
4 -1.8388  &0.0000&  0.5000& -0.5000&  0.0000&  0.5000& -0.5000  \\
3 -6.9317  &0.5774&  0.2887& -0.2887& -0.5774& -0.2887&  0.2887  \\
2 -6.9317  &0.0000&  0.5000&  0.5000&  0.0000& -0.5000& -0.5000  \\
1 -14.8670 &0.4082&  0.4082&  0.4082&  0.4082&  0.4082&  0.4082  \\ \hline
\end{tabular}
\end{center}
These form a normalized, orthogonal set. Under the NDDO approximation,
overlaps between different atomic orbitals are ignored, i.e.,
$<\varphi_i|\varphi_j>=\delta(i,j)$, so instead of
$$
<\psi_i|\psi_j> = \sum_{\lambda}\sum_{\sigma}c_{\lambda i}c_{\sigma j}<\varphi_{\lambda}|\varphi_{\sigma}>
$$
we have
$$
<\psi_i|\psi_j> = \sum_{\lambda}c_{\lambda i}c_{\lambda j} = \delta(i,j).
$$

\subsection{Exercises involving eigenvectors}
In the following exercises `verify' means using a hand calculator. They are
intended to confirm understanding of the theory involved. Work though one or
more examples to confirm the validity of the statement that follows.
\begin{enumerate}
\item Verify that the eigenvectors are normalized.
\item Verify that the eigenvectors are orthogonal to each other.
\item Verify that the eigenvalues are correct.
\item Verify that the eigenvectors diagonalize the Fock matrix.
\item Verify that the diagonal sum rule is obeyed; i.e., that the sum of the
eigenvalues is equal to the sum of the diagonal matrix elements (the trace) of
the Fock matrix.
\end{enumerate}

\subsection{Iterating density matrix}
\index{Density matrix!iterating}
The density matrix is then reformed using the occupied set of eigenvectors,
i.e., the lowest three levels. This yields:
\begin{center}
Density Matrix (eV) \nopagebreak \\
\begin{tabular}{rrrrrrr} \hline
 Atom &   1    &  2   &   3    &  4    &  5  &    6 \\ \hline
 1& 1.0000  \\
 2& 0.6667& 1.0000  \\
 3& 0.0000& 0.6667& 1.0000  \\
 4& -0.3333& 0.0000& 0.6667& 1.0000  \\
 5& 0.0000& -0.3333& 0.0000& 0.6667& 1.0000  \\
 6& 0.6667& 0.0000 &-0.3333& 0.0000& 0.6667& 1.0000  \\ \hline
\end{tabular}
\end{center}
 Verify that the density matrix is correct.

\subsection{Iterating Fock matrix}
\index{Fock matrix!iterating}
The second Fock matrix can then be constructed using this density matrix. The
on-diagonal terms are identical to those in the first Fock matrix, since the
atomic orbital electron densities are unchanged, but the off-diagonal terms are
now changed. The off-diagonal terms are modified to allow for exchange
interactions. (Note that not all exchange terms are stabilizing.)

Let us evaluate the matrix element F(1,2):
$$
  F(1,2) = -3.2457 - \frac{1}{2}(0.6667)(9.6583){\rm eV}.
$$
The second Fock matrix is thus:
\begin{center}
Second Fock Matrix (eV) \nopagebreak \\
\begin{tabular}{rrrrrrr} \hline
 Atom &   1    &  2   &   3    &  4    &  5  &    6 \\ \hline
 1& -5.4823  \\
 2& -6.4652& -5.4823  \\
 3& -1.0970& -6.4652& -5.4823  \\
 4& +0.3611& -1.0970& -6.4652 &-5.4823  \\
 5& -1.0970& +0.3611& -1.0970& -6.4652& -5.4823  \\
 6& -6.4652& -1.0970 &+0.3611& -1.0970 &-6.4652& -5.4823  \\ \hline
\end{tabular}
\end{center}

Diagonalization of this matrix yields the same set of eigenvectors as we had
initially. In general, several iterations are necessary in order to obtain an
SCF; however, a few systems exist for which symmetry restrictions on the form
of the eigenvectors allow them to achieve an SCF in one iteration. Hexagonal
H$_6$ is one such system. Although the eigenvectors are the same, the
eigenvalues obviously have to be different.

Exercise: Verify that the SCF energy levels of H$_6$ are -20.2457, -11.2116,
-11.2116, 2.4411, 2.4411, and 4.8929 eV.

Once an SCF is achieved we need to calculate the heat of formation.

\subsection{Calculation of heat of formation}
\index{$\Delta H_f$!worked example}\index{Heat of Formation!worked example}

The heat of formation is defined as:
$$
 \Delta H_f = E_{elect} + E_{nuc}-E_{isol}+E_{atom},
$$
where $E_{elect}$ is the electronic energy, $E_{nuc}$ is the
nuclear-nuclear repulsion energy, $-E_{isol}$ is the energy
required to strip all the valence electrons off all the
atoms in the system, and $E_{atom}$ is the total heat of
atomization of all the atoms in the system.

$E_{elect}$ is calculated from  $\frac{1}{2}{\bf P(H + F)}$, or
$$
E_{elect} = \frac{1}{2}\sum_{\lambda=1}^6\sum_{\sigma=1}^6P_{\lambda\sigma}(H_{\lambda\sigma} + F_{\lambda\sigma}).
$$
Using the data we have already derived, we can calculate $E_{elect}$ as
\begin{eqnarray}
   E_{elect} & = &  3(+1.0000)(-51.7124 + -5.4823)   \nonumber  \\
             &   &+ 6(+0.6667)( -3.2457 + -6.4652)   \nonumber  \\
             &   &+ 3(-0.3333)( -0.6992 + -0.3611)   \nonumber
\end{eqnarray}
or
$$
   E_{elect}  = -210.0898 {\rm eV}.
$$
$E_{nuc}$ is a relatively straightforward calculation, and is
equal to 130.2902eV. The total energy of the system is thus
-79.7996 eV.

We are now ready to calculate the $\Delta H_f$. As the total energy and
$E_{isol}$ are in eV, we must first convert them into kcal/mol:
$$
 \Delta H_f = 23.061( -79.7996 + 71.4377) + 6(52.1020){\rm \ kcal/mol}
$$
or
$$
\Delta H_f= 119.780{\rm  kcal/mol}.
$$
It is convenient to combine $E_{isol}$ and $E_{atom}$ together, to simplify
this calculation. In order to convert any total energy $(E_{elect} + E_{nuc})$
into a $\Delta H_f$, the following operation must be performed:
$$
\Delta H_f = 23.061(E_{elect} + E_{nuc}  -\sum_i E_{(isol-atom)}  )
$$
in which the index $i$ is over all atoms in the system.

Users of MOPAC may wish to verify this calculation for a system of their own
choice. To facilitate this, the data in Table~\ref{eisol} may prove useful.

\begin{table}
\caption{\label{eisol}Values of $E_{(isol-atom)}$ }
\begin{center}
\begin{tabular}{lrrrr} \hline
Element  & \multicolumn{4}{c}{$E_{(isol-atom)}$ (eV/atom)} \\ \cline{2-5}
& \multicolumn{1}{c}{MINDO/3} & \multicolumn{1}{c}{MNDO} &
\multicolumn{1}{c}{AM1} & \multicolumn{1}{c}{PM3} \\
\hline
Hydrogen            &-14.764312&   -14.165588& -13.655739& -15.332633  \\
Lithium             && -6.793583  \\
Beryllium           &&-27.541992  \\
Boron               &-67.584394   &-70.200344 &-69.601659  \\
Carbon            & -126.880346  &-127.910952&-128.226140&-118.640263  \\
Nitrogen           &-192.410048 & -207.466249&-207.307791&-162.513823  \\
Oxygen &-309.652672&-320.451178 &-318.682192&-291.924879  \\
Fluorine& -475.817831&-477.502913&-483.109715&-438.336301  \\
Aluminum& &  -47.931017 & &-50.311708  \\
Silicon &-95.576505&-87.539565&-83.701885&-72.488357  \\
Phosphorus &-154.270388&-156.236921&-124.436836&-121.236135  \\
Sulfur& -231.996798&-228.891710  &&-186.333060  \\
Chlorine& -347.185366&-354.374768&-373.455532&-316.452049  \\
Zinc  &  &-31.231065  \\
Germanium& &  -80.129955  \\
Bromine&  & -347.840783&-353.473742&-353.699430  \\
Tin  & &-95.454929  \\
Iodine&  &-341.704860&-347.970786&-289.422586  \\
Mercury & &-29.456154  \\
Lead & &-107.856099  \\ \hline
\end{tabular}
\end{center}
\end{table}

These numbers may be used in conjunction with the semiempirical electronic and
nuclear energies to calculate the heat of formation.
              %  Worked example - 1SCF
\section{Localized Molecular Orbital Theory}\label{lmot}
\index{MOZYME!theory}\index{Localized MOs!theory of}
\subsection{Why Use Localized Molecular Orbitals?}
Using conventional SCF methods, the time required for a SCF calculation rises
as the third or higher power of the size of the system (number of atoms,
orbitals, or electrons).  For semiempirical methods, this places a practical
limit on the number of atoms in a molecule:   about 500--1,000, using a
supercomputer.  

\index{Conventional SCF!N$^3$ dependence} The origin of the $N^3$ dependence
lies in the way conventional molecular orbitals are generated and used.  They
are generated from a Fock matrix  either  by a diagonalization, or, if
approximate M.O.s exist, by a  pseudo-diagonalization.  Both processes are of
type $N^3$.  Once generated, they are used in the construction of a density
matrix.   Because every M.O. extends over every atom, this process is also of
order $N^3$.   Most of the time in a semiempirical calculation is spent in
these two operations, therefore the overall time dependency of conventional
semiempirical calculations rises as $N^3$.

To understand how limiting this dependency is, consider how the time
requirement rises with increasing numbers of atoms.  For the sake of
discussion, assume that the time required for an SCF calculation of an amino
acid is one second.  Table~\ref{t-amino} shows the time required for larger
systems.

\begin{table}
\begin{center}
\caption[Times for One SCF Calculation]{\label{t-amino}Predicted Times Required 
for One SCF Calculation Using Conventional SCF Methods*}
\begin{tabular}{rrl}\\ \hline
No.\ of Residues  &   \multicolumn{2}{c}{Time} \\ \hline
2        &   8.0 & seconds \\
5        &   2.1 & minutes \\
10       &  16.7 & minutes \\
20       &   2.2 & hours   \\
50       &  1.4 & days   \\
100      & 11.6 & days  \\
200      & 3.1 & months \\
500      & 4.0 & years  \\
1,000    & 31.7 & years \\
2,000    & 253.6 & years \\
\hline
\end{tabular} \\
$*$: Using Conventional SCF Methods, \\ and assuming that one
amino acid \\ runs in one second.
\end{center}
\end{table}

Many enzymes containing thousands of residues are known.  It is obvious that
these systems cannot be studied using conventional SCF methods.  By using
localized molecular orbitals, the time dependency can be changed from $N^3$ to
$N^1$.  LMO methods require much more arithmetic manipulation, in that every
array which holds atomic orbital information has one or more associated integer
arrays which specify the atoms involved and their location. For this reason,
LMO methods are much slower than conventional methods for small systems.  
%LMO and conventional methods run at about the same speed for
%systems of about 100 atoms, and for larger systems LMO methods are faster.
%If the $N^3$ to $N^1$ ratio is followed exactly, then a system of 1,000 atoms 
%should run about 100 times faster using LMO methods, and a system of 
%10,000 atoms should run 10,000 times faster.

\subsubsection{What is a Localized Molecular Orbital?}
Unlike a conventional molecular orbital, which extends over all atoms, a
localized molecular orbital is   localized in a small region of the molecule. 
Almost all of a localized M.O.\ can be found on one, two, or, at most, three
atoms. The nature and behavior of the LMO is dictated by these few atoms. Small
amounts of the LMO can be found on the surrounding atoms.  Although the effect
on these small contributions on the nature of the LMO is very small, their
existence is responsible for almost all of the time spend on mathematical
manipulation of the LMOs.

\subsubsection{Energy Considerations.}\index{Energy considerations}
The electronic energy of a system is given by 
\begin{equation}
E_{elect} = \frac{1}{2}\sum_{\lambda}\sum_{\sigma},
P_{\lambda\sigma}(H_{\lambda\sigma}+F_{\lambda\sigma})
\end{equation}
where $P_{\lambda\sigma}$ is the density matrix element connecting atomic
orbitals $\phi_{\lambda}$ and $\phi_{\sigma}$.

In practice, the density matrix elements rapidly become small as the
interatomic distance increases.  Only for atoms which are chemically bonded
together will $P_{\lambda\sigma}$ be large.  As an example, in propanolamine,
the smallest bond order between two chemically bonded atoms is 0.98. Between
any two neighboring but non-chemically bound atoms (that is, atoms  separated
by one atom), the largest bond order is 0.015, and for all other  interactions,
the bond-orders are very small.

Because the electronic energy is determined by the density matrix, the
contribution to the electronic energy arising from density matrix elements
between atoms which  are not near to each other is small, and becomes very
small with increasing  distance.  The contributions from atoms separated by
more than about 10~\AA ngstroms is quite negligible, and can safely be
ignored.  From this it follows that the only density matrix terms which need to
be considered are those arising from atoms separated by less that 10~\AA
ngstroms.  Since the density matrix is constructed from the M.O.s only, then,
in LMO theory, the LMOs need only extend over a distance of about 10~\AA
ngstroms from the center of the LMO.  

\subsubsection{Limitations on Systems.}
The use of localized molecular orbitals to help solve the self-consistent
field equations is limited to what we will call `normal' compounds.  In
order for a system   to be classified as `normal', it must be possible to
sketch the molecular structure using only the following drawing elements:
\index{Compounds!`normal'!definition}

\begin{enumerate}
\item Chemical symbols to represent the atoms.
\item A maximum valency of 4 for any non-hydrogen atom.
\item Lines to represent $\sigma$ bonds.
\item Pairs of parallel lines to represent double bonds, and three
parallel lines to represent triple bonds.
\item Pairs of dots to represent lone-pairs.
\item `+' and `--' signs, to represent charges.
\end{enumerate}

\index{Aromatic rings}\index{Delocalized $\pi$ systems}
\index{pi@$\pi$ systems!delocalized}
This definition allows aromatic rings and delocalized $\pi$ systems to be
present, as these structures can be represented as localized $\pi$ bonds.

\index{Hypervalent systems!allowed}
What is {\em not} allowed are radicals, open-shell systems, electronic
excited states, etc.  Hypervalent systems are allowed, as they can always be
written as non-hypervalent Lewis structures, for example, SF$_6$ can be submitted
as a system of point group $O_h$, in which case the Lewis structure generated
would be represented by SF$_4^{++}$ plus 2F$^-$.

\subsection{Memory Management}\index{Memory!management}
\subsubsection{The Nature of the Problem.}
Conventional semiempirical software is unsuitable for the calculation of large
systems, due to the heavy memory demands made.  For example, in MOPAC~93, the
memory requirement (in Kb) rises as $2249+17.15N+7.395N^2$,  where $N$ is the
number of heavy (non-hydrogen) atoms in the molecule, assuming that the number
of hydrogen and non-hydrogen atoms are equal.  In proteins, the number of
hydrogen and non-hydrogen atoms are roughly equal, so the expression for
MOPAC~93 would be applicable to proteins.  Table~\ref{n2size} illustrates how
rapidly the memory requirement increases for proteins.

\begin{table}
\caption{\label{n2size} MOPAC Memory Requirements for Proteins}
\begin{center}
\begin{tabular}{rrr}
\hline 
$N$   &  No.$^{\dag}$ of &   Memory   \\
      & Residues & Needed (Mb)  \\ \hline
1000  &  65  &  7,414 \\
2000  & 131  & 29,616 \\
3000  & 196  & 66,609 \\
4000  & 262  &118,389 \\
5000  & 327  &184,961 \\
\hline
\end{tabular} \\
\dag : Number of residues is approximate only;\\ the number will vary
from protein to protein
\end{center}
\end{table}

Clearly, calculation of systems of only a few hundred residues is
impractical.   By making approximations, by altering the way arrays are
specified, and by use of localized molecular orbitals, the memory
requirements can be reduced. As a result of these changes, the memory
requirement for MOZYME is considerably less than that for MOPAC.  For
example, a MOPAC calculation of a system of 3,686 atoms
(bacteriorhodopsin)  would require over 100Gb; the same calculation
using MOZYME would require only 408Mb.  This represents less than
0.5\%\ of the memory required by MOPAC.  A more detailed
\htmlref{description of array specification}{newmat} appears
\begin{htmlonly}
elsewhere.
\end{htmlonly}
\begin{latexonly}
on p.~\pageref{newmat}.
\end{latexonly}


\subsubsection{Distance Cutoffs.}\label{cutoff}\index{Cutoff|ff}\label{m2el}
In conventional SCF calculations, all interactions regardless of distance are
calculated.  This is both impractical and unnecessary in LMO work. Impractical,
because the storage required would rapidly become very large: the storage
necessary for merely the two-electron two-center integrals of  a system of $M$
non-hydrogen and $N$ hydrogen atoms  would be $50M(M-1)+10M\times
N+\frac{1}{2}N(N-1)$ array elements. Unnecessary, because many of the integrals
would never be used in the LMO calculation, anyway.  Consider the electronic
energy terms.  These all depend on the  density matrix elements.  If a density
matrix element, $P_{\lambda\sigma}$  is zero, then the associated  energy term
is independent of $H_{\lambda\sigma}$ and $F_{\lambda\sigma}$. Similarly, the
Fock element $F_{\lambda\sigma}$ depends on $P_{\lambda\sigma}$; therefore, if
$P_{\lambda\sigma}$ is zero, then  $F_{\lambda\sigma}$ is independent of the
value of the exchange integral 
$<\!\lambda\sigma|\frac{e^2}{r}|\lambda\sigma\!>$.

\begin{table}
\caption{\label{bo}Bond-Orders in GLY-GLY-GLY-GLY}
\begin{center}
\begin{tabular}{lrlrrr}
\hline
Bond to  &  Order & Bond to & Order\\
\hline
C$_2$& 1.017090  &C$_8$ & 0.000009   \\
C$_3$& 0.015249  &C$_9$ & 0.000001 \\
N$_4$& 0.004989  &N$_{10}$ & 0.000000  \\
C$_5$& 0.001219  &C$_{11}$ & 0.000000  \\
C$_6$& 0.000086  &C$_{12}$ & 0.000000 \\
N$_7$& 0.000031  \\
\hline
\end{tabular}

Bonds are between the terminal nitrogen and the backbone atoms.

GLY-GLY-GLY-GLY has the structure:
$\chem H_2N\!-\!CH_2\!-\!CO\!-\!NH\!-\!CH_2\!-\!CO\!-\!NH\!-\!CH_2\!-\!CO\!-\!NH\!-\!CH_2\!-\!COOH  $.
\end{center}
\end{table}

The bond-order is a measure of the electron density between two atoms.  
Table~\ref{bo} shows the bond-orders between the terminal nitrogen and the
backbone atoms in the tetrapeptide GLY-GLY-GLY-GLY. \label{cutofs}  For atoms
separated by more than 3 - 4 bonds, that is, by more than  5 - 7 \AA ngstroms,
the bond-orders rapidly become very small.  Also, at such distances the overlap
of two atomic orbitals becomes extremely   small.  Since  the one-electron
interactions depend on the overlap 
\begin{equation}
H_{\lambda\sigma} = S_{\lambda\sigma}(\beta_{\lambda}+\beta_{\sigma}),
\end{equation}
where $S_{\lambda\sigma}$ is the atomic orbital overlap, and $\beta_{\lambda}$
and $\beta_{\sigma}$ are atomic orbital parameters, it follows that the
one-electron interactions at distances greater than about 6--7 \AA ngstroms will
also be vanishingly small.

By calculating and storing only the one-electron integrals representing
interactions of less than a given distance, a considerable saving in storage
requirements can be achieved.

Unlike the one-electron integrals, the value of some two-electron two-center 
integrals does not fall off rapidly with increasing distance.  The 100
two-electron  integrals involving two atoms can be divided into five groups, as
shown in Table~\ref{multipoles}.  Some of these integrals, such as the
$<\!pp|pp\!>$ integrals, can be expressed in terms of two multipoles, here a
monopole and a quadrupole; however, for the purpose of this discussion the
effect of the lower multipole---the monopole in this case---dominates.  

\begin{table}
\caption{\label{multipoles} Multipolar Representation of the Two Electron Two Center Integrals}
\begin{center}
\begin{tabular}{lcrcc} \hline
Multipole    & Example & No. & \multicolumn{2}{c}{Distance Dependence} \\
             &         &     & Energy & Gradient \\ \hline
Monopole     & $<\!ss|ss\!>$& 16  & $r^{-1}$& $r^{-2}$\\
Dipole       & $<\!ss|sx\!>$ & 24  & $r^{-2}$& $r^{-3}$\\
Quadrupole   & $<\!ss|xy\!>$ & 33  & $r^{-3}$& $r^{-4}$\\
Octapole     & $<\!sx|xy\!>$ & 18  & $r^{-4}$& $r^{-5}$\\
Hexadecapole & $<\!xy|xy\!>$ &  9  & $r^{-5}$& $r^{-6}$\\
\hline
\end{tabular}
\end{center}
\end{table}

At distances greater than about 5 or 6 \AA ngstroms, the behavior of these
integrals becomes very simple.  Of the 100 integrals, 60 represent quadrupoles 
or higher multipoles and have a negligible value.  The 16 monopole terms --
representing simple electrostatic repulsion -- are all composed of the same
monopole term and various quadrupolar components, and can be set equal. The 24
dipole terms can be expressed as a dipole interacting with either a simple
monopole (an $<\! ss|sx\!>$, for example), or a monopole plus quadrupole,
e.g.,   $<\! xx|sx\!>$. Since the quadrupolar terms can be ignored, the 24
dipolar terms can be represented by 6 simpler terms, of generic form  $<\!
sp|ss\!>$  and  $<\! ss|sp\!>$.  Thus, out of the 100 integrals needed at small
distances, only 7 are needed at larger distances. At still larger distances, 30
\AA ngstroms or more, even the dipolar terms become negligible.  Therefore, at
such large distances only the single monopole term, representing simple
electrostatic repulsion, need be used.

Two-electron two-center integrals must therefore be represented in a different
way from the one-electron two-center integrals.  As with the one-electron
integrals, a cutoff distance can be specified.  For convenience, the
one-electron and two-electron cutoff distances are set equal. As a result, all
one- and two-electron  integrals which represent interactions between atoms
that are separated by less than the cutoff distance would then be treated using
standard NDDO approximations. The simplest electrostatic repulsion 
($<\!ss|ss\!>$) between electrons on atoms  that are separated by more than the
cutoff distance is approximated by the conventional NDDO term 
\begin{equation}
<\!\phi_{\lambda}\phi_{\lambda}|e^2/r_{AB}|\phi_{\sigma}\phi_{\sigma}\!> =
\frac{1}{\left (R_{AB}^2+\frac{1}{2}(
\frac{1}
{<\!\phi_{s}\phi_{s}|e^2/r_{AA}|\phi_{s}\phi_{s}\!>} +
\frac{1}{<\!\phi_{s}\phi_{s}|e^2/r_{BB}|\phi_{s}\phi_{s}\!>})
\right )^{\frac{1}{2}}}.
\end{equation}

This was chosen in order to minimize the discontinuity at  the cutoff
distance.  Similar NDDO approximations were used for the other dipolar terms.

In recognition of this fact, SCF calculations can be simplified considerably 
by including only those NDDO terms which apply to atoms separated by less than
a preset limit, and including only the electrostatic term for interactions
between atoms separated by more than that limit.

\label{cutof1} 
\label{cutof2} 
The various cutoffs used have specific names and defaults.  These are summarized
in Table~\ref{cutoffs}.
\begin{table}
\caption{\label{cutoffs} Distance Cutoffs for Various Types of Interaction}
\begin{center}
\begin{tabular}{|l|l|l|l|r|cc}  \hline
Type &Less than Cutoff & Greater than Cutoff & Cutoff Name & Default \\ \hline
Two electron & All multipoles & Dipole plus Monopole & \comp{CUTOF2} & 12\AA \\
Two electron & Dipole plus Monopole & Monopole & \comp{CUTOF2} & 30\AA \\
One electron & Overlap calculated & Overlap not calculated & \comp{CUTOF2} & 7\AA \\ \hline
\end{tabular}
\end{center}
\end{table}

By default, the cutoff distances are set sufficiently large that any  $\Delta
H_f$ calculated using these values will agree with that obtained by using
conventional methods  within 0.1 kcal/mol. \comp{CUTOF2} can be reduced, under
user control, to save computational time and to reduce the memory demand.  To
avoid serious errors, \comp{CUTOF2} should not be set below about 5~\AA
ngstroms.

Once \comp{CUTOF2} is set, all one-electron integrals involving atoms separated
by less than \comp{CUTOF2} can be evaluated.  To save space, only those
integrals which are evaluated are stored an an array. For the one-electron
integral,  this array is called $H$.  All integrals relating to any pair of
atoms are stored contiguously.  Given the starting address of an atom-pair, the
sequence in which the integrals occur is determined only by the number of
atomic orbitals on each atom. However, the order in which atom pairs are stored
is not so simple.  The only way to find the starting address of any atom pair
is to have that address stored in a new array.  This array is called
\comp{IJBO}, and has the following structure:

\comp{IJBO} is a square array, of size $N$ by $N$, where $N$ is the
number of atoms. The starting address of integral string for atoms 
$A$ and $B$ is stored in array element IJBO(A,B).  If $A \ne B$, then the
starting address is also stored in IJBO(B,A).  All pairs of atoms
separated by more than \comp{CUTOF2} do not have associated integral
strings.  To recognize this, the relevant array elements in \comp{IJBO}
are set to `$-1$', and if the interatomic distance is greater than
\comp{CUTOF1} then the array element is set to `$-2$'. Zero cannot be
used, because of \hyperref[pageref]{the way starting addresses are
defined}{ (see Page~}{)}{starting-addresses}.

To reiterate: All integrals involving atoms, $A$ and $B$,  separated by  less
than \comp{CUTOF2} are stored, and the starting address of the integrals  is
given by IJBO(A,B) and IJBO(B,A).  With the exception of some electrostatic
terms, no integrals involving atoms separated by more than \comp{CUTOF2} are
calculated or stored.  The relevant array element in \comp{IJBO} is set to
`$-1$' or `$-2$'. 


To summarize:

There are three regions around each atom.  The sizes of these regions are
determined by \comp{CUTOF2=$m.mm$} (default: $m.mm$=12) and
\comp{CUTOF1=$n.nn$} (default $n.nn$=30). In the first region, 0.0 to $m.mm$
\AA ngstroms, all  NDDO interactions are used.  Between $m.mm$ and $n.nn$ \AA
ngstroms, only monopolar and dipolar electrostatic terms are used, and beyond
$n.nn$ \AA ngstroms, the only term considered is the monopolar electrostatic
interaction.

\subsection{Lewis Structures.}
\index{Lewis structure! checking}
One of the more difficult operations involved in using localized molecular
orbitals is the generation of a Lewis structure.  While this operation is
almost trivially easy for a competent chemist, setting up the instructions so
that a program can do the same operation has proved to be a daunting task.

In the following section, the steps involved in calculating the Lewis structure
are described.  This description is intended to be definitive, in the sense
that it should allow the Lewis structure for {\em any} compound to be
generated. At the same time, any deficiency in the description should be
reflected in the inability of MOZYME to generate the Lewis structure for
certain systems.

Because of its complexity, the main sequence involved will be given first,
followed by a more detailed explanation of the individual steps.

\subsubsection{Lewis Structure---Main Sequence.}
\begin{enumerate}
\item The connectivity is calculated.  This determines which atom is bonded to
which atom.
\item All atoms that have explicit charges are identified, and the charges
assigned.
\item The $\sigma$ framework is determined.
\item Most of the lone pairs are identified.
\item Open-ended (non-aromatic) $\pi$-bonds are identified.
\item Aromatic $\pi$-bonds are identified.
\item All cations, anions, and any remaining lone pairs are identified.
\end{enumerate}

\subsubsection{Detailed Description of Lewis Structure.}
\begin{description}
\item[Calculation of Connectivity]~\\
Hydrogen atoms are monovalent.  Because of this, they can only bond to one
other atom.  Therefore, the first set of bonds formed are the X--H $\sigma$
bonds.  The criterion used is that each hydrogen atom is connected to the atom
nearest to it, except that a hydrogen atom is not allowed to be bonded to
another hydrogen atom.

The connectivity of all other atoms is determined.  Atoms are considered as
being connected (bonded together) if the interatomic distance is less than
110\% of the sum of their Van der Waals radii.

Any bridging hydrogen bonds are identified.  These usually indicate a faulty
geometry.  If any are present, then the SCF calculation will not be run, unless
\comp{LET} is present.

Any user-defined chemical bonds are identified. This is useful in cases where a
Lewis structure could not otherwise be created.  An example is the simple
system HNO$_3$.

Finally, the number of atoms bonded to each atom is checked.  Conventional
Lewis structures do not allow more than four bonds to each atom (assuming an
$sp^3$ basis set), so if there are more than four atoms bonded to any atom, a
conventional Lewis structure cannot be generated, and the SCF cannot be run. 
Before abandoning the calculation, an attempt is made to reduce the number of
bonds to a hypervalent atom.  First, any bonds from a hypervalent atom to a
halogen are broken.  If this makes the hypervalent atom normal, then the
calculation can proceed.  If that does not work, then bonds to elements of
group VI are broken.  If that still does not make the atom normal-valent, then
the  calculation is stopped.

\item[$\sigma$-framework]~\\
This is the simplest set of bonds to identify.  A $\sigma$ bond exists for
every pair of atoms that are connected.  The number of $\sigma$ bonds is equal
to the number of connections in the system.  For water, this would be 3; for
benzene, 12; and for ethylene, 5.

Each time a bond is formed, the number of available atomic orbitals on the
atoms involved is decremented by 1 and the number of available electrons is
decremented by 1.  For example, before the $\sigma$ framework  of ammonia is
formed, the number of available orbitals on nitrogen is 4 and the number of
electrons is 5. After the $\sigma$ bonds are formed, there is 1 orbital and 2
electrons left.

\item[Lone pairs]~\\
The next set of Lewis elements formed are the lone pairs.  The rule used here
is that if there are more electrons than orbitals on an atom, the extra
electrons are used in the construction of lone pairs.  Each lone pair uses up
two electrons and one orbital.  Thus, one lone pair would be assigned to the
nitrogen in ammonia, two lone pairs would be assigned to oxygen in water, and
three lone pairs would be assigned to chlorine in HCl.

The lone pairs could only assigned after the $\sigma$ bonds were created.  If
they were assigned before the $\sigma$ bonds, then some ionic systems, such as
NH$_4$, could not be represented.

\item[Open-ended $\pi$-bonds]~\\
Once the lone pairs are assigned, the number of unused atomic orbitals on an
atom will be equal to the number of unused electrons on that atom.  These
unused orbitals are then available for forming multiple bonds between atoms. 
If two atoms, that are $\sigma$ bonded together, both have unused orbitals,
then they can form a $\pi$ bond.  

The order in which $\pi$ bonds are generated is important.  If in styrene,  for
example, a $\pi$ bond is assigned to the ring-C$_1$ as in Figure~\ref{styrene},
$A$, then when the remaining double bonds are created, there are two unused
atomic orbitals on atoms that are not bonded together.  In order for a Lewis
structure to be generated, two electrons are put into one of these unused
orbitals,  creating an anionic center, and no electrons are put into the other
orbital, making it a cationic center.

\begin{figure}
\begin{makeimage}
\end{makeimage}
\begin{center}  
\includegraphics{styrene}
\end{center}  
\caption{\label{styrene} Examples of Initial Choice of Double Bond}  
\end{figure} 


A better choice is to identify open-ended $\pi$ systems, and to assign these first,
option $B$ in Figure~\ref{styrene}.  

The order in which the $\pi$ bonds are assigned in an open-ended $\pi$ system is
important.  In the case of a simple conjugated polyene, the order is simple -
the carbon atom that has only one atom $\pi$ bonding to it is identified.
A $\pi$ bond is constructed between the two atoms.  This is repeated until
all $\pi$ bonds in the polyene are identified.

Problems arise in more complicated systems, such as buta-1,3-diyne.  If the
simple rule just described is used, then a zwitterionic cumulated polyene results,
Figure~\ref{butdiyne} $A$,
instead of a diyne, Figure~\ref{butdiyne} $B$.

\begin{figure}
\begin{makeimage}
\end{makeimage}
\begin{center}
\includegraphics{butdiyne}
\end{center}
\caption{\label{butdiyne} Generation of Yne Bond}
\end{figure}

Also, if both ends of the olefinic group are connected to aromatic rings, as
in stilbene, then identification of the olefin group is not obvious.

To allow for this, the following two rules are used:
\begin{enumerate}
\item Where there is the possibility of forming a triple bond, do so.
\item When a delocalized $\pi$ system is opened, the $\pi$ bond formed should
involve atoms that are $\pi$ bonded to exactly two other atoms.  
\end{enumerate}
In the case of stilbene ($\phi$--CH=CH--$\phi$), this prevents a $\pi$
bond forming between the ring and a carbon atom of the olefin.

Whenever a delocalized $\pi$ system is encountered, as soon as a $\pi$ bond
is formed and the delocalization destroyed, then the rest of the $\pi$ system
is treated as a simple conjugated polyene.  This ensures that the maximum
number of $\pi$ bonds is formed, and prevents unconnected $\pi$ bonds
from being created.  Thus benzene would have the three $\pi$ bonds:
C$_1$-C$_2$, C$_3$-C$_4$, and C$_5$-C$_6$, and not the quinoidal C$_1$-C$_2$
and C$_4$-C$_5$ bonds.

To allow compounds that contain several fused delocalized $\pi$-systems, such 
as the higher buckyballs (specifically, C$_{960}$) to be modeled, two extra rules
are needed.  These rules can be regarded as minor qualifications to the earlier
rules:
\begin{enumerate}
\item If any atoms attached to an open-ended $\pi$ system belong to a delocalized
$\pi$ system, then when delocalized $\pi$ systems are opened, the opening is done
using these atoms.
\item If a five-membered $\pi$ ring is attached to another delocalized 
$\pi$ system, then when the $\pi$ system is opened, at least one atom must be
in the other delocalized system.
\end{enumerate}

The effect of the first rule is that as soon as a graphitic network is encountered,
all the $\pi$ bonds in the network are assigned in one pass.  Without this rule,
individual parts of the network could be assigned separately, and at the junctions
of the various domains, the potential for isolated $\pi$ orbitals exists. This would
lead to charges that would cause severe problems with the SCF calculation.

\begin{figure}
\begin{makeimage}
\end{makeimage}
\begin{center}
\includegraphics{ful_fac}
\end{center}
\caption{\label{ful_fac}Kekule Structure for a Facet of Fullerene C$_{1500}$}
\end{figure}

For extended graphite-like systems, two more rules are needed.  These are:
\begin{enumerate}
\item  If a six-membered ring has two $\pi$ bonds already, then
         add a third $\pi$ bond to make it an aromatic ring.
\item If a six-membered ring has one $\pi$ bond, then add another
      $\pi$ bond to the same ring, so that the previous rule can be used.
\end{enumerate}
The effect of these rules is that when a graphitic lattice is encountered, all
the atoms in the lattice will be assigned in such a way as to maximize the
number of aromatic rings.  An example of such a system is provided by the large
icosahedral fullerene C$_{1500}$.  A facet of this system is shown in
Figure~\ref{ful_fac}.

\item[Remaining unused atomic orbitals]~\\
All that remains is to identify any unused orbitals and to assign them to
either the occupied or virtual sets.  The action taken depends mainly on
the group in the periodic table to which the atom belongs, to a lesser degree
on the nature of its environment, and sometimes on the charge on the system.

Each group has different properties. Thus:

\begin{description}
\item[Group I]~\\
The alkali metals are extremely electropositive.  Therefore,
without exception, the unused orbital is assigned to the virtual
set, and the atom is charged unipositive.

\item[Group II]~\\
The alkaline earth elements are very electropositive.  If the
atom does not form any bonds, then the charge is set to +2,
and all orbitals are assigned to the virtual set.

An atom that has formed $N$ bonds will have a charge of 2-$N$.
This can allow the atom to have a negative charge, in which case
the data set should indicate that the atoms attached to it have a positive charge.

\item[Group III]~\\
These elements are electropositive.  The action taken depends on the
number of unused valence electrons:
\begin{description}
\item[1 unused electron]  The atom is unipositive.
\item[2 unused electrons]  The atom is neutral:  the two unused electrons
are used in the formation of a `lone pair'.  
\item[3 unused electrons]  The atom is unipositive.
\end{description}

\item[Group IV]~\\
This is the most complicated group, with the charge on the atom depending on
many factors.  If there are two unused valence electrons, then the atom will be
neutral (a carbene, for example). Otherwise, if the first or second nearest
neighboring atom is of Group~6, then the atom is assigned a negative charge; if
the first or second nearest neighboring atom is of Group~5, then the atom is
assigned a positive charge. If the charge is still not determined, then the
assignment of charge is deferred until all other atoms have been assigned.  At
that point, the charge is assigned as either +1 or --1, depending on the
calculated charge on the system and the charge supplied by the data set.  The
charges are assigned so that the calculated charge equals the supplied charge.

\item[Group V]~\\
If there is one unused valence electron, then the charge assigned to Group V
elements is +1.  If there are 2, then the atom is neutral.

\item[Group VI]~\\
If oxygen  and there are two unused orbitals, then both are filled, and the
charge is --2.  If the atom has one unused orbital, the charge is --1;
otherwise the atom is neutral.

\item[Group VII]~\\
A very electronegative group, the charge is invariably --1.
\end{description}
\end{description}


\subsection{Construction of Starting Localized Molecular Orbitals}
\index{LMOs! construction of starting}
\index{Construction of starting LMOs}
In conventional semiempirical methods, molecular orbitals are created by the
diagonalization of an atomic orbital secular determinant.  Because of the size
of the matrices involved in proteins, this operation is impractical; therefore
the starting orthogonal localized molecular orbitals must be constructed in a
different way.

Monatomic and diatomic LMOs are constructed using all the atomic orbitals in
the system.  Care must be taken to ensure that a realistic set of starting LMOs
are generated.  If the set is not realistic, then, although an SCF could be
calculated, more work would be involved.  The most obvious set of LMOs is that
set corresponding to the classical line-drawing of a molecule: a sigma
framework, lone pairs, $\pi$-bonds, and charges, localized on specific atoms. 
Without loss of rigor, delocalized $\pi$ systems can be represented  as
localized $\pi$ bonds.  The purpose in constructing the mono and di-atomic LMOs
is to have a starting set of LMOs, which can then be used as the basis for a
self-consistent set of LMOs.

The steps involved in constructing the LMOs are as follows:
\begin{enumerate}
\item Hybrid atomic orbitals are constructed for each heavy atom.
\item All sigma bonds are identified.
\item Diatomic $\sigma$ LMOs are constructed.
\item All lone pairs are identified, and monatomic LMOs constructed.
\item All non-cyclic $\pi$ systems are identified, and diatomic $\pi$ LMOs
 bonds constructed.
\item All cyclic $\pi$ systems are identified, and diatomic $\pi$ LMOs
constructed.
\item Any unused atomic orbitals are identified as cationic or anionic centers,
and monatomic LMOs constructed.
\end{enumerate}

Inspection of this sequence shows that the resulting set of LMOs will, in fact,
correspond to the classical line-sketch of the molecular structure.

There are one or two less than obvious details involved in this sequence, which
will now be described.

\subsubsection{Construction of Hybrids}\label{make_hybrid}
\index{Hybrid orbitals!construction of}
\index{Construction of hybrid orbitals}

With the exception of hydrogen, all atoms have an $s-p$ basis set, which must
be mixed to form four hybrid atomic orbitals.  Ideally, the main lobes of the
hybrid orbitals involved in $\sigma$ bonding would point towards the other
atoms involved in those $\sigma$ bonds.  To achieve this, information regarding
the immediate environment of the atom must be used.  As only the 
direction of the neighboring atoms is important, the energy terms relating
to the one-electron integrals between the atom and the $s$ orbital of the
neighboring atoms can be used.

The process of using these energy terms to generate hybrid orbitals involves
two steps.  First, a set of M.O.s which involve the atomic orbitals of an
atom and the $s$ orbitals of its immediate neighbors is constructed by
diagonalizing the interaction matrix.  Consider a carbon atom in 
ethane.  Using the numbering scheme and orientation shown in 
Figure~\ref{c2h6pic}, the complete MNDO one-electron matrix for ethane is 
shown in Table~\ref{c2h6f}.

\begin{figure}
\begin{makeimage}
\end{makeimage}
% \begin{picture}(120,130)
% \setlength{\unitlength}{0.07cm}
% \put(40 ,-20){\includegraphics{c2h6}}
% \put(120,45){
% \begin{picture}(10,10)
% \put(40,-20){\vector(0,1){20}}
% % \put(40,-20){\line(0,1){20}}
% \put(38,2){Y}
% \put(40,-20){\vector(1,0){20}}
% \put(61,-22){X}
% \end{picture}
% }
% \end{picture}
\includegraphics{c2h6}
\caption{\label{c2h6pic} Ethane, Showing Orientation and Numbering System}
\end{figure}

The one-electron matrix is used in the construction of the interaction matrix.
This matrix is of size 4+$n$, where $n$ is the number of neighbors, i.e, $n$ =
1, 2, 3, or 4.  The first four functions are the $s$ and $p$ atomic orbitals
of the atom; the remaining functions are the $s$ orbitals of the neighbors.
The only non-zero terms are those representing the interaction of the 
atom with its neighbors.  Therefore, for the first carbon in ethane, the
interaction matrix would be that shown in Table~\ref{c2h6c1}.

\begin{table}
\caption{\label{c2h6f}One-Electron Integral Matrix for Ethane}
\begin{center}
\begin{tabular}{l|rrrrrrr} \hline
A.O.&C$(1)_s$&C$(1)_{p_x}$&C$(1)_{p_y}$&C$(1)_{p_z}$&C$(2)_s$&C$(2)_{p_x}$&C$(2)_{p_y}$\\
\hline
C$(1)_s$&-126.7891\\
C$(1)_{p_x}$ &  -6.5701 &-114.2822\\
C$(1)_{p_y}$ &   0.0000&  0.0000&-111.8502\\
C$(1)_{p_z}$ &   0.0000&  0.0000&  0.0000&-111.8502\\
C$(2)_s$     &  -5.3977& -4.2826&  0.0000&  0.0000&-126.7891\\
C$(2)_{p_x}$ &   4.2826&  2.4743&  0.0000&  0.0000&  6.5701&-114.2822\\
C$(2)_{p_y}$ &   0.0000&  0.0000& -1.1990&  0.0000&  0.0000&  0.0000 &-111.8502\\C$(2)_{p_z}$ &   0.0000&  0.0000&  0.0000& -1.1990&  0.0000&  0.0000 &  0.0000 \\H$(1)$       &  -1.0116& -0.6430& -0.3460&  0.0000& -5.7834& -1.2013 & -3.0984\\
H$(1)$       &  -1.0116& -0.6430&  0.1730& -0.2996& -5.7834& -1.2013 &  1.5492\\
H$(1)$       &  -1.0116& -0.6430&  0.1730&  0.2996& -5.7834& -1.2013 &  1.5492\\
H$(1)$       &  -5.7834&  1.2013& -1.5492&  2.6833& -1.0116&  0.6430 & -0.1730\\
H$(1)$       &  -5.7834&  1.2013&  3.0984&  0.0000& -1.0116&  0.6430 &  0.3460 \\H$(1)$       &  -5.7834 &  1.2013& -1.5492& -2.6833& -1.0116&  0.6430 & -0.1730\\\rule[-0.0cm]{0cm}{0.6cm}\\
\hline
A.O.&C$(2)_{p_z}$&H$(1)$&H$(2)$&H$(3)$&H$(4)$&H$(5)$&H$(6)$\\
\hline
C$(2)_{p_z}$&-111.8502\\
H$(1)$ &0.0000    & -99.6727\\
H$(1)$ &2.6833    &  -0.9455& -99.6727\\
H$(1)$ &2.6833    &  -0.9455&  -0.9455& -99.6727\\
H$(1)$ &0.2996    &  -0.2444&  -0.0814&  -0.2444& -99.6727\\
H$(1)$ &0.0000    &  -0.0814&  -0.2444&  -0.2444&  -0.9455& -99.6727\\
H$(1)$ &0.2996    &  -0.2444&  -0.2444&  -0.0814&  -0.9455&  -0.9455& -99.6727\\
\hline
\end{tabular}
\end{center}
\end{table}

\begin{table}
\caption{\label{c2h6c1}Interaction Matrix used in Constructing Hybrid Orbitals for C$_1$ in Ethane}
\begin{center}
\begin{tabular}{l|rrrrrrrr} 
\hline
A.O. &   C$(1)_s$ &C$(1)_{p_x}$ & C$(1)_{p_y}$ &  C$(1)_{p_z}$ &  C$(2)_s$ &H$(1)$ &H$(2)$ & H$^3$ \\
\hline
C$(1)_s$ & 0.0000  &  \\
C$_{p_x}$& 0.0000  & 0.0000  &  \\
C$_{p_y}$& 0.0000  & 0.0000  & 0.0000 &      \\
C$_{p_z}$& 0.0000  & 0.0000  & 0.0000 &  0.0000 &   \\
C$(2)_s$ &-5.3977  &-4.2826  & 0.0000 &  0.0000 &  0.0000  \\
H$(1)$   &-5.7834  & 1.2013  &-1.5492 &  2.6833 &  0.0000 &  0.0000  \\
H$(1)$   &-5.7834  & 1.2013  & 3.0984 &  0.0000 &  0.0000 &  0.0000 &  0.0000  \\
H$(1)$   &-5.7834  & 1.2013  &-1.5492 & -2.6833 &  0.0000 &  0.0000 &  0.0000 &  0.0000  \\
\hline
\end{tabular}
\end{center}
\end{table}

\begin{table}
\caption{\label{low4}Eigenvectors of Lowest Energy from Interaction Matrix}
\begin{center}
\begin{tabular}{r|rrrrrrrr}  \hline
$\Psi$ &   C$(1)_s$ &C$(1)_{p_x}$ & C$(1)_{p_y}$ &  C$(1)_{p_z}$ &  
C$(2)_s$ &H$(1)$ &H$(2)$ & H$(3)$ \\
\hline
1&  0.7069& 0.0150& 0.0000& 0.0000& 0.3409& 0.3577& 0.3577& 0.3577 \\
2& -0.0150& 0.7069& 0.0000& 0.0000& 0.6195&-0.1968&-0.1968&-0.1968 \\
3&  0.0000& 0.0000& 0.6112& 0.3556& 0.0000&-0.0019&-0.4990& 0.5009 \\
4&  0.0000& 0.0000&-0.3556& 0.6112& 0.0000&-0.5773& 0.2903& 0.2870 \\
\hline
\end{tabular}
\end{center}
\end{table}

\begin{table}
\caption{\label{low4l}Localized Diatomic Molecular Orbitals for C$_1$ in Ethane}
\begin{center} 
\begin{tabular}{r|rrrrrrrr} \hline
$\Psi$ &   C$(1)_s$ &C$(1)_{p_x}$ & C$(1)_{p_y}$ &  C$(1)_{p_z}$ &  
C$(2)_s$ &H$(1)$ &H$(2)$ & H$(3)$ \\
\hline
1&   0.3277& 0.6266& 0.0000& 0.0000& 0.7071& 0.0000& 0.0000& 0.0000  \\
2&  -0.3617& 0.1892&-0.2887& 0.5000& 0.0000&-0.7071& 0.0000& 0.0000  \\
3&   0.3618&-0.1892& 0.2887& 0.5000& 0.0000& 0.0000& 0.0000& 0.7071  \\
4&   0.3618&-0.1892&-0.5774& 0.0000& 0.0000& 0.0000& 0.7071& 0.0000  \\
\hline
\end{tabular}
\end{center}
\end{table}

\begin{table}
\caption{\label{c1hyb}Hybrid Atomic Orbitals for C$_1$ in Ethane}
\begin{center}  
\begin{tabular}{c|rrrrrrrr}\hline
Hybrid &   C$(1)_s$ &C$(1)_{p_x}$ & C$(1)_{p_y}$ &  C$(1)_{p_z}$  \\
1 & 0.46345 &   0.88612 &   0.00002 &  -0.00004  \\
2 &-0.51159 &   0.26761 &  -0.40825 &   0.70711  \\
3 & 0.51161 &  -0.26756 &   0.40825 &   0.70711  \\
4 & 0.51161 &  -0.26756 &  -0.81650 &   0.00000  \\
\hline
\end{tabular}
\end{center} 
\end{table}

The four eigenvectors of lowest energy (Table~\ref{low4}) are used to form the
hybrid orbitals.  A simple unitary transformation converts these eigenvectors
into localized di-atomic bonds (Table~\ref{low4l}).  This transformation is 
the localization procedure, which is very rapid. Finally, before the hybrid
functions can be used, they must be normalized to unity.  For C$_1$ in ethane,
the final hybrid atomic orbitals are presented in Table~\ref{c1hyb}.

\index{sp$^3$ hybrid atomic orbitals} This simple procedure works not just for
atoms of $sp^3$ hybridization. Consider an $sp^2$ system.  Now, with only three
ligands, the interaction  matrix for the atom will be of order seven.  The
hybrid orbitals are constructed from the lowest four eigenvectors resulting
from diagonalization  of this matrix. As with the $sp^3$ system, the hybrid
orbitals are first localized, then re-normalized.  This gives four hybrid
orbitals, three of which are the $sp^2$ hybrids, with the fourth being a pure
$p$ orbital.

This  procedure is quite general, as can be illustrated by oxygen, first in a
hydroxy and then in a double-bonded oxygen environment.  In a hydroxy
environment an oxygen will have two ligands: so the interaction  matrix will be
of order six.  Two $sp$ hybrids  and two $p$ orbitals will be  created.  Later
on, we will see that the two $p$ orbitals become lone-pairs.

In the case of the double-bonded oxygen, the interaction matrix is of  order
five.  The lowest eigenvector of this matrix corresponds to the oxygen-ligand
$\sigma$-bond, and consists of a mixture of oxygen $s$ and $p$ orbitals.  To
satisfy orthogonality requirements, two other $s-p$ hybrids are also made, so
the overall hybridization is three $sp^2$ orbitals plus one  pure $p$ orbital.
As with the hydroxy oxygen, two of these orbitals will eventually be used to 
create lone pairs. 

When there is only one or two atoms bonded to an atom, the non-bonding hybrid
orbitals generated are ill-defined by a unitary transform.  This ill-definition
is unimportant in that at self-consistency the density matrix is independent of
the starting LMOs.  It is important, however, in that the starting LMOs are
very sensitive to minute numerical differences, of the type encountered on
porting the program to different platforms.

To prevent this, the ill-definition is removed by adding an extra constraint,
subroutine \comp{MINLOC}. If the atom has only one ligand, oxygen in carbonyl,
for example, then the three non-bonding degenerate hybrid orbitals are mixed so
that the $p_x$ and $p_y$ orbitals on one hybrid have zero coefficients, and the
$p_x$ orbital on another hybrid also has a zero coefficient.

If the atom has two ligands, then the two non-bonding degenerate hybrids are
mixed so that $p_x$ orbital on one of the hybrids has a zero coefficient.

Hybrid atomic orbitals form an orthonormal set.  This is a natural consequence
of the fact that the hybrids are obtained by a unitary transform of the
starting orthonormal atomic orbitals.  

\subsubsection{Construction of Starting Localized Molecular Orbitals}
In order to illustrate the procedure for generating starting LMOs, we will use
benzaldehyde as the test example.  The line-sketch of benzaldehyde is shown in
Figure~\ref{c6h5cho}.  Before starting the procedure, let us spend a little
time examining the sketch.  From the empiric formula of  benzaldehyde,
C$_7$H$_6$O, we see that there are ($7\times4+6\times1+1\times6$) = 40 valence 
electrons, and ($7\times4+6\times1+1\times4$) = 38 valence orbitals.  Since
each  diatomic bond involves two electrons and two orbitals, and each lone-pair
involves two electrons and one orbital, we see that there must be 18 diatomic
bonds and two lone pairs. The diatomic bonds can be divided into three sets: 14
$\sigma$ bonds, one  localized $\pi$ bond and three $\pi$ bonds in the aromatic
ring. Every atom contributes both atomic orbitals and electrons to the system. 
These contributions are shown in Table~\ref{contrib}.

\begin{table}
\caption{\label{contrib} Atomic Orbital and Electron Contributions in
Benzaldehyde}
\begin{center}
\begin{tabular}{lcccccccccccccc} \hline
Quantity & C$_1$ & C$_2$ & C$_3$ & C$_4$ & C$_5$ & C$_6$ & H$_2$ & H$_3$ & 
H$_4$ & H$_5$ & H$_6$ & C$_7$ & O$_1$ & H$_7$  \\ \hline
Orbitals  & 4 &  4 & 4 & 4 & 4 & 4 & 1 & 1 &  1 &  1 &  1 &  4 & 4 & 1  \\
Electrons & 4 &  4 & 4 & 4 & 4 & 4 & 1 & 1 &  1 &  1 &  1 &  4 & 6 & 1  \\
\hline
\end{tabular}
\end{center}
\end{table}

\begin{figure}
\begin{makeimage}
\end{makeimage}
% \begin{center}
\begin{picture}(120,150)( -60,-40)
\setlength{\unitlength}{0.03cm}
\thicklines
 \put(00, 00){\line(1,1){20}}
 \put(20, 20){\line(0,1){20}}
 \put(20, 40){\line(-1, 1){20}}
 \put(00, 60){\line(-1,-1){20}}
 \put(-20, 40){\line(0,-1){20}}
 \put(-20, 20){\line( 1,-1){20}}
 \put( 00, 60){\line( 0, 1){18}}
 \put(-04, 86){\line(-1, 1){12}}  %  to H
 \put( 07, 91){\line( 1, 1){12}}  %  to O
 \put( 09, 88){\line( 1, 1){12}}  %  to O
 \put(-07, 80){ C} \put( 05,79){$_7$}
 \put( 17,102){ O} \put( 32,101){$_1$}
 \put( 31,104){\circle*{2}}
 \put( 31,108){\circle*{2}}
 \put( 26,112){\circle*{2}}
 \put( 22,112){\circle*{2}}
 \put(-27,100){ H} \put(-15,99){$_7$}
 \put(00, 05){\line(1,1){16}}
 \put(16, 40){\line(-1,1){15}}
 \put(-17,20){\line(0,1){19}}
 \put(00, 00){\line(0,-1){13}}
 \put(20, 20){\line(1,-1){14}}
 \put(20, 40){\line(1, 1){14}}
 \put(-20, 40){\line(-1,1){12}}
 \put(-20, 20){\line(-1,-1){12}}
 \put(-8,-23){ H} \put( 04,-24){$_4$}
 \put(32,00){ H} \put(44,-01 ){$_3$}
 \put(32,55){ H} \put(44,54  ){$_2$}
 \put(-45,55){ H} \put(-33,54 ){$_6$}
 \put(-45,00){ H} \put(-33,-01){$_5$}
\put(100, 20){\begin{minipage}[b]{3.0in}
\begin{center}
Benzaldehyde - Electronic Structure

\begin{tabular}{lr} \\  \hline
Number of Atomic Orbitals & 38 \\
Number of Electrons       & 40 \\
Number of Diatomic Bonds  & 18 \\
Number of Lone Pairs      & 2  \\ \hline
\end{tabular}
\end{center}
\end{minipage}}
\end{picture}
\end{center}

\begin{center}
\includegraphics{l_c6h5cho}
\end{center}
\caption{\label{c6h5cho} Electronic Structure of Benzaldehyde}
\end{figure}

\subsubsection{Construction of Sigma Framework}
\index{Construction of sigma orbitals}
\index{Sigma Orbitals!construction of}
Because it is the simplest set of bonds to identify,  the $\sigma$ framework 
is constructed first.  $\sigma$ bonds are defined as existing between every
pair of atoms which are chemically bonded together.  In our example, there are
14 $\sigma$ bonds.  Each di-atomic $\sigma$ bond is constructed by identifying
which pair of unused hybrid orbitals on the atoms involved has the largest
interaction energy.  When one of the atoms is a hydrogen, only  the hybrid
orbitals on the other atom need to be examined at this point.

Once the appropriate orbitals have been identified, bonding and antibonding
diatomic LMOs are constructed.   This is accomplished by diagonalizing the
associated two by two secular matrix.  The elements of this matrix are shown in
Figure~\ref{2by2}.
\begin{figure}
\begin{makeimage}
\end{makeimage}
\caption{\label{2by2}Secular Equation used in Constructing Starting LMOs}
\[
C \begin{array}{cc}
<\mathit{Hybrid}_1|H|\mathit{Hybrid}_1>-E_{1,1} & \mathit{Hybrid}_1|H|\mathit{Hybrid}_2> \\
<\mathit{Hybrid}_2|H|\mathit{Hybrid}_1> & \mathit{Hybrid}_2|H|\mathit{Hybrid}_2> -E_{2,2}
\end{array} C^{t} =0
\]
\end{figure}

For every $\sigma$ bond formed, two electrons are used, one from each of the
atoms involved in forming the bond, and two atomic orbitals are used - again,
one from each atom.  After all 14 $\sigma$ bonds are formed, the number of
still unused atomic orbitals and electrons is much reduced.  The remaining
unused atomic orbitals and electrons are shown in Table~\ref{after-sig}. The
oxygen forms only one $\sigma$ bond, so  there are still three unused hybrid
orbitals and five valence electrons left.

\begin{table}
\caption{\label{after-sig} Unused Electrons and Orbitals after Construction of
$\sigma$ Framework}
\begin{tabular}{lcccccccccccccc}\\ \hline
Quantity & C$_1$ & C$_2$ & C$_3$ & C$_4$ & C$_5$ & C$_6$ & H$_2$ & H$_3$ &
H$_4$ & H$_5$ & H$_6$ & C$_7$ & O$_1$ & H$_7$  \\ \hline
Orbitals  & 1 &  1 & 1 & 1 & 1 & 1 & 0 & 0 &  0 &  0 &  0 &  1 & 3 & 0  \\
Electrons & 1 &  1 & 1 & 1 & 1 & 1 & 0 & 0 &  0 &  0 &  0 &  1 & 5 & 0  \\
\hline
\end{tabular}
\end{table}

\subsubsection{Identification of Lone Pairs}
\index{Lone pairs!construction of}
\index{Construction of lone pairs}
After the $\sigma$ framework has been constructed, all lone pairs are
identified.  They could not be identified earlier, because sometimes a lone
pair may be used up in making the $\sigma$ framwork.  An example of this would
be the lone pair on an -\"{N}H$_2$ group, which is used up in forming a bond
when a proton is added to form a -NH$_3^+$ ion.

The maximum number of lone pairs on each atom is given by the difference 
between the number of electrons and the number of orbitals on an atom.   Since
a lone pair consists of two electrons in one atomic orbital, the remaining
atomic orbitals are used up, one at a time, until either all the atomic
orbitals on the atom are accounted for or all the potential lone pairs have
been formed.

In benzaldehyde, only the oxygen atom has the potential to form lone pairs:
five electrons and three atomic orbitals imply a maximum of two lone pairs. The
fact that there are three available atomic orbitals means that both lone pairs
can in fact be constructed.  This uses up two atomic orbitals and four
electrons, to leave the remaining orbital and electron as shown in
Table~\ref{after-lp}.

\begin{table}
\caption{\label{after-lp} Unused Electrons and Orbitals after Construction of
Lone Pairs}
\begin{center}
\begin{tabular}{lcccccccccccccc} \hline
Quantity & C$_1$ & C$_2$ & C$_3$ & C$_4$ & C$_5$ & C$_6$ & H$_2$ & H$_3$ &
H$_4$ & H$_5$ & H$_6$ & C$_7$ & O$_1$ & H$_7$  \\ \hline
Orbitals  & 1 &  1 & 1 & 1 & 1 & 1 & 0 & 0 &  0 &  0 &  0 &  1 & 1 & 0  \\
Electrons & 1 &  1 & 1 & 1 & 1 & 1 & 0 & 0 &  0 &  0 &  0 &  1 & 1 & 0  \\
\hline
\end{tabular}
\end{center}
\end{table}
When a lone pair is constructed, an occupied LMO is formed, but, unlike the
$\sigma$ bonds, no unoccupied LMO is formed at the same time.

\subsubsection{Construction of Non-Cyclic $\pi$ Bonds}
\index{Construction of acyclic $\pi$ bonds}
\index{pi@$\pi$ bonds, acyclic!construction of}
A potential complication exists when the $\pi$ bonds are constructed, in that
it is possible to start constructing $\pi$ bonds in such a way that a
di-radical is formed.  For example, in butadiene, if a $\pi$ bond is
constructed between carbon atoms 2 and 3, then radical sites would be formed at
atoms 1 and 4.  To avoid this, the first set of $\pi$ bonds constructed is that
set in which the way the $\pi$ bond is drawn is unambiguous.  The rule for
assigning  $\pi$ bonds is simple: a $\pi$ bond is constructed if an atom can
make  a $\pi$ bond with exactly one other atom.  

In benzaldehyde, this situation exists only for the carbon and the oxygen of
the -CHO group.  The $\pi$ bonds in the ring cannot be assigned using this
rule, because {\em every} atom in the ring can potentially make a $\pi$ bond
with {\em two} other atoms.  In more complicated systems, repeated application 
of the rule automatically results in the correct assignment of $\pi$ bonds  in
all non-cycle conjugated systems.

As with the $\sigma$ bonds, when a $\pi$ bond is made, two electrons and two
atomic orbitals are used up, and an occupied and an unoccupied LMO are formed. 
After the non-cyclic $\pi$ bonds are formed, the number of available orbitals
and electrons is as shown in Table~\ref{after-pi}.

\begin{table}
\caption{\label{after-pi} Unused Electrons and Orbitals after Construction of 
Non-Cyclic $\pi$ Bonds}
\begin{center}
\begin{tabular}{lcccccccccccccc} \hline
Quantity & C$_1$ & C$_2$ & C$_3$ & C$_4$ & C$_5$ & C$_6$ & H$_2$ & H$_3$ & 
H$_4$ & H$_5$ & H$_6$ & C$_7$ & O$_1$ & H$_6$  \\ \hline 
Orbitals  & 1 &  1 & 1 & 1 & 1 & 1 & 0 & 0 &  0 &  0 &  0 &  0 & 0 & 0  \\ 
Electrons & 1 &  1 & 1 & 1 & 1 & 1 & 0 & 0 &  0 &  0 &  0 &  0 & 0 & 0  \\ 
\hline 
\end{tabular} 
\end{center}
\end{table} 

\subsubsection{Construction of Cyclic $\pi$ Bonds}
\index{Construction of cyclic $\pi$ bonds}
\index{pi@$\pi$ bonds, cyclic!construction of}
At this point, the only remaining bonds are those in cyclic $\pi$ systems. The
sequence in which these are assigned is similar to that for the non-cyclic
$\pi$ bonds.  One of the remaining $\pi$ bonds is assigned arbitrarily, then
the rule for assigning non-cyclic $\pi$ bonds is used until no more $\pi$ bonds
can be made.  This two-step sequence is repeated until all cyclic $\pi$ systems
have been accounted for.

In the case of benzaldehyde, this completes the construction of the occupied
and virtual starting LMOs.  Two more structures are possible in enzymes,
however.  These will be considered next.

\subsubsection{Identification of Ions} 
\index{Ions!identification of}
\index{Identification of ions}

For un-ionized systems, all the atomic orbitals on all atoms are  accounted for
at this point.  For ionized systems, however, one or more atomic orbitals will
still be unused.  At most, there would be one unused atomic orbital on any
given atom.  The only problem left is to decide whether to assign  these unused
orbitals as belonging to the  occupied or unoccupied sets of LMOs.

In proteins, charged sites occur in only a limited number of environments.
Table~\ref{ions} lists the more common species.
\begin{table}
\index{Charged sites in proteins}
\index{Proteins!charged sites in}
\caption{\label{ions} Charged Sites in Proteins}
\begin{center}
\begin{tabular}{ccc} \hline
Neutral  &   Ion      &   Charge \\ \hline
-COOH    &  -COO      &$-$ \\
-NH$_2$  &  -NH$_3$   &$+$ \\
=NH      &  =NH$_2$   &$+$ \\
-SH      &  -S        &$-$ \\
\hline
\end{tabular}
\end{center}
\end{table}
The following general rules can be derived from this Table:
\begin{enumerate}
\item If the immediate environment has an oxygen or a sulfur atom, the
unused atomic orbital is assigned to the occupied set.
\item If the immediate environment has a nitrogen atom,  the
unused atomic orbital is assigned to the unoccupied set. 
\end{enumerate}
In other words, all ions associated with oxygen or sulfur are anions, and
all ions associated with nitrogen are cations.  

As with the lone pairs, LMOs representing ionized sites involve only one
orbital, an occupied LMO in the case of an anion, and an unoccupied LMO for a
cation.

If the calculated number of cationic and anionic sites is different, then the
entire system will have a net charge.  The numerical value of that  charge is
given by subtracting the number of anionic sites from the number of cationic
sites.

\subsubsection{Properties of Starting LMOs}
\index{LMOs!properties of starting}
The starting LMOs have two properties which are essential for the SCF
calculation.  First, since they are constructed from atomic or hybrid atomic
orbitals, either directly or as a linear combination of two hybrids, they form
an orthogonal set.  Because of the way bonding and antibonding  LMOs are
constructed, all occupied LMOs are orthogonal to all unoccupied LMOs. The
second essential property is that every LMO is normalized to unity.

Both properties are essential for the SCF calculation, in that their existence
means that any unitary transform involving any two LMOs (usually one occupied
and one unoccupied LMO) will preserve the orthonormality of the entire set of
occupied plus virtual LMOs.

\subsection{Other Considerations}

The first time a large system is run, there is a high probability that  errors
will exist in the data-set which would make the full calculation useless.  To
save time, a check is run on the system to determine the number of each type of
LMO.    If an atom bonds to more than four neighbors, then the $\sigma$
framework cannot be made, and the run is stopped with a warning message.  

If the calculated net charge is different from the charge specified on the
keyword line, then the assigned number of occupied LMOs will not match the
calculated number of occupied LMOs, i.e., the number of  $\sigma$ and $\pi$
LMOS plus the lone pairs plus the anions. If that happens,  then the job is
stopped, and a warning message printed out.

\subsubsection{Re-Orthogonalization of LMOs}\label{reorth}
During the course of a long calculation, the LMOs will become increasingly
non-orthogonal.  Subroutine \comp{CHECK} ensures that the LMOs remain
normalized; this is a very rapid calculation.  Ensuring that the LMOs are all
orthogonal is not so simple.  Re-orthogonalizing the LMOs is a lengthy
calculation, and is not routinely performed.  However, the option exists to
re-orthogonalize the LMOs, and this can be done either during a \comp{1SCF}
calculation (the preferred way), or periodically during a geometry
optimization.

Re-orthogonalizing consists of taking pairs of LMOs, $\psi_i$ and $\psi_j$, and
forming linear combinations such that the overlap ($<\!\psi_i|\psi_j\!>$) is
zero.  The LMOs involved form the full set, that is, both occupied  plus
virtual sets are used.

The two LMOs can be regarded as unit vectors that are almost at 90$^{\circ}$ to
each other.  Let the difference from  90$^{\circ}$ be $\alpha$.  If the vectors
are each rotated by -$\frac{1}{2}\alpha$, then they will become perfectly
orthogonal. This operation is most conveniently performed using perturbation
theory. Let:
$$
<\!\psi_i|\psi_i\!> \simeq 1 \simeq <\!\psi_j|\psi_j\!>
$$
and 
$$
|<\!\psi_i|\psi_j\!>| = S_{ij} \ll 1,
$$
then
$$
\psi_i' = \psi_i - \frac{1}{2}S_{ij}\psi_j
$$
and 
$$
\psi_j' = \psi_j - \frac{1}{2}S_{ij}\psi_i.
$$

That the new LMOs are orthogonal can readily be demonstrated:
\begin{eqnarray}
<\!\psi_i'|\psi_j'\!> &=&<(\psi_i - \frac{1}{2}S_{ij}\psi_j)|
(\psi_j - \frac{1}{2}S_{ij}\psi_i)\!> \nonumber \\
 &=&<\psi_i|\psi_j>-\frac{1}{2}S_{ij}<\!\psi_i|\psi_i>-
\frac{1}{2}S_{ij}<\!\psi_j|\psi_j\!>+\frac{1}{4}S_{ij}^2<\!\psi_j|\psi_i\!> 
\nonumber \\
&=& S_{ij}-\frac{1}{2}S_{ij}-\frac{1}{2}S_{ij} +\frac{1}{4}S_{ij}^3 \nonumber \\
&=& 0 \nonumber
\end{eqnarray}

The calculation of the overlaps, $S_{ij}$, is most conveniently done for one
LMO, $\phi_i$, with all other LMOs.  Because of this, $\phi_i$ should not be
modified while the re-orthogonalization is done.  In order to avoid modifying
$\phi_i$, the rotation is changed so that $\phi_i$ remains stationary and all
the rotation is incurred by $\phi_j$, thus:
$$
\psi_i' = \psi_i 
$$
and
$$
\psi_j' = \psi_j - S_{ij}\psi_i.
$$

Before the re-orthogonalization, the LMOs are almost orthogonal, and the use of
perturbation theory here is fully justified.


\subsubsection{Localized Molecular Orbitals:  Storage Considerations}
\index{LMOs!storage considerations}
When they are first made, the LMOs are highly compact functions.  Each LMO
consists of one or two atom labels and a set of four to eighteen atomic orbital
coefficients.  However, as soon as the LMOs start to interact,  mixing of the
LMOs will occur, and the size of the LMOs will increase rapidly.  This
expansion must be allowed for.  Later on, further mixing of the LMOs may result
in the contribution to an atom to a certain LMO becoming negligible, in which
case the atom can be deleted from the LMO.    This results in the LMO becoming
smaller.  A mechanism must be constructed to allow the size of the LMOs  to be
changed.  Because the occupied and unoccupied LMO sets are treated in a similar
manner, only the storage of the occupied set will be described. Therefore, for
the rest of this section, all reference to LMOs should be understood as
applying to the occupied  set only.

\label{starting-addresses}

Before going further, we need to spend a little time to understand how LMOs are
stored.   The atomic orbital coefficients of all LMOs are stored in  a single
REAL array, \comp{COCC}.  A second array, \comp{NCOCC}, holds the starting 
location in \comp{COCC} of each LMO.   For convenience, all starting locations
are defined as the location immediately {\em before} the  location of the first
element of an array.  The atom numbers (not the atomic numbers)  used in all
LMOs are stored in an INTEGER array, \comp{ICOCC}, and a second INTEGER  array,
\comp{NNCF}, holds the starting location in \comp{ICOCC} of  each LMO.  
Finally, the  number of atoms in each LMO are stored in an array \comp{NCF}.  
Figure~\ref{store-slmo} illustrates these arrays for methane.

\begin{figure}
\begin{makeimage}
\end{makeimage}
\begin{center}
\setlength{\unitlength}{0.1in}
\begin{picture}(50,30)(0,-24)
 \put(00, 02){\makebox(2.7,2){1}\makebox(2.7,2){2}\makebox(2.7,2){3}
\makebox(2.7,2){4}\makebox(2.7,2){5}\makebox(2.7,2){6}\makebox(2.7,2){7}
\makebox(2.7,2){8}\makebox(2.7,2){9}\makebox(2.7,2){10}\makebox(2.7,2){11}
\makebox(2.7,2){12}\makebox(2.7,2){13}\makebox(2.7,2){14}\makebox(2.7,2){15}
\makebox(2.7,2){16}\makebox(2.7,2){17}\makebox(2.7,2){18}\makebox(2.7,2){19}
\makebox(2.7,2){20}}
 \put(00, 00){\framebox(14,2){$H_1,C_s,C_{px},C_{py},C_{pz}$}}
 \put(14, 00){\framebox(14,2){$C_s,C_{px},C_{py},C_{pz},H_2$}}
 \put(28, 00){\framebox(14,2){$C_s,C_{px},C_{py},C_{pz},H_3$}}
 \put(42, 00){\framebox(14,2){$C_s,C_{px},C_{py},C_{pz},H_4$}}
\put(-6,00){\makebox(4,2){COCC:}}
%
 \put(00,-04){\makebox(2.7,2){1}\makebox(2.7,2){2}\makebox(2.7,2){3}\makebox(2.7,2){4}}
\put(-6,-6){\makebox(4,2){NCOCC:}}
\put(00,-6){\framebox(2.7,2){0}\framebox(2.7,2){5}\framebox(2.7,2){10}\framebox(2.7,2){15} }
%
 \put(00,-10){\makebox(2.7,2){1}\makebox(2.7,2){2}\makebox(2.7,2){3}
\makebox(2.7,2){4}\makebox(2.7,2){5}\makebox(2.7,2){6}\makebox(2.7,2){7}
\makebox(2.7,2){8}}
\put(-6,-12){\makebox(4,2){ICOCC:}}
\put(00,-12){\framebox(2.7,2){1}\framebox(2.7,2){2}\framebox(2.7,2){2}\framebox(2.7,2){3}\framebox(2.7,2){2}\framebox(2.7,2){4}\framebox(2.7,2){2}\framebox(2.7,2){5}}
%
 \put(00,-16){\makebox(2.7,2){1}\makebox(2.7,2){2}\makebox(2.7,2){3}
\makebox(2.7,2){4}}
\put(-6,-18){\makebox(4,2){NNCF:}}
\put(00,-18){\framebox(2.7,2){0}\framebox(2.7,2){2}\framebox(2.7,2){4}\framebox(2.7,2){6} }
%
 \put(00, 02){\makebox(2.7,2){1}\makebox(2.7,2){2}\makebox(2.7,2){3}
\makebox(2.7,2){4}}
\put(-6,-24){\makebox(4,2){NCF:}}
\put(00,-24){\framebox(2.7,2){2}\framebox(2.7,2){2}\framebox(2.7,2){2}\framebox(2.7,2){2} }
   \end{picture}
\end{center}
\caption{\label{store-slmo}Storage of Starting LMOs in Methane}
\end{figure}

During the SCF calculation, the number of atoms in each LMO will increase.   To
allow for this, some space must be left between the end of one  LMO and the
start of the next LMO. The amount of space is determined by dividing all the
unused space by the number of LMOs.  If the size of the vector \comp{COCC} is
40, and  the size of \comp{ICOCC} is 25, then the storage of the LMO vectors,
after adjusting to allow for expansion, is as shown in
Figure~\ref{store-slmoe}.

\begin{figure}
\begin{makeimage}
\end{makeimage}
\begin{center}
\setlength{\unitlength}{0.1in}
\begin{picture}(50,30)(0,-24)
\put(00, 02){\makebox(2.7,2){1}\makebox(2.7,2){2}\makebox(2.7,2){3}
\makebox(2.7,2){4}\makebox(2.7,2){5}\makebox(2.7,2){6}\makebox(2.7,2){7}
\makebox(2.7,2){8}\makebox(2.7,2){9}\makebox(2.7,2){10}\makebox(2.7,2){11}
\makebox(2.7,2){12}\makebox(2.7,2){13}\makebox(2.7,2){14}\makebox(2.7,2){15}
\makebox(2.7,2){16}\makebox(2.7,2){17}\makebox(2.7,2){18}\makebox(2.7,2){19}
\makebox(2.7,2){20}}
\put(00, 00){\framebox(28,2){$H_1,\ C_s,C_{px},C_{py},C_{pz}$}}
\put(28, 00){\framebox(28,2){$C_s,C_{px},C_{py},C_{pz},\ H_2$}}
\put(56, 00){\makebox ( 4,2){\ldots.}}
\put(-6,00){\makebox(4,2){COCC:}}
%
\put(00,-04){\makebox(2.7,2){1}\makebox(2.7,2){2}\makebox(2.7,2){3}\makebox(2.7,2){4}}
\put(-6,-6){\makebox(4,2){NCOCC:}}
\put(00,-6){\framebox(2.7,2){0}\framebox(2.7,2){10}\framebox(2.7,2){20}\framebox(2.7,2){30} }
%
\put(00,-10){\makebox(2.7,2){1}\makebox(2.7,2){2}\makebox(2.7,2){3}\makebox(2.7,2){4}\makebox(2.7,2){5}\makebox(2.7,2){6}\makebox(2.7,2){7}\makebox(2.7,2){8}\makebox(2.7,2){9}\makebox(2.7,2){10}\makebox(2.7,2){11}\makebox(2.7,2){12}\makebox(2.7,2){13}\makebox(2.7,2){14}\makebox(2.7,2){15}\makebox(2.7,2){16}\makebox(2.7,2){17}\makebox(2.7,2){18}\makebox(2.7,2){19}\makebox(2.7,2){20}\makebox(2.7,2){21}\makebox(2.7,2){22}}
\put(-6,-12){\makebox(4,2){ICOCC:}}
\put(00,-12){\framebox(2.7,2){1}\framebox(2.7,2){2}\framebox(2.7,2){ }\framebox(2.7,2){ }\framebox(2.7,2){ }\framebox(2.7,2){ }\framebox(2.7,2){2}\framebox(2.7,2){3}\framebox(2.7,2){ }\framebox(2.7,2){ }\framebox(2.7,2){ }\framebox(2.7,2){ }\framebox(2.7,2){2}\framebox(2.7,2){4}\framebox(2.7,2){ }\framebox(2.7,2){ }\framebox(2.7,2){ }\framebox(2.7,2){ }\framebox(2.7,2){2}\framebox(2.7,2){5}\framebox(2.7,2){ }\framebox(2.7,2){ }\makebox(2.7,2){\ldots}}
%
\put(00,-16){\makebox(2.7,2){1}\makebox(2.7,2){2}\makebox(2.7,2){3}\makebox(2.7,2){4}}
\put(-6,-18){\makebox(4,2){NNCF:}}
\put(00,-18){\framebox(2.7,2){0}\framebox(2.7,2){6}\framebox(2.7,2){12}\framebox(2.7,2){18}}
%
\put(00,-22){\makebox(2.7,2){1}\makebox(2.7,2){2}\makebox(2.7,2){3}\makebox(2.7,2){4}}
\put(-6,-24){\makebox(4,2){NCF:}}
\put(00,-24){\framebox(2.7,2){2}\framebox(2.7,2){2}\framebox(2.7,2){2}\framebox(2.7,2){2} }
\end{picture}
\end{center}
\caption{\label{store-slmoe}Storage of Starting LMOs in Methane, Showing Space Available for Expansion}
\end{figure}

After a few iterations, the number of atoms in each LMO will have increased. In
general, this number will be less than the number of atoms in the system, but
in the example we are using, every LMO will contain all the atoms.  This is
shown in Figure~\ref{store-slmoes}.  For large systems, the number of atoms in
any one LMO will vary.  Very large LMOs may have 300 or more atoms, while small
LMOs may have only 25 atoms.  On average, LMOs will have about 150--200 atoms.

For every array used in describing the occupied LMOs, there exists an 
equivalent  array for the virtual LMOs.  Table~\ref{arrays} gives the names  of
the  arrays for the virtual LMOs.

\begin{figure}
\begin{makeimage}
\end{makeimage}
\setlength{\unitlength}{0.1in}
\begin{center}
\begin{picture}(50,30)(0,-24)
 \put(00, 02){\makebox(2.7,2){1}\makebox(2.7,2){2}\makebox(2.7,2){3}
\makebox(2.7,2){4}\makebox(2.7,2){5}\makebox(2.7,2){6}\makebox(2.7,2){7}
\makebox(2.7,2){8}\makebox(2.7,2){9}\makebox(2.7,2){10}\makebox(2.7,2){11}
\makebox(2.7,2){12}\makebox(2.7,2){13}\makebox(2.7,2){14}\makebox(2.7,2){15}
\makebox(2.7,2){16}\makebox(2.7,2){17}\makebox(2.7,2){18}\makebox(2.7,2){19}
\makebox(2.7,2){20}}
 \put(00, 00){\framebox(28,2){$H_1,\ C_s,C_{px},C_{py},C_{pz},\ H_2,\ H_3,\ H_4$}}
 \put(28, 00){\framebox(28,2){$C_s,C_{px},C_{py},C_{pz},\ H_2,\ H_1,\ H_3,\ H_4$}}
 \put(56, 00){\makebox ( 4,2){\ldots.}}
\put(-6,00){\makebox(4,2){COCC:}}
%
 \put(00,-04){\makebox(2.7,2){1}\makebox(2.7,2){2}\makebox(2.7,2){3}\makebox(2.7,2){4}}
\put(-6,-6){\makebox(4,2){NCOCC:}}
\put(00,-6){\framebox(2.7,2){0}\framebox(2.7,2){10}\framebox(2.7,2){20}\framebox(2.7,2){30} }
%
 \put(00,-10){\makebox(2.7,2){1}\makebox(2.7,2){2}\makebox(2.7,2){3}\makebox(2.7,2){4}\makebox(2.7,2){5}\makebox(2.7,2){6}\makebox(2.7,2){7}\makebox(2.7,2){8}\makebox(2.7,2){9}\makebox(2.7,2){10}\makebox(2.7,2){11}\makebox(2.7,2){12}\makebox(2.7,2){13}\makebox(2.7,2){14}\makebox(2.7,2){15}\makebox(2.7,2){16}\makebox(2.7,2){17}\makebox(2.7,2){18}\makebox(2.7,2){19}\makebox(2.7,2){20}\makebox(2.7,2){21}\makebox(2.7,2){22}}
\put(-6,-12){\makebox(4,2){ICOCC:}}
\put(00,-12){\framebox(2.7,2){1}\framebox(2.7,2){2}\framebox(2.7,2){3}\framebox(2.7,2){4}\framebox(2.7,2){5}\framebox(2.7,2){ }\framebox(2.7,2){2}\framebox(2.7,2){3}\framebox(2.7,2){1}\framebox(2.7,2){4}\framebox(2.7,2){5}\framebox(2.7,2){ }\framebox(2.7,2){2}\framebox(2.7,2){4}\framebox(2.7,2){1}\framebox(2.7,2){3}\framebox(2.7,2){5}\framebox(2.7,2){ }\framebox(2.7,2){2}\framebox(2.7,2){5}\framebox(2.7,2){1}\framebox(2.7,2){3}\makebox(2.7,2){\ldots}}
%
 \put(00,-16){\makebox(2.7,2){1}\makebox(2.7,2){2}\makebox(2.7,2){3}\makebox(2.7,2){4}}
\put(-6,-18){\makebox(4,2){NNCF:}}
\put(00,-18){\framebox(2.7,2){0}\framebox(2.7,2){6}\framebox(2.7,2){12}\framebox(2.7,2){18}}
%
 \put(00,-22){\makebox(2.7,2){1}\makebox(2.7,2){2}\makebox(2.7,2){3}\makebox(2.7,2){4}}
\put(-6,-24){\makebox(4,2){NCF:}}
\put(00,-24){\framebox(2.7,2){5}\framebox(2.7,2){5}\framebox(2.7,2){5}\framebox(2.7,2){5} }
\end{picture}
\end{center}
\caption{\label{store-slmoes}Storage of Iterating LMOs in Methane}
\end{figure}

\begin{table}
\caption{\label{arrays} Names of Arrays Used to Represent LMOs}
\begin{center}
\begin{tabular}{lll} 
\hline
Occupied      &  Unoccupied      &  Contents \\ \hline
NCF           &  NCE             &  Number of Atoms in LMO \\
NNCF          &  NNCE            &  Starting Address of Atoms\\
ICOCC         &  ICVIR           &  Atom Numbers\\
NCOCC         &  NCVIR           &  Starting Address of LMO \\
COCC          &  CVIR            &  LMO Atomic Orbital Coefficients \\ \hline
\end{tabular}
\end{center}
\end{table}

\subsection{Diagonalization}
\index{Diagonalization}
At self-consistency, all matrix elements connecting every occupied LMO with
every unoccupied LMO is zero.  Annihilation of these matrix elements is
performed by the subroutine DIAGG.  The only matrix elements which need to be
considered are those which involve an occupied LMO with an  unoccupied LMO
which shares one or more common atoms.  With increasing size of system, the
number of interactions of this type compared with the product of the number of
occupied times unoccupied LMOs  becomes increasingly small.

Although DIAGG looks quite formidable, the working of this subroutine can be
easily understood in terms of the principles involved in matrix element
annihilation.  Therefore, before going into the details of the LMO matrix
annihilation, the conventional procedure used in MOPAC will be reviewed.

\subsubsection{Conventional Matrix Annihilation}
The raw material DIAGG is supplied with consists of the occupied LMOs and the
Fock matrix over atomic orbitals.  The first step, therefore, is to calculate
the occupied and virtual energy levels, and the potentially non-zero matrix
elements connecting the occupied and virtual sets.

In a system of $N$ atoms, let a molecular orbital, $\psi_i$, be represented by
\begin{equation}
\psi_i = \sum_{A=1}^N\sum_{\lambda\epsilon A}c_{\lambda i}.
\end{equation}
Then the energy of a M.O.\ can be evaluated from
\begin{equation}
\varepsilon_{ii}=\sum_A^N\sum_B^N\sum_{\lambda\epsilon A}\sum_{\sigma\epsilon B}c_{\lambda i}F_{\lambda\sigma}c_{\sigma i}
\end{equation}
and the matrix element representing the energy term between occupied M.O.\
$\psi_i$ and virtual M.O.\ $\psi_j$ would be 
\begin{equation}
\varepsilon_{i j}=\sum_A^N\sum_B^N\sum_{\lambda\epsilon A}\sum_{\sigma\epsilon B}c_{\lambda  i}F_{\lambda \sigma}c_{\sigma  j}.
\end{equation}

Annihilation of a matrix element  is achieved by performing a unitary 
transform on the M.O.s involved 
\begin{equation}
\begin{array}{llrr}
\psi_i^{'}  & = & \alpha\psi_i & + \beta\psi_j \\
\psi_j^{'}  & = &-\beta\psi_i  & +\alpha\psi_j 
\end{array}
\end{equation}
where the rotation angles $\alpha$ and $\beta$ are calculated from the molecular
orbital energy matrix elements
\begin{equation}
\begin{array}{lll}
\alpha &=& \sqrt{\frac{1}{2}(1+D/\sqrt{4\varepsilon_{i j}^2+D^2}} \\
\beta  &=& \phi\sqrt{1-\alpha^2} 
\end{array}
\end{equation}
where $D= \varepsilon_{j j}-\varepsilon_{i i}$ and $\phi$ = 1 if 
$\varepsilon_{i j}$ is negative, $\phi =-1$, otherwise.

By evaluating partial sums, the calculation of  $\varepsilon_{i j}$ can 
be made more efficient. If $\varepsilon_{i j}$ is re-written as
\begin{equation}
\varepsilon_{i j} = \sum_A^N\sum_B^N\sum_{\lambda\epsilon A}c_{\lambda  i}\sum_{\sigma\epsilon B}F_{\lambda \sigma}c_{\sigma  j};
\end{equation}
then the partial sum
\begin{equation}
F_{\lambda}^{'}(j) = \sum_{\sigma\epsilon B}F_{\lambda \sigma}c_{\sigma  j}
\end{equation}
can be used to simplify $\varepsilon_{i j}$:
\begin{equation}
\varepsilon_{i j} = 
\sum_A^N\sum_{\lambda\epsilon A}c_{\lambda  i}F_{\lambda}^{'}(j).
\end{equation}

Because the vector $F^{'}(j)$ is evaluated once, and then used for all $i$, the
calculation of the $\varepsilon_{i j}$ is changed from an $n^3$ process to a
$n^2$ process.  The use of the partial sum $F^{'}(j)$ also speeds up the
evaluation of the virtual M.O.\ energies.  Unfortunately, it does not speed up 
the calculation of the occupied M.O.\ energies.


\subsubsection{Localized Molecular Orbital Matrix Element Evaluation}
It is important in LMO work that only those matrix elements which are
potentially non-zero be evaluated; therefore, modifications need to be made to
the equations just described. For the purpose of the following discussion, the
system being studied should be assumed to be very large, i.e., to contain
thousands of atoms.

To assist in understanding how the modifications are made, it is convenient at
this time to write the occupied LMO as:
\begin{equation}
\psi_i = \sum_{j=1}^{NCF(i)}\sum_{\lambda\epsilon ICOCC(j+NNCF(i))}COCC(\lambda+NCOCC(i))
\end{equation}
and the virtual LMO as:
\begin{equation}
\psi_j = \sum_{l=1}^{NCE(j)}\sum_{\sigma \epsilon ICVIR(l+NNCE(j))}CVIR(\sigma 
+NCVIR(j)).\end{equation}
These expressions also illustrate the relationships of the five arrays which
are used to represent the LMOs.

Then the partial sum, $F^{'}(j)$, can be represented as:
\begin{equation}
F_{\lambda}^{'}(j) = 
\sum_{l=1}^{NCE(j)}\sum_{\sigma \epsilon ICVIR(l+NNCE(j))}CVIR(\sigma +
NCVIR(j))F_{\lambda \sigma}.
\end{equation}

The number of terms in this sum is already much smaller than in conventional
matrix annihilation in that the number of atoms represented in each LMO, 
$NCE(j)$, is much less than the total number of atoms.

Because all Fock matrix elements connecting atoms which are separated by  more
than \comp{CUTOF2} are automatically zero, only those terms which refer to
atoms separated by less that \comp{CUTOF2} need be evaluated.

All LMOs consist of a central part consisting of one to three atoms, which 
accounts for almost all of the wavefunction.  In the regions of an LMO far away
from the center, the contribution of any atom to the wavefunction becomes very
small.  If the Fock matrix elements connecting this distant atom to any other
atom is also very small, then quantities which depend of both of these terms
becomes quite negligible.    To test whether this condition exists, two scalar
quantities need to be calculated.  The first is the contribution of each atom
to the LMO:
\begin{equation}
\rho_A(j) =\sum_{\lambda\epsilon A}\psi_{\lambda j}^2
\end{equation}
and second is the magnitude  of the Fock matrix vector connecting each pair of
atoms:
\begin{equation}
F_{A B} = \sum_{\lambda\epsilon A \sigma\epsilon B}F_{\lambda \sigma}^2.
\end{equation}

During the evaluation of  $F_{\lambda}^{'}(j)$, the quantity $\rho_AF_{A B}$ is
computed. Only if it is above a preset limit are the terms involving atoms $A$
and $B$ used.  

As soon as $F_{\lambda}^{'}(j)$ is calculated, the magnitude of each atom's
contribution to $F_{\lambda}^{'}(j)$ is determined, and stored:
\begin{equation}
F_A(j)=\sum_A\sum_{\lambda\epsilon A}F_{\lambda}^{'}(j)^2.
\end{equation}

Together, these three modifications result in a large reduction in the time
necessary to compute the partial sum $F_{\lambda}^{'}(j)$.

The virtual energy levels are calculated using $F_{\lambda}^{'}(j)$:
\begin{equation}
\varepsilon_{jj}= \sum_{l=1}^{NCE(j)}\sum_{\sigma \epsilon ICVIR(l+NNCE(j))}
CVIR(\sigma +NCVIR(j))F_{\sigma}^{'}(j).
\end{equation}
In the same manner as $\rho_AF_{A B}$ was used in deciding whether the terms
involving atoms $A$ and $B$ should be evaluated in the calculation of 
$F_{\lambda}^{'}(j)$,  the quantity $F_A(j)\rho_A$ is evaluated and used to
decide which terms in the current summation should be used.

Now the calculation of the occupied-virtual matrix elements can be performed. 
The quantity to be calculated is
\begin{equation}
\varepsilon_{i j} =
\sum_{k=1}^{NCF(i)}\sum_{\lambda\epsilon ICOCC(k)+NNCF(i)}COCC(\lambda+NCOCC(i))
F_{\lambda}^{'}(j).
\end{equation}

Because this is a single sum over atoms, not much time is saved by testing to
see if any terms can be omitted.  However, the test is a simple one, and  it
does result in a small increase in speed by evaluating the quantity
$F_A(j)\rho_A(i)$, and comparing it to a preset limit.  Since both $F_A(j)$ and
$\rho_A(i)$ have already been calculated, this test is very rapid.

Calculation of the occupied-virtual matrix elements presents a new problem.
There are a large number of these elements, many of which involve only atoms
which are far from the LMO centers.  To see why this is so, consider two LMOs
which have exactly one atom in common.  For this to happen, the atom in
question must be very far from the centers of both LMOs.  The energy terms
arising from such an atom must, of necessity, be very small. In the interest of
efficiency, all calculations involving such atoms should be ignored. 
Unfortunately, it is not possible to {\it a priori} determine which terms to
include and which to leave out.  Therefore, at the start of the SCF
calculation, all terms must be evaluated.  As soon as  large changes in the
LMOs have stopped, a list can be constructed of those occupied-virtual matrix
elements which need to be considered for annihilation.


\subsubsection{Localized Molecular Orbital Matrix Element Annihilation}
Once the matrix elements have been calculated, annihilation is
relatively straightforward.  As with conventional SCF matrix annihilation,
the operation to be performed is:
\begin{equation}
\begin{array}{llrr}
\psi_i^{'}  & = & \alpha\psi_i & + \beta\psi_j \\
\psi_j^{'}  & = &-\beta\psi_i  & +\alpha\psi_j
\end{array}
\end{equation}
where $\alpha$ and $\beta$ are calculated using the matrix elements just
derived.

Because of the way LMOs are represented (the five arrays for each set, occupied
and virtual), care must be taken to ensure that the two by two rotation is done
correctly.  For the rotation to be performed, each atomic orbital in $\psi_i$
must be matched with the same atomic orbital in $\psi_j$.  Three distinct
situations occur: the same atom might be present in both $\psi_i$ and $\psi_j$;
$\psi_i$ might have an atom which $\psi_j$ does not have; and $\psi_j$ might
have an atom which $\psi_i$ does not have. The procedure for dealing with these
situations is as follows:
\begin{itemize}
\item Both LMOs have the same atom.

$\ \ $Simple rotation of the coefficients is performed:
\begin{equation}
\begin{array}{llrr}
\sum_{\lambda\epsilon A} \psi_i^{'}(\lambda)  & = 
& \alpha\psi_i(\lambda) & + \beta\psi_j(\lambda) \\
\sum_{\lambda\epsilon A}\psi_j^{'}(\lambda)  & = 
&-\beta\psi_i(\lambda)  & +\alpha\psi_j(\lambda)
\end{array}
\end{equation}

\item Occupied LMO $\psi_i$ has atom $A$, virtual LMO $\psi_j$ does not 
have atom $A$.

$\ \ $If the product $\beta\rho_A(i)$ is above a threshold, then the rotation
\begin{equation}
\begin{array}{llrr}
\sum_{\lambda\epsilon A} \psi_i^{'}(\lambda)  & =
& \alpha\psi_i(\lambda)  \\
\sum_{\lambda\epsilon A}\psi_j^{'}(\lambda)  & =
&-\beta\psi_i(\lambda)  
\end{array}
\end{equation}
is performed. If it is less than the threshold, no rotation is done.
The most important effect of this rotation is to place an atom in the virtual
LMO $\psi_j$, an atom which was not in $\psi_j$ before the two by two rotation
was started.  That is, the number of atoms in $\psi_j$ is increased by one.

\item Virtual  LMO $\psi_j$ has atom $A$, occupied LMO $\psi_i$ does not 
have atom $A$.

$\ \ $Again, if the product $\beta\rho_A(i)$ is above a threshold, then the rotation
\begin{equation}
\begin{array}{llrr} 
\sum_{\lambda\epsilon A} \psi_i^{'}(\lambda)  & =
&  \beta\psi_j(\lambda) \\
\sum_{\lambda\epsilon A}\psi_j^{'}(\lambda)  & =
&\alpha\psi_j(\lambda)
\end{array}
\end{equation}
is performed.  The number of atoms in the occupied LMO is increased by one.
\end{itemize}

At the start of the first SCF calculation, each LMO contains at most two atoms,
so the most dramatic effect of matrix element annihilation is to cause the
LMOs to expand so that they involve more atoms.  In the first few iterations,
the number of atoms in each LMO increases rapidly.  After about 10-30 iterations,
the rate of increase becomes very small, as the size of the LMOs becomes stable.

During the first SCF calculation, the number of atoms in certain LMOs may 
increase so rapidly that the gap between LMOs might vanish.  Any further
expansion would then cause vector overwriting to occur.  To avoid this
happening, the annihilation step is modified as follows:
The situation where vector overwriting is about to occur is detected.  The
degree of mixing of the two LMOs is then halved, and the annihilation is attempted
again.  If this procedure does not work, then the degree of mixing 
is halved again.  This
is done repeatedly until the danger of vector overwriting has been removed.
This technique is only used near the end of the SCF calculation; at other
times the degree of mixing is simply set to zero.

\subsection{Reducing the Size of Localized Molecular Orbitals}
If, as a result of matrix element annihilation,  the LMOs are allowed  to
expand so that they include more and more atoms without limit, then  eventually
the efficiency of the calculation would be severely impaired. To prevent this
occurring, a procedure is needed which will allow the size  of LMOs to be
reduced.  

Atoms can be added at an LMO as a result of matrix element annihilation. What
is not so obvious is that  annihilation can also make the contribution of an
atom to the LMO so small that the atom can subsequently be deleted from that
LMO without harm. Of course, if an atom is deleted from an LMO, changes should
be made to the  five arrays representing the LMOs.  The operation of removing
atoms from LMOs and adjusting the  appropriate arrays is done in subroutine
\comp{TIDY}.

This compression is carried out by copying all the information on each atom in
the LMOs.  During the copy operation, any atom whose contribution to an LMO is
insignificant is  not copied.  In addition, unused space at the end of each LMO
is also not copied.  The result of the copy is to produce a set of LMOs in
which only atoms which contribute significantly are present, and with no unused
space between the LMOs. 

Before the LMOs can be used once more, they must be redistributed in their
arrays so that each LMO has some unused space at the end to allow for expansion
during the next matrix annihilation operation.

\subsubsection{Size of LMOs}\label{size_of_lmo}
When a SCF is achieved, the LMOs extend over more atoms than might be
expected.  Each LMO is about 90--99\% on one or two atoms, and if the
surrounding few atoms are included, almost 100\% of the LMO can be accounted
for.   From this, it would appear that the intensity of the LMO would continue
to decrease rapidly with distance from the center.  This is normally not the
case. Instead, LMOs usually have intensity on a large number of atoms,
sometimes several  hundred atoms.  As a result, the calculations take a much
longer time than would otherwise be expected.

Some effort has been expended in trying to find ways of reducing the size of
LMOs. These attempts have not been successful.  The definitive failure was
provided as follows:

Using MOPAC, the LMOs for a large system were generated using \comp{PRECISE}.
The resulting LMOs were as expected, over 90\%\ on two atoms, with most of the
rest of the wavefunction on the nearby atoms.  However, the intensity did not
drop rapidly to zero with increasing distance.  Instead, it held  more or less
constant at about 10$^{-5}$ to 10$^{-8}$ for a large number of atoms before
finally dropping to a negligible value.

This behavior did not change on increasing the precision of the localization.

Because of this result, it was obvious that further localization of the MOZYME 
LMOs would not be useful.


\subsection{Density Matrix Construction}
The general expression for constructing the density matrix from the occupied
set of molecular orbitals is:
\begin{equation}
P_{\lambda \sigma} = 2\sum_{i=1}^{occ}c_{\lambda i}c_{\sigma i}.
\end{equation}

This same basic equation is used in the construction of the density matrix in
LMO theory.  The main difference is that not all density matrix elements are,
or need to be, evaluated.  

The only elements of the density matrix which need to be evaluated are
those relating to atoms which are \hyperref[pageref]{separated by less than
\comp{CUTOF2}}{ (see Page~}{)}{cutoff}.  To understand why, consider how
the density matrix is used in the SCF calculation. The Fock matrix is
constructed from products involving the density matrix, the
one-electron matrix, and the two-electron integrals. All terms which
involve  density matrix elements connecting any two atoms only, involve
one electron integrals and two electron exchange integrals for  the
same two atoms. All one-electron integrals and two-electron exchange
integrals for atoms separated by more than \comp{CUTOF2} are zero.
Therefore, the value of density matrix elements for atoms separated by
more than \comp{CUTOF2} is unimportant.  To save time and to reduce
array size, therefore, these density matrix elements are not
calculated.

Having stated that the construction of the density matrix is the same in
principle as that used in conventional M.O.\ theory, the way in which LMOs are
stored does introduce a technical difference.  The sequence in which the atoms
are represented in LMOs changes from LMO to LMO.  While it is possible for two
LMOs to have the same sequence   (a $\sigma$ and $\pi$ LMO involving the same
two atoms is an example),  in general the LMOs should be considered as
consisting of a random set of atoms, and should be treated as such.

This unpredictable nature of the composition of the LMOs means that the order
in which density matrix elements are calculated is determined by the  LMOs, and
not by the sequence of atoms in the molecule.  

\subsection{Energy Effects of \comp{CUTOF2}}
Energy terms arise from pairs of atoms which are separated by distances greater
than the cutoff distance.  These terms are purely electrostatic in origin:
their magnitude is simply proportional to the net charge, $Q_A$, on the atoms
involved.  Net charge is defined as 
\begin{equation}
Q_A = Z_A-Ne_A
\end{equation}
where $Ne_A$ is the total electron population on atom $A$ 
\begin{equation}
Ne_A=\sum_{\lambda\varepsilon A}P_{\lambda \lambda}.
\end{equation}
From simple electrostatics, the overall effect of these net charges is to
contribute a long-range energy term, $E_{lre}$ to the energy of the system
thusly 
\begin{equation}
 E_{lre} = \frac{1}{2}\sum_A\sum_B\phi_{AB}Q_AQ_B\gamma_{AB},
\end{equation}
where $\phi_{AB}$ \ is 1 \ if $R_{AB}$ \ is greater \ than the
\ cutoff distance but
0 otherwise, and $\gamma_{AB}$ is the $<ss|1/R_{AB}|ss>$ integral.
This simple description must be modified because of the effect of net
charges on the electron density.  Each distant net positive charge will induce
a small, but in general not negligible, stabilizing effect on the electron
distribution, while a distant net negative charge will have the opposite
effect.  This can be expressed formally in terms of the Fock matrix 
\begin{equation}
F_{\lambda\varepsilon A \lambda\varepsilon A} =
F_{\lambda\varepsilon A \lambda\varepsilon A}^{'}-\sum_B\phi_{AB}Q_B\gamma_{AB}.
\end{equation}
The effect on the nuclear term is similar, but opposite in sign:
\begin{equation}
E_{A(nuc)}=E_{A(nuc)}^{'}+\sum_B\phi_{AB}Q_BZ_A\gamma_{AB}.
\end{equation}
The total energy is given by the sum of the electronic plus nuclear energies.
Given that the electronic energy is 
\begin{equation}
E_{ee}=\frac{1}{2}\sum_{\lambda}\sum_{\sigma}P_{\lambda \sigma}
(H_{\lambda \sigma}+F_{\lambda \sigma}),
\end{equation}
the electronic term due to distant nuclei is
\begin{equation}
E_{lre(ee)}=\frac{-1}{2}\sum_A\sum_{\lambda\varepsilon A}P_{\lambda \lambda}\sum_B\phi_{AB}Q_B\gamma_{AB}
\end{equation}
or
\begin{equation}
E_{lre(ee)}=\frac{-1}{2}\sum_ANe_A\sum_B\phi_{AB}Q_B\gamma_{AB}.
\end{equation}
Likewise, given that the nuclear energy is 
\begin{equation}
E_{ne}=\sum_A\sum_{B<A} Z_AZ_B\gamma_{AB},
\end{equation}
the nuclear energy term due to distant nuclei is
\begin{equation}
E_{lre(ne)}=\sum_A\sum_{B<A}\phi_{AB} Q_AZ_B\gamma_{AB}.
\end{equation}
This can be re-written in a more symmetric form as
\begin{equation}
E_{lre(ne)}=\frac{1}{4}\sum_A\sum_{B}\phi_{AB}(Q_AZ_B +Q_BZ_A)\gamma_{AB}.
\end{equation}
Together, the total contribution due to the electronic and nuclear terms
arising from distant atoms is 
\begin{equation}
E_{lre(ee)}+E_{lre(ne)}= \frac{1}{4}\sum_A\sum_{B}\phi_{AB}(Q_A(Z_B-Ne_B)+Q_B(Z_A-Ne_A))\gamma_{AB}.
\end{equation}
Rearranging gives 
\begin{equation}
E_{lre(ee)}+E_{lre(ne)}= \frac{1}{4}\sum_A\sum_{B}\phi_{AB}(Q_AQ_B +Q_BQ_A)\gamma_{AB}
\end{equation}
or
\begin{equation}
E_{lre(ee)}+E_{lre(ne)}= \frac{1}{2}\sum_A\sum_{B}\phi_{AB}Q_AQ_B\gamma_{AB},
\end{equation}
which is identical to the simple equation this discussion began with.

In the 1996 version of MOZYME, this sum was calculated explicitly.  However, 
evaluation of the sum can be avoided by adding into the one-electron matrix
the  electrostatic stabilization terms arising from atoms beyond
\comp{CUTOF2}.  This concisely and effectively takes into account the
quantities just discussed.

Although the point-charge electrostatic effects are important, they are not the
only long-range effect.  The effect of a point-charge on a lone pair of
electrons is also significant.  Consider a lone pair on an atom pointing in the
direction of a positive charge.  Clearly this will lead to a stabilizing
effect. A lone pair pointing in the opposite direction will be destabilized.
Similarly, a lone pair pointing at 90$^{\circ}$ to a charge will be subjected
to a torque.  In order to include these effects in the Hamiltonian some extra
terms are needed.

The size of a lone pair on an atom is represented in the density matrix by the
value of the associated  $s-p$ terms.  For each atom, there are three such
terms: $P_{sp_x}$, $P_{sp_y}$, and $P_{sp_z}$. The effect of  distant charges
on $F_{sp_x}$ is given by 
\begin{equation}
F_{sp_x}' = F_{sp_x}-\sum_{B}Q_B\! <\!ss|sx\!>\!P_{sp_x}
\end{equation}
Similar terms exists for $F_{sp_y}$  and $F_{sp_z}$.

%\section{Timing Considerations}


            %  Localized M.O. theory (MOZYME)
%
%   Now comes sections on theory other than simple SCF theory.  (Mainly
%   geometry theory)
%
\section{Geometry optimization}\index{Geometry!BFGS optimizer}
The default geometry optimizer in MOPAC uses Baker's EigenFollowing method.  If
this is {\em not} wanted, for example, if there is a need to reduce memory
demands, then the Broyden Fletcher Goldfarb Shanno method can be used.

The most common use of MOPAC is for geometry optimization. This involves
starting with an approximation to the desired geometry and, by calculating the
forces acting on the system, changing the geometry so as to lower the total
energy. The objective of geometry optimization is to achieve a structure in
which all the atoms are at equilibrium, that is, one in which the forces acting
on every atom are very small, and in which the second derivatives are
everywhere positive.  Such a geometry is called a ground state stationary
point.

\input{t_ef}
\subsection{The BFGS function optimizer}
The alternative heat of formation minimization routine in  MOPAC is a modified 
Broyden~\cite{bfgs1}-Fletcher~\cite{bfgs2}-Goldfarb~\cite{bfgs3}-Shanno~\cite{bfgs4}
or BFGS method. Minor changes were made necessary by the presence of phenomena
peculiar to chemical systems.

Starting with a user-supplied geometry $x_o$, MOPAC computes an estimate to the
inverse Hessian $H_o$. The geometry optimization proceeds by
$$
x_{k+1} = x_k+\alpha d_k,
$$
where
$$
d_k=H\,g_k ,
$$
and each element of $H$ is defined by
$$
H_{k+1}=H_k-\frac{H\ y_k\ p_k^t + p_ky_k^tH}{S}+\frac{Q(p_k\ p_k^t)}{S},
$$
where  
$$
Q=1+\frac{y_k^t\ H\ y_k}{p_k^t\ y_k},
$$
and $g_k$ is the gradient vector on step $k$.

\index{Hessian!in BFGS optimizer} Although this expression for the update of
the Hessian matrix looks very complicated, the operation can be summarized as
follows:

The initial Hessian matrix used in geometry optimization is chosen as a
diagonal matrix, with the diagonal elements determined by a simple formula
based on the gradients at two geometries.  As the optimization proceeds, the
gradients at each point are used to improve the Hessian.  In particular, the
off-diagonal elements are assigned based on the old elements and the current
gradients.

Two different methods are used to calculate the displacement of $x$ in the
direction $d$. During the initial stages of geometry optimization, a line
search is used. This proceeds as follows:

\index{Line search|(}\index{NOTHIEL} The geometry is displaced by $(\alpha/4)d$
and the energy evaluated via an SCF calculation. If this energy is lower than
the original value, then a second step of the same size is made. If it is
higher, then a step of -$(\alpha/4)d$ is made. The energy is then re-evaluated.
Given the three energies, a prediction is made as to the value of $\alpha$
which will yield the  minimum  value  of the energy in the direction $d$. Of
course, the size of the steps are constrained so that the system would not
suddenly become unrealistic (e.g., break bonds, superimpose atoms, etc.).
Similarly, the contingency in which the energy versus $\alpha$ function is
inverse parabolic is considered, as are rarely-encountered curves, e.g., almost
perfectly linear regressions. By default, Thiel's FSTMIN \index{Thiel@{\bf
Thiel, Walter}} technique is used~\cite{fstmin}.  This uses gradient
information from the starting point of the search, and the calculated $\Delta
H_f$, to decide when to end the line search.  If \comp{NOTHIEL} is specified,
the older line-search is used, in which case the search is stopped when the
drop in energy on any step becomes less than 5\% of the total drop or 0.5
kcal/mol, whichever is smaller.

An  important modification has  been made to the BFGS routine.  For the
line-search, Thiel's FSTMIN technique is used. This  modification make the
algorithm run faster most of the time.  However, one unfortunate result of
these changes is that there is no guarantee that as  the cycles increase, the
energy will drop monotonically.  If the calculation does not converge on a
stationary point, then re-run the job with \comp{NOTHIEL}.

As the geometry converges on a local minimum, the prediction of the search
direction becomes less accurate. There are many reasons for this. For example,
the finite precision of the SCF calculation may lead to errors in the density
matrix, or finite step sizes in the derivative calculation (if analytical
derivatives are not used) may result in errors in the derivatives. For whatever
reason, the gradient norm and energy minimum may not coincide. The difference
is typically less than 0.00001 kcal/mol and less than 0.05 units of gradient
norm. 

Normally, the initial guess to $H$, the inverse Hessian, is the unit matrix.
However, in chemical systems where the second derivatives are very large, use
of the unit matrix would result in large changes in the geometry. Thus a
slightly elongated bond length could, in the first step, change from 1.6\AA\ 
to -6.5\AA  . To prevent this catastrophe, the initial geometry is perturbed by
a small amount, thus
$$
   x_1=x_0+0.01\times {\rm sign}(g_0),
$$
from which a trial inverse Hessian can be constructed:
$$
  H_1(i,i)=0.01\times {\rm sign}(g_0(i))/y_1(i).
$$
A negative value for $H_1(i,i)$ would lead to difficulties
within the BFGS optimization. To avoid this, $H_1(i,i)$ is set
to $0.06/{\rm abs}(g(i))$ whenever sign$(g_0(i))/y_1(i)$ is negative.

As the optimization proceeds, the inverse Hessian matrix becomes more accurate.
However, as the geometry steadily changes, the inverse Hessian will contain
information which does not reflect the current point. This can lead to the
predicted search direction vector making an angle of more than $90^{\circ}$
with the gradient vector. In other words, the search direction vector may point
uphill in energy. To guard against this, the inverse Hessian is re-initialized
whenever the cosine of the angle between the search direction and the gradient
vector drops below 0.05.

Originally the Davidson-Fletcher-Powell technique was used, but in rare
instances it failed to work satisfactorily. The BFGS formula appears to work as
well as or better than the DFP method most of the time. In the infrequent case
when the DFP is more efficient, the increase in efficiency of the DFP can
usually be traced to a fortuitous choice of a search direction. Small changes
in starting conditions can destroy this accidental increased efficiency and
make the BFGS method appear more efficient. A keyword, \comp{DFP}, is provided
to allow the DFP optimizer to be used.




\subsection{Optimization of one unknown}
If a system has exactly one coordinate to be optimized, then obviously one
line-search will optimize the geometry.  Because of this, the geometry
optimization is done a little differently.  Given the initial geometry, the
$\Delta H_f$ is calculated, and the line-search started. Unlike the normal
line-search, however, the search is not stopped when the minimum is almost
reached, instead, the minimum is located with quite high precision.  After the
line-search is complete, the gradients are not \index{GRADIENTS} calculated
(unless requested by \comp{GRADIENTS}). Instead, it is assumed that the
gradient is small, and the results are output.  This saves some time. However,
if \comp{GRADIENTS} is {\em not} present, and the geometry is not at a
stationary point (because other coordinates are not optimized), then the
warning message that the geometry is not at a stationary point will not be
printed.

\index{Line search|)}

\subsection{Considerations in Geometry Optimization}
The default settings in MOPAC are designed to allow most systems to be
optimized in an efficient way.  Quite often, however, problems arise.  The
following  notes are intended as background material for use when things go
wrong.

\subsubsection{Overriding the default options}
In the EigenFollowing geometry optimization method, the geometry is changed on
each cycle; if the $\Delta H_f$ decreases, the cycle is completed.  If it does
not drop, the step-size is reduced, and the $\Delta H_f$ recalculated.  Only
when the $\Delta H_f$ decreases, compared to the previous cycle, is the current
cycle considered to be successful.  During the calculation, the confidence
level or trust radius is continuously checked.  If this becomes too small, the
calculation will be stopped.  This can readily happen if (a) the geometry was
already almost  optimized; (b) a reaction path or grid calculation is being
performed; (c) if the geometry is in internal coordinates and ``big rings'' are
involved; or (d) if the  gradients are not correctly calculated (in a
complicated C.I., for example).

For cases (a) and (b), add \comp{LET} and \comp{DDMIN=0}.  In case (c) use
either mixed coordinates or entirely Cartesian coordinates. Case (d) is
difficult---if nothing else works, add \comp{NOANCI}; this will always cause
the derivatives to be correctly calculated, but will also use a lot of time.

Adding \comp{LET} and \comp{DDMIN=0} is often very effective, particularly when
reaction paths are being calculated.  The first geometry optimization might
take more cycles, but the resulting Hessian matrix is better tempered, and
subsequent steps are generally more efficient.

\subsubsection{Locating Transition States}\index{Locating transition states}
\index{Transition states!locating}\index{Narcissistic reactions}
Unlike optimizing ground states, locating transition states involves deciding on 
an efficient strategy.  In general, there are three stages in locating 
transition states:
\begin{enumerate}
\item Generating a geometry in the region of the transition state.
\item Refining the transition state geometry.
\item Characterizing the transition state.
\end{enumerate}
Of these three, the first is by far the most difficult.  The following 
approaches are suggested as potential strategies for generating a geometry in 
the region of the transition state.

{\bf For narcissistic reactions (reactions in which the reactants and 
products are the same, e.g.\ the inversion of ammonia.}
\begin{itemize}
\item Use  geometry constrains, e.g.\ \comp{SYMMETRY}, to lock the geometry in
the symmetry of the potential transition state.
\item Minimize the $\Delta H_f$.
\item Verify that the system is a transition state.  If it has more than one 
negative force constant, use another method.
\end{itemize}

{\bf For a bond making-bond breaking reaction (e.g., an S$_{N^2}$ reaction)}
\begin{itemize}
\item  Use \comp{SYMMETRY} to set the two bonds equal.  If does {\em not} 
matter that the bonds are of different type. For example, to locate the
transition  state for Br$^-$ reacting with CH$_4$ to give CH$_3$Br, the C--Br
and C--H bonds  would be set equal.
\item  Optimize the geometry, to minimize the $\Delta H_f$.  Any geometry
optimizer could be used, but of course the default optimizer should be tried
first.
\item Remove the symmetry constraint, and locate the transition state using
\comp{TS}.
At this point, the main geometric change is to adjust the two bond lengths
involved in the reaction.
\end{itemize}

{\bf For barriers to rotation, inversion, or other simple reaction that 
does not involve making or breaking bonds}
\begin{itemize}
\item Optimize the starting geometry.
\item Optimize the final geometry.
\item Identify the coordinate that corresponds to the reaction. This is likely
to be an angle or a dihedral.
\item Starting with the higher energy geometry, use a \hyperref[pageref]{path option}{ (see 
p.~}{)}{rpaths} to drive the reaction in the direction of the other
geometry.   Use about 20 points, and go about half way to the other
geometry---the transition state is likely to be between the higher energy
geometry and the half-way point.
\item From the output, locate the highest energy point---this will be near to
the transition state.
\item Starting with the geometry of the highest energy point, repeat the path 
calculation.  Use smaller steps (0.1 times the previous step is usually OK),
and again do 20 points.
\item Inspect the reaction gradient.  It should drop as the transition state is
approached.  If it does, then use \comp{TS} to refine the transition state.
\end{itemize}

{\bf For bond making or bond breaking reactions}
\begin{itemize}
\item Identify the reaction coordinate (the bond that makes or breaks)
\item Use a \hyperref[pageref]{path calculation}{ (see 
p.~}{)}{rpaths} to drive the reaction.
\item The geometry of the highest point on the reaction path should then be 
used to start a \comp{TS} calculation.
\end{itemize}

{\bf For complicated reactions (e.g.\ Diels Alder)}
For these systems, the \comp{SADDLE} calculation is a suitable method.
\begin{itemize}
\item Optimize the reactant geometry.
\item Using the same atoms in the same sequence, optimize the product geometry.
\item Run the \comp{SADDLE} calculation.
\item If the calculation ends because ``both reactants and products are on
 the same side of the transition state,'' use two of the geometries to set
up a new \comp{SADDLE} calculation.  Use a smaller \comp{BAR=$n.nn$}, e.g.,
\comp{BAR=0.03}, and re-run the calculation.  If CPU time is not important,
run the original data set with \comp{BAR=0.03}.
\item Use the final geometry, or the highest energy geometry, if the 
\comp{SADDLE} does not run to completion, as the starting point for a 
\comp{TS} calculation.
\end{itemize}


\section{``Size'' of a molecule}\index{Size of molecule}\index{Molecular size}
A useful measure of a molecule is its size.  There are several possible ways of
defining the size of a molecule.  The definition used in MOPAC is as follows:

The first dimension is the maximum distance between any pair of atoms.  

For systems  of 20 or fewer atoms, this distance, and the atoms involved, is
worked out explicitly.  For systems of 21 or more atoms, an atom is selected;
the atom, $K$,  most distant from it is then identified, then the atom, $L$,
most distant  from $K$ is identified.  In most systems, the distance R($K-L$)
is the first dimension.  To ensure that it is, the point half-way between $K$
and $L$ is  selected, and atom $K$ is then re-defined as the atom most distant
from  that point.  A new $L$ is determined.  This sequence in repeated up to 10
times, or until atoms $K$ and $L$ no longer change. There is no guarantee that
the first dimension is, in fact, the largest distance, but it is likely to be
close to the largest possible value.

The second dimension is the maximum distance in the plane perpendicular to the
first dimension between any pair of atoms. 

The technique that was used in determining the first dimension for systems of
over 21 atoms is used here.

The third dimension is the maximum distance between any two atoms on the line
perpendicular to the plane of the first two atoms.

This quantity is explicitly calculated.

Note that the second and third dimensions do {\em not} define the smallest 
rectangular slot that a molecule would go through; it will normally be slightly
larger than the minimum slot. Nevertheless, the ``dimensions'' of a molecule
can be regarded as a good measure of the size of hole that the molecule could
pass through.  Of course, allowance must be made for the finite size of atoms.

Monatomic systems have no ``dimension'', linear systems have two zero
``dimensions'', and flat systems have one zero ``dimension''.




\section{Solid state capability}\label{solid-state}\index{Solid state!description}
%{\sc Users are warned that the two and three dimensional calculations are
%relatively new, and only limited testing has been done to verify their
%validity.  Extreme caution should be exercised when these calculations are
%being done.  In particular, care should be taken to ensure that the cluster
%is sufficiently large that the translation vector conditions described
%below are met. Many other considerations, such as the significance of using
%only one point in $k$-space, have not been fully explored.  The solid-state
%capability should be regarded as a research tool, and should not be used
%except be researchers familiar with solid-state concepts.}

\subsection{Constructing Data Sets}
Setting up the data set for a solid is much more complicated than
setting up the equivalent data set for a molecule. For example,
the data set for iodine, I$_2$, might take say 20 seconds to type
up. For crystalline I$_2$, the data set might take 20 minutes to
an hour to set up. The reason for this is the presence in solids
of the translation vectors. In MOPAC, these are represented by the
symbol ``Tv", short for Translation Vector. Translation vectors
must be the last entries in the Z-matrix, are defined in terms of
the positions of atoms or dummy atoms.

The recommended procedure for constructing a data set for a solid
is as follows:

Build a primitive unit cell. In this work, the action of simple
translations on a primitive unit cell would make the solid. That
is, other operations, such as rotation, non-primitive translation,
or reflection should not be necessary.

Use MAKPOL to build a cluster. The number of primitive unit cells
in each direction should be sufficiently large that the shortest
distance between two opposite faces in the cluster is at least 10
\AA ngstroms.

Using an editor (vi, Notepad, or WORD, in order of preference),
add symmetry to the system.

Run using MOPAC

Worked examples of this procedure, with explanation of each step
are shown at the end of this section, see P. \pageref{diamond}.


\subsection{The Cluster}
Unlike more conventional methods, MOPAC does not normally use a fundamental
unit cell.  Neither does it sample the Brillouin Zone in order to model the
electronic structure.  Instead, it uses a large unit cell, called a `cluster',
\index{Born-von K\'{a}rm\'{a}n! periodic boundary conditions} \index{Periodic
boundary conditions|ff} and applies the Born-von K\'{a}rm\'{a}n
~\cite{bornvon1,bornvon2} periodic  boundary conditions.  In this discussion,
the term `solid' is intended to include polymers, layer systems, and true
solids, unless otherwise indicated by the context.

If a unit cell of a solid is large enough,  then  a  single  point  in
\index{Gamma@{$\Gamma$ point!in solid state}} $k$-space,  the $\Gamma$ point,
is sufficient to specify the entire Brillouin zone.  The secular determinant
for this  point  can  be  constructed  by adding together the Fock matrix for
the central unit cell plus those for the adjacent unit cells.  The periodic
boundary conditions are satisfied, and diagonalization yields the correct
density matrix for the $\Gamma$ point.

At this point  in  the  calculation,  conventionally,  the  density matrix
for  each  unit  cell  is constructed.  Instead, the $\Gamma$-point density
and  one-electron  density  matrices  are   combined   with  \index{Coulomb
integrals} ``$\Gamma$-point-like'' Coulomb  and  exchange integral strings to
produce a new Fock matrix.  The  calculation  can  be  visualized  as  being
done entirely in reciprocal space, at the $\Gamma$ point.

Most  solid-state  calculations  take  a  very  long  time.   These
calculations,   called ``Cluster'' calculations   after  the  original
publication, require between 1.3 and 2 times  the  equivalent  molecular
calculation.\index{Cluster model}

A minor `fudge'  is  necessary  to  make  this  method  work.   The
contribution  to  the  Fock  matrix  element  arising  from the exchange
integral between an atomic orbital and all atomic orbitals which are more than
half a unit cell away must be ignored.

The unit cell must be large enough that an atomic  orbital  in  the
\index{Translation vector!requirements} center  of  the  unit  cell has an
insignificant overlap with the atomic orbitals at the ends of the  unit
cell.   In  practice,  a  translation vector  of more that about 7 or 8\AA\ is
sufficient.  For one rare group of compounds a larger translation vector is
needed.  Solids with delocalized  $\pi$--systems,  and  solids  with  very
small band-gaps will require a larger translation  vector,  in  order  to
accurately  sample k-space.   For these systems, a translation vector in the
order of  15--20 \AA ngstroms is needed.


\subsection{Derivatives}
Solid-state derivatives with respect to geometry are handled differently from
molecule derivatives.  If the Cartesian coordinate derivatives are printed,
using \comp{DEBUG} and \comp{DCART}, then for a molecule with an  optimized
geometry all the derivatives will be zero.  This is  not the case for an
infinite system.

An infinite system is represented by cell supplied by the user, called the
Central Unit Cell, or the CUC, and the cells surrounding this CUC. When
\comp{DCART}, \comp{LARGE}, and \comp{DEBUG} are used in an infinite system
calculation for which  the geometry has been optimized, the Cartesian
derivatives for all unit cells are output. Many of these will be quite large,
up to about 60 kcal/mol/\AA .  This is not an error, rather it is a peculiarity
of the way solid-state derivatives are stored.

The Cartesian derivatives of the CUC represent the sum of all forces acting on
the atoms of the CUC due to all the atoms in the CUC.  Thus, if the atoms in
the CUC are the set ($a$,$b$,$c$,$d$,$e$,$f$), then the Cartesian  derivatives
for atom $a$ represent the forces on $a$ due to the set
($b$,$c$,$d$,$e$,$f$).  The Cartesian derivatives of atom $a$ do {\em NOT}
include terms from  the surrounding unit cells.  Because of this, those atoms
in the CUC  which are at the cell boundaries are likely to have large
derivatives.

The Cartesian derivatives of the surrounding unit cells represent the  forces
acting on the atoms in those cells arising from the atoms of the CUC.  Again,
this is an unbalanced set of forces, and those atoms near to the cell
boundaries are likely to have large resultant forces.

It is possible to evaluate the total, balanced, forces acting on the atoms of
the CUC.  This is done by simply adding the forces acting on the atoms of the
three unit cells.  When the keywords given above are used, the last part of the
derivative output consists of the forces acting on the CUC itself.

Only by representing the forces in this unusual manner can the information
necessary for calculating the derivative of the translation vector be
generated.


\subsection{Geometry Specification for Band Structure Calculations}\label{polygeo}
\index{Band structure!geometry requirements}
Before electronic band structure calculations can be done, the sequence of
atoms in the polymer must be supplied in a highly specific order.  For a simple
polymer, the coordinates of all the atoms in the first fundamental unit cell
are given.  These atoms can be in any order.  The next set of atoms defined are
those for the next unit cell.  These atoms {\em must} be in the same order as
the atoms in the first unit cell. For band structures at least two unit cells
must be defined.  If more than two unit cells are defined, the atoms in the
other  unit cells must be defined in the same order as those in the first unit
cell.

For all polymer calculations {\em except} band structures, the order of
atoms is not important.  \htmlref{An example of such a data set}{pthf}
is shown
\begin{latexonly}
on p~\pageref{pthf}
\end{latexonly} for
polytetrahydrofuran.  When band structures are to be calculated,
\hyperref[pageref]{the order of atoms is important}{. For an example,
see p.~}{}{polyc2h4}.

Because of the difficulty in generating data sets for band-structure work,
program \comp{BZ} was written.  Given a suitable data-set, BZ will generate a
MOPAC data set which can then be used for the calculation of band structures.

\subsection{Electronic Band Structure}
\index{Band structure!electronic}
In a normal cluster calculation, the Fock matrix is diagonalized to yield
eigenvalues corresponding to various points  in the Brillouin zone.  For
$m$ unit cells, the points generated  are $0, 1/m, 2/m$, \ldots  up to $1/2$.  If
$m$ is odd, the upper bound  becomes $(m-1)/(2m)$.  No other points in the
Brillouin zone can be generated by diagonalization.

In order to represent a general point, $k$, in the Brillouin  zone, a complex
secular determinant, $F_{k}$, of size $n$ must be  constructed.  The elements
of this matrix are
$$
 F_{k}(\lambda,\sigma) = \sum_{r=-\infty}^{r=
\infty}E(\lambda,\sigma+nr)e^{-ikr2\pi}.
$$
Because interactions between atomic orbitals fall off  rapidly with distance,
the limits of $r$ can be  truncated to include all non-vanishing elements of
$E$, for the sake of convenience. However,  these elements are precisely those
which were used in the  construction of the Fock matrix.  Using this, and the
fact that  periodic boundary conditions were employed in the construction of
the Fock matrix, this summation can be simplified to
$$
 F_{k}(\lambda,\sigma) = \sum_{r=0}^{r=m-1}E(\lambda,\sigma+nr)exp(-ikr'2\pi),
$$
where $r'$, the index of the unit cell, equals $r$ while $r$ is less  than
$m/2$, otherwise $r' = m-r$.  Band structures can then readily be constructed
by varying  the wave-vector $k$ over the range 0--0.5.  Units of $k$ are
$2\pi/a$,  where $a$ is the fundamental unit cell repeat distance.  The band
structure is then constructed by simply joining the points in the
\index{Crossing, bands in solid state} order in which they are generated.
Within band structures, bands  of different symmetry are allowed to cross.
Simply joining the  points does not allow for band crossing.  However, when
the  resulting bands are represented graphically, visual inspection  readily
reveals which bands should, in fact, cross.

\subsection{Electronic Density of States}
\index{Density of states!electronic}\index{Electronic density of states}
The density of states, DoS, is the spectrum of the number of  energy levels per
eV versus energy.  While the energy levels  resulting from the calculation of
the band structure could be  used directly for the calculation of the DoS, the
resulting DoS  would be very rough as a result of the relatively coarse mesh
used.  A better procedure is to assume continuity of the bands, and, by using
an interpolation procedure, numerically integrate.   A possible complication
arises from the incorrect representation  of bands which should cross.  In
practice, however, errors due to  such causes are so small as to not show up in
a normal graphical  representation of the DoS.

At present, the DoS is calculated in MOPAC (not in BZ), and only
for one-dimensional systems, i.e., polymers.

\subsection{Brillouin Zone: Generation of Band Structures}
Using a modified cluster technique, band structures of  polymers can readily be
calculated.  When a sufficiently large  repeat unit is used,  errors introduced
due to  the methodology of the cluster procedure become vanishingly small.
\index{Polyacetylene}\index{Delocalized $\pi$ systems} Even for delocalized
$\pi$ systems, such as polyacetylene, accurate band structures can be
generated  using repeat units of about 20\AA .  For less highly conjugated
systems, a shorter cluster length should be sufficient.   \index{Oligomers} In
contrast to earlier oligomer work, no allowance need be  made for end-effects.
In addition, the set of points in the BZ  to be used is determined explicitly
by the step-size.

The technique outlined here is very fast compared to earlier
methods~\cite{mosol}.

Geometry optimization of clusters of the size reported  here (i.e., having
translation vectors of about 25\AA\,) require  only a little more time than
molecules of similar size, the extra time  being used to calculate the
inter-unit cell interactions.  Band  structure calculations are also very
fast.  The time required  depends on the size of the fundamental unit cell.
For  polyacetylene, this amounted to 3\% of the time for a single self
consistent field calculation of the cluster.

Band structures calculated using the program BZ are accurate in the sense  that
any errors are due to the Hamiltonian used.  A more accurate  \index{ab
initio@{{\em ab initio}}!band structures} method, for example a large basis set
ab initio calculation,  should yield highly accurate band structures.  In
addition,  limited use of symmetry in the construction of the cluster  secular
determinant and in the geometry optimization should  increase the speed of such
a calculation considerably.   Electrical conductivity in semiconductors is
caused by holes  in the valence band and electrons in the conduction band.
Conductivity also depends on the hole and electron effective  masses, which are
readily calculable from the second derivative  of the energy of the band with
respect to wave-vector.  Band  structures for linear polymers, calculated using
semiempirical  methods, should be suitable for calculation of effective
masses,  \index{Electron!effective mass} and consequently electrical
conductivity.  Unfortunately, NDDO  type semiempirical methods have not proven
very accurate at  predicting conduction band levels.  As a result, in order to
rapidly calculate electrical phenomena, it is likely that a  combination of ab
initio methods and the cluster technique will  be necessary.

As generated by MOPAC, the Fock matrix is unsuitable for band-structure work.
First, the matrix represents the cluster, not the unit cell, and second, the
Fock matrix will not exhibit the high symmetry of the associated space-group.
The perturbation is small, but fortunately it can readily be eliminated.

% 56 lines, including this line
\begin{table}
\caption{\label{spgoh7}Space-group operations for $O_h^7$ (diamond)}
\compresstable
\begin{center}
\begin{tabular}{cccc} \hline
 $\{E|000\}$
&$\{C_2(1,1,0)|\frac{1}{2},\frac{1}{2},\frac{1}{2}\}$
&$\{I|\frac{1}{2},\frac{1}{2},\frac{1}{2}\}$
&$\{\sigma_d(0,1,-1)|000\}$ \\
 $\{C_3(1,1,1)|000\}$
&$\{C_2(1,0,1)|\frac{1}{2},\frac{1}{2},\frac{1}{2}\}$
&$\{S_4(0,0,1)|000\}$
&$\{S_6(1,1,1)|\frac{1}{2},\frac{1}{2},\frac{1}{2}\}$ \\
 $\{C_3(1,-1,\! -1)|000\}$
&$\{C_2(0,1,1)|\frac{1}{2},\frac{1}{2},\frac{1}{2}\}$
&$\{S_4(0,1,0)|000\}$
&$\{S_6(1,-1,\! -1)|\frac{1}{2},\frac{1}{2},\frac{1}{2}\}$ \\
 $\{C_3(-1,1,\! -1)|000\}$
&$\{C_2(0,1,-1)|\frac{1}{2},\frac{1}{2},\frac{1}{2}\}$
&$\{S_4(1,0,0)|000\}$
&$\{S_6(-1,1,\! -1)|\frac{1}{2},\frac{1}{2},\frac{1}{2}\}$ \\
 $\{C_3(-1,\! -1,1)|000\}$
&$\{C_2(1,0,-1)|\frac{1}{2},\frac{1}{2},\frac{1}{2}\}$
&$\{S_4^2(0,0,1)|000\}$
&$\{S_6(-1,\! -1,1)|\frac{1}{2},\frac{1}{2},\frac{1}{2}\}$ \\
 $\{C_3^2(1,1,1)|000\}$
&$\{C_2(1,-1,0)|\frac{1}{2},\frac{1}{2},\frac{1}{2}\}$
&$\{S_4^2(0,1,0)|000\}$
&$\{S_6^5(1,1,1)|\frac{1}{2},\frac{1}{2},\frac{1}{2}\}$ \\
 $\{C_3^2(1,-1,\! -1)|000\}$
&$\{C_4(0,0,1)|\frac{1}{2},\frac{1}{2},\frac{1}{2}\}$
&$\{S_4^2(1,0,0)|000\}$
&$\{S_6^5(1,-1,\! -1)|\frac{1}{2},\frac{1}{2},\frac{1}{2}\}$ \\
 $\{C_3^2(-1,1,\! -1)|000\}$
&$\{C_4(0,1,0)|\frac{1}{2},\frac{1}{2},\frac{1}{2}\}$
&$\{\sigma_d(1,1,0)|000\}$
&$\{S_6^5(-1,1,\! -1)|\frac{1}{2},\frac{1}{2},\frac{1}{2}\}$ \\
 $\{C_3^2(-1,\! -1,1)|000\}$
&$\{C_4(1,0,0)|\frac{1}{2},\frac{1}{2},\frac{1}{2}\}$
&$\{\sigma_d(1,0,1)|000\}$
&$\{S_6^5(-1,\! -1,1)|\frac{1}{2},\frac{1}{2},\frac{1}{2}\}$ \\
 $\{C_2(0,0,1)|000\}$
&$\{C_4^3(0,0,1)|\frac{1}{2},\frac{1}{2},\frac{1}{2}\}$
&$\{\sigma_d(0,1,1)|000\}$
&$\{\sigma_h(0,0,1)|\frac{1}{2},\frac{1}{2},\frac{1}{2}\}$ \\
 $\{C_2(0,1,0)|000\}$
&$\{C_4^3(0,1,0)|\frac{1}{2},\frac{1}{2},\frac{1}{2}\}$
&$\{\sigma_d(1,-1,0)|000\}$
&$\{\sigma_h(0,1,0)|\frac{1}{2},\frac{1}{2},\frac{1}{2}\}$ \\
 $\{C_2(1,0,0)|000\}$
&$\{C_4^3(1,0,0)|\frac{1}{2},\frac{1}{2},\frac{1}{2}\}$
&$\{\sigma_d(1,0,-1)|000\}$
&$\{\sigma_h(1,0,0)|\frac{1}{2},\frac{1}{2},\frac{1}{2}\}$ \\ \hline
\end{tabular}
\end{center}
\end{table}
The steps involved in converting the MOPAC Fock matrix into one suitable
for band-structure work are as follows:
\begin{description}
\item[Generation of solid-state Fock matrix]~\\ BZ assumes that the unit cells
used in constructing the MOPAC cluster were \index{MAKPOL|ff} supplied in the
order defined in MAKPOL.  Based on this assumption, the first unit cell will
have the index (0,0,0).  If there are N atomic functions in a unit cell, then
the first N rows of the MOPAC Fock matrix will correspond to the central unit
cell (CUC).  Of all the unit cells, this one is the only one for which the
entire Fock matrix is not present; instead only the lower-half triangle is
available.  However, since the CUC is symmetric, the missing data are generated
by forming the transpose, i.e., H$_{i,j}$ = H$_{j,i}$.

The Fock matrix representing the interaction of the CUC with the next
\index{BCC} unit cell, (0,0,1), or (0,0,2) if BCC is specified, is then
extracted, as are all the small Fock matrices.  Each Fock matrix, representing
the CUC interacting with each unit cell, is stored in a large array, of size
N$^2$ times the number of unit cells used.  As phrases of the type ``The Fock
matrix  representing the interaction of the CUC with unit cell (i,j,k)'' are
cumbersome, from here on, the term ``unit cell (i,j,k)'' should be understood
as having the same meaning.

The indices of each unit cell is also generated and stored.  However, the
cluster theory assumes that the interaction matrix relating two unit cells
which  are separated by more than half the distance of the translation vector
does not represent that interaction. Rather, it represents the interaction of
two unit cells which are separated by less that half the translation vector
distance.  In order to conform with this definition, all unit cell indices more
that half of the number \index{MERS} of unit cells specified by the \comp{MERS}
keyword are changed.  For example, if \comp{MERS(4,4,4)} is used, the unit
cells (0,1,1) and (2,2,2) would be unchanged, but unit cells(0,1,3) and (0,0,4)
would become (0,1,-2) and (0,0,-1), respectively.

As a result of this operation, most of the unit cells surrounding the CUC are
generated.  The next step is to symmetrize the Fock matrices so that they have
the symmetry of the space group.  Note that if symmetrization is not done, the
band-structures generated would be almost, but not quite, identical to those
which use symmetrized Fock matrices.

\item[Symmetrization of Fock matrices]~\\
\index{Fock matrix!symmetrization in solid state calcs.}
From group-theory we know that if a matrix is operated on by every operation of
a group exactly once, the resulting matrix will have the symmetry of that
group.  In other words,
$$
F_{sym} = \frac{1}{M}\sum_{i=1}^M<R_i|F_{unsym}|R_i^T>.
$$
The index $i$ covers all operations of the group, including the identity.

\end{description}

\subsubsection{Space-group operations}
\index{Space group!theory} Space-group operations differ from point-group
operations in that in addition \index{Non-primitive translations|ff} to the
point-group operation, a non-primitive translation may be involved. Thus far,
we have been using as our example the diamond lattice which is suitable for
illustrating space-group operations.  For convenience, \index{Space
group!operations|ff} we will specify a space-group operation thus: \{R$|$T\},
where ``R'' is a point-group operation, e.g.\ C$_2$(0,0,1) or S$_6$(1,1,-1),
and ``T'' is a non-primitive translation, e.g.\
($\frac{1}{2}$,$\frac{1}{2}$,$\frac{1}{2}$), or (0,0,0). The term in
parentheses following the point group operation indicates the axis about which
the operation is to be performed.  Finally, to complete the specification of a
space-group operation, the point about which the operation acts must be
defined.  As this is most conveniently done in fractional unit
\index{Fractional!unit cell coordinates} \index{Coordinates!fractional unit
cell} cell coordinates, or crystal coordinates, the Cartesian coordinates are
converted at this time into fractional unit cell coordinates.

\index{oh7@{$O_h^7$}}\index{Fd3m@{$Fd3m$}} Diamond belongs to the
$Fd3m$ or $O_h^7$ space-group, and has octahedral symmetry; its
associated point-group is O$_h$.  The space-group operations are
given in Table~\ref{spgoh7}.
\subsection{Examples of Solid State Data Set Construction}
 \label{diamond}
 \subsubsection{Diamond}\index{Diamond}\index{Data!Diamond!construction of}\index{Solid state!data sets}

The unit cell of diamond consists of two carbon atoms. If symmetry operations were allowed,
 only one atom would be needed, but only translation operations are allowed, so two atoms
 must be used.

Each carbon atom forms four single bonds with other carbon atoms. 
This can be used in defining the translation vectors: the effect 
of a translation vector acting on atom 1 would be to move it to 
one of the atoms attached to atom 2. 
 Using this fact, the data set can easily be made:

First attempt at a data set for MAKPOL. \begin{verbatim}
MERS=(4,4,4)  Diamond
C
C  1.545 1 0 0 0 0 1
Tv 2.523 1 35.26439 10 0 1 2
Tv 2.523 1 35.26439 1 120 1 1 2 3
Tv 2.523 1 35.26439 1 240 1 1 2 3
\end{verbatim}

In this description, the number of unit cells to be used is too
small: 4 by 4 by 4.  Ideally, at least a 6 by 6 by 6 system
 should be use. The smaller number is used here purely for the
 purposes of illustration.



This data set is, however, not ideal, for the following reasons:

The angle between the translation vectors is 60 degrees. This 
means that the unit cell must be very large in order for the 
distance between opposite faces to be at least 10 \AA ngstroms. 
Diamond is a Body-Centered-Cubic lattice. This means that every 
odd cell (e.g. 001,111,012, etc.) is missing. By specifying BCC, 
the angle between the translation vectors can be increased to 90 
degrees. This is the ideal angle to maximize the distance between 
faces. The system has a lot of symmetry. By adding ``SYMMETRY" two 
objectives can be met: First, symmetry can be used in defining the 
Tv. This is not very important in this unit cell, but does make it 
easier to change the bond-length of atom 2, in that any change in 
this distance is automatically made in the Tv. Second, if SYMMETRY 
is present, MAKPOL will automatically add symmetry to the data 
set.

\begin{verbatim}
Data set for MAKPOL. File name: Make_diamond.dat
 SYMMETRY MERS=(4,4,4) BCC
 Diamond, 64 atoms 
C 
C 1.545 1 0 0 0 0 1 
Tv 1.784 0 54.73561 0 0 0 1 2 
Tv 1.784 0 54.73561 0 120 0 1 2 3 
Tv 1.784 0 54.73561 0 240 0 1 2 3 

2 19 1.1547005 3 4 5 
\end{verbatim}

When this data set is run using MAKPOL, the following data set is 
generated: 
\begin{verbatim}
SYMMETRY MERS=(4,4,4) BCC 
Diamond, 64 atoms 

C  0.000000 0   0.000000 0    0.000000 0 
C  1.545000 1   0.000000 0    0.000000 0 1 
C  3.568025 1  54.735610 1    0.000000 0 1 2 
C  1.545000 0 125.264390 1    0.000000 1 3 1 2 
C  2.522974 1  35.264390 1   60.000000 1 1 2 3 
C  1.545000 0 144.735610 1    0.000000 0 5 1 2 
C  2.522974 0 135.000000 1   45.000000 1 3 1 2 
C  1.545000 0 144.735610 0  -90.000000 1 7 3 1 
C  2.522974 0  90.000000 1   90.000000 0 5 1 2 
C  1.545000 0  90.000000 0 -144.735610 1 9 5 1  

(many lines deleted) 

C  1.545000 0  90.000000 0  160.528779 1 61 57 41 
C  2.522974 0  60.000000 0  -35.264390 0 59 43 26
C  1.545000 0  90.000000 0  160.528779 0 63 59 43 
XX 2.522974 0  90.000000 0  180.000000 0 7   3 1 
XX 2.522974 0  90.000000 0 -125.264390 0 29  9 5 
XX 2.522974 0 180.000000 0    0.000000 0 53 41 42 
Tv 7.136049 1   0.000000 0    0.000000 0 1  65 2 
Tv 7.136049 0   0.000000 0    0.000000 0 1  66 2 
Tv 7.136049 0   0.000000 0    0.000000 0 1  67 2 

2 1 4 6 8 10 12 14 16 18 20 22
2 1 24 26 28 30 32 34 36 38 40 42
2 1 44 46 48 50 52 54 56 58 60 62
2 1 64
4 3 6 11 12 15 16 18 33 34 35 36
4 3 39 40 43 44 45 51 52 53 54 55
4 3 56 59 60 67
5 1 7 9 11 13 15 17 19 21 23 25
5 1 27 29 31 33 35 37 39 41 43 45
5 1 47 49 51 53 55 57 59 61 63 65
5 1 66 67
5 2 17
5 14 17

(many lines deleted) 

\end{verbatim}
Before running this data set, more symmetry can be added 
``by hand." Every angle is symmetry defined, and does not need to 
be optimized, therefore every angle and dihedral optimization flag 
can be set to zero. Every angle and dihedral symmetry relation 
defined at the end of the data set can also be deleted. These are 
the lines that start with a number followed by a 2, a 3, or a 14. 
Every distance can be related to the bond-length of atom 2. The 
position of atoms 3 and 5, and the length of the translation 
vector Tv can be defined by distances that are exactly Sqrt(16/3). 
Sqrt(8/3) and Sqrt(64/3) times the C$_2$-C$_1$ 
distance, respectively. These symmetry relations can be defined 
using MOPAC symmetry function 19. This has the form: Defining-atom 
19 multiplier dependent atom(s) When these changes are made to the 
data set, the final data set is produced. This is: 
\begin{verbatim}
 SYMMETRY MERS=(4,4,4) BCC 
 Diamond, 64 atoms

  C    0.000000  0   0.000000  0    0.000000  0
  C    1.545000  1   0.000000  0    0.000000  0     1
  C    3.568025  0  54.735610  0    0.000000  0     1     2
  C    1.545000  0 125.264390  0    0.000000  0     3     1     2
  C    2.522974  0  35.264390  0   60.000000  0     1     2     3
  C    1.545000  0 144.735610  0    0.000000  0     5     1     2
  C    2.522974  0 135.000000  0   45.000000  0     3     1     2
  
  (many lines deleted) 
 
  C    1.545000  0  90.000000  0  160.528779  0    61    57    41
  C    2.522974  0  60.000000  0  -35.264390  0    59    43    26
  C    1.545000  0  90.000000  0  160.528779  0    63    59    43
 XX    2.522974  0  90.000000  0  180.000000  0     7     3     1
 XX    2.522974  0  90.000000  0 -125.264390  0    29     9     5
 XX    2.522974  0 180.000000  0    0.000000  0    53    41    42
 Tv    7.136049  0   0.000000  0    0.000000  0     1    65     2
 Tv    7.136049  0   0.000000  0    0.000000  0     1    66     2
 Tv    7.136049  0   0.000000  0    0.000000  0     1    67     2

   2 19 2.3094  3
   2 19 1.6330  5
   2 19 4.6188 68
   2  1    4    6    8   10   12   14   16   18   20   22
   2  1   24   26   28   30   32   34   36   38   40   42
   2  1   44   46   48   50   52   54   56   58   60   62
   2  1   64
   5  1    7    9   11   13   15   17   19   21   23   25
   5  1   27   29   31   33   35   37   39   41   43   45
   5  1   47   49   51   53   55   57   59   61   63   65
   5  1   66   67
  68  1   69   70
\end{verbatim} 

Is the use of symmetry worth all this effort? Most 
definitely! If symmetry is NOT used, then for this small system of 
64 atoms, 195 parameters would need to be optimized. That is, 
3*64-6 for the 64 atoms plus 9 parameters for the three 
translation vectors. Optimization of a system with 195 unknowns 
would take much longer than for a system with precisely one 
unknown. For a more realistic system, involving 6 by 6 by 6 
primitive unit cells, symmetry would lower the complexity of the 
calculation from 651 unknowns to precisely 1. 


\section{Point Group Theory}
\index{Groups|(} \index{Point-group!theory} \index{Symmetry!in group theory}
This Section is based on the original work of Peter Bischof in the UMNDO
program, and made available to me by Dr David Danovich.

Some point-group theory has been added to MOPAC.  The main functionalities
added are: \index{AUTOSYM}\index{Bischof@{\bf Bischof, Peter}}
\index{UMNDO}\index{Danovich@{\bf Danovich, David}}

\begin{itemize}
\item `Normal' symmetry relationships are now automatically recognized if
\comp{AUTOSYM} is specified.

\item The symmetry of the system is printed both at the start of the run and at
the end.  If the point-group changes, the change will be shown in the different
point-group symbols.

\item Molecular orbitals will be characterized by Irreducible Representation 
(I.R.). 

\item Normal coordinates generated in the vibrational calculation will be
characterized by I.R.

\item State functions will be characterized by I.R.

\item All rotation groups up to order 8, except D$_{8d}$, are available.

\item The cubic groups T, T$_h$, T$_d$, O, O$_h$, I, and I$_h$ are available.

\item The infinite groups C$_{\infty v}$, D$_{\infty h}$ and R$_3$ are
available.

\item In \comp{FORCE} or vibrational frequency calculations, symmetry will be
used to accelerate the calculation, thus a calculation of benzene would involve
two atoms, a C and a H atom, to be calculated, rather than the normal 12 atoms.

\item In vibrational frequency calculations, the Hessian or force matrix
will be symmetrized.  \index{sym\_force} \label{sym_force}
$$
F_{ij}=\frac{1}{h}\sum_hR(h)^TF_{ij}'R(h)
$$

This eliminates the normal small deviations from exact symmetry
\begin{htmlonly}
(\htmlref{a qualification appears
elsewhere}{fc})
\end{htmlonly}
\begin{latexonly}
(see also p.~\pageref{fc}
for a qualification)
\end{latexonly}.

\end{itemize}


\subsection*{Limitations}
\begin{itemize}
\index{Point Group D$_{8d}$ missing}\index{D$_{8d}$ missing}
\item Group D$_{8d}$ is missing.  This group is characterized by the presence
of a 16-fold S$_n$ axis.  Only S$_n$ operations up to S$_{12}$  are  checked
for.  As a  result, D$_{8d}$ would not be recognized.  However, this is a rare
point-group, and its loss should not be important.

\item Some systems which are insufficiently near to a given point group will be
assigned to the nearest sub-group.  For example, if SF$_6$ is distorted so that
two opposite F atoms are at a different distance to the other four, the system
might be classified as O$_h$ or D$_{4h}$, depending on the degree of
distortion.  This shows up mainly in methyl groups, e.g.\ neopentane, in which
optimization normally stops before the angles of the hydrogens are fully
optimized.
\end{itemize}


\subsection{Representation of Point Groups}
The 57 groups recognized in MOPAC are given in Table~\ref{pgs}.
\begin{table}
\caption{\label{pgs} Point Groups available within Symmetry Code}
\begin{center}
\begin{tabular}{lllllllll} \hline
 C$_1$&C$_{s} $ & C$_{i }$   &       &      &      &     &  O             \\
 C$_2$&C$_{2v}$ & C$_{2h}$   & D$_2$ & D$_{2d}$  & D$_{2h}$  &     &  T   \\
 C$_3$&C$_{3v}$ & C$_{3h}$   & D$_3$ & D$_{3d}$  & D$_{3h}$  &     &  T$_d$   \\
 C$_4$&C$_{4v}$ & C$_{4h}$   & D$_4$ & D$_{4d}$  & D$_{4h}$  & S$_4$  &  T$_h$   \\
 C$_5$&C$_{5v}$ & C$_{5h}$   & D$_5$ & D$_{5d}$  & D$_{5h}$  &     &  O$_h$   \\
 C$_6$&C$_{6v}$ & C$_{6h}$   & D$_6$ & D$_{6d}$  & D$_{6h}$  & S$_6$  &  I & I$_h$   \\
 C$_7$&C$_{7v}$ & C$_{7h}$   & D$_7$ & D$_{7d}$  & D$_{7h}$  &  &   C$_{\infty v}$  
& D$_{\infty h}$ \\
 C$_8$&C$_{8v}$ & C$_{8h}$   & D$_8$ &      & D$_{8h}$ & S$_8$  &   R$_3$   \\ \hline
\end{tabular}
\end{center}
\end{table}

Each point group is represented by a subset of the associated point-group
table. For example, the group D$_{2h}$ is represented by the subset shown in
Table~\ref{d2h}. The operations selected for the subgroup are the identity, E,
and that minimum set of operations which is sufficient to allow all the
operations to be generated as products of these operations.  Thus, for the
highest finite point group, I$_h$, the generating operations are: E, I, C$_3$,
and C$_5$.  Although it is not obvious, all 120 operations of the group can be
generated as products of these four operations.

\begin{table} 
\caption{\label{d2h}Subset of Group D$_{2h}$}
\begin{center}
\begin{tabular}{lrrrr} \\ \hline
  $\Gamma$ &  E  &  C$_{2y}$ & C$_{2z}$& I \\ \hline
A$_g$   \\
B$_{1g}$ &1 &  1  &-1  &  1  \\
B$_{2g}$ &1 & -1  & 1  &  1  \\
B$_{3g}$ &1 & -1  &-1  &  1  \\
A$_{u} $ &1 &  1  & 1  & -1  \\
B$_{1u}$ &1 &  1  &-1  & -1  \\
B$_{2u}$ &1 & -1  & 1  & -1  \\
B$_{3u}$ &1 & -1  &-1  & -1  \\ \hline
\end{tabular}\end{center}
\end{table}

Each point-group is assumed to contain the totally symmetric representation,
here \index{Euler matrices} A$_{1g}$.  Operations are represented as $3\times3$
Euler matrices, thus C$_{2x}$, C$_{2y}$  and C$_{2z}$  would be represented as
in Figure~\ref{c2op} \ All operations not given can be generated as products of
operations already known, thus C$_{2x}$ = C$_{2y}$ $\times$ C$_{2z}$.

% 9 lines, including this line
\begin{figure}
\begin{makeimage}
\end{makeimage}
\begin{center}\hfil
C$_{2x}$: \begin{tabular}{|rrr|}1&0&0\\0&-1&0\\0&0&-1\end{tabular}\hfil
C$_{2y}$: \begin{tabular}{|rrr|}-1&0&0\\0&1&0\\0&0&-1\end{tabular}\hfil
C$_{2z}$: \begin{tabular}{|rrr|}-1&0&0\\0&-1&0\\0&0&1\end{tabular}\hfil
\end{center}
\caption{\label{c2op} Representation of Symmetry Operations}
\end{figure}

In order to minimize storage, the characters in  character tables are stored
separately from the point groups.  This allows, e.g., C$_{2v}$, C$_{2h}$, and
D$_{2}$ to use the same character table. 

\subsection{Identification of Point-Groups}
\index{Infinite groups}
\subsubsection*{Infinite Groups}
In order to identify the molecular point-group the system must be oriented in a
specific way.  Four families of point-groups are checked for: (1) the infinite
groups, (2) the cubic groups, (3)  groups with one high-symmetry axis, and (4)
the Abelian groups. Each family is treated differently.  First, the moments of
inertia are calculated.  If all are zero, the system is a single atom, and the
associated group is R$_3$.  If two moments are zero, the system is  either
C$_{\infty v}$ or D$_{\infty h}$; the presence of a horizontal plane of
symmetry distinguishes between them.

\index{Cubic Groups}
\subsubsection*{Cubic Groups}
Having eliminated the infinite groups, the three moments of inertia are checked
to see if they are all the same.  If they are, then the system is cubic.  Cubic
systems are oriented by identifying atoms of the set nearest to the center of
symmetry.  If there are 4, 6, 8, 12, or 20 of these, and the number of
equidistant nearest neighbors  is 3, 4, 3, 5, or 3, respectively, then the
atoms are 
%probably 
at the vertices of one of the Platonic solids (tetrahedron, octahedron, cube,
icosahedron, dodecahedron),
%Platonic solids (tetrahedron, octahedron, cube, icosahedron, dodecahedron).
%There are a few cases in which these two conditions are true, but the
%solid is nevertheless not platonic.
%To exclude thses rare case, use is made of the ratio of the edge distance 
%to the (vertex to center) distance.  This ratio is unique for each of the 
%platonic solids, if the
%computed value is correct the solid is unambiguously platonic,
and therefore all atoms of the set lie on high-symmetry axes.  The first  atom
is selected and used to define the $z$ axis.

\index{Platonic solids} \index{Buckminsterfullerene} If the number of atoms in
the set does not correspond to any of the Platonic solids, then the set is
checked for the existence of a equilateral triangle, a square, or a regular
pentagon.  When one of these is found, the center of the polygon is used to
define the $z$ axis.  An example of this type of system is C$_{60}$,
Buckminsterfullerene, which has a five-fold axis going through the center of a
pentagonal face.

Once the $z$ axis is identified, the system is checked for C$_n$ axes, $n$=3 to
$n$=8.  To complete the orientation, the system is rotated about the $z$ axis
so that two atoms, having equal $z$ coordinates, have equal $y$ coordinates.
The existence of rotation axes which are not coincidental with the $z$ axis and
the presence or absence of a center of inversion are then used to identify
which cubic group the system belongs to.

\index{Degenerate groups}
\subsubsection*{Other Degenerate Groups}
If the system has still not been identified, then the two equal moments of
inertia indicate a degenerate point group.  As with the cubic groups, the $y$ and
$z$ axes (and, by implication, the $x$ axis) are identified. The system is
oriented, and the C$_n$ and S$_n$ axes identified.

The degenerate groups, C$_n$, C$_{nv}$, C$_{nh}$, D$_n$, D$_{nd}$, D$_{nh}$, 
S$_n$, are distinguished by the existence or absence of C$_2$ axes
perpendicular to the $z$ axis, and by planes of symmetry.

\index{Abelian groups}\index{Orientation!for symmetry}
\index{Symmetry!orientation of molecules}

\subsubsection*{Abelian Groups}
All that remains are the Abelian groups, C$_1$, C$_2$, C$_i$, C$_s$,  C$_{2v}$,
C$_{2h}$, D$_2$,  and D$_{2h}$.  After orienting the molecule, the axes are
swapped around so that the normal convention for orienting Abelian systems is
obeyed.  For groups C$_1$, C$_2$, C$_i$, and C$_s$, there is no possibility for
ambiguity.  For C$_{2v}$ and D$_2$, however, the orientation of the system
affects the labels of the irreducible representations.  To prevent ambiguity,
the convention for orienting Abelian molecules is:
\begin{itemize}
\item The axis with the largest number of atoms is the $z$  axis.
\item The plane with the largest number of atoms that includes the $z$ axis is 
the $yz$  plane.
\end{itemize}
Thus for ethylene, the $\pi$ orbitals point along the $x$  axis.

\subsubsection*{Tolerance}
Normally, molecular geometries do not exactly correspond to the idealized
point-group.  Thus benzene might have slightly different bond-lengths and
angles.  Of course, symmetry could be used to prevent this, but in the
discussion here we assume that the symmetry of the system is unknown.  To allow
for these slight distortions, a small tolerance is built in to the tests for
symmetry elements. This starts off at 0.1\AA\ , but may be tightened
automatically if ambiguities are detected.  An example of such an ambiguity is
found in tropylium, C$_7$H$_7^+$ ion, where the C-C distance is 1.4 \AA .
Rotating the ring by 45 degrees (a C$_8$ operation) would place the atoms at a
distance of only 0.18\AA\  from equivalent positions.  C$_7$ and C$_8$ would
thus  give almost identical results.  To resolve such ambiguities, when they
arise, the tolerance is reduced, and the test re-run.

Even with this feature, some systems still resist classification.   A distorted
geometry might have some, but not all, elements of a high point group.  Perhaps
a distorted benzene has a C$_2$(z) and a C$_3$(z), but  not a C$_6$(z), 
impossible in a real system.  As such it would appear to be different from all
real point groups.  To accommodate such defects a descent in symmetry is
carried out.  This consists of checking each point-group in turn, in order of
decreasing symmetry.  Once all of the elements of a point group are satisfied,
the system is assigned to that point group, even if the system contains more
symmetry than the point group.

By these two devices, a variable tolerance and the descent in symmetry, most
systems should be identified correctly, or at least as a sub-group of the full
point group.

\subsection{Orientation of the Abelian groups C$_{2v}$ and D$_{2h}$}
Unlike all other groups, two of the Abelian groups, C$_{2v}$ and  D$_{2h}$,
present novel problems in assigning the irreducible representations.  For most
groups, the symmetry axis is obvious, or of there are several axes, the
principal axis is obvious.  For C$_{2v}$ and D$_{2h}$ an ambiguity exists. 
Consider, for example, ethylene, a system of point-group D$_{2h}$.  Should the
$z$ axis be perpendicular to the plane of the molecule---that is the unique
direction, or should it go through the two carbon atoms---that is also a unique
direction, but for a different reason, or should it be the third orthogonal
direction---which is also unique.  The choice of $z$ axis is important in order
to correctly assign the B$_{1g}$ and B$_{1u}$ of point-group D$_{2h}$.  For
both C$_{2v}$ and D$_{2h}$ the $x$ and $y$ axes must also be unambiguously
defined in order to distinguish  between B$_{2g}$ and B$_{3g}$ and between
B$_{2u}$ and B$_{3u}$ of  D$_{2h}$, and between B$_{1}$ and B$_{2}$ of
C$_{2v}$. Clearly a convention has to be decided upon, otherwise one persons
B$_{1g}$ might be a second persons B$_{2g}$ and a third persons B$_{3g}$.

The convention used in MOPAC is the following:

If there are three C$_2$ axes, the one with the largest number of atoms unmoved
by a C$_2$ operation is $z$.  If there is only one C$_2$ axis, that is $z$.

Once $z$ is defined, the $y$ axis is defined as the axis of the remaining two
axes which has the larger number of atoms unmoved by  the $\sigma$ symmetry
operations.  

The $x$ axis is the remaining axis.

To see how this works, consider ethylene, with the C--C axis being along the
$x$ direction, and the plane of the system being $xy$. Under the eight
operations of D$_{2h}$, E, C$_{2z}$, C$_{2y}$, C$_{2x}$, $\sigma_{xy}$,
$\sigma_{xz}$, $\sigma _{yz}$, and $i$, the number of atoms unmoved are 6, 0,
0, 2, 6, 2, 0, and 0 respectively.

From this it follows that the old $x$ axis is now re-defined as the $z$ axis. 
The new $y$ axis has to be chosen based on the number of atoms unmoved under
the $\sigma_{xy}$ and $\sigma_{xz}$ operations (6 and 2). The new $y$ axis is
defined as being the old $y$ axis.  The remaining new axis $x$ therefore is the
old $z$ axis.

The overall result is that the symmetry axes in ethylene are defined as: $z$ -
along the C--C bond; $y$ - in the molecular plane, perpendicular to the C--C
bond, and $x$ - out-of-plane.

In MOPAC the orientation of the molecule is defined by the user, therefore the
assignment of the symmetry axes might be confusing.  If the irreducible
representations of ethylene are assigned, and the atoms are defined  using
internal coordinates in the order C, C, H, H, H, H, then the $p$ orbitals will
reflect the orientation used in the previous discussion, but the
representations will be correct according to the conventions just defined.

\subsection{Molecular Orbitals}
Each M.O.\ is subjected to the operation
$$
\psi' = |R|\psi>
$$
from which the expectation value
$$
\chi=<\psi'|\psi>
$$
can readily be calculated.

\index{Degenerate M.O.s}
All $\chi$'s within a given degenerate manifold are summed:
$$
\chi_i = \sum_j\chi_j^{(i)}
$$
where $j$ runs over all components of the degenerate manifold $i$.

This results in a set of characters which can be compared to those stored in
the character tables.  

Since molecular orbitals involve single electrons, the irreducible
representations \index{Irreducible representations} are rendered into lower
case before printing. 

\subsection{Normal Coordinates}
\index{States!vibration} \index{Normal coordinates}\index{Coordinates!normal}
Analysis of normal coordinates is a little simpler than that of molecular
orbitals in that every atom contributes precisely three components to each
normal coordinate, an $x$, $y$, and $z$ component.  These transform in the same
way as the p$_x$, p$_y$, and p$_z$ atomic orbitals.

Because normal coordinates are states, the first letter of the irreducible
representation is capitalized.

\subsection{States}
\index{States!electronic}\index{States!symmetry of}
Calculating the characters for electronic states is much more complicated than
that for M.O.s or normal coordinates.  

Consider the effect of an operation, $R$, on a state, $\Phi_a$.  The 
character of the operation is given by
$$
\chi_{R,a} = <\Phi_a|R|\Phi_a>.
$$
A state function can be expressed as a linear combination of microstates:
$$
\Phi_a = \sum_jC_{ja}\Psi_j,
$$
so the character of the operation on the state function can be written in 
terms of microstates as
$$
\chi_{R,a} =\sum_i\sum_jC_{ia}C_{ja}<\Psi_i|R|\Psi_j>.
$$
Each microstate, $\Psi_j$, can be represented by a \mi{Slater determinant}
of $N$ molecular orbitals
:
$$
\Psi_j = \frac{1}{\sqrt{N!}}\sum_{P=1}^{N!}(-1)^PP(\prod_{k=1}^N\psi_k^j)
$$
\index{Microstates} where the molecular orbitals in the microstate consist of a
selection of the M.O.s in the  active space\index{Active space!in C.I.}. 
Before we continue, let us examine this idea:

Consider a full set of M.O.s:
$$
\psi_1\psi_2\psi_3\psi_4 \psi_5 \psi_6 \psi_7 \psi_8 \psi_9 
\psi_{10}\psi_{11}\psi_{12}\psi_{13}\psi_{14}\psi_{15}\psi_{16}.
$$
Let the active space be the M.O.s  from 8 to 11.  Then microstates containing
two electrons would be:
\begin{center}
\hfil
$\psi_8\psi_9$
\hfil
$\psi_8\psi_{10}$
\hfil
$\psi_8\psi_{11}$
\hfil
$\psi_9\psi_{10}$
\hfil
$\psi_9\psi_{11}$
\hfil
$\psi_{10}\psi_{11}$.
\hfil
\end{center}
These microstates could be represented by M.O.\ orbital occupancies.
\begin{center}
\hfil
1100
\hfil
1010
\hfil
1001
\hfil
0110
\hfil
0101
\hfil
0011.
\hfil
\end{center}
Remember that the M.O.s here can be of either $\alpha$ or $\beta$ spin.

To continue, we need to evaluate $<\Psi_i|R|\Psi_j>$.  This can be expressed
in terms of M.O.s as:
$$
<\Psi_i|R|\Psi_j> = \frac{1}{N!}\sum_{P=1}^{N!}(-1)^PP(<\prod_{k=1}^N\psi_k^i)|
R|
\sum_{Q=1}^{N!}(-1)^QQ(\prod_{l=1}^N\psi_l^j)>.
$$

For convenience, we will represent the integral $<\psi_k^i|R|\psi_l^j>$ by 
$\chi_{kl}^{ij}$.  This integral can be described as ``The integral over M.O.\
$\psi_k$ in microstate $\Psi_i$ with the result of operator $R$ acting
on M.O.\ $\psi_l$ in microstate $\Psi_j$.''

Using this abbreviation, $<\Psi_i|R|\Psi_j>$ can be written as:
$$
<\Psi_i|R|\Psi_j> =\frac{1}{N!}\sum_{P=1}^{N!} \sum_{Q=1}^{N!}(-1)^P(-1)^Q
P\prod_{k=1}^{N!}Q\prod_{l=1}^{N!}\chi_{kl}^{ij}.
$$
Although it is not immediately obvious, the right-hand term is a determinant,
of order $N$:
$$
<\Psi_i|R|\Psi_j> =\left|
\begin{array}{cccc}
\chi_{11}^{ij} & \chi_{21}^{ij} & \chi_{31}^{ij} & \ldots \\
\chi_{12}^{ij} & \chi_{22}^{ij} & \chi_{32}^{ij} & \ldots \\
\chi_{13}^{ij} & \chi_{23}^{ij} & \chi_{33}^{ij} & \ldots \\
 \ldots           &  \ldots           &   \ldots          & \ldots 
\end{array}
\right|.
$$

For our purposes, solution of the determinant is best done explicitly. To see
why, note that the number of M.O.s involved in the C.I.\ (the active space) is
very small.  Because of this, the number of electrons, $N$, in the Slater
determinants is also small; $N$ has a maximum value of  20.  Next, use can be
made of the fact that no point-group operation can mix $\alpha$ and $\beta$
electrons.  This allows the integral to be split into two parts, each of which
has a maximum value  of $N$=10. Finally, remember that $N$ is the number of
electrons, not M.O.s, used in the active space.  A system of $N$ electrons has
the same symmetry as a system in which all the M.O.s which were occupied were
replaced with all the M.O.s which were not occupied (the positron
equivalent)\index{Positron equivalent}. (This assumes that if every M.O.\ were
occupied, then the state of the system would be totally symmetric.)   Using
this fact, we can replace the $N$ occupied M.O.s with $N'$ unoccupied M.O.s, if
$N' < N$.

When these three points are considered, we see that $N$ has a maximum value  of
5 (for a system of 10 M.O.s).  Each case can be considered separately.
\begin{description}
\item{For $N$ = 1:}
$$
<\Psi_i|R|\Psi_i> = \frac{1}{1}\sum_{P=1}^{1} \sum_{Q=1}^{1}(-1)^P(-1)^Q
P\prod_{k=1}^{1}Q\prod_{l=1}^{1}\chi_{kl}^{ii}
$$
or
$$
<\Psi_i|R|\Psi_i> = <\psi_1^i|R|\psi_1^i> = \chi_{11}^{ii}.
$$
\item{For $N$=2:}
$$
<\Psi_i|R|\Psi_j> = \frac{1}{2!}\sum_{P=1}^{2!} \sum_{Q=1}^{2!}(-1)^P(-1)^Q
P\prod_{k=1}^{2}Q\prod_{l=1}^{2}\chi_{kl}^{ij}
$$
or 
$$
<\Psi_i|R|\Psi_j> = <\psi_1^i|R|\psi_1^j>-<\psi_2^i|R|\psi_2^j>
<\psi_1^i|R|\psi_2^j><\psi_1^i|R|\psi_2^j>
$$
or
$$
<\Psi_a|R|\Psi_a> = \chi_{11}^{ij}\chi_{22}^{ij}-\chi_{12}^{ij}\chi_{21}^{ij}.
$$        
\item{For $N$=3:}
$$
<\Psi_i|R|\Psi_j> = \frac{1}{3!}\sum_{P=1}^{3!} \sum_{Q=1}^{3!}(-1)^P(-1)^Q
$$
\begin{eqnarray}
<\Psi_i|R|\Psi_j>& = &\ <\psi_1^i|R|\psi_1^j><\psi_2^i|R|\psi_2^j><\psi_3^i|R|\psi_3^j> \nonumber   \\
&&-<\psi_1^i|R|\psi_1^j><\psi_2^i|R|\psi_3^j><\psi_3^i|R|\psi_2^j> \nonumber  \\
&&-<\psi_1^i|R|\psi_2^j><\psi_2^i|R|\psi_1^j><\psi_3^i|R|\psi_3^j> \nonumber  \\
&&+<\psi_1^i|R|\psi_2^j><\psi_2^i|R|\psi_3^j><\psi_3^i|R|\psi_1^j> \nonumber  \\
&&+<\psi_1^i|R|\psi_3^j><\psi_2^i|R|\psi_1^j><\psi_3^i|R|\psi_2^j> \nonumber  \\
&&-<\psi_1^i|R|\psi_3^j><\psi_2^i|R|\psi_2^j><\psi_3^i|R|\psi_1^j>  \nonumber
\end{eqnarray}
or
\begin{eqnarray}
<\Psi_i|R|\Psi_j>&=&\chi_{11}^{ij}\chi_{22}^{ij}\chi_{33}^{ij}+
                            \chi_{12}^{ij}\chi_{23}^{ij}\chi_{31}^{ij}+
                            \chi_{13}^{ij}\chi_{21}^{ij}\chi_{32}^{ij} \nonumber \\
&&-\chi_{11}^{ij}\chi_{23}^{ij}\chi_{32}^{ij}
-\chi_{12}^{ij}\chi_{21}^{ij}\chi_{33}^{ij}
-\chi_{13}^{ij}\chi_{22}^{ij}\chi_{31}^{ij}). \nonumber
\end{eqnarray}
\end{description}

For higher numbers of electrons, the associated determinant is solved using
standard methods. 

The total character, $<\Psi_a|R|\Psi_a>$, is obtained by multiplying the
characters for the $\alpha$ and $\beta$ parts together:
$$
<\Psi_a|R|\Psi_a> = <\Psi_a^{\alpha}|R|\Psi_a^{\alpha}>
<\Psi_a^{\beta}|R|\Psi_a^{\beta}>.
$$
If the positron equivalent is taken for only one set of electrons, e.g.\ either
the $\alpha$ or the $\beta$ set, but not both, then the  character has  to be
multiplied by the determinant of the M.O.\ transform.

These expressions can then be used in 
$$
\chi_{R,a} =\sum_i\sum_jC_{ia}C_{ja}<\Psi_i|R|\Psi_j>.
$$
to give the expectation value for the state. Finally, if the state is
degenerate, the character is given by summing the components of the state.

For the atom, the \mi{Russell-Saunders} coupling scheme can be reproduced. 
States allowed are $S$, $P$, $D$, $F$, $G$, $H$, $I$, $K$, $L$, and $M$. This
set is more than sufficient to allow all possible Russell-Saunders states
spanned by a basis set of $s$, $p$, and $d$ orbitals to be represented. The
highest angular momentum achievable with such a basis set is 8, i.e. $L$. For
simpler atoms (ones with only a $s-p$ basis set) the allowed states are
$p^0,p^6$: $^1S_g$,  $p^1,p^5$: $^2P_u$, $p^2,p^4$: $^1S_g + ^3\!\!P_g +
^1\!\!D_g$,  $p^3$: $^4\!S_u +^2\!\!P_u + ^2\!\!D_u$.

For the axial infinite groups, allowed states are: $\Sigma$, $\Pi$, $\Delta$,
$\Phi$, and $\Gamma$.  Even quite simple systems can achieve quite high angular
momentum, \index{Angular momentum} thus acetylene, with a \comp{C.I.=4}  (the
HOMO $\pi$ and LUMO $\pi^*$) will contain a $^1\Gamma_g$ state,  i.e., the
angular momentum will be 4.

At present J-J coupling is not supported.
\index{Groups|)}

\section{Level of Precision within MOPAC}
\index{Precision} \index{Criteria}
Several users have criticized the  tolerances  within  MOPAC.   The point 
made  is  that significantly different results have been obtained when
different starting conditions have been used, even  when  the  same conformer 
should  have  resulted.  Of course, different results must be expected---there
will always be small differences---nonetheless  any differences   should   be 
small,  for example,  heats  of  formation   ($\Delta H_f$) \index{$\Delta
H_f$!precision} \index{Heat of Formation!precision} differences should be less
than about 0.1  kcal/mol.   MOPAC  has  the flexibility  to  allow  users  to 
specify a much higher precision than the default when circumstances warrant it.

\subsection{Fundamental Physical Constants}
\index{Fundamental Physical Constants}
The fundamental physical constants used in MOPAC were updated in 1993 to
conform with the 1986 CODATA recommendations~\cite{codata}.  The constants used
in MOPAC are given in Table~\ref{codata}.  As a result of this update, all
calculated quantities in MOPAC, except molecular weight, will change slightly
when compared to earlier MOPACs (MOPAC 6 and earlier).  Most of the time,
changes in $\Delta H_f$ are less than 0.1 kcal/mol.  It is not anticipated that
the physical constants will change again.  If they do, however, the effect on
calculated properties should be very small.

The derived quantities, AM, AD, AQ, EISOL, DD, and QQ are functions of the
fundamental constants.  Rather than change all of these, starting with MOPAC
93, they are evaluated  at the start of each calculation.  This is a quick
operation, taking only about 0.1s, and prevents any mistakes being introduced
due to human error. 

\begin{table}
\index{Boltzmann constant}
\index{Constants!physical}
\index{Definitions!Boltzmann constant}
\index{Definitions!velocity of light}
\index{Gas constant, R}
\caption{\label{codata} Fundamental Physical Constants}
\begin{center}
\begin{tabular}{|ccll|}\hline
Physical Constant & Symbol &Value &Units \\ \hline 
Speed of Light    &     c  &  299 792 458 &m sec$^{-1}$ (Definition)\\
Planck constant   &   $h$  & 6.626 075 5(40)  $\times$ 10$^{-34}$ &J sec \\
&&  6.626 075 5(40) $\times$ 10$^{-27}$ & erg s\\
Avogadro constant & $N$    & 6.022 136 7(36)  $\times$  10$^{23}$&mol$^{-1}$ \\
Molar gas constant & $R$   & 1.987 215 6  & cal/mol/degree\\
            & & 8.314 510(70) & J/mol/K \\
Volume of 1 mol of gas& $V_0$ & 22.414 10(19) & l/mol (at 1 atm, 25 C) \\
Electron volt     & eV     & 1.602 177 33(49) $\times$ 10$^{-19}$&J \\
Electron charge   & e      & 1.602 177 33(49) $\times$ 10$^{-19}$&C \\
Hartree           & $E_h$  & 27.211 396 1(81) & eV \\
Electrostatic energy & $E_ha_{\circ}$ & 14.399 651 782 565 &eV \\
Bohr radius       & $a_{\circ}$ & 0.529 177 249(24) $\times$  10$^{-10}$ & m \\
Boltzmann constant& $k$ = $R/N$ & 1.380 658(12) $\times 10^{-23}$& J/K \\
                  &              & 1.380 658(12) $\times 10^{-16}$& erg/K\\
pi & $\pi$ & 3.141 592 653 589 79 & \\
Joule             & J/cal  & 4.184        & J/cal (Definition)\\
 cm$^{-1}$ &  $h$c/eV   & 1.239 842 4  $\times$  10$^{-4}$ & eV\\
 cm$^{-1}$ & $h$c$N$/(1000J/cal)& 2.859 144  $\times$  10$^{-3}$ & kcal/mol\\
 cm$^{-1}$&$h$c$N$/(J/cal)& 2.859 144 &cal/mol \\
cm$^{-1}$   & & 1.196 266 $\times$  10$^{8}$ & erg = dyne \AA $^{-1}$\\
Atomic unit (a.u.)     &        & 8.657 10 $\times$  10$^{-33}$ &e.s.u. \\
a.u.      &  &  2.541 747 8 $\times$  10$^{-40}$ & Debye \\
a.u.        & & 51.422 082     & V m$^{-1}$\\
kcal/mol    & & 6.947 700 $\times$  10$^{-3}$ & erg\\
1 J &&1.$ \times 10^7$ & erg\\
1 eV &&23.060 542 301 389  &kcal/mol\\
&& 627.509 6& kcal/mol\\
1 atm && 1.013 25 $\times 10^5$ &Pa \\
&& 1.013 25 $\times 10^6$ &dyn/cm$^2$ (Definition)\\ \hline
\end{tabular}~\\
Note: The precision of derived constants should {\em not} be used as an
indication of their accuracy. The uncertainty in the fundamental constants is
given in parenthesis after the value. 
\end{center}
\end{table}


\subsection{Various precision levels}
In normal (non-publication quality) work the default  precision  of MOPAC  is
recommended.  This will allow reasonably precise results to be obtained  in  a 
reasonable  time.    Unless   this   precision   proves unsatisfactory, use
this default for all routine work.

The  best  way  of  controlling  the  precision  of  the   geometry
optimization  and gradient minimization is by specifying a gradient norm
\index{GNORM|ff} which must be satisfied.  The gradient norm is the scalar of
the vector of derivatives of the energy with respect to the geometric variables
flagged for optimization. I.e.,
$$
{\rm  GNORM}=\sqrt{\sum_i\left(\frac{d(\Delta H_f)}{dx_i}\right)^2}
$$
where $x_i$ refers to coordinates flagged for optimization.  Note that the
calculated GNORM may be very small and at the same time the geometry might not
be at a stationary point.  This can easily happen when (a) less than 3$N$-6
coordinates are flagged for optimization, (b) \comp{SYMMETRY} has been used
incorrectly, or (c) (less common) only 3$N$-6 coordinates are flagged for
optimization and dummy atoms are used.  If any one of these three conditions
occurs, then the warning message,  ``WARNING -- GEOMETRY IS NOT AT A STATIONARY
POINT''  \index{Error message!WARNING -- GEOM\ldots } will be printed.

A less common, but not unknown, situation arises when internal
coordinates are used.  In this strange situation, the internal derivatives might
all be zero, but the Cartesian derivatives are large.  An example of such a
system is shown in Figure~\ref{xch2o}.
\begin{figure}
\begin{makeimage}
\end{makeimage}
\begin{verbatim}
Line 1:  GRADIENTS 1SCF DEBUG DCART
Line 2:  EXAMPLE OF INTERNAL COORDINATE DERIVATIVES ZERO
Line 3:  AND CARTESIAN DERIVATIVES LARGE
Line 4: O
Line 5: C   1.2075664 1
Line 6: H   1.2114325 1   103.347931 1    0 0   2 1
Line 7: H   1.0904182 1   180.000000 1   90 1   2 1 3
\end{verbatim}
\caption{\label{xch2o} Example of Spurious Stationary Point}
\end{figure}

In this strange system, the bond-angle of the second hydrogen is 180$^\circ$,
and the dihedral is 0$^\circ$.  Obviously, the derivative of the dihedral will
be zero. The derivative of the angle is not so obvious.  If the angle changes,
then the fourth atom will move out of the plane of the other three atoms. The
energy will change in the same way regardless of whether the angle increases or
decreases; therefore, the derivative of the angle must be zero.

Because of the unusual nature of this type of system, users may be unaware of
the danger.  If such a system is detected, a warning will be given and the job
stopped.  For all other cases, the ``WARNING -- GEOMETRY IS NOT AT A 
STATIONARY POINT'' message will be printed on completion of the calculation. 
In the unlikely event that the calculation should not be stopped when the
strange system is detected, calculation can be continued by specifying
\comp{GEO-OK}.

Modification of  GNORM is done via the keyword \comp{GNORM=$n.nn$}.  Altering
the   GNORM  automatically disables the other termination tests resulting in
the gradient norm dominating the calculation.  This works both  ways:
\comp{GNORM=20}  will give a very crude optimization while \comp{GNORM=0.01} 
will give a very precise optimization.  The default is \comp{GNORM=1.0}.

When the highest precision is needed, such as in exacting  geometry work,  or 
when  you want results whose precision cannot be improved, then use the
combination keywords \comp{GNORM=0.0} and either \comp{RELSCF=0.01} or
\comp{SCFCRT=1.D-NN};  \index{SCFCRT} (\comp{NN} should  be  in  the range 
5--15).   By default, EigenFollowing is used in geometry optimization.  One
reason is that EigenFollowing is nearer to a gradient minimizer than it is to
an  energy minimizer.  Because of this, if there is any difference between the
gradient minimum and the energy minimum, it  will give better reproducibility
of the optimized geometry than the alternative \comp{BFGS} method.

In practice, optimized geometries for ``well behaved'' systems can be obtained
with \comp{GNORM}s of less than 0.0001.

Increasing the SCF criterion (the default is \comp{SCFCRT=1.D-4}) improves the
precision of the gradients; however, it can lead to excessive run times, so
take care.  Also, there is an increased chance of  not  achieving  an SCF when
the SCF criterion is excessively increased.

Superficially, requesting \comp{GNORM=0}   might  seem  excessively stringent, 
but  as soon as the run starts, it will be cut back to 0.01. Even that might
seem too  stringent.   The  geometry  optimization  will continue to lower the
energy, and hopefully the GNORM, but frequently it will not prove possible to
lower  the   GNORM to  0.01.   If,  after  10 cycles,  the energy does not drop
then the job will be stopped.  At this point you have the best geometry that
MOPAC, in its  current  form,  can give.

If a slightly less than highest precision is needed,  such  as  for normal
publication quality work, set the \comp{GNORM} to the limit wanted.  For
example, for a flexible system, a \comp{GNORM} of 0.1 to 0.5 will  normally  be
good enough for all but the most demanding work.

If higher than the default, but still not very  high  precision  is wanted, 
then  use  the  keyword  \comp{PRECISE}.  This will tighten up various criteria
so that higher-than-routine precision will be given.

If high precision is used, so that the printed  GNORM is 0.000,  and the  
resulting   geometry   resubmitted   for  one  SCF  and  gradients calculation,
then normally a  GNORM higher than 0.000 will result.   This is  {\em not}  an
error in MOPAC:  the geometry printed is only precise to eight figures after
the decimal point.  Geometries may need  to  be  specified  to more than eight
decimals in order to drive the  GNORM to less than 0.000.

If you want to test MOPAC, or use it  for  teaching  purposes,  the
\comp{GNORM}  lower limit of 0.01 can be overridden by specifying \comp{LET},
in which case you can specify any limit for \comp{GNORM}.  However, if it is
too low the job  may  finish  due to an irreducible minimum in the heat of
formation being encountered.  If this happens, the ``STATIONARY POINT'' message
will be printed.

Examples of highly optimized geometries can be found in the \comp{port.dat}
file.  When this job is run, most gradients will be less than 0.001
kcal/mol/\AA .  A few will be larger. These exceptions fall into two classes:
diatomics, for which a simple line-search is sufficient to locate the optimum
geometry, in which case the  GNORM criterion is {\em not} used; and
non-variationally optimized systems, where the analytical C.I.\ derivatives 
are used.  These derivatives are of lower precision than the variational
derivatives, but are still much better than finite difference derivatives using
full SCFs. \index{Analytical derivatives}\index{Derivatives!analytical|ff}
Finally there is  a  full  analytical  derivative  function~\cite{analyt}
within \index{STO-6G}\index{Gaussian!wavefunctions} MOPAC.  These use STO-6G
Gaussian wavefunctions because the derivatives of the overlap integral are
easier to calculate  in  Gaussians  than  in STOs.  Consequently, there will be
a small difference in the calculated $\Delta H_f$ when analytical derivatives
are used.  If  there  is  any  doubt \index{Analytical derivatives} about  the 
accuracy of the finite derivatives, try using the analytical derivatives.  They
are a bit slower than finite derivatives but are more precise  (a  rough 
estimate is 12 figures for finite difference, 14 for analytical).

Some calculations, mainly open shell RHF or closed shell  RHF  with C.I., have
untracked errors which prevent very high precision.  For these systems
\comp{GNORM} should be in the range 1.0 to 0.1.

\subsection{Reasons for low precision}
There are several reasons for obtaining low quality  results,   the most 
obvious  of which is that for general work the default criteria will result in 
a  difference  in  $\Delta H_f$   of  less  than  0.1 kcal/mol.    This   is  
only  true  for  fairly  rigid  systems,  e.g.\ formaldehyde and benzene.  For
systems with low barriers to rotation  or \index{Flat potential surfaces}
\index{Aniline, dimer}\index{Water, dimer} flat  potential  surfaces,  such
as   aniline  or  water dimer, quite large $\Delta H_f$  errors can result.

\subsection{How large can a gradient be and still be acceptable?}
A common source of confusion is the limit to which the \comp{GNORM} should be 
reduced  in  order  to  obtain acceptable results.  There is no easy answer.
However, a few guidelines can be given.

\index{LET}\index{ANALYT} First of all, setting  the \comp{GNORM} to an
arbitrarily  small  number  is not  sensible.   If \comp{GNORM=0.000001} and
\comp{LET} are used, a geometry can be obtained which is precise to about 
0.000001 \AA. \ If \comp{ANALYT} is also used, the results obtained will be
slightly different. Chemically, this change is meaningless, and no 
significance  should  be attached  to  such  numbers.   In  addition,  any 
minor  change  to the algorithm, such as porting it to a new machine, will give
rise to  small changes  in  the optimized geometry.  Even the small changes
involved in going from one version of MOPAC to another causes  small  changes 
in  the optimized geometry of test molecules.

As a guide, a \comp{GNORM} of 0.1 is sufficient for all  heat-of-formation
work,  and  a  \comp{GNORM}  of  0.01 for most geometry work.  If the system is
large, you may need to settle for a \comp{GNORM} of 1.0--0.5.

This whole topic was raised by Dr.\ Donald B. Boyd while he was at Lilly
Research  Laboratories,  who provided unequivocal evidence for a failure of
MOPAC and convinced me of the importance of increasing  precision  in certain
circumstances.\index{boyd@{\bf Boyd, Donald B.}}

\subsection{Convergence tests in subroutine ITER}
\index{Convergence tests!SCF}
\subsubsection{Self-consistency test}
\index{SCF!test}
The SCF iterations are stopped when two tests are satisfied.  These are (1)
when the difference in electronic energy, in eV, between any two consecutive
iterations drops below the adjustable parameter, \comp{SELCON}, and the 
difference between any three consecutive iterations drops below ten times
\comp{SELCON}, and (2) the difference in density matrix elements  on  two
successive iterations falls below a preset limit, which is a multiple of
\comp{SELCON}. \index{SELCON}

\comp{SELCON} is set initially to 0.0001 kcal/mol; this can be  made  100
times  smaller by specifying \comp{PRECISE} or \comp{FORCE}.   It can be
over-ridden by explicitly defining the SCF criterion {\em via}
\comp{SCFCRT=1.D-12}, or by use of \comp{RELSCF=0.1}.

\comp{SELCON} is further modified by the value of the  gradient  norm,  if
known.   If  \comp{GNORM} is large, then a more lax SCF criterion is
acceptable, and \comp{SCFCRT} can be relaxed up to 50 times its  default  value
(using \comp{RELSCF=50}).   As  the gradient norm drops, the SCF criterion
returns to its default value.

The SCF test is performed using the energy calculated from the Fock matrix 
which  arises  from  a  density matrix, and not from the density matrix which
arises from a Fock.  In the limit, the two  energies  would be  identical, 
but  the first converges faster than the second, without loss of precision.

\section{Torsion or Dihedral Angle Coherency}\label{coherency}
\index{Dihedral angles!coherency}\index{Chirality}\index{Enantiomers}
MOPAC  calculations  do  not   distinguish   between   enantiomers,
consequently  the  sign of the dihedrals can be multiplied by $-1$ and the
calculations will be unaffected.  However, if chirality is important,  a user
should be aware of the sign convention used.

The dihedral angle convention used in  MOPAC  is  that  defined  by Klyne and
Prelog~\cite{klyne}. \index{klyne@{\bf Klyne and Prelog}}  In this convention,
four atoms, AXYB, with a dihedral angle of 90 degrees, will have atom  B
rotated  by 90 degrees clockwise relative to A when X and Y are lined up in the
direction of sight, X being nearer to the eye.  In  their  words, ``To
distinguish between enantiomeric types the angle $\tau$ is considered as
positive when it is measured clockwise from the front  substituent  A to  
the   rear   substituent  B,  and  negative  when  it  is  measured
anticlockwise.'' The alternative  convention  was  used  in  programs which
preceded MOPAC.

\section{Gradients}\label{derivs}
\index{Gradients|ff}\index{Derivatives|(}
By ``gradients'' we generally mean ``the derivative of the energy with respect
to coordinates''.  The two most commonly used gradients are with respect to
Cartesian coordinates, in which case the units are kcal/mol/\AA ngstrom, or
with respect to internal coordinates, in which case the units are either 
kcal/mol/\AA ngstrom or kcal/mol/radian, depending on whether the coordinate is
a distance (in which case it would be kcal/mol/\AA ngstrom) or an angle or
dihedral (in which case it would be kcal/mol/radian). The particular gradient
actually being used at any given point should be clear from the context. In all
cases, the gradient can be regarded as the following derivative
$$
g_i = \frac{d(\Delta H_f)}{dx_i}
$$
In discussion ``gradient'' will be reserved for the derivative with respect to
coordinates flagged for optimization (internal or Cartesian), and
``derivative'' will be used for both gradients and terms which are used to
calculate gradients, such as Cartesian derivatives which are used to calculate
internal coordinate gradients.

There are four very different ways to calculate gradients, although all four
result in the same type of derivative.  The four ways are:
\begin{description}
\item[Frozen density matrix finite difference derivatives]~\\
\index{Wavefunctions!variational}
In these procedures, once an SCF has been achieved, the derivatives can be 
calculated using the density matrix from the SCF calculation.  These methods 
can only be used with variationally optimized wavefunctions. 

By default, the derivatives are worked out by calculating the energy of each
pair of atoms, then re-calculating the energy after a small displacement has
been made, and then calculating the derivative from the differences in the
energies and the step.  This is the default, and is the fastest.  If this
method is {\em not} wanted, specify \comp{ANALYT}.

\item[Analytical derivatives, using frozen density matrix approximation]~\\
Not as fast as the first method, but more accurate.  Useful when finite
difference derivatives are suspected to be of insufficient accuracy.  When
analytical derivatives are wanted, specify \comp{ANALYT}.  Analytical
derivatives cannot be used with non-variational finite difference derivatives.

\item[Non-variational finite difference derivatives]~\\ 
For non-variational \index{Wavefunctions!non-variational}wavefunctions (systems
for which the electronic energy is modified after the SCF calculation is done,
e.g.\ C.I.\ calculations), a  sophisticated derivative routine in \comp{DERNVO}
calculates the effect on the  derivative of the post-SCF energy terms.  This
method is used automatically in RHF C.I.\ calculations.  If this method is {\em
not} wanted, specify \comp{NOANCI}.

\item[Brute force gradients]~\\
These should be avoided whenever possible.  To calculate the gradient, a small
change is made in the desired coordinate, then a full SCF is done, and the
gradient calculated from
$$ 
g_i = \frac{\Delta H_f-\Delta H_f'}{x-x^{'}}.
$$ 
These gradients are very slow, and are of poor accuracy, but sometimes they
are the only way to obtain  gradients.  These derivatives cannot be used
with variationally optimized wavefunctions, but can be used with 
non-variational wavefunctions by specifying \comp{NOANCI}.  
\end{description}

Note that \comp{ANALYT} and \comp{NOANCI} apply to two very different things:
\comp{ANALYT} applies to the derivatives using a frozen density matrix
approximation, and uses true analytical methods.  \comp{NOANCI} prevents
Liotard's C.I.\ derivative method being used.  Of course \comp{NOANCI} has no
meaning for variationally optimized wavefunctions.

\subsection{Frozen density matrix finite difference derivatives}

The first step in calculating the gradients is to calculate the derivatives
with respect to Cartesian coordinates.  This is done in subroutine DCART.
\index{DCART}

DCART calculates the energy of each pair of atoms, then moves one atom a small
distance ($10^{-4}$\AA ) in each of the three Cartesian directions.  The
density matrix for the atom-pair is not changed during this calculation, but is
set equal to the SCF density matrix. The derivative for each atom is then
calculated  from:
$$\left (\frac{dE}{dx}\right )_A = \sum_{B\neq A}\frac{E_{AB}-E_{AB}^{'}}{\delta_x}, $$
$$\left (\frac{dE}{dy}\right )_A = \sum_{B\neq A}\frac{E_{AB}-E_{AB}^{'}}{\delta_y}, $$
$$\left (\frac{dE}{dz}\right )_A = \sum_{B\neq A}\frac{E_{AB}-E_{AB}^{'}}{\delta_z}. $$
where $E_{AB}^{'}$ is the energy of the pair of atoms after displacement in the
appropriate direction.  For a stationary point, these derivatives are zero.

To convert from Cartesian coordinate (c.c.) derivatives into gradients (i.c.),
the sum
$$
g_i = \sum_j\frac{dE}{d({\rm c.c.}_j)}\frac{d({\rm c.c.}_i)}{d({\rm i.c.}_j)} 
$$
must be evaluated.  Evaluation of  
$\frac{d({\rm c.c.}_i)}{d({\rm i.c.}_j)}$  
is quite simple, and in done in routine JCARIN.

\subsection{Hessian matrix in \comp{FORCE} calculations}\index{Hessian|(}\label{ssd}
\index{Single-sided derivatives}
The Hessian matrix is the matrix of second derivatives of the energy with
respect to geometry. The most important Hessian is that used in the
\comp{FORCE} calculation.  Normal modes are expressed as Cartesian
displacements, consequently the Hessian is based on Cartesian rather than
internal coordinates.

\index{Derivatives!``single-sided''} 
Although first derivatives are relatively easy to calculate, second derivatives
are not.  The simplest, although not an elegant, way to calculate~\cite{pulayf}
second derivatives is to calculate first derivatives for a given geometry, then
perturb the geometry, do an SCF calculation on the new geometry, and
re-calculate the derivatives. The second derivatives can then be calculated
from the difference of the two first derivatives divided by the step size. 
This method, which is used in the EigenFollowing routine, is called
`single-sided' derivatives.

The Hessian is quite sensitive to geometry, and should only be evaluated at
stationary points.  Because of this sensitivity, ``double-sided'' derivatives
\index{Derivatives!``double-sided''} are used:
$$
H_{i,j} = \frac{g_i^{+\delta_j}-g_i^{-\delta_j}}{2\delta}.
$$
Note the asymmetry in the treatment of the Cartesian coordinates $i$ and $j$.
It can be shown that 
$$
\frac{g_j^{+\delta_i}-g_j^{-\delta_i}}{2\delta} = \frac{g_i^{+\delta_j}-g_i^{-\delta_j}}{2\delta}.
$$
To help improve precision, the Hessian is calculated from
$$
H_{i,j} = \frac{1}{2}\left ( \frac{g_j^{+\delta_i}-g_j^{-\delta_i}}{2\delta} + \frac{g_i^{+\delta_j}-g_i^{-\delta_j}}{2\delta} \right ).
$$
\index{Hessian|)}
\index{Derivatives|)}

\section{Normal Coordinate Calculation}\index{Normal coordinates!calculation of|(}
\subsection{Calculation of Vibrational Frequencies}
For a simple harmonic oscillator the period $r$ is given by:
$$
r = 2 \pi \sqrt{\frac{\mu}{k}}
$$
where $k$ is the force constant.  A molecule can absorb a photon that vibrates
at the same frequency as one of its normal vibrational modes.  That is, if a
molecule, initially in its ground vibrational state, could be excited so that
it vibrated at   a given frequency, then that molecule could absorb a photon
that vibrates at the same frequency.  Although vibrational frequencies are
usually expressed as kilohertz or megahertz, in chemistry vibrational
frequencies are normally  expressed in terms of the number of vibrations that
would occur in the time that light travels one centimeter, i.e., $\bar{\nu} =
1/cr$ Using this equation for simple harmonic motion, the vibrational frequency
can be written as:
$$
\bar{\nu} = \frac{1}{2\pi c}\sqrt{\frac{k}{\mu}}.
$$
In order for $\bar{\nu}$ to be in cm$^{-1}$, $c$, the speed of light must be in
cm.sec$^{-1}$, $k$, the force constant in erg/cm$^2$, and $\mu$ the reduced
mass in grams.

For a molecule, the force constants are obtained by diagonalization of the
mass-weighted Hessian matrix.  Most of the work in calculating vibrational
frequencies is spent in constructing the Hessian.

\subsubsection*{Calculation of the Hessian matrix}
The elements of the Hessian are defined as:
$$
H_{i,j} = \frac{\delta^2E}{\delta x_i\delta x_j}
$$
and  are generated by use of finite displacements, that is, for each  atomic
coordinate $x_i$, the coordinate is first incremented by a small amount, the
gradients  calculated, then the coordinate is decremented and the gradients
re-calculated. The second derivative is then obtained from the difference of
the two derivatives and the step size:
$$
H_{i,j} = \frac{(\frac{\delta E}{\delta x_i})_{_{+0.5\Delta x_j}}-
                (\frac{\delta E}{\delta x_i})_{_{-0.5\Delta x_j}}}
          {\Delta x_j}.
$$
This is done for all $3N$ Cartesian coordinates.  Because the Hessian is
symmetric, that is $H_{i,j}=H_{j,i}$, the random errors that occur in the
gradient calculation can be reduced (by a factor of $\sqrt{2}$) by re-defining
the Hessian as:
$$
H_{i,j} = \frac{1}{2}\left(\frac{(\frac{\delta E}{\delta x_i})_{_{+0.5\Delta x_j}}-
                (\frac{\delta E}{\delta x_i})_{_{-0.5\Delta x_j}}}
          {\Delta x_j}+
 \frac{(\frac{\delta E}{\delta x_j})_{_{+0.5\Delta x_i}}-
                (\frac{\delta E}{\delta x_j})_{_{-0.5\Delta x_i}}}
          {\Delta x_i}\right).
$$
A call to the energy - gradient function \comp{COMPFG} will generate the
gradients in kcal/mol/\AA ngstrom at a given geometry.  These can then be
converted into millidynes/\AA ngstrom (or 10$^5$ dynes/cm) as follows:
$$
H_{i,j} (\mathrm{millidynes/\AA}) = 10^5\frac{({\mathrm{Kcal\ to\
ergs}})}{(\mathrm{\AA\ to\ cm})^2(\mathrm{Mole\ to\ molecule})}H_{i,j}(\mathrm{kcal/mol/\AA ^2 })
$$
or
$$
H_{i,j} (\mathrm{millidynes/\AA}) = 10^5\frac{4.184*10^3*10^7}{(10^{-8*2})
(6.022*10^{23})}H_{i,j}(\mathrm{kcal/mol/\AA ^2}).
$$
Diagonalization of this matrix yields the force constants of the system.

In order to calculate the vibrational frequencies, the Hessian matrix is first
mass-weighted:
$$
H^m_{i,j} = \frac{H_{i,j}}{\sqrt{M_i*M_j}}.
$$

Then the Hessian is converted from millidynes per \AA ngstrom to dynes per
centimeter by multiplying by 10$^5$.

Diagonalization of this matrix yields eigenvalues, $\epsilon$, which represent
the quantities $\sqrt{k/\mu}$, from which the vibrational frequencies can be
calculated:
$$
\bar{\nu}_i = \frac{1}{2\pi c}\sqrt{\epsilon_i}.
$$

\subsection{Mechanism of the frame in FORCE calculation}
\index{Frame!description of}
The FORCE calculation uses Cartesian coordinates, and all 3N  modes are
calculated, where N is the number of atoms in the system.  Clearly, there will
be 5 or 6 ``trivial'' vibrations,  which  represent  the  three translations
and two or three rotations.  If the molecule is exactly at a stationary point,
then these ``vibrations'' will have a  force  constant and  frequency  of
precisely  zero.   If the force calculation was done correctly, and the
molecule was not exactly at a stationary point,  then the  three  translations
should be exactly zero, but the rotations would be non-zero.  The extent to
which  the  rotations  are  non-zero  is  a measure of the error in the
geometry.

If  the  distortions  are  non-zero,  the  trivial  vibrations  can interact
with  the  low-lying genuine vibrations or rotations, and with the transition
vibration if present.

To prevent this the analytic form of the rotations  and  vibrations is
calculated,  and arbitrary eigenvalues assigned; these are 500, 600, 700, 800,
900, and 1000 millidynes/\AA ngstrom for Tx, Ty, Tz, Rx,  Ry  and Rz  (if
present),  respectively.  The rotations are about the principal axes of inertia
for the system, taking  into  account  isotopic  masses. The ``force matrix''
for these trivial vibrations is determined, and added on to the calculated
force matrix.  After diagonalization the  arbitrary eigenvalues are subtracted
off the trivial vibrations, and the resulting numbers are the ``true'' values.
Interference with genuine vibrations  is thus avoided.

\subsection{Vibrational Analysis}\index{Vibrational analysis}
Analyzing normal coordinates is very tedious.  Users  are  normally familiar
with the internal coordinates of the system they are studying, but not familiar
with the Cartesian coordinates.  To  help characterize the  normal
coordinates,  a very simple analysis is done automatically, and users are
strongly encouraged to use this analysis first,  and  then to look at the
normal coordinate eigenvectors.

In the analysis, each pair of bonded atoms is examined  to  see  if there  is
a  large  relative  motion  between them.  By bonded is meant \index{Van der
Waals|ff} within the van der Waals' distance.  If there  is  such  a  motion,
the indices  of  the  atoms,  the  relative  distance  in \AA ngstroms, and the
percentage radial motion are printed.   Radial  plus  tangential  motion adds
to  100\%,  but  as there are two orthogonal tangential motions and only one
radial, the radial component is printed.

Vibrations in the range +50 to {\em i}50 cm$^{-1}$ cannot be described
accurately in the vibrational analysis, due to numerical difficulties. However,
the normal coordinates and frequencies are printed in the section above
``DESCRIPTION OF VIBRATIONS''.

\subsection{Reduced masses}\index{Reduced mass|ff}\index{Simple harmonic motion}
\index{Mass!reduced}
A molecular vibration normally involves all the atoms moving simultaneously.
This is clearly very different from the  simple harmonic motion of a mass
attached to a spring that is attached to an immovable object.  Nevertheless, it
is convenient to visualize a molecular vibration as consisting of a single
mass, $M$, on the end of a spring of force constant $k$. For such a  system,
the period of vibration, $T$, is given by:
$$
T=2\pi\sqrt{\frac{M}{k}}.
$$

How, then, do we relate the complicated motion of a molecular vibration to the
mass and spring model?

During a molecular vibration, each atom follows a simple harmonic motion. So
the problem is, to what extent does each atom contribute to the mass, and to
what extent does each atom contribute to the spring?

In order to answer this, first consider some simple systems.  In the system
H--X, where X has a very large mass, compared to that of the H,  the effective
mass is obviously that of H.  In H$_2$, the effective mass is half that of a
single H.   Why is this so?  In H--X, particle X is stationary, and particle H
contributes 100\% of the energy to the vibration.  In H$_2$, each particle
obviously contributes 50\%, but now the center of mass is half way between the
two particles.  If the force constants are the same in H--X and in H--H, then
the period of vibration of H--X will be $\sqrt{2}$ times that of H--H.  This is
the same period as for a system of two particles, each of which having a mass
twice that of a H particle.  For a system of two particles, $A$ and $B$, having
masses $M_A$ and $M_B$,  the vibrational wavefunction, $\psi_v$, is:
$$
\psi_v=\sqrt{\frac{M_B}{M_A+M_B}}\psi_A-\sqrt{\frac{M_A}{M_A+M_B}}\psi_B.
$$
This can be interpreted as particle $A$ moves $(\sqrt{\frac{M_B}{M_A+M_B}})^2$
in the time particle $B$ moves $(\sqrt{\frac{M_A}{M_A+M_B}})^2$.  The center of
mass, $\rho$, stays constant:
$$
\rho=\sum_iM_i\delta x_i = M_A\frac{M_A}{M_A+M_B}- M_B\frac{M_A}{M_A+M_B} = 0.
$$
The square of the coefficients of the wavefunction represent the contribution
to the motion.  The effective mass, $\mu$, for this system is:
$$
\mu = \frac{M_A\times M_B}{M_A+M_B}.
$$
What fraction is due to $A$ and what fraction is due to $B$?  From the
wavefunction, the intensity of $A$ is $\frac{M_B}{M_A+M_B}$, and the relative
rate of motion is also $\frac{M_B}{M_A+M_B}$, so the contribution to the
effective mass due to $A$ is:
$$
(\frac{M_B}{M_A+M_B})M_A;
$$
likewise, for $B$:
$$
(\frac{M_A}{M_A+M_B})M_B.
$$
Consider two particles, $A$ and $B$, of mass 1 and 4, respectively.   The
wavefunction for the vibration is:
$$
\psi_v = \sqrt{\frac{4}{5}}\psi_A-\sqrt{\frac{1}{5}}\psi_B,
$$
where $A$ contributes
$$
\frac{16}{25}\times 1 = 0.64
$$
and $B$ contributes
$$
\frac{1}{25}\times 4 = 0.16
$$
to the effective mass of $\frac{4}{5}$.

In other words, the contribution to the effective mass is equal to the
intensity of the wavefunction on each atom, times the mass of the atom, times
the intensity of the wavefunction.  This is intuitively correct:  the total
vibration is composed of contributions from each particle, and the amount each
particle contributes is proportional to its intensity in the wavefunction.  The
mass of each particle is also proportional to its intensity in the
wavefunction.

Extension to polyatomic molecules is now trivial.  The effective mass is given
by:
$$
\mu = \sum_A <\!\psi_AM_A\psi_A\!>\times <\!\psi_A|\psi_A\!>.
$$
When written in this way, the quantum nature of the expression is obvious.
However, for computational convenience, the effective mass is rewritten as:
$$
\mu = \sum_A(\psi_A^2)^2\times M_A
$$
or
$$
\mu = \sum_A(\sum_{i=1}^3c_{A_i}^2)^2M_A.
$$
This expression is suitable for most systems.  However, it is not a
well-defined quantity.  Under certain circumstances involving degenerate
vibrations, the quantity $\mu$ can become ill-defined.  This phenomenon can be
attributed to the  fact that the reduced mass is not an observable.

\subsection{Effective masses}\index{Mass!effective}\index{Effective mass}
Another way of regarding the effective mass of a mode can be derived from
consideration of the simple harmonic oscillator:
$$
   E =  \sqrt{\frac{k}{\mu}}.
$$
Diagonalization of the mass-weighted Hessian yields the energies, and from the
normal coordinates the force-constants can readily be derived.  From these two
quantities, the effective mass can readily be calculated:
$$
\mu = \frac{k}{E^2}.
$$
For a homonuclear diatomic, the effective mass calculated this way is equal
to the mass of one atom.

\subsection{Travel}\index{Travel}
To continue the idea of representing a normal mode as a simple harmonic
oscillator, the distance the atoms move through can be represented as the
distance the idealized mass moves through.  This can be calculated knowing the
energy of the mode and the force constant:
$$
     E = \frac{1}{2}kx^2.
$$
Here $k$ is the force-constant for the mode, and is given by
$$
k = <\psi|Hessian|\psi>;
$$
$E$ is the energy of the mode.

From this, the distance, $x$, which the system moves through, can be calculated
from
$$
x =\sqrt{\frac{2\times 1.196266\times 10^8 \times 1000 \times 10^8 \nu}{N k}},
$$
where $1.196266\times 10^8$ is the conversion factor from cm$^{-1}$ to ergs,
1000 converts from millidynes to dynes, $10^8$ converts from cm to \AA , and
$N$ converts from moles to molecules.

Note that $x$, which in the output is called TRAVEL, is in mass weighted
space, not simple space.  This quantity can also be calculated using the DRC,
by depositing one quantum of energy into a vibrational mode.  For a system at a
stationary point, the relevant keywords would be \comp{IRC=1 DRC t=1m}.  For
larger systems, the time may need to be increased.  At least one coordinate
must have an optimization flag set to 1.  This is required in order to instruct
the DRC to print the turning points. \index{Normal coordinates!calculation
of|)}

\section{A Note on Thermochemistry}\index{Thermochemistry|(}
\label{tsuneo}
\begin{center}
by
\ \\
Tsuneo Hirano\index{Hirano@{\bf Hirano, Tsuneo}}\\
Department of Chemistry  \\
Faculty of Science  \\
Ochanomizu University  \\
2-1-1 Otsuka  \\
Bunkyo-ku  \\
Tokyo 112  \\
Japan
\end{center}

\subsection{Basic Physical Constants}
All basic physical constants should be taken from: ``Quantities, Units and
Symbols in Physical Chemistry,'' Blackwell Scientific Publications Ltd, Oxford
OX2 0EL, UK, 1987 (IUPAC, based on CODATA of ICSU, 1986~\cite{codata}).  pp.\
81--82. A summary of these constants can be found in
Table~\ref{codata}
\begin{latexonly}
(p.~\pageref{codata})
\end{latexonly}.

Some derived quantities, which will be used in this section only are:

Moment of inertia: $I$ 1 amu \AA ngstrom$^2$ =
$1.660 540 \times 10^{-40}$ g cm$^2$.

Rotational constants: $A$, $B$, and $C$ (e.g.\ $A = h/(8\pi^2I)$)\\
With $I$ in amu \AA ngstroms$^2$ then:
$A$ (in MHz)  = $5.053 791 \times 10^5 / I$ \\
$A$ (in cm$^{-1}$) = $5.053 791 \times 10^5/ cI = 16.857 63 / I$


\subsection{Thermochemistry from {\em ab initio} MO methods}
Ab initio MO methods provide total energies, $E_{\rm eq}$,  as  the  sum  of
electronic   and   nuclear-nuclear  repulsion  energies  for  molecules,
isolated in vacuum, without vibration at 0  K. \index{ab initio@{{\em ab
initio}}!total energies}
\begin{equation}
E_{\rm eq} = E_{\rm el} + E_{\rm nuclear-nuclear} \nonumber
\end{equation}
From the 0 K potential surface and using  the  harmonic  oscillator
\index{Harmonic oscillator} approximation,  we can calculate the vibrational
frequencies, $\nu_i$, of the normal modes of vibration.  Using these, we can
calculate  vibrational, rotational   and   translational   contributions  to
the  thermodynamic quantities such as the partition function and heat capacity
which  arise \index{Partition function|ff}\index{Heat capacity|ff}\index{C$_p$}
from heating the system from 0 to T K.

$Q$: partition function, $E$: energy, $S$: entropy,
\index{Energy}\index{Entropy|(}
and $C$: Heat capacity at constant pressure = C$_p$. In {\em ab initio}
calculations,\index{Ab initio!C$_v$} the heat capacity calculated is C$_v$.
The relationship between C$_p$ and C$_v$ (in cal.degree$^{-1}$.mol$^{-1}$) is:
$$
C_p=C_v+R=C_v+1.987.
$$

\subsubsection{Vibrational terms}
\begin{equation}
Q_{\rm vib} = \prod_i{\frac{1}{[1 - \exp(-h\nu_i/kT)]}} \label{eq2}
\end{equation}
$E_{\rm vib}$, for a molecule at the temperature $T$ as:
\begin{equation}
E_{\rm vib} = \sum_i\left\{\frac{h\nu_i}{2} +
\frac{h\nu_i\exp(-h\nu_i/kT)}{[1 - \exp(-h\nu_i/kT)]}\right\}  \label{eq3} \nonumber
\end{equation}
where $h$ is Planck's constant, $\nu_i$ the $i$--th normal  vibration
frequency, and $k$  the Boltzmann constant.  For 1 mole of molecules, $E_{\rm
vib}$  should be multiplied by the Avogadro number $N_a = R/k$.  Thus:
\begin{equation}
E_{\rm vib} = N_a  \sum_i\left\{ \frac{h\nu_i}{2}
 + \frac{h\nu_i\exp(-h\nu_i/kT)}{[1-\exp(-h\nu_i/kT)]}\right\}  \label{eq4}
\end{equation}

Note that the first term  in equation~(\ref{eq4})  is   the  zero-point
vibration energy.   Hence,  the second term  in eq.~(\ref{eq4}) is the
additional vibrational contribution due to the temperature increase from 0 K to
T K.  Namely,
\begin{eqnarray}
E_{\rm vib}                & = & E_{\rm zero} + E_{\rm vib}(T)
\label{eq5}\nonumber\\
E_{\rm zero}               & = & N_a \sum_i {\frac{h\nu_i}{2}} \nonumber\label{eq6}\\
E_{\rm vib}( T) & = & N_a  \sum_i
{\frac{h\nu_i\exp(-h\nu_i/kT)}{[1 - \exp(-h\nu_i/kT)]}} \label{eq7}
\end{eqnarray}
The value of $E_{\rm vib}$ from GAUSSIAN 82 and 86 includes
$E_{\rm zero}$ as defined by Eqs.~(\ref{eq4},\ref{eq7}).
\begin{eqnarray}
S_{\rm vib} & = & R \sum_i\left\{
\frac{(h\nu_i/kT)\exp(-h\nu_i/kT)}{[1 - \exp(-h\nu_ii/kT)]}
- \ln[1 - \exp(-h\nu_i/kT)]\right\} \nonumber\label{eq8}\\
C_{\rm vib} & = & R \sum_i\left\{
\frac{(h\nu_i/kT)^2 \exp(-h\nu_i/kT)}{[1 - \exp(-h\nu_i/kT)]^2}\right\}
\nonumber\label{eq9}
\end{eqnarray}

At temperature $T>0$ K, a molecule rotates about  the  x,  y,  and z-axes  and
translates  in  x,  y,  and  z-directions.  By assuming the
\index{Equipartition of energy} equipartition of energy, energies for rotation
and translation,  $E_{\rm rot}$ and $E_{\rm tr}$, are calculated.

\subsubsection{Rotational terms}
\index{Moments of inertia}
$\sigma$ is symmetry number (Examples of symmetry numbers are shown
in Table~\ref{rot}). $I$ is moment of inertia.
$I_A$, $I_B$, and $I_C$ are moments of inertia about A, B, and C axes.
\begin{table}
\caption{\label{rot} Table of Symmetry Numbers }
\begin{center}
\begin{tabular}{lllcrclllcrclrr}
\hline
C$_1$&C$_I $    &C$_S $:&&   1&&D$_2$&D$_{2d}$&D$_{2h}$:&&4 &&  C$_{\infty v}$:& 1  \\
C$_2$&C$_{2v}$&C$_{2h}$:&&  2 &&D$_3$&D$_{3d}$&D$_{3h}$:&&6 &&  D$_{\infty h}$:&
2  \\
C$_3$&C$_{3v}$&C$_{3h}$:&&  3 &&D$_4$&D$_{4d}$&D$_{4h}$:&&8 && T, T$_h$ T$_d$: &12  \\
C$_4$&C$_{4v}$&C$_{4h}$:&&  4 &&D$_5$&D$_{5d}$&D$_{5h}$:&&10  &&  O, O$_h $:   &24  \\
C$_5$&C$_{5v}$&C$_{5h}$:&&  5 &&D$_6$&D$_{6d}$&D$_{6h}$:&&12  &&  I, I$_h$:  &24
\\
C$_6$&C$_{6v}$&C$_{6h}$:&&  6 &&D$_7$&D$_{7d}$&D$_{7h}$:&&14  && S$_4$:&2
\\
C$_7$&C$_{7v}$&C$_{7h}$:&&  7 &&D$_8$&D$_{8d}$&D$_{8h}$:&&16  && S$_6$: &3 \\
C$_8$&C$_{8v}$&C$_{8h}$:&&  8 &&      &       &         && &   &  S$_8$: &4 \\\hline
\end{tabular}
\end{center}
\end{table}

\paragraph*{Linear molecule}
\begin{eqnarray}
Q_{\rm rot} & = & \frac{8\pi^2 IkT}{\sigma h^2} \nonumber\label{eq10} \\
E_{\rm rot} & = & (2/2)RT  \nonumber\label{eq11} \\
S_{\rm rot} & = &  R \ln \left[ \frac{8\pi^2 IkT}{\sigma h^2} \right] + R
\nonumber\label{eq12} \\
            & = & R \ln I + R \ln T - R \ln\sigma - 4.349 203  \nonumber
\end{eqnarray}
where $ -4.349 203 = R \ln[8\times 10^{-16} \pi^2 k/(N_a h^2)] + R$.
\begin{equation}
C_{\rm rot} = (2/2)R    \nonumber\label{eq14}
\end{equation}

\paragraph*{Non-linear molecule}
\index{Non-linear molecules}
\begin{eqnarray}
 Q_{\rm rot} & = & \left( \frac{\sqrt{\pi}}{\sigma}\right)
 \left[ \frac{8\pi^2kT}{h^2}\right]^{3/2} \sqrt{I_AI_BI_C} \nonumber\\
& = & \left( \frac{\sqrt{\pi}}{\sigma}\right)
\left[ \left( \frac{8\pi^2cI_A}{h} \right)
       \left( \frac{8\pi^2cI_B}{h} \right)
       \left( \frac{8\pi^2cI_C}{h} \right) \right]^{1/2}
\left(\frac{kT}{hc}\right)^{3/2} \nonumber\label{eq15} \\
 E_{\rm rot} & = & (3/2)RT \nonumber\label{eq16} \\
 S_{\rm rot} & = & \frac{R}{2} \ln\left\{
 \left(\frac{\pi}{\sqrt{\sigma}}\right) \left(\frac{8\pi^2cI_A}{h}\right)
 \left(\frac{8\pi^2cI_B}{h}\right)      \left(\frac{8\pi^2cI_C}{h}\right)
 \left(\frac{kT}{hc}\right)^3\right\} + (3/2)R  \nonumber\label{eq17} \\
& = & (R/2) \ln{(I_A I_B I_C)} + (3/2) R\ln{T}  - R \ln{\sigma} - 5.3863921
\nonumber
\end{eqnarray}

Here, $-5.386 3921$ is calculated as:
$$ R \ln\left\{\frac{1}{h^3} \left(\frac{10^{-16}}{N_a}\right)^{3/2}
\sqrt{(3\times 2^9 \times \pi^7 \times k)}\right\} + (3/2)R.
$$

\begin{equation}
C_{\rm rot} = (3/2)R \nonumber\label{eq18}
\end{equation}

\subsubsection{Translational terms}
$M$ is Molecular weight.
\begin{eqnarray}
Q_{\rm tra} & = & \left( \frac{\sqrt{2\pi MkT/N_a}}{h}\right)^3
\nonumber\label{eq19} \\
E_{\rm tra} & = & (3/2)RT                            \nonumber\label{eq20} \\
S_{\rm tra} & = & R\left\{ \frac{5}{2} +
\frac{3}{2}\ln\left(\frac{2\pi k}{h^2}\right) + \ln k +
\frac{3}{2}\ln\left(\frac{M}{N_a}\right) + \frac{5}{2}\ln T - \ln p \right\}
\nonumber\label{eq21} \\
            & = & (5/2)R\ln T + (3/2)R\ln M - R\ln p - 2.31482 \nonumber\label{eq22}\\
C_{\rm tra} & = & (5/2)R                              \nonumber\label{eq23}
\end{eqnarray}
or  $H_{\rm tra} = (5/2)RT$  due to the $pV$ term (cf.  $H = U + pV$).
The internal energy $U$ at $T$ is:
\begin{equation}
U = E_{\rm eq} + [E_{\rm vib} + E_{\rm rot} + E_{\rm tra}] \nonumber\label{eq24}
\end{equation}
or
\begin{equation}
U = E_{\rm eq} + [(E_{\rm zero} + E_{\rm vib}(T))
  + E_{\rm rot} + E_{\rm tra}]  \label{eq25}
\end{equation}
Enthalpy $H$ for one mole of gas is defined as
\begin{equation}
H = U + pV     \label{eq26}
\end{equation}
Assumption of an ideal gas (i.e.,  $pV = RT$) leads to
\begin{equation}
H = U + pV = U + RT  \label{eq27}
\end{equation}
Thus, \mi{Gibbs free energy} $G$ can be calculated as:
\begin{equation}
           G = H - T S(T) \label{eq28}
\end{equation}

\subsection{Thermochemistry in MOPAC}
\index{$\Delta H_f$|ff} \index{Heat of Formation|ff}
It should be noted that M.O. parameters for MINDO/3,  MNDO,  AM1, PM3, and
MNDO-$d$  are optimized so as to reproduce the experimental  heat of formation
(i.e., standard enthalpy of formation or the enthalpy change to  form  a mole
of  compound  at  25$^\circ$C from its elements in their standard state) as
well as observed geometries (mostly at 25$^\circ$C), and  not to reproduce the
$E_{\rm eq}$ and equilibrium geometry at 0 K.

In  this  sense, $E_{SCF}$  (defined  as  Heat  of  formation, $\Delta H_f$),
force constants,  normal  vibration  frequencies, etc.\  are  all related to
the values at 25$^\circ$C, not to 0 K.  Therefore, the $E_{\rm zero}$
calculated in FORCE is not the true $E_{\rm zero}$. Its use as $E_{\rm zero}$
should be made at your own risk, bearing in mind the situation discussed above.

Since $E_{SCF}$ is standard enthalpy of formation (at 25$^\circ$C):
\begin{equation}
E_{SCF} = E_{\rm eq} + E_{\rm zero} +
E_{\rm vib}(298.15) + E_{\rm rot} + E_{\rm tra} + pV
 + \sum\left[ - E_{\rm elec}({\rm atom}) + \Delta H_f({\rm atom})\right].
\label{eq29}
\end{equation}
To avoid the complication arising from the definition of $E_{SCF}$, within
\index{Standard  Enthalpy  of Formation}
the  thermodynamics  calculation  the  Standard  Enthalpy  of Formation,
$\Delta H$, is calculated by
\begin{equation}
\Delta H = E_{SCF} + (H_T - H_{298}). \label{eq30}
\end{equation}

Here, $E_{SCF}$ is the heat of formation (at 25$^\circ$C)  given in the output
list, and $H_T$ and $H_{298}$ are the enthalpy contributions for the increase
of the temperature from 0 K to $T$ and 298.15, respectively.  In other words,
the enthalpy of formation is corrected for the difference in temperature from
298.15~K to $T$.

There is a problem in that $H_T$ is the heat of formation at $T$ relative to
the heat of formation of the elements in their standard state at 298K. This
involves mixing standard and not standard terms. There is no easy way to get
 the correct value for $H_T$, but for rough work $H_T$ is useful. For more correct
 work, calculate $\Delta H$ for the elements in their standard state at $T$, and use
 these $\Delta H$'s to get the $\Delta H$ for the compound you're working with (or use
 tables from the literature).

 The method of calculation for $T$ and $H_{298}$ will  be
given below.

In MOPAC, the variables defined below are used:
\begin{equation}
C_1 = \frac{hc}{kT}.   \nonumber\label{eq31}
\end{equation}
The wavenumber, $\omega_i$, in cm$^{-1}$:
\begin{equation}
\nu_i = \omega_i c    \nonumber\label{eq32}
\end{equation}
\begin{equation}
E_{\rm \omega_i} = \exp( -h\nu_i/kT) = \exp(-\omega_i hc/kT) = \exp(-\omega_i C_1) \nonumber\label{eq33}
\end{equation}
The rotational constants $A$, $B$, and $C$ in cm$^{-1}$:
\begin{equation}
A = \frac{h}{(8\pi^2 cI_A)}   \nonumber\label{eq34}
\end{equation}

\index{S} Energy and Enthalpy in cal/mol, and Entropy in cal/mol/K. Thus, eqs.\
(\ref{eq2}--\ref{eq28}) can be written as follows:

\subsubsection{Vibration}
\begin{eqnarray}
Q_{\rm vib} & = & \prod_i \frac{1 }{ (1 - E_{\rm \omega_i})}  \nonumber\label{eq35}\\
E_0         & = & \frac{0.5 N_a hc}{4.184 \times 10^7}\sum_i {\omega_i}
\nonumber\label{eq36}\\
            & = & 1.429 572 \sum_i {\omega_i}    \nonumber\label{eq37}\\
E_{\rm vib}(T) & = &
N_a h c\sum_i{\frac{\omega_i E_{\rm \omega_i}}{1 - E_{\rm \omega_i}}}
= (R/k) h c\sum_i{\frac{W_ iE_{\rm \omega_i}}{1 - E_{\rm \omega_i}}}     \nonumber\label{eq38}\\
S_{\rm vib} & = & R (hc/kT)
\sum_i\left\{\frac{\omega_i E_{\rm \omega_i}}{(1 - E_{\rm \omega_i})}\right\}
                - R\sum_i {\ln (1 - E_{\rm \omega_i})} \nonumber \\
& = & R C_1\sum_i\left\{\frac{\omega_i E_{\rm \omega_i}}{(1 - E_{\rm \omega_i})}\right\}
- R\sum_i {\ln(1 - E_{\rm \omega_i})}            \nonumber\label{eq39}\\
C_{\rm vib} & = & R (hc/kT)^2
\sum_i\left\{\frac{\omega_i^2 E_{\rm \omega_i}}{(1- E_{\rm \omega_i})^2} \right\} \nonumber \\
& = & R C_1^2
\sum_i\left\{\frac{\omega_i^2 E_{\rm \omega_i}}{(1- E_{\rm \omega_i})^2}\right\}
\nonumber\label{eq40}
\end{eqnarray}

\subsubsection{Rotation}

\paragraph*{Linear molecule}
\begin{eqnarray}
Q_{\rm rot} & = & (1/\sigma) (1/{\rm \AA}) (kT/hc) = \frac{1}{\sigma A C_1}
\nonumber\label{eq41}\\
E_{\rm rot} & = & (2/2)RT   \nonumber\label{eq42} \\
S_{\rm rot} & = & R\ln\left(\frac{kT}{\sigma hc{\rm \AA}}\right) + R
= R\ln \left(\frac{1}{\sigma {\rm \AA} C_1}\right) + R
= R\ln\left(\frac{kT}{\sigma hc{\rm \AA}}\right) + R \nonumber\label{eq43}\\
C_{\rm rot} & = & (2/2)R       \nonumber\label{eq44}
\end{eqnarray}

\paragraph*{Non-linear molecule}
\begin{eqnarray}
Q_{\rm rot} & = & \frac{1}{\sigma}
\left[\frac{\pi}{(A B C C_1^3)}\right]^{1/2} \nonumber\label{eq45}\\
E_{\rm rot} & = & (3/2)RT        \nonumber\label{eq46} \\
S_{\rm rot} & = & \frac{R}{2}\ln\left\{\frac{\pi}{\sigma^2ABC}
\left(\frac{kT}{hc}\right)^3 \right\} + (3/2)R \nonumber \\
& = & 0.5R { 3\ln(kT/hc) - 2\ln\sigma +\ln\left(\frac{\pi}{A B C}\right) + 3}
\nonumber\label{eq47} \\
& = & 0.5R { -3\ln C_1 - 2\ln\sigma + \ln\left(\frac{\pi}{A B C}\right) + 3}
\nonumber \\
C_{\rm rot} & = & (3/2)R     \nonumber\label{eq48}
\end{eqnarray}

\subsubsection{Translation}
\begin{eqnarray}
Q_{\rm tra} & = &
   \left( \frac{\sqrt{2\pi MkT/N_a}}{h} \right)^3
=  \left( \frac{\sqrt{1.660540\times^{-24}\times 2\pi MkT}}{h} \right)^3
\nonumber\label{eq49} \\
E_{\rm tra} & = &  (3/2)RT                    \nonumber\label{eq50} \\
H_{\rm tra} & = &  (3/2)RT + pV = (5/2)RT \; {\rm cf.}
\;pV = RT  \nonumber\label{eq51}\\
S_{\rm tra} & = & (R/2) [ 5\ln T + 3\ln M ] - 2.31482
\;{\rm cf.}\; p = 1 {\rm atm} \nonumber \\
& = & 0.993608 [ 5\ln T + 3\ln M] - 2.31482    \nonumber\label{eq52}
\end{eqnarray}

In MOPAC:
\begin{equation}
H _{\rm vib} = E_{\rm vib}(T)     \nonumber\label{eq53}
\end{equation}

(Note: $E_{\rm zero}$ is {\em not\/} included in $H_{\rm vib}$;
$\omega_i$ is not derived from force-constants at 0 K)
and for $T$:
\begin{equation}
H_T   = [H_{\rm vib} + H_{\rm rot} + H_{\rm tra}]   \nonumber\label{eq54}
\end{equation}
while for $T=298.15$~K:
\begin{equation}
H_{298} = [H_{\rm vib} + H_{\rm rot} + H_{\rm tra}]  \nonumber\label{eq55}
\end{equation}

Note that $H_T$ (and $H_{298}$) is equivalent to:
\begin{equation}
(E_{\rm vib} - E_{\rm zero}) + E_{\rm rot} + (E_{\rm tra} + pV) \nonumber\label{eq56}
\end{equation}
except that  the  normal  frequencies  are  those  obtained  from  force
constants at 25$^\circ$C, or at least not at 0 K.

Thus, Standard Enthalpy of Formation, $\Delta H$,  can  be  calculated
according to Eqs.~(\ref{eq25},\ref{eq26}) and (\ref{eq29}),
as shown in Eq.~(\ref{eq30});
\begin{equation}
\Delta H = E_{SCF} + (H_T - H_{298})   \nonumber\label{eq57}
\end{equation}
Note that $E_{\rm zero}$ is already counted in $E_{SCF}$,
see Eq.~(\ref{eq29}).

\index{Standard Internal Energy of Formation}
By using Eq.~(\ref{eq27}), Standard Internal Energy of Formation, $\Delta U$,
can be calculated as:
\begin{equation}
\Delta U = \Delta H - R(T - 298.15) \nonumber\label{eq58}
\end{equation}


Standard Gibbs Free-Energy of Formation, $\Delta G$, can be calculated
\index{Standard Gibbs Free-Energy of Formation}
by  taking  the difference from that for the isomer or that at different
temperature:
\begin{equation}
\Delta G = [\Delta H - TS] \;(\mbox{for the state under consideration})
- [\Delta H - TS]\; (\mbox{for reference state})  \nonumber\label{eq59}
\end{equation}

Taking the difference is necessary  to  cancel  the  unknown  values  of
standard entropy of formation for the constituent elements.

\index{Entropy|)}\index{Thermochemistry|)}

\section{Force Constants}\index{Force constants}\index{Internal!coordinate!force constants}\label{fc}
Internal coordinate force constants, $f_{ic}$, can be derived from the
Cartesian coordinates and the Cartesian force constant matrix by use of:
$$
f_{ic}(l) = \sum_j\sum_k
\frac{d({\rm c.c.}_j)}{d({\rm i.c.}_l)}
\frac{d^2E}{d({\rm c.c.}_j)d({\rm c.c.}_k)}
\frac{d({\rm c.c.}_k)}{d({\rm i.c.}_l)}.
$$
As with the gradients, the calculation of $\frac{d({\rm c.c.}_j)}{d({\rm
i.c.}_l)}$ is quite simple, and in done in routine JCARIN.

During the testing of this function, a minor fault in the conventional
force calculation was revealed.  To reduce any error introduced by
finite arithmetic, the Hessian matrix is symmetrized before the
vibrational frequencies and normal coordinates are calculated.  This is
done by \hyperref[pageref]{operating on the Hessian matrix}{ as
described on p.~}{}{sym_force}.

In addition to the requirement that the symmetry of the Hessian should be the
same as that of the nuclear coordinates, a second requirement is that the
diagonal elements of the Hessian should be equal to the negative of the sum of
the off-diagonal elements, that is, that:
$$
F_{ii} = -\sum_{j\ne i}F_{ij}.
$$
During the testing of the internal force constants, very small variations in
the force constants were found where no variation was expected.  This was
traced back to a failure of the above expression.  To correct this, the
diagonal terms of the force constant matrix were modified.  This resulted in a
perfect equivalence of equivalent force constants.  An incidental benefit would
be that the associated error in the calculated frequencies would be eliminated.

\section{Reaction paths}\index{Reaction paths}\index{Reaction path calculation}
\label{rpaths}
\index{Path calculations}

\begin{figure}
\begin{makeimage}
\end{makeimage}
\begin{verbatim}
 CHARGE=-1
 SN2 reaction, Cl(-) + CH3F = CH3Cl + F(-)
   C
   F   1.4  1
   H   1.1  1   109.5 1     0    0    1 2
   H   1.1  1   109.5 1   120.0  0    1 2 3
   H   1.1  1   109.5 1   120.0  0    1 2 3
   Cl 20.0 -1   127.3 1   180.0  0    1 2 3
   0   0.00 0     0.0 0     0.0  0    0 0 0
  10.0 5.0 4.0 3.0 2.9 2.8 2.7 2.6 2.5
   2.4 2.3 2.2 2.1 2.0 1.9 1.8 1.7 1.6
\end{verbatim}
\index{S$_{N^2}$!reaction}
\caption{\label{sn2} Example of an S$_{N^2}$ reaction path calculation}
\end{figure}

\begin{figure}
\begin{makeimage}
\end{makeimage}
\begin{verbatim}
step=5 points=13  SYMMETRY
Ethane, Barrier to Rotation
C
C  1.5 1    0 0    0  0   1 0 0
H  1.0 1  111 1    0  0   2 1 0
H  1.0 0  111 0  120  0   2 1 3
H  1.0 0  111 0 -120  0   2 1 3
H  1.0 0  111 0   60 -1   1 2 3
H  1.0 0  111 0  180  0   1 2 3
H  1.0 0  111 0  -60  0   1 2 3
0  0.0 0    0 0    0  0   0 0 0
3 1 4 5 6 7 8
3 2 4 5 6 7 8
6 7 7
6 11 8
\end{verbatim}
\index{Rotation barrier calculation}
\caption{\label{c2h6p} Example of a rotation barrier calculation}
\end{figure}

\begin{figure}
\begin{makeimage}
\end{makeimage}
\begin{verbatim}
step=0.05 points=20
Trans-polyparaphenylene benzobisthiazole
Stretching the polymer
  C    0.0  0      0  0      0  0    0  0  0
  N    1.3  1      0  0      0  0    1  0  0
  S    1.7  1    115  1      0  0    1  2  0
  C    1.6  1     92  1      0  1    3  1  2
  C    1.4  1    109  1     -0  1    2  1  3
  C    1.4  1    124  1   -180  1    5  2  1
  C    1.4  1    116  1    180  1    6  5  2
  C    1.4  1    121  1      0  1    7  6  5
  C    1.4  1    129  1    180  1    4  3  1
  S    1.6  1    129  1    180  1    7  6  5
  C    1.7  1     92  1    180  1   10  7  6
  N    1.4  1    113  1   -180  1    8  7  6
  C    1.4  1    121  1   -180  1   11 10  7
  C    1.4  1    120  1    -90  1   13 11 10
  C    1.4  1    120  1    180  1   14 13 11
  C    1.4  1    120  1      0  1   15 14 13
  C    1.4  1    118  1     -0  1   16 15 14
  C    1.4  1    120  1      0  1   17 16 15
  H    1.0  1    121  1     -0  1    6  5  2
  H    1.0  1    121  1      0  1    9  4  3
  H    1.0  1    120  1     -0  1   14 13 11
  H    1.0  1    119  1   -180  1   15 14 13
  H    1.0  1    120  1   -180  1   17 16 15
  H    1.0  1    119  1    180  1   18 17 16
 xx    1.4  1    120  1    180  1   16 15 14
 Tv   12.6 -1      0  0      0  0    1 25 24
\end{verbatim}
\index{Polymers!stretching}
\index{PBT}
\caption{\label{hook} Data set to stretch a polymer}\index{Hook's law calculation}
\end{figure}

MOPAC has the capability to model the effects of changing an internal
coordinate. In the data-set, the relevant internal coordinate is flagged with a
`-1' rather than a `1' or `0'.  Two options then exist to allow the values of
the changing coordinate to be defined. 

First, the various values of the coordinate can be supplied after the geometry
and any symmetry data have been entered.  An example for the S$_{N^2}$ reaction
Cl$^-$ + CH$_3$F $\rightarrow$ CH$_3$Cl + F$^-$  is given in Figure~\ref{sn2}.

\index{Symmetry!use in reaction paths}
\index{Rotation barriers|ff}
\index{Ethane!barrier in}
Second, if the step-size is a constant, then the step-size and number of steps
can be defined on the keyword line.  An example of such a ``reaction'' would be
the rotation of a methyl group in, e.g., ethane, Figure~\ref{c2h6p}. Here,
symmetry is used to maintain D$_3$ symmetry as the rotation takes place. Note
that \comp{SYMMETRY} can be used to relate coordinates to the reaction
coordinate. The path calculations work by optimizing the geometry while the
reaction coordinate is fixed at the starting value.  Once the geometry is
optimized, the reaction coordinate is changed, and the geometry re-optimized. 
This is done for all points on the reaction path.  

Reaction paths can be used for calculating mechanical properties.  For example,
to calculate Hook's force constant for stretching polyethylene, the translation
vector could be steadily increased, Figure~\ref{hook}.

\section{Grid Calculation}
\label{grid}
\index{Grid Calculation|ff}
The GRID calculation is the two-dimensional analog of the PATH calculation.  In
a PATH calculation, one coordinate is flagged with a `-1'.  In a GRID
calculation, two coordinates are flagged by `-1's.  An example of a  GRID
calculation is shown in Figure~\ref{exgrid}. Note that the keywords
\comp{STEP1=$n.nn$} and \comp{STEP2=$m.mm$} are essential.

\begin{figure}
\begin{makeimage}
\end{makeimage}
\begin{verbatim}
SYMMETRY STEP1=0.01 STEP2=1
Water, potential energy surface for

 H
 O   0.92 -1     0  0    0 0   1
 H   0.92  0   104 -1    0 0   2 1

 2 1 3
\end{verbatim} 
\caption{\label{exgrid} Example of a GRID Calculation}
\end{figure}

In this example, the potential energy surface for water is generated.  For one
axis of the 2-D plot, the O--H bond length is varied from 0.92\AA\ to 1.02\AA ,  in 11 steps of 0.01\AA , and in the other axis, the H--O--H angle is varied
from 104  to 114$^{\circ}$ in 11 steps of 1.0 degree.  Because of the use of
symmetry, there are no variables to be optimized.  If symmetry were not used,
then the second O--H bond length could either be optimized, by setting its flag
to 1, or held constant, by setting its flag to 0.  

Other keywords which can be used with the GRID option are: \comp{MAX}: to use
23 points in each direction, rather than the default 11; \comp{POINT1=n} and
\comp{POINT2=m}: to use $n$ and $m$ points in directions 1 and 2; and keywords
to specify how the geometry should be optimized.

\section{Dynamic and Intrinsic Reaction Coordinates}\index{Coordinates!reaction}
\label{t_irc}\index{Molecular dynamics}\index{Time-dependent phenomena}
The Intrinsic Reaction Coordinate method pioneered and developed by Mark
Gordon, North Dakota State University,\index{gordon@{\bf Gordon, Mark}}   has
been incorporated into MOPAC in a modified form.  As this facility is quite
complicated all the keywords associated with  the IRC have been grouped
together in this section (these can be seen later on  in this section).
\index{DRC|(} \index{DRC!definition}

The Dynamic Reaction Coordinate is the path  followed  by  all  the atoms  in
a  system  assuming  conservation  of  energy;  i.e.,  as the potential energy
changes the kinetic energy of  the  system  changes  in exactly  the  opposite
way  so  that  the  total  energy  (kinetic plus potential) is a constant.  It
is equivalent to the molecular mechanics molecular dynamics calculation, except
that bond-breaking and bond-making are supported, as are all the electronic
phenomena of the semiempirical methods.

If started at a  ground  state  geometry,  no \index{Transition state!use with
DRC} significant  motion should be seen.  Similarly, starting at a transition
state geometry should not produce  any  motion---after  all  it  is  a
stationary point and during the lifetime of a calculation it is unlikely to
accumulate enough momentum to travel far from the starting position.

In order to calculate the DRC path from a transition state,  either an
initial  deflection  is  necessary  or some initial momentum must be supplied.

Because of the time-dependent nature of the DRC  the  time  elapsed since the
start of the reaction is meaningful, and is printed.

\subsection*{Description}
The course of a molecular vibration can be followed by  calculating the
potential  and  kinetic  energy  at  various  times.   Two  extreme conditions
can be identified:  (a) gas phase, in which the total  energy is a constant
through time, there being no damping of the kinetic energy allowed, and (b)
liquid phase, in which kinetic energy is always set  to zero, the motion of the
atoms being infinitely damped.\index{Liquids}

All possible degrees of damping  are  allowed.   In  addition,  the facility
exists  to  dump  energy into the system, appearing as kinetic energy.  As
kinetic energy is a function of velocity, a vector quantity, the  energy
appears  as  energy of motion in the direction in which the molecule would
naturally move.  If the system  is  a  transition  state, then  the  excess
kinetic  energy  is added after the intrinsic kinetic energy has built up to at
least 0.2 kcal/mol.\index{Kinetic energy!damping}

For ground-state systems, the excess energy sometimes  may  not  be added;  if
the  intrinsic kinetic energy never rises above 0.2 kcal/mol then the excess
energy will not be added.


\subsection*{Equations used}
Force acting on any atom:
$$ g(i) + g'(i)t + g''(i)t^2 = \frac{dE}{dx(i)} +
        \frac{d^2E}{dx(i)^2} + \frac{d^3E}{dx(i)^3} $$
Acceleration due to force acting on each atom:
$$ a(i) = \frac{1}{M(i)} (g(i) + g'(i)t + g''(i)t^2) $$
New velocity:
$$ V(o) + \frac{1}{M(i)} \left(\Delta t g(i) + (1/2) \Delta  t^2 g'(i) +
(1/3) \Delta t^3g''(i)\right) $$
or:
$$ V(i) = V(i) + V'(i)t + V''(i)t^2 + V'''(i)t^3 $$
That is, the change in velocity is equal to the integral  over  the
time interval of the acceleration.

New position of atoms:
$$ X(i) = X(o) + V(o)t + (1/2) V't^2 + (1/3) V''t^3 + (1/4) V'''t^4 $$
That is, the change in position is equal to the integral  over  the
time interval of the velocity.

The velocity vector is accurate to the extent that  it  takes  into account
the  previous velocity, the current acceleration, the predicted acceleration,
and the change in predicted  acceleration  over  the  time interval.    Very
little  error  is  introduced  due  to  higher  order contributions to the
velocity; those that do occur  are  absorbed  in  a re-normalization of the
magnitude of the velocity vector after each time interval.

The magnitude of $\Delta t$, the time interval, is determined mainly by the
factor   needed   to   re-normalize  the  velocity  vector.   If  it  is
significantly different from unity, $\Delta t$ will be reduced; if  it  is
very close to unity, $\Delta t$ will be increased.

Even with all this, errors creep in and a system,  started  at  the transition
state,  is  unlikely  to  return precisely to the transition state  unless  an
excess  kinetic  energy  is  supplied,  for   example 0.2 kcal/mol.

The calculation  is  carried  out  in  Cartesian  coordinates,  and converted
into   internal  coordinates  for  display.   All  Cartesian coordinates must
be allowed to vary, in order to  conserve  angular  and translational
momentum.\index{DRC!conservation of momentum}

\subsection*{IRC}
\index{IRC|(}
The Intrinsic Reaction Coordinate is the path followed by  all  the atoms  in
a  system  and assumes that  all kinetic energy is completely lost at every
point; i.e., as the potential energy changes  the  kinetic  energy generated
is  annihilated  so  that  the  total  energy  (kinetic  plus potential) is
always equal to the potential energy only.

The IRC is intended for use in calculations in which the starting geometry is
that of the transition state.  A   normal  coordinate  is  chosen,  usually
the  reaction coordinate, and the system  is  displaced  in  either  the
positive  or negative  direction  along  this  coordinate.   The  internal
modes are obtained by calculating the mass-weighted  Hessian  matrix  in  a
force calculation   and   translating  the  resulting  Cartesian  normal  mode
eigenvectors to conserve  momentum.   That  is,  the  initial  Cartesian
coordinates  are  displaced  by  a  small  amount  proportional  to  the
eigenvector coefficients plus a translational constant; the constant  is
required  to  ensure that the total translational momentum of the system is
conserved as zero.  At the present time there may be  small  residual
rotational  components  which  are not annihilated; these are considered
unimportant, and will not materially affect the calculation.

\subsection*{General description of the DRC and IRC}
As the IRC usually requires a normal coordinate, a  force  constant
calculation  normally  has to be done first.  If IRC is specified on its own, a
normal coordinate is not used and the IRC calculation is performed on the
supplied geometry.

A recommended sequence of operations to start an IRC calculation is
as follows:
\begin{enumerate}
\item Calculate the transition state geometry.  If  the  transition state  is
not first  optimized,  then  the  IRC  calculation  may  give  very misleading
results.  For example, if NH$_3$ inversion  is  defined as  the  planar
system  but  without the N--H bond length being optimized, the first normal
coordinate might be for N--H  stretch rather  than  inversion.   In  that case
the IRC will relax the geometry to the optimized planar structure.

\index{ISOTOPE!use with IRC}
\index{FORCE!use with IRC}
\item Do a normal FORCE calculation, specifying \comp{ISOTOPE} in  order  to
save  the  FORCE  matrices.   (Note: Do  not  attempt  to  run the IRC at this
point directly unless you have confidence that  the FORCE  calculation will
work as expected.  If the IRC calculation is run directly, specify
\comp{ISOTOPE} anyway:  that will save the FORCE matrix and if the
calculation  has  to  be  re-done  then  \comp{RESTART} will work correctly.)

\item Using \comp{IRC=$n$} and \comp{RESTART}, run the IRC calculation.   If
\comp{RESTART} is specified with \comp{IRC=$n$} then the restart is assumed to
be from the FORCE calculation.  If, in an \comp{IRC} calculation,
\comp{RESTART}  is specified, and \comp{IRC=$n$} is {\em not} present,  then
the restart is  assumed to be from an earlier IRC calculation that was  shut
down  before  going  to completion.
\end{enumerate}

A DRC calculation is simpler, in that a force calculation is  not  a
prerequisite;  however,  most  calculations of interest normally involve use of
an internal coordinate.  For this reason IRC=$n$  can  be  combined with  DRC
to  give  a  calculation in which the initial motion (0.3kcal worth of kinetic
energy) is supplied by  the  IRC,  and  all  subsequent motion  obeys
conservation of energy.  The DRC motion can be modified in three ways:
\begin{enumerate}
\item It is possible to calculate the reaction  path  followed  by  a system
in  which  the  generated  kinetic energy decays with a finite half-life.  This
can  be  defined  by  DRC=$n.nnn$,  where $n.nnn$  is  the  half-life in
femtoseconds.  If $n.nn$ is 0.0 this corresponds  to  infinite  damping
simulating  the   IRC.    A limitation  of  the  program is that time only has
meaning when DRC is specified without a half-life.

\item Excess kinetic energy can be added to the calculation by use of
KINETIC=$n.nn$.   After  the  kinetic  energy  has  built  up  to 0.2 kcal/mol
or if IRC=$n$ is used then $n.nn$ kcal/mol of kinetic energy  is  added  to
the  system.   The excess kinetic energy appears as a velocity vector  in  the
same  direction  as  the initial motion.

\index{RESTART!use with IRC/DRC}
\item The RESTART file \verb/<filename>.res/ can be edited to allow the user
to  modify the velocity vector or starting geometry.  This file
is formatted.
\end{enumerate}

Frequently, the  DRC leads to a periodic, repeating orbit.   One  special
type---the  orbit in which the direction of motion is reversed so that the
system retraces its own path---is sensed for  and  if  detected  the
calculation  is  stopped after exactly one cycle.  If the calculation is to be
continued,  the  keyword  \comp{GEO-OK}  will  allow  this  check  to  be
by-passed.

\index{GNORM!use with IRC/DRC}  Sometimes the system will enter a stable state
in which the geometry is always changing, but nothing new is occurring.  One
example would be a system which decomposed into fragments, and the fragments
were moving apart. If all forces acting on the atoms become small, then the
calculation will be stopped.  If the calculation should be continued, then
specify \comp{GNORM=0 LET}.

Due to the potentially very large output files  that  the  DRC  can generate,
extra  keywords  are  provided  to allow selected points to be printed.  Two
types of control are provided:  one controls which points to  print, the other
controls what is printed.

By default, every point calculated is printed.  Often, this is not desirable,
and three keywords are provided to allow printing to be done whenever the
system changes by a preset amount.  These keywords are:

\begin{center}
\begin{tabular}{cll}\hline
          KeyWord &       Default         &   User Specification  \\ \hline
\comp{X-PRIO}  &   0.05 \AA ngstroms   &         \comp{X-PRIORITY=$n.nn$}  \\
\comp{T-PRIO}  &   0.10 Femtoseconds   &      \comp{T-PRIORITY=$n.nn$}  \\
\comp{H-PRIO}  &   0.10 kcal/mol      &      \comp{H-PRIORITY=$n.nn$}\\ \hline
\end{tabular}
\end{center}

By default, only the energies involved are printed (one line per point).  To
allow the geometry to be printed, \hyperref[pageref]{\comp{LARGE} is provided}{, see
p.~}{ for more detail}{large}.  Using \comp{LARGE} a wide range of control
is provided over what is printed.

\subsection*{Option to allow only extrema to be output}
In the geometry specification, if an internal coordinate is  marked for
optimization  then  when that internal coordinate passes through an extremum a
message will be printed and the geometry output.

Difficulties can  arise  from  the  way  internal  coordinates  are
processed.   The  internal  coordinates are generated from the Cartesian
coordinates, so an internal coordinate supplied  may  have  an  entirely
different  meaning  on  output.  In particular the connectivity may have
changed.  For obvious reasons dummy atoms should  not  be  used  in  the
supplied  geometry  specification.   If  there  is  any  doubt about the
internal coordinates or if the starting geometry  contains  dummy  atoms then
run  a  \comp{1SCF} calculation specifying \comp{INT}.  This  will produce an
ARC file with the ``ideal'' numbering---the internal numbering system used  by
MOPAC. \ Use this ARC file to construct a data file suitable for the DRC or
IRC.\index{DRC!dummy atoms in}\index{Dummy atoms!in DRC}

Notes:
\begin{enumerate}
\item Any coordinates marked for optimization  will  result  in  only extrema
being printed.
\item If extrema are being printed then kinetic energy  extrema  will also be
printed.
\end{enumerate}

\subsection*{Keywords for use with the IRC and DRC}
\index{IRC!keywords for}
\label{drckeys}
\begin{enumerate}
\item Setting up the transition state:  \comp{NLLSQ}, \comp{SIGMA}, or \comp{TS}.
\item Constructing the FORCE matrix:  \comp{FORCE} or \comp{IRC=$n$},
\comp{ISOTOPE}, \comp{LET}.
\item Starting an IRC:  \comp{RESTART} and \comp{IRC=$n$},  \comp{X-PRIO}, \comp{H-PRIO}.
\item Starting a DRC:  \comp{DRC} or \comp{DRC=$n.nn$}, \comp{KINETIC=$n.nn$},
\comp{T-PRIO}, etc..
\item Starting a DRC from a transition state:   (\comp{DRC}  or  \comp{DRC=$n$})  and
            \comp{IRC=$n$}, \comp{KINETIC=$n$}.
\item Restarting an IRC:  \comp{RESTART} and \comp{IRC}.
\item Restarting a DRC:  \comp{RESTART} and (\comp{DRC} or \comp{DRC=$n.nn$}).
\item Restarting a DRC starting from a transition state:  \comp{RESTART} and
            (\comp{DRC} or \comp{DRC=$n.nn$}).
\end{enumerate}
Other keywords, such as \comp{T=$nnn$} or \comp{GEO-OK} can be used any time.


\subsection*{Examples of DRC/IRC data}
Use of the IRC/DRC facility is quite complicated.  In the following examples
various `reasonable' options are illustrated for a calculation on water. It is
assumed  that  an  optimized  transition-state  geometry  is available.

Example  1:   Figure~\ref{h2odrc} illustrates a  Dynamic  Reaction   Coordinate
calculation,  starting  at   the transition  state  for  water  inverting, the
initial motion being opposite to the transition normal mode, with 6kcal of
excess kinetic  energy  added  in. Every point calculated is to be printed
(Note all coordinates are marked with a zero, and T-PRIO, H-PRIO and X-PRIO are
all absent).  The results of  an  earlier calculation using the same keywords
is assumed to exist. The earlier calculation would have constructed the force
matrix.   While the  total  cpu  time  is specified, it is in fact redundant in
that the calculation will run to completion in less than 600 seconds.

\index{IRC!example of}
\begin{figure}
\begin{makeimage}
\end{makeimage}
\begin{verbatim}
 KINETIC=6 RESTART  IRC=-1 DRC T=600
 WATER

      H   0.000000 0   0.000000 0   0.000000 0  0 0 0
      O   0.911574 0   0.000000 0   0.000000 0  1 0 0
      H   0.911574 0 180.000000 0   0.000000 0  2 1 0
      0   0.000000 0   0.000000 0   0.000000 0  0 0 0
\end{verbatim}
\caption{\label{h2odrc} Example of DRC calculation}
\end{figure}

Example 2:  Figure~\ref{h2oirc} shows an Intrinsic Reaction Coordinate
calculation.  Here the restart  is from a previous IRC calculation which was
stopped before the minimum was reached.  Recall that RESTART with IRC=$n$
implies  a  restart from  the FORCE calculation.  Since this is a restart from
within an IRC calculation the keyword IRC=$n$ has been replaced by IRC. \  IRC
on its  own (without the ``=$n$'') implies an IRC calculation from the starting
position---here the RESTART position---without initial
displacement.\index{IRC!example of restart}

\begin{figure}
\begin{makeimage}
\end{makeimage}
\begin{verbatim}
 RESTART  IRC  T=600
 WATER

      H   0.000000 0   0.000000 0   0.000000 0  0 0 0
      O   0.911574 0   0.000000 0   0.000000 0  1 0 0
      H   0.911574 0 180.000000 0   0.000000 0  2 1 0
      0   0.000000 0   0.000000 0   0.000000 0  0 0 0
\end{verbatim}
\caption{\label{h2oirc} Example of IRC calculation}
\end{figure}

\subsection*{Output format for IRC and DRC}
The IRC and DRC can produce  several  different  forms  of  output. Because of
the large size of these outputs, users are recommended to use search functions
to extract information.  To facilitate  this,  specific lines  have specific
characters.  Thus, a search for the ``\%'' symbol will summarize the energy
profile while a search  for ``AA'' will  yield  the coordinates of atom 1,
whenever it is printed.  The main flags to use in searches are:

\begin{description}
\item[\comp{\%}] Energies for all points calculated,    excluding extrema
\item[\comp{\%M}] Energies for all turning points
\item[\comp{\%MAX}] Energies for all maxima
\item[\comp{\%MIN}] Energies for all minima
\item[\comp{\%}] Energies for all points calculated
\item[\comp{AA*}] Internal coordinates for atom 1 for every point
\item[\comp{AE*}] Internal coordinates for atom 5 for every point
\item[\comp{123AB*}] Internal coordinates for atom 2 for point 123
\end{description}

As the keywords for the IRC/DRC are interdependent,  the  following list of
keywords illustrates various options.\index{DRC!keyword options}
\index{KINETIC}

\begin{description}
\item[\comp{DRC}] The Dynamic Reaction Coordinate is calculated.
Energy is conserved, and no initial impetus.
\item[\comp{DRC=0.5}] In the DRC kinetic energy is lost with a half-life of
0.5 femtoseconds.
\item[\comp{DRC=1.0}] Energy is put into a DRC with an half-life of
-1.0 femtoseconds, i.e., the system gains   energy.
\item[\comp{IRC}] The Intrinsic Reaction Coordinate is
calculated.  No initial impetus is given.
Energy not conserved.
\item[\comp{IRC=4}] The IRC is run starting with an impetus in the
negative of the 4th normal mode direction. The
impetus is one quantum of vibrational energy.
\item[\comp{IRC1 KINETIC=1}] The first normal mode is used in an IRC, with
the initial impetus being 1.0 kcal/mol.
\item[\comp{DRC KINETIC=5}] In a DRC, after the velocity is defined, 5 kcal
of kinetic energy is added in the direction of
the initial velocity.
\item[\comp{IRC=1 DRC KINETIC=4}] After starting with a 4 kcal impetus in the
direction of the first normal mode, energy is
conserved.
\item[\comp{DRC VELOCITY KINETIC=10}] Follow a DRC trajectory which starts with an
initial velocity read in, normalized to a
kinetic energy of 10 kcal/mol.
\end{description}


Instead of every point being printed, the option  exists  to  print specific
points  determined  by the keywords \comp{T-PRIORITY}, \comp{X-PRIORITY} and
\comp{H-PRIORITY}.  If any one of these words is specified, then the calculated
points  are used to define quadratics in time for all variables normally
printed.  In addition, if the flag for the first atom is set to  ``T''  then
all  kinetic  energy  turning  points  are printed.  If the flag for any other
internal coordinate is set to ``T'' then, when that coordinate  passes through
an extremum, that point will be printed.  As with the PRIORITY's, the point
will be calculated via  a  quadratic  to  minimize  non-linear errors.

N.B.:  Quadratics are unstable in the regions of inflection points; in  these
circumstances linear interpolation will be used.  A result of this is that
points printed in the  region  of  an  inflection  may  not correspond  exactly
to those requested.  This is not an error and should not affect the quality of
the results.

\subsection*{Test of DRC---verification of trajectory path}
Introduction:  Unlike  a  single-geometry  calculation  or  even  a geometry
optimization, verification of a DRC trajectory is not a simple task.  In this
section  a  rigorous  proof  of  the  DRC  trajectory  is presented;  it  can
be used both as a test of the DRC algorithm and as a teaching exercise.  Users
of the DRC are asked to  follow  through  this proof in order to convince
themselves that the DRC works as it should.

\subsection*{The nitrogen molecule}
For the nitrogen molecule (using MNDO) the equilibrium  distance is  $1.103816$
\AA, the heat of formation is 8.25741 kcal/mol and the vibrational frequency is
$2738.8$ cm$^{-1}$.   For  small  displacements, the  energy curve versus
distance is parabolic and the gradient curve is approximately linear, as is
shown in Table~\ref{n2}.         A  nitrogen molecule is thus a good
approximation to a harmonic oscillator.

% 40 lines, including this line
\begin{table}
\caption{\label{n2}Stretching Curve for Nitrogen Molecule}
\begin{center}
\begin{tabular}{rrr}
\multicolumn{1}{c}{N--N DIST} & \multicolumn{1}{c}{$\Delta H_f$}  & \multicolumn{1}{c}{GRADIENT}\\
\multicolumn{1}{c}{(\AA ngstroms)} &\multicolumn{1}{c}{(kcal/mol)} & \multicolumn{1}{c}{(kcal/mol/\AA ngstrom)}\\
\hline
1.11800   &  8.69441  &   60.84599 \\
1.11700   &  8.63563  &   56.70706 \\
1.11600   &  8.58100  &   52.54555 \\
1.11500   &  8.53054  &   48.36138 \\
1.11400   &  8.48428  &   44.15447 \\
1.11300   &  8.44224  &   39.92475 \\
1.11200   &  8.40444  &   35.67214 \\
1.11100   &  8.37091  &   31.39656 \\
1.11000   &  8.34166  &   27.09794 \\
1.10900   &  8.31672  &   22.77620 \\
1.10800   &  8.29611  &   18.43125 \\
1.10700   &  8.27986  &   14.06303 \\
1.10600   &  8.26799  &    9.67146 \\
1.10500   &  8.26053  &    5.25645 \\
1.10400   &  8.25749  &    0.81794 \\
1.10300   &  8.25890  &   -3.64427 \\
1.10200   &  8.26479  &   -8.12993 \\
1.10100   &  8.27517  &  -12.63945 \\
1.10000   &  8.29007  &  -17.17278 \\
1.09900   &  8.30952  &  -21.73002 \\
1.09800   &  8.33354  &  -26.31123 \\
1.09700   &  8.36215  &  -30.91650 \\
1.09600   &  8.39538  &  -35.54591 \\
1.09500   &  8.43325  &  -40.19953 \\
1.09400   &  8.47579  &  -44.87745 \\
1.09300   &  8.52301  &  -49.57974 \\
1.09200   &  8.57496  &  -54.30648 \\
1.09100   &  8.63164  &  -59.05775 \\
1.09000   &  8.69308  &  -63.83363 \\
\end{tabular}
\end{center}
\end{table}

\subsubsection{Period of vibration}
The period of vibration (time taken for the oscillator to undertake one
complete vibration, returning to its original position and velocity) can be
calculated in three ways.  Most direct is  the  calculation  from the  energy
curve; using the gradient constitutes a faster, albeit less direct, method,
while calculating it from the vibrational  frequency  is very  fast  but
assumes  that the vibrational spectrum has already been calculated.

\begin{enumerate}

\item From the energy curve. For a simple harmonic oscillator the period $r$ is
given by: $$ r = 2 \pi \sqrt{\frac{\mu}{k}}  $$ where $k$ is the
force constant.  \index{Reduced mass}\index{Force constant} The
reduced  mass, $\mu$,  (in amu)   of   a   nitrogen  molecule  is
$14.0067/2  =  7.00335$, and  the force-constant, $k$, can be
calculated from: $$E-c = (1/2) k(R-R_o)^2. $$ Given $R_o =
1.1038$, $R = 1.092$, $c = 8.25741$ and $E = 8.57496$~kcal/mol
then:
 \begin{eqnarray*}
k &=& 2*0.31755/(0.0118)^2 \; \mbox{(per mole)}\\
k &=& 4561.2 \mbox{ kcal/mol/A$^2$  (per mole)}\\
k &=& 1.9084*10^{30} \; \mbox{ ergs/cm$^2$ (per mole)}\\
k &=& 31.69*10^5  \; \mbox{ dynes/cm (per molecule)}\\
\end{eqnarray*}

(Experimentally, for N$_2$, k = $23*10^5$ dynes/cm )

Therefore:
$$ r = 2 \times 3.14159 \times \sqrt{\frac{7.0035}{1.9084\times 10^{30}}}
\;{\rm seconds} = 12.037 \times 10^{-15}\;{\rm s} = 12.037\;{\rm fs}. $$
If the frequency is calculated using the other half of the curve ($R=1.118,
E=8.69441$), then $k=12.333$ fs, or $k$, average, = 12.185 fs.

\item From the gradient curve. The force  constant  is  the  derivative  of
the  gradient  wrt distance:
$$ k = \frac{dG}{dx}. $$
Since we are using discrete points,  the  force  constant  is  best
obtained from finite differences:
$$ k = \frac{(G_2-G_1)}{(x_2-x_1)}. $$
For $x_2 = 1.1100$, $G_2 = 27.098$ and for $x_1 =  1.0980$,
$G_1  =  -26.311$,
giving rise to $k = 4450.75$ kcal/mol/\AA$^2$ and a period of $12.185$~fs.

\item From the vibrational frequency. Given a ``frequency'' (wavenumber) of
vibration of N$_2$ of $\bar{\nu}=2738.8$   cm$^{-1}$,  the period of
oscillation, in seconds, is given directly by:
$$ r = \frac{1}{c\bar{\nu}} = \frac{1}{2738.8 \times 2.998 \times 10^{10}} ,$$
or as $12.179$ fs.
\end{enumerate}

Summarizing, by three different methods the period  of  oscillation of N$_2$
is calculated to be $12.1851$, $12.185$ and $12.179$~fs, average $12.183$~fs.

\subsubsection{Initial dynamics of \mbox{N$_{2}$} with N--N distance = 1.094 \AA}
A useful check on the dynamics of N$_2$ is to  calculate  the  initial
acceleration  of  the  two  nitrogen  atoms  after releasing them from a
starting interatomic separation of 1.094 \AA.

At R(N-N) = 1.094 \AA, $G = -44.877$~kcal/mol/\AA\ or $-18.777 \times
10^{19}$~erg/cm. Therefore acceleration, $f = -18.777 \times 10^{19}
/14.0067$~cm/sec/sec, or $-13.405 \times 10^{18}$~cm/s$^2$, which is $ -13.405
\times 10^{15} \times$ Earth surface gravity.

Distance from equilibrium  $= 0.00980$ \AA. After $0.1$ fs, velocity is
$0.1\times  10^{-15} (-13.405 \times  10^{18})$ cm/sec or $1340.5$ cm/s.

In the  DRC  the  time-interval  between  points  calculated  is  a complicated
function of the curvature of the local surface.  By default, the first
time-interval is 0.105fs, so the calculated velocity  at  this time should be
$0.105/0.100 \times 1340.5 = 1407.6$ cm/s, in the DRC calculation the predicted
velocity is $1407.6$ cm/s.

The option is provided to allow sampling of the system at  constant
time-intervals,  the  default being $0.1$~fs.  For the first few points the
calculated velocities are given in Table~\ref{tdrc}.

\begin{table}
\caption{\label{tdrc} Velocities in DRC for N$_2$ Molecule}
\begin{center}
\begin{tabular}{rrrr}\\ \hline
      Time  & Calculated &  Linear &    Diff. in \\
            & Velocity  & Velocity &  Velocity \\ \hline
      0.000 &      0.0  &    0.0   &    0.0  \\
      0.100 &   1340.6  & 1340.5   &   -0.1  \\
      0.200 &   2678.0  & 2681.0   &   -3.0  \\
      0.300 &   4007.0  & 4021.5   &  -14.5  \\
      0.400 &   5325.3  & 5362.0   &  -36.7  \\
      0.500 &   6628.4  & 6702.5   &  -74.1  \\
      0.600 &   7912.7  & 8043.0   & -130.3  \\ \hline
\end{tabular}
\end{center}
\end{table}

As the calculated velocity is  a  fourth-order  polynomial  of  the
acceleration,   and  the  acceleration,  its  first,  second  and  third
derivatives, are all changing, the predicted velocity rapidly becomes  a poor
guide to future velocities.

For simple harmonic motion the velocity at any time is given by:
$$ v = v_0 \sin(2\pi t/r). $$
By fitting the computed velocities to simple harmonic motion, a much better fit
is obtained (Table~\ref{tdrc2}).

\begin{table}
\caption{\label{tdrc2} Modified  Velocities in DRC for N$_2$ Molecule}
\begin{center}
\begin{tabular}{rrrr} \hline
 & Calculated & Simple Harmonic  &    Diff. \\
Time  & Velocity  & 25325.Sin(0.5296t) &\\ \hline
            &           &                   &    \\
     0.000  &     0.0   &       0.0         &    0.0  \\
     0.100  &  1340.6   &    1340.6         &    0.0  \\
     0.200  &  2678.0   &    2677.4         &   +0.6  \\
     0.300  &  4007.0   &    4006.7         &   +0.3  \\
     0.400  &  5325.3   &    5324.8         &   +0.5  \\
     0.500  &  6628.4   &    6628.0         &   +0.4  \\
     0.600  &  7912.7   &    7912.5         &    0.0  \\ \hline
\end{tabular}
\end{center}
\end{table}

The repeat-time required for this  motion  is  $11.86$~fs,  in  good agreement
with  the  three  values calculated using static models.  The repeat time
should not be calculated from the time required to go from a minimum  to  a
maximum and then back to a minimum---only half a cycle. For all real systems
the potential energy is a skewed parabola, so  that the  potential energy
slopes are different for both sides; a compression (as in this case) normally
leads to a higher force-constant, and shorter apparent  repeat  time  (as in
this case).  Only the addition of the two half-cycles is meaningful.

\subsubsection{Conservation of normal coordinate}
So far this analysis has only considered a homonuclear diatomic.  A detailed
analysis  of  a  large  polyatomic  is  impractical,  and  for simplicity a
molecule of formaldehyde will be studied.

In polyatomics, energy can  transfer  between  modes.   This  is  a result  of
the non-parabolic nature of the potential surface.  For small displacements the
surface can be considered as  parabolic.   This  means that  for small
displacements interconversion between modes should occur only very slowly.  Of
the six normal modes, mode 1, at 1209.5~cm$^{-1}$, the in-plane C--H asymmetric
bend, is the most unsymmetric vibration, and is chosen to demonstrate
conservation of vibrational purity.

Mode 1 has a  frequency  corresponding  to  3.46  kcal/mol  and  a predicted
vibrational time of $27.58$~fs.  By direct calculation, using the DRC, the
cycle time is $27.59$~fs.  The rate of decay of this mode  has  an estimated
half-life of a few thousands femtoseconds.

\subsubsection{Rate of decay of starting mode}
For trajectories initiated by an IRC=$n$  calculation,  whenever  the
potential  energy is a minimum the current velocity is compared with the
supplied velocity.  The square of the cosine of the  angle  between  the two
velocity vectors is a measure of the intensity of the original mode in the
current vibration.

\subsubsection{Half-Life for decay of initial mode}
Vibrational purity is assumed to decay according to  zero'th  order kinetics.
The  half-life is thus $-0.6931472t/\log(<\!\psi^2\!>^2)$~fs, where
$<\!\psi^2\!>^2$ is the square of  the overlap integral of the wavefunction for
the  original vibration with that of the current  vibration.   Due to the  very
slow rate of decay of the starting mode, several half-life calculations
should  be  examined.   Only  when successive  half-lives  are  similar  should
any confidence be placed in their value.

\subsubsection{DRC print options}
The amount of output in the DRC is  controlled  by  three  sets  of
options.  These sets are:\index{H--PRIORITY}
\begin{itemize}
\item Equivalent Keywords \comp{H-PRIORITY}, \comp{T-PRIORITY}, and
\comp{X-PRIORITY}.
\item Potential Energy Turning Point option.
\item Geometry Maxima Turning Point options.
\end{itemize}
If \comp{T-PRIORITY} is used then  turning  points  cannot  be  monitored.
%Currently  \comp{H-PRIORITY} and \comp{X-PRIORITY} are not implemented,
%but will be as soon as practical.

\index{``T'' - Optimization flag} To monitor geometry turning points, put  a
``T'' in  place  of  the geometry optimization flag for the relevant geometric
variable. In the example shown in Figure~\ref{t}, the geometry of formaldehyde
would first be optimized, then a \comp{FORCE} calculation run, then a DRC
calculation started, using the first normal mode for the starting velocity.
Whenever the C=O bond length becomes a maximum or a minimum, a message is
printed.

\begin{figure}
\begin{makeimage}
\end{makeimage}
\begin{verbatim}
 IRC=1 DRC T=20
 Formaldehyde
 Monitoring the C=O Bond-length turning points
  O    0.0 0    0.0 0    0.000000 0   0 0 0
  C    1.2 T    0.0 0    0.000000 0   1 0 0
  H    1.0 1  120.0 1    0.000000 0   2 1 0
  H    1.0 1  120.0 1  180.000000 0   2 1 3
  0    0.0 0    0.0 0    0.000000 0   0 0 0
\end{verbatim}
\caption{\label{t} Example of DRC calculation, monitoring a geometric variable}
\end{figure}

To monitor the potential energy turning points, put a ``T'' for  the flag for
atom 1 bond length (Do not forget to put in a bond-length (zero will do)!).

To monitor the geometry, use \comp{LARGE=$n$}.  This will cause the geometry to
be printed once every $n$ steps.

The effect of using these flags together is as follows.
\begin{enumerate}
\item No options:  All calculated points will be printed.  No turning points
will be calculated.

\item Atom 1 bond length flagged with a ``T'': If  \comp{T-PRIO},  etc.\ are
NOT  specified,  then  potential  energy turning points will be printed.

\item Internal coordinate flags set to ``T'':  If \comp{T-PRIO}, etc.  are NOT
specified,  then geometry extrema will be printed.  If only one coordinate is
flagged, then the turning point will be displayed in  chronologic  order; if
several are flagged then all turning points occurring in a given time-interval
will  be  printed  as they  are  detected.   In  other  words,  some  may  be
out of chronologic order.  Note that each coordinate flagged will give rise  to
a different geometry:  minimize flagged coordinates to minimize output.

\item Potential and geometric flags set:  The effect is equivalent to the sum
of the first two options.

\item \comp{T-PRIO} set:  No turning points will be  printed,  but  constant
time-slices  (by  default  $0.1$~fs)  will  be used to control the print.
\end{enumerate}
\index{DRC|)}\index{IRC|)}

\section{Use of \comp{SADDLE} Calculation}
\index{SADDLE!description|(}
The \comp{SADDLE} technique is used for locating a transition state, given two
geometries, one on each side of the transition state. In order for the
\comp{SADDLE} technique to work, the Z-matrix must be specified as follows:

\begin{itemize}
\item The first geometry, defining one geometry is defined as usual.   If
symmetry data is supplied, it should follow the first geometry.  After the
geometry, or geometry and symmetry data, there should be a blank line to
indicate the end of the data
\item The second geometry should then be specified. There must be a one-to-one
correspondence of the atoms in the second geometry to those of the first
geometry.
\end{itemize} 

From this specification, it follows that if two molecules react to form one
molecule, then the first geometry must contain all the atoms of the two
molecules. The easiest way of defining such a geometry is to define one
molecule, then have an unusually long bond-length from one atom in the first
molecule to the first atom in the second molecule.  The two molecules together
form the first geometry.  Likewise, if a molecule decomposes, e.g.\
C$_2$H$_5$OH $\rightarrow$ C$_2$H$_4$ + H$_2$O, every atom in the product must
be defined in the same order as the atoms in the reactant.

An example of a data-set for a \comp{SADDLE} calculation, modeling the
\index{Ethyl radical} ethyl radical hydrogen migration from one methyl group to
the other is given in Figure~\ref{c2h5s}. \index{Data! for ethyl radical
(SADDLE calcn.)}

\begin{figure}
\begin{makeimage}
\end{makeimage}
\begin{verbatim}
Line  1:   UHF  SADDLE
Line  2:   ETHYL RADICAL HYDROGEN MIGRATION
Line  3:
Line  4:    C   0.000000 0    0.000000 0    0.000000 0   0 0 0
Line  5:    C   1.479146 1    0.000000 0    0.000000 0   1 0 0
Line  6:    H   1.109475 1  111.328433 1    0.000000 0   2 1 0
Line  7:    H   1.109470 1  111.753160 1  120.288410 1   2 1 3
Line  8:    H   1.109843 1  110.103163 1  240.205278 1   2 1 3
Line  9:    H   1.082055 1  121.214083 1   38.110989 1   1 2 3
Line 10:    H   1.081797 1  121.521232 1  217.450268 1   1 2 3
Line 11:    O   0.000000 0    0.000000 0    0.000000 0   0 0 0
Line 12:    C   0.000000 0    0.000000 0    0.000000 0   0 0 0
Line 13:    C   1.479146 1    0.000000 0    0.000000 0   1 0 0
Line 14:    H   1.109475 1  111.328433 1    0.000000 0   2 1 0
Line 15:    H   1.109470 1  111.753160 1  120.288410 1   2 1 3
Line 16:    H   2.109843 1   30.103163 1  240.205278 1   2 1 3
Line 17:    H   1.082055 1  121.214083 1   38.110989 1   1 2 3
Line 18:    H   1.081797 1  121.521232 1  217.450268 1   1 2 3
Line 19:    O   0.000000 0    0.000000 0    0.000000 0   0 0 0
Line 20:
\end{verbatim}
\caption{\label{c2h5s} Example of data for \comp{SADDLE} calculation}
\end{figure}

Details of the mathematics of \comp{SADDLE} appeared  in  print  in  1984 (M.\
J.\ S.\ Dewar,  E.\ F.\ Healy,  J.\ J.\ P.\ Stewart,  {\em J.\ Chem.\ Soc.\ 
Faraday Trans.\ II}, 3, 227, (1984)), so only a  superficial  description 
will  be given here.

The main steps in the saddle calculation are as follows:
\begin{enumerate}
\item The heats of formation of both systems are calculated.

\item A vector $R$ of length $3N$ defining the difference  between  the two
geometries in Cartesian coordinates is calculated.  The scalar of the vector is
called the \comp{BAR}, and represents the distance between the two geometries.

\item  The \comp{BAR}  is  reduced  by  some fraction, normally about 5 to 15
percent. Normally, the default step is used, but this can be changed by use of
\comp{BAR=$n.nn$}, where $n.nn$ is the fraction.  \comp{BAR=0.15} is the
default.

\item  The geometry of lower energy is identified; call this $G$.

\item Geometry $G$ is optimized, subject to  the  constraint  that  it 
maintains  a constant distance $P$ from the other geometry.

\item If the newly-optimized geometry is higher in  energy  than  the other 
geometry,  and the last two steps involved the same geometry  moving,  make 
the  other geometry $G$ without modifying $P$, and go to 5.

\item Otherwise go back to 2.
\end{enumerate}

The mechanism of 5 involves the coordinates of the moving  geometry being 
perturbed  by  an  amount equal to the product of the discrepancy between the
calculated and required $P$ and the vector $R$.

\comp{SADDLE} works with Cartesian coordinates, so before the calculation
starts, the two geometries are superimposed as much as possible.  This is done
as follows:
\begin{enumerate}
\item Both geometries are converted into Cartesian coordinates.

\item Both geometries are centered about the origin of Cartesian space.

\item One geometry is  rotated  until  the  difference  vector  is  a minimum 
---  this  minimum  is  within  1 degree of the absolute bottom.

\item The \comp{SADDLE} calculation then proceeds as described above.
\end{enumerate}

The two geometries must be related by a continuous  deformation  of the  
coordinates. For this, internal  coordinates  are unsuitable in that while 
bond  lengths  and  bond  angles  are \index{Dihedral angles!ambiguities in|ff}
unambiguously  defined (being both positive), the dihedral angles can be either
positive or  negative.   Clearly  300  degrees  could  equally  well  be
specified  as  $-60$ degrees.  A wrong choice of dihedral would mean that
instead  of  the  desired  reaction  vector  being  used,  a  completely
incorrect vector was used, with disastrous results.

To prevent this, a \comp{SADDLE} calculation will always convert coordinates
into Cartesian before starting the run.  If symmetry is to be used, then the
geometry must be supplied in Cartesian coordinates, because internal symmetry
relations are not meaningful here.

\subsection{How to escape from a hilltop}
\index{Hilltop, escaping from}
\index{SIGMA}\index{NLLSQ}\index{TS}
A  particularly  irritating  phenomenon  sometimes  occurs  when  a transition 
state is being refined.  A rough estimate of the geometry of the transition
state has been obtained by either a  \comp{SADDLE}  or  reaction path or by
good guesswork.  This geometry is then  refined by \comp{TS}, \comp{SIGMA} or
by \comp{NLLSQ}, and the system characterized by a force calculation.  Remember
that \comp{NLLSQ} is preferred over \comp{SIGMA} when the GNORM is large, so
\comp{NLLSQ} is probably the method of choice, if for any reason \comp{TS} does
not work.    It  is  at this  point  that  things  often go wrong.  Instead of
only one negative force constant, two or more are found.  In the past, the 
recommendation has been to abandon the work and to go on to something less
masochistic. It is possible, however, to  systematically  progress  from  a 
multiple maximum to the desired transition state.  The technique used will now
be described.

If a multiple maximum is identified, most likely one negative force constant 
corresponds  to  the  reaction  coordinate,  in which case the objective  is 
to  render  the  other  force  constants  positive.   The associated  normal 
mode  eigenvalues are complex, but in the output are printed as negative
frequencies, and for the sake of simplicity will  be
\index{Frequencies!imaginary}\index{DRAW} described  as  negative  vibrations. 
Use a graphical user interface program to display the negative vibrations, 
and  identify  which  mode  corresponds  to  the   reaction coordinate.  This
is the one we need to retain.

Hitherto, simple motion in the direction of  the  other  modes  has \index{DRC}
proved  difficult.   However the DRC provides a convenient mechanism for
automatically following a normal coordinate.  Pick the  largest  of  the
negative  modes to be annihilated, and run the DRC along that mode until a
minimum is reached.  At that point,  refine  the  geometry  once  more using 
\comp{TS}  and  repeat  the  procedure  until  only one negative mode exists.

To be on the safe  side,  after  each  \comp{DRC}+\comp{TS}   sequence  do  the
\comp{DRC}+\comp{TS}  operation  again,  but use the  negative of the initial
normal coordinate to start the trajectory.  After both  stationary  points  are
reached,  choose  the  lower  point  as  the starting point for the next
elimination.  The lower point is chosen  because  the  transition  state
wanted  is  the  highest  point  on  the  lowest  energy path connecting
reactants to products.  Sometimes the two points will have equal energy: this 
is normally a consequence of both trajectories leading to the same point or
symmetry equivalent points.

After  all  spurious  negative  modes  have  been  eliminated,  the remaining 
normal  mode  corresponds to the reaction coordinate, and the transition state
has been located.

This technique is relatively rapid, and relies on starting  from  a stationary 
point to begin each trajectory.  If any other point is used, the trajectory
will  not  be  even  roughly  simple  harmonic.   If,  by mistake,  the
reaction coordinate is selected, then the potential energy will  drop  to 
that  of  either  the  reactants  or  products,   which, incidentally,  forms a
handy criterion for selecting the spurious modes: if the potential energy only
drops by  a  small  amount,  and  the  time evolution  is  roughly  simple 
harmonic,  then  the  mode is one of the spurious modes.  If there is any doubt
as to whether a minimum is in the vicinity  of  a stationary point, allow the
trajectory to continue until one complete cycle is executed.  At that point
the  geometry  should  be near to the initial geometry.

Superficially, a line-search might appear more attractive than  the relatively 
expensive  DRC.   However,  a line-search in Cartesian space will normally not
locate the minimum in a mode.  An obvious  example  is the mode corresponding
to a methyl rotation.

\subsection*{Keyword Sequences to be Used}
\begin{enumerate}
\item To locate the starting stationary point  given  an  approximate
transition state: \comp{TS}

\item To define the normal modes: \comp{FORCE ISOTOPE}

At this point, copy all the files to a second filename, for use later.

\item Given vibrational frequencies of $-654$, $-123$,  234,  and  456,
identify  {\em via} a GUI the normal coordinate mode,  say the  $-654$ mode. 
Eliminate the second mode by:

\comp{IRC=2 DRC T=30M RESTART LARGE}

Use is made of the FORCE restart file.

\item Identify  the  minimum  in  the  potential  energy  surface  by
inspection or using the ``grep'' command, of form:

\comp{grep '\%'  $<$Filename$>$.out }

\item Edit out of the output file the data file corresponding to  the lowest
point, and refine the geometry using: \comp{TS}

\item Repeat the last three steps but for the negative of the  normal mode, 
using  the  copied files.  The keywords for the first of the two jobs are:
\comp{IRC=--2 DRC T=30M RESTART LARGE}


\item Repeat the last four steps  as  often  as  there  are  spurious
modes.

\item Finally, carry out a DRC to confirm that the  transition  state does, in
fact, connect the reactants and products.  The drop in potential energy 
should  be  monotonic.   If  you  are  unsure whether  this  last  operation
will work successfully, do it at any time you have a stationary point.  If it
fails at the  very start,  then we are back where we were before---give up
and go home!!
\end{enumerate}

\index{SADDLE!description|)}

\section{Polarizability and Hyperpolarizability Calculation}\label{t_polar}
\index{Polarizability|ff} \index{Non-Linear Optics}
\index{Hyperpolarizability}
\index{Dipole!induced}
\index{Applied electric fields}\index{Electric fields!applied}
\index{Kurtz@{\bf Kurtz, Henry, A.}}
%
%Finite Field Procedure:
%
%By default, an electric field gradient of 0.001 is used. This can
%be modified by specifying POLAR=n.nnnnn, where n.nnnnn is the new field.
%
%POLAR calculates the polarizabilities from the heat of formation and
%from the dipole. The degree to which they agree is a measure of the
%precision (not the accuracy) of the calculation. The results from the
%heat of formation calculation are more trustworthy than those from the
%dipole.
%
%Users should note that the hyperpolarizabilities obtained have to be
%divided by 2.0 for beta and 6.0 for gamma to conform with experimental
%convention.
%
%Two sets of results are printed: a set (labeled E4) derived from
%the effect of the applied electric field on the heat of formation, and a
%set (labeled DIP) derived from the value of the dipole in various
%electric fields.
%
This section was written based on material provided by:
\begin{center} Henry Kurtz and Prakashan Korambath\\
Department of Chemistry\\Memphis State University\\
Memphis TN 38152.
\end{center}

\subsection{Time-Dependent Hartree-Fock}
This procedure is based on the detailed description given by
M.\ Dupuis and S.\ Karna (J.\ Comp.\ Chem.\ 12, 487 (1991)). The
program is capable of calculating the quantities shown in Table~\ref{nlo}.
\begin{table}
\index{Pockels effect}\index{Electro-optic Pockels effect}
\index{Optical!rectification}
\index{Third Harmonic Generation}
\index{Harmonic, third, generation}
\index{DC-EFISH}
\index{Kerr effect}
\index{Optical!rectification}
\index{Refractive index}
\index{Frequency-dependent NLO}
\caption{\label{nlo} Quantities Calculable using \comp{POLAR}}
\begin{center}
\begin{tabular}{ll}\hline
\multicolumn{1}{c}{Type of Phenomenon} & \multicolumn{1}{c}{Symbol} \\ \hline
Frequency Dependent Polarizability   &   $\alpha(-\omega;\omega)$ \\
Second Harmonic Generation           &   $\beta(-2\omega;\omega,\omega)$ \\
Electrooptic Pockels Effect          &   $\beta(-\omega;0,\omega)$ \\
Optical Rectification                &   $\beta(0;-\omega,\omega)$ \\
Third Harmonic Generation            &   $\gamma(-3\omega;\omega,\omega,\omega)$  \\
DC-EFISH                             &   $\gamma(-2\omega;0,\omega,\omega)$  \\
Optical Kerr Effect                  &  $\gamma(-\omega;0,0,\omega)$  \\
Intensity Dependent Index of Refraction & $\gamma(-\omega;\omega,-\omega,\omega)$ \\
\hline
\end{tabular}
\end{center}
\end{table}

Keywords for the \comp{POLAR} calculation are given inside the \comp{POLAR}
keyword.  Quantities under user-control are:
\begin{description}
\item[\comp{IWFLB=$n$}] The type of $\beta$ calculation to be performed.  
This variable is only important if iterative beta calculations are chosen.
\begin{description}
\item[\comp{IWFLB=0}] static (This is the default)
\item[\comp{IWFLB=1}] second harmonic generation
\item[\comp{IWFLB=2}] electrooptic Pockels effect
\item[\comp{IWFLB=3}] optical rectification
\end{description}
\item[\comp{E=($n_1, n_2, n_3, \ldots$)}] The energies, in eV, of 
the radiation to be used. Up to 10 energies can be specified.
If this option is not used, the default energies of 0.0, 0.25, and 0.50 eV will
be used.

\item[\comp{BETA=$n$}] Type of beta calculation.
\begin{description}   
\item[\comp{BETA=0}] $\beta$(0;0) static (This is the default)
\item[\comp{BETA=1}] iterative calculation with type of $\beta$ chosen by \comp{IWFLB}
\item[\comp{BETA=-1}] Noniterative calculation of second harmonic generation
\item[\comp{BETA=-2}] Noniterative calculation of electrooptic Pockels effect
\item[\comp{BETA=-3}] Noniterative calculation of optical rectification
\end{description}
\item[\comp{GAMMA=$n$}] Type of gamma calculation:
\begin{description}
\item[\comp{GAMMA=0}] No gamma calculation
\item[\comp{GAMMA=1}] third harmonic generation (This is the default)
\item[\comp{GAMMA=2}] DC-EFISH
\item[\comp{GAMMA=3}] intensity dependent index of refraction
\item[\comp{GAMMA=4}] optical Kerr effect
\end{description}
\item[\comp{TOL=$n.nn$}] Cutoff tolerance for $\alpha$ calculations, 
default=0.001.
\item[\comp{MAXITU=$nnn$}] Maximum number of interactions for beta, default:
500.
\item[\comp{MAXITA=$nnn$}] Maximum number of iterations for $\alpha$
calculations, default: 150.
\item[\comp{BTOL=$n.nn$}]  Cutoff tolerance for $\beta$ calculations
The default is 0.001.
\end{description}

\subsubsection*{Examples of \comp{POLAR} keyword}

To calculate the NLO quantities $\alpha, \beta,$ and $\gamma$ at 1.0eV:\\
  \comp{POLAR(E=(1.0)) }\\
This same calculation can be set up by setting all the variables to their
default value:
\begin{verbatim}
POLAR(IWFLB=0,E=(1.),BETA=0,GAMMA=1,TOL=1.D-3,MAXITU=500,MAXITA=150,BTOL=1.D-3)
\end{verbatim}

This takes up the entire keyword line.  If more than one line is needed to hold
the keyword, use the \comp{+} option, as in:
\begin{verbatim}
+ symmetry 1scf uhf POLAR(IWFLB=0,E=(1.),BETA=0,GAMMA=1,TOL=1.D-3,MAXITU=501,
 MAXITA=151,BTOL=1.D-3)
\end{verbatim}

Note: This is not a recommended way of writing a keyword.  In order for a
keyword to be recognized, the `join' of the two lines must be perfect.  In
other words, the last character of the first line must be in column 80, unless
character 1 was not blank, in which case the last character must be in column
79. Anyhow, it is unlikely that such long keywords would be used very often.




\section{COSMO (Conductor-like Screening Model)}\index{COSMO|ff}
\index{Klamt@{\bf Klamt, Andreas}}
\index{Solvation model!COSMO}
Based on materials provided by
\begin{center}
Dr Andreas Klamt \\
Burscheider Strasse 524 \\
D-51381 Leverkusen \\
Germany \\
\end{center}

Unlike the Self-Consistent Reaction Field model~\cite{scrf}, the  {\bf
Co}nductor-like {\bf S}creening {\bf Mo}del (COSMO) is a  continuum
approach~\cite{cosmo} which, while more  complicated, is computationally quite
efficient.  The expression for the total screening energy is simple enough to
allow the first derivatives of the energy with respect to atomic  coordinates
to be easily evaluated.

The COSMO procedure generates a conducting polygonal surface around the system
(ion or molecule), at the van der Waals' distance.  By introducing a
$\varepsilon$-dependent correction factor,
$$
f(\varepsilon)=\frac{(\varepsilon-1)}{(\varepsilon+\frac{1}{2})}, 
$$
into the expressions for the screening energy and its gradient, the theory can
be extended to finite dielectric constants with only a small error.

The accuracy of the method can be judged by how well it reproduces known
quantities, such as the heat of solution in water (water has a dielectric
constant of 78.4 at 25$^{\circ}$C), Table~\ref{cosmo_tab}.   Here, the keywords
used were:

\comp{NSPA=60 GRADIENTS 1SCF  EPS=78.4 AM1 CHARGE=1}

From the Table we see that the glycine zwitterion becomes the stable form in
water, while the neutral species is the stable gas-phase form.

(After the COSMO paper was published, improvements in the method made the
results shown in Table~\ref{cosmo_tab} invalid.  However, the general
conclusion that the method is of useful accuracy is still true.)

The COSMO method is easy to use, and the derivative calculation is of
sufficient precision to allow gradients of 0.1 to be readily achieved.

\begin{table}
\caption{\label{cosmo_tab} Calculated and Observed Hydration Energies}
\begin{center}
\begin{tabular}{llrrrr}
\hline
Compound   &   Method  &  \multicolumn{2}{c}{$\Delta H_f$ (kcal/mol)} & \multicolumn{2}{c}{Hydration} \\
           &           & gas phase & solution phase & $\Delta H$(calc.)  & Enthalpy(exp.)~\dag  \\
\hline
NH$_4^+$ & AM1 & 150.6 & 59.5 & 91.1 & 88.0 \\
N(Me)$_4^+$  & AM1 & 157.1 & 101.1 & 56.0 & 59.9 \\
N(Et)$_4^+$  & AM1 & 132.1 &  84.2 & 47.9 & 57.0 \\
Glycine \\
neutral & AM1 & -101.6 & -117.3 & 15.7 & $--$ \\
zwitterion & AM1 & -59.2 & -125.6 & 66.4 & $--$ \\
\hline
\end{tabular} \\
\dag : Y. Nagano, M. Sakiyama, T. Fujiwara, Y. Kondo, J. Phys. Chem., {\bf 92},
5823 (1988).
\end{center}
\end{table}

\subsection{A Walk Through COSMO}
To explain the COSMO program, it is best to do a walk through the program first.
%(FORTRAN symbols will be written in {\bf BOLD}, and algebraic quantities
%written in $math$ font.)

\subsection*{\comp{INITSV} (INITialize SolVation)}
This subroutine is called by the main program, if the \comp{EPS=$nn.n$} keyword
is set.  Here, all initializations for the COSMO calculation are done.

\begin{itemize}
\item In the \comp{DATA} statement, the van der Waals (VDW) radii are set.
\item The dielectric constant $\epsilon$ (\comp{EPSI}) is read in from 
\comp{KEYWRD}, and transformed to the dielectric factor \comp{FEPSI} = 
($\epsilon$-1)/($\epsilon$+0.5).
\item The number of interatomic density matrix elements \comp{NDEN} is 
calculated from \comp{NORBS} and \comp{NUMAT}:

\comp{NDEN=3$\times$NORBS-2$\times$NUMAT}

\item The solvent radius, \comp{RSOLV}, for the construction of the SAS is
read in off  \comp{RSOLV=$n.nn$} (default: 1.0 \AA ngstroms).
\item \comp{DISEX} is set (default = 2).  \comp{DISEX} controls the  distance
within which  the interaction of two segments is calculated accurately using
the basic grid.
\item The solvation radius \comp{SRAD} is set for each atom.  Be careful:  the
distance of the \comp{SAS} will be \comp{SRAD-RDS}.
\item \comp{NSPA} (Number of Segments Per Atom) is set (default = 42).
\comp{NSPA} is the number of segments created on a full VDW sphere.
\item To guarantee the most homogeneous initial segment distribution, it is
best to create one of the ``magic numbers'' (see \comp{DVFILL}) of segments.
Therefore the next smallest ``magic number'' compared to \comp{NSPA}  is
evaluated and the corresponding set of direction vectors is stored in
\comp{DIRSM}.

For hydrogens the number of segments can safely be reduced by a factor of three
or four, since the potential is rather homogeneous on hydrogen spheres. This is
done and the corresponding direction vectors are stored in \comp{DIRSM}.

\item By the equation:
$$
\comp{NSPA}\times \pi\times \comp{RSEG}^2 = 4\pi R^2
$$
the mean segment radius can easily be found to be 
$$
\comp{RSEG}=2R/\sqrt{\comp{NSPA}}.
$$

With a mean VDW-radius of 1.5 \AA ngstroms, the mean radius is 
$R=1.5+\comp{RSOLV}-\comp{RDS}$.
Thus \comp{DISEX2} is set to the squared mean distance of 
two segments multiplied by
\comp{DISEX}.
\end{itemize}

\subsection*{\comp{DVFILL} (Direction Vector FILLing)}
This routine constructs  a homogeneous set of points on a unit sphere
(direction vectors).  It starts with  12 face-centers on a regular
dodecahedron, i.e.\ the 12 corners of a regular icosahedron. These form 20
regular triangles.  If the centers of these 20 triangles are added to the 12
initial directions, a new set of triangles is created, which, by a Wigner-Seitz
construction on the unit sphere corresponds to 32 faces, 12 pentagons, and 20
hexagons.  Thus we have constructed a soccerball.  By iterative addition of the
triangle centers, any number $N$ of homogeneously distributed points can be
generated, which can be written in the form $N=3^i\times 10+2$,  with $i$ being
an integer ($N=12, 32, 92, 272, \ldots$).

But there is a second way of generating a finer mesh of regular triangles from
a cruder one.  Instead of additional centers, the midpoints of the edges can be
added. This roughly corresponds to an increase in the number of directions by a
factor  of four, instead of three.  By combining these two procedures, a number
$N=3^i\times 4^j\times 10+2$ of directions can be created.  These  are the
allowed values of \comp{NPPA} in the \comp{DVFILL} routine (``magic numbers'').
The smallest are 12, 32, 42, 92, 122, 162, \ldots . The default for the
construction of the fine grid is $1082 = 3^3\times 4\times 10+2$.

\subsection*{\comp{COSCAV/COSCAN} (COSmo CAVity)}
This routine constructs the SAS and calculates and inverts the surface charge
interaction matrix $A$. 

Then the new transformation matrix is built from the local coordinate system 
defined by the nearest neighbors.  The transformation to a local coordinate 
system guarantees a higher stability of the segmentation during geometry
optimization.  Then the basic grid of direction vectors is transformed 
according to the new transformation matrix.

\subsection*{\comp{MKBMAT}}
This routine calculates the $B$-matrix, i.e., the density-surface charge
interaction matrix.
\begin{itemize}
\item First the array \comp{COSURF} is filled with the explicit segment center
coordinates instead of directions only.

\item Then in a loop over all atoms, all segments for all density matrix
elements of each atom the Coulomb interaction of the density with the segment
charge is calculated.  With $r$ being the atom-segment distance vector, the
various elements are:
\begin{center}
\begin{tabular}{ll}\hline
$ss$ & $r^{-1}$ \\
$sp_k$ & $\frac{DD(i)r_k}{r^3}$ \\
$p_kp_k$ & $r^{-1}+QQ(i)^2(\frac{3r_k^2}{r^5}-r^{-3})$ \\
$p_kp_l$ & $6QQ(i)^2\frac{r_kr_l}{r^5}$ \\ 
\hline
\end{tabular}
\end{center}
  
Here, $DD(i)$ and $QQ(i)$ are the atomic dipole and quadrupole lengths, 
respectively, as coded in MOPAC.
\item The $B$-matrix is stored in \comp{BMAT}.
\end{itemize}


\subsection*{\comp{DIEGRD} (DIElectric GRaDient)}
The dielectric part of the gradient is calculated and added to \comp{DXYZ}.
There are two dielectric gradient contributions:
$$
d\ E_{die}/d\ R_k^{\alpha} = Q(\frac{d}{d\ R_k^{\alpha}}B)G -
 \frac{1}{2}G(\frac{d}{d\ R_k^{\alpha}}A)G
$$
with $G=A^{-1}BQ$ being the vector of charges on the segments.
\begin{itemize}
\item First the \comp{COSURF} transformation is done.
\item Then \comp{Q} and \comp{QS} are calculated. (If you are interested in 
a visualization of the screening charges, you may write out \comp{QS} and 
\comp{COSURF} at this point.)
\item Then the `second part' of the gradient is calculated.
\item In the next part the first part is calculated.  Here the gradient of the
density-segment-charge interactions have to be evaluated.

\item Finally, \comp{COSURF} is re-transformed.
\end{itemize}

\subsection{COSMO Keywords}
\index{Dielectric energy}
\begin{description}
\item[\comp{EPS}=$n.n$] Defines the dielectric constant of the solvent.   This
keyword triggers the whole COSMO.
\item[\comp{NSPA}=$nn$] Controls the number of segments, default = 42.
\item[\comp{DISEX}=$n.n$] Controls the radius, up to which the segment-segment
interactions are evaluated on the basis of the basic grid points.  Default =
2.0.  For accurate calculations or very high  dielectric
energies\footnote{\samepage The dielectric energy is the energy of
stabilization arising from the  interaction of the charges in the solute with
the induced charges on the  solvent accessible surface plus the electrostatic
energy due to the charges on the SAS interacting with each other.}, (e.g.\
ions) larger values may be preferable.  The calculation time may increase  as  
\comp{DISEX$^2$}  (until all interactions are calculated accurately).
\item[\comp{RSOLV}=$n.n$] Effective VDW radius of the solvent molecule. 
Default = 1.0\AA .
\end{description}
\subsection{Solvent Accessible Surface}\index{SAS|ff}
\index{Solvent Accessible Surface|ff}
The solvent accessible surface is a continuous surface of the molecule which
can be reached by the center of charge of a solvent molecule.  The calculation 
of the SAS is carried out as follows:

\begin{itemize} 
\item Each atom is assigned a van der Waals' radius.  VdW radii used in COSMO
are given in Table~\ref{vdw}.
\begin{table}
\caption{\label{vdw} Van der Waals radii (\AA ) used in COSMO}
\begin{center}
\begin{tabular}{llllllllllllll}
\hline
 I  & R    & II & R    &III& R    &IV & R    & V & R    &VI & R    &VII& R \\
\hline
 H  & 1.08 & \\
 Li & 1.80 &    &      &   &      & C & 1.53 & N & 1.48 & O & 1.36 & F & 1.30\\
 Na & 2.30 &    &      & Al& 2.05 & Si& 2.10 & P & 1.75 & S & 1.70 & Cl& 1.65\\
 K  & 2.80 & Ca & 2.75 &   &      &   &      &   &      &   &      &Br & 1.80\\
    &      &    &      &   &      &   &      &   &      &   &      &I  &  2.05\\
\hline
\end{tabular}
\end{center}
\end{table}

\item To each radius is added a distance equal to the radius of the solvent. By
default, this is 1.0\AA, but may be changed by the user using 
\comp{RSOLV=$n.nn$}. This gives the distance from the nucleus to the center of
a solvent molecule.
\item A set of points is generated on this surface. These points produce a
basic grid.
\item All points which are inside the surface of any other atom are excluded.
\item The remaining points are moved towards the center of the atom.  The
distance moved is equal to the distance of the center of charge of the solvent
molecule from the center of the solvent molecule.  By default, this distance is
set to \comp{RSOLV}, but may be set explicitly by keyword \comp{RSOLV=$n.nn$}.
\item Each of the remaining points represents a small area of the solvent
accessible surface.  The total SAS is calculated from the number of points.
\end{itemize}

From this definition of the SAS we see that the SAS of each atom is a  surface
of radius equal to the van der Waals' radius plus the radius of the solvent
molecule minus the distance of the center of charge of the solvent molecule to
the center of the solvent molecule.  In other words, the radius is the VdW
radius plus the distance from the surface of the solvent molecule to the center
of charge of the solvent molecule.  By default, this extra distance is zero.
Only that part of the atom surface which can be touched by the solvent molecule
is used.  This means that only those atoms on the surface of the molecule can
contribute to the SAS.  Of those atoms  that are on the surface of the molecule
there will be parts of the surface which cannot be reached by the solvent
because the solvent molecule is too bulky.


\subsubsection{Some hints on the use of COSMO}
\begin{itemize}
\item \comp{1SCF} calculations run in general without problems. On  gas-phase
geometries they give useful solvation energies for  neutral rigid molecules.
\item For geometry optimization Eigenvector Following has proved to be most
efficient in combination with COSMO. Gradient norms of about 1\% of the
dielectric energy should be reachable, in many cases even less. Nevertheless,
don't use a too small \comp{GNORM} criterion, since the calculation may have
convergence problems.
\item Keep in mind that energy differences of about 1\% of the dielectric
energy may arise due to small differences in the segmentatation.
\item Dr Klamt does not recommend the use of COSMO in \comp{FORCE} calculations
at the present time.
\item \comp{UHF} calculations should run without additional problems.
\item \comp{C.I.} calculations can now be done, and C.I.\ gradients are now 
valid; this has been the result of recent work by Dr Klamt.
\end{itemize}

\section{Parametric Molecular Electrostatic Potential (PMEP)}\index{PMEP|ff}
\index{Ford@{\bf Ford, George}}
\index{Wang@{\bf Wang, Bingze}}
\index{Electrostatic Potential!PMEP model}
The PMEP procedure~\cite{pmep1,pmep2} is a technique for rapidly calculating 
the electrostatic properties of a molecule.  Written by Prof.\ George Ford and
Dr.\ Bingze  Wang\footnote{Current address: IRBM, Via Pontina Km.30.600, 00040
Pomezia (Roma), Italy} at Southern Methodist University, Dallas, Texas, the
procedure is ideally suited for large systems.  

The PMEP procedure has two main functionalities: first, to generate a 2-D grid
of points giving the Electrostatic Potential (ESP) in a cross-section through a
system, and second, to generate atomic charges based on the calculated ESP. At
present, the method is limited to AM1 systems containing H, C, N, O, F, Cl,
only. 

\subsection{2-D Electrostatic Potential Plots}
ESP plots are generated in two steps.  First, a MOPAC calculation generates a
2-D grid of points.  This grid is then converted into a picture by the utility
program \comp{ESPLOT}.  \comp{ESPLOT} is very simple to use: the command is
\comp{esplot $<$filename$>$}, where \comp{$<$filename$>$} is the name of the
data-set.  \comp{ESPLOT} generates an on-line picture of the PMEP, and a HPGL
file suitable for use in generating hard-copy.  Because \comp{ESPLOT} is so
simple, it will not be discussed further.  Instead, the rest of this discussion
applies to the MOPAC calculation.

The grid generated by MOPAC consists of a 2-D array of points representing a
cross-section through the system.  The distance between points is a constant
0.1 \AA ngstroms.  The size represented by the grid is roughly 4 \AA ngstroms 
plus twice the size of the system.  For example, N$_2$ has a N--N distance of
1.1 \AA , and the default associated grid represents a rectangular area of 5.8
by 4.6 \AA .    Each grid point represents the potential in kcal/mol which a
unit positive charge would experience due to the electrostatic field of the
system.

ESP grids are generated by specifying 
\hyperref[pageref]{\comp{PMEP}}{(see p.~}{)}{pmep} and \comp{PRTMEP}.
An example of a data-set for the PMEP procedure is shown in
Figure~\ref{pmepdata}. The PMEP plot for this data set is shown in
Figure~\ref{pmepplot}. This plot can be compared with the
\hyperref[pageref]{\comp{MEP} plot}{ on page~}{}{mep1plot}.

\begin{figure}
\begin{makeimage}
\end{makeimage}
\begin{verbatim}
 1scf  AM1 PMEP MINMEP PRTMEP
 Formaldehyde (Cross-section in plane of molecule)
 Generate a 2-D grid of PMEP potentials for `esplot' to use
  O    0.00000000 0     0.0000000 0    0.0000000 0    0 0 0     -0.2759
  C    1.22732374 1     0.0000000 0    0.0000000 0    1 0 0      0.1384
  H    1.11047287 1   122.2253516 1    0.0000000 0    2 1 0      0.0688
  H    1.11048351 1   122.2158646 1  179.9998136 1    2 1 3      0.0687
\end{verbatim}
\caption{\label{pmepdata}Data Set for PMEP Calculation of Formaldehyde}
\end{figure}

\begin{figure}
\begin{makeimage}
\end{makeimage}
\begin{center}
\includegraphics{ch2o-ford}
\end{center}
\caption{\label{pmepplot}Parametric Molecular Electrostatic Potential around
Formaldehyde}
\end{figure}

\subsubsection{Choice of Plane to be Calculated}
By default, the grid is centered on the center of the molecule, and the X-Y
plane at Z=0 is selected.  Other grids can be chosen using \comp{PMEPR}.
\comp{PMEPR} uses three atoms and an optional offset to define the plane to be
used.  It has enough options to allow any plane to be easily specified.

\subsection{Atomic Charges}
By use of \comp{QPMEP}, a set of atomic charges can be calculated.  This set of
charges is the best least squares fit to the charges which reproduce the ESP of
the Connolly or Williams surfaces.

\section{Miertus-Scrocco-Tomasi Solvation Model}
\index{MST Solvation Model}\label{mst}
\index{Orozco@{\bf Orozco, Modesto}}
\index{Luque@{\bf Luque, F. J.}}
\index{Solvation Model!MST}
\begin{center}Based on materials provided by\end{center}

{\bf Modesto Orozco}, Dep.\ Bioqu\'{\i}mica, Facultat de Qu\'{\i}mica,
Universitat de Barcelona, C/ Mart\'{\i} i Franqu\'{e}s 1, Barcelona 08028. 
SPAIN. E-mail: modesto@far1.far.ub.es.

and

{\bf F. J. Luque}, Dep.\ Farmacia, Unitat Fisicoqu\'{\i}mica, Facultat de
Farmacia,  Universitat de Barcelona, Avgda Diagonal sn,  Barcelona 08028,
SPAIN. E-mail: javier@far1.far.ub.es.

\subsection{Outline of the MST Method}
\index{MST solvation model}
\index{Solvation model!MST model}
Solvation is computed using a modified version of Miertus, Scrocco and 
Tomasi~\cite{mst,mt} self-consistent reaction field method (MST/SCRF), which 
has been modified~\cite{lno,glo,nol,lbo,obl} to allow semiempirical
Hamiltonians  to be used. Following this strategy the  solute is placed in a
molecular-shaped cavity, which is built using GEPOL  routines~\cite{pastb}
which use optimized van der Waals' radii~\cite{ojl,blo,rap}.  The solvent is 
represented as a continuum which reacts against the solute charge distribution 
{\em via} a perturbation operator. The perturbation operator is obtained by 
solving the Laplace equation at the solute/solvent interface (the solute
cavity).  The solute electrostatic potential, which is necessary to solve the
Laplace equation, is rigorously computed as the expectation value of the 
$r^{-1}$ operator~\cite{mst,mt,lno,glo,nol,lbo,obl}.

The free energy of hydration is computed as the addition of three
contributions: 

\begin{enumerate}
\item The electrostatic term, which is computed from the linear free energy
response  theory~\cite{mst,mt,lno,glo,nol,lbo,obl}.

\item The cavitation contribution, which is computed from Pierotti's  scaled
particle theory~\cite{ojl,blo}.\index{Pierotti's scaled particle theory}

\item The van der Waals terms, which is computed using a linear relation with 
the solute accessible surface, and optimized ``hardness''
parameters~\cite{rap}. \index{Van der Waals!in MST theory}
\end{enumerate}

In addition to the free energy of hydration a ``solvent-adapted'' wavefunction 
is obtained. Such a wavefunction can be used to determine changes in  solute
properties due to the solvent~\cite{lobg,lo,lo2,lobg2}.

\index{Electrostatic potential!Orozco-Luque model}
\index{MEP}
\index{Self Consistent Reaction Field method}
\index{SCRF Method}

The program allows the use of two different strategies to compute the 
semiempirical Molecular Electrostatic Potential (MEP). The first 
one~\cite{lio,alo,lo3} computes the MEP via the deorthogonalization of the
semiempirical  wavefunction (no keyword), while the second one~\cite{frr,alo2}
maintains the  orthogonality of the wavefunction in the calculation of the MEP
(keyword \comp{ORT}). \label{ort}

The strategy has been tested in different systems. Optimized versions for 
MNDO, AM1 and PM3 are available. The comparison of experimental and 
theoretical free energies of hydration for 23 small neutral molecules yields
to  RMS deviations in the range of 1  kcal/mol. The best results  (RMS 1.0
kcal/mol, r=0.94) are obtained with the AM1 Hamiltonian with  orthogonal MEP's.
The solvent-induced dipoles matches Monte Carlo/QM and {\em ab initio} SCRF
values.

Semiempirical/SCRF methods have been successfully applied to several problems. 
Examples: i) tautomerization of nitroimidazole, and pyridones, ii) solvent 
effects in the isomerization of amides, and iii) H-bond dimerization.

\subsubsection*{Warnings}
\begin{itemize}
\item The solute geometry cannot be properly optimized in solution.
Optimization flags  at the different coordinates should be set to 0

\item The program has been carefully parametrized to work with neutral
molecules.  A reduction of the cavity for charged solutes may be
necessary~\cite{ol}.

\item We cannot recommend the use of the program for solvents other than 
water, chloroform, or carbon tetrachloride at the present moment. A simple 
change in the dielectric constant cannot  be enough to reproduce the
characteristics of other solvents.
\end{itemize}

\subsection{Data Requirements for MST Model}\label{mstdata}
\index{TOM@{\bf TOM}}
For the three solvent for which parameters have been published, special
keywords exist.  These are \comp{H2O}, \comp{CHCL3}, and \comp{CCL4}. If other
solvents are to be modeled, extra data must be provided at  the end of the data
set.  This extra data is described by example, using water as the solvent.
(This extra data could be avoided by using \comp{H2O}.)

For water, the data needed is shown in Figure~\ref{tomdata}.
\begin{figure}
\begin{makeimage}
\end{makeimage}
\begin{verbatim}
Line
  *   PM3 TOM 
  *
  *   O   .0000000 0     .000000 0     .000000 0    0 0 0
  *   H   .9512971 0     .000000 0     .000000 0    1 0 0
  *   H   .9512971 0  107.457468 0     .000000 0    1 2 0
  *
  1   78.5  1.0  1  1  0  3
  2   298.15  18.07  2.77  0.00026  71.690  0.657  1.277
  3   40.0  1.5  0.2  0.7  0.02  5  3  1
  4   1  1.75000  8
  5   2  1.00000  99
  6   3  1.00000  99
  7   1.00
\end{verbatim}
\caption{\label{tomdata}Data set for Calculation of Water using the Miertus-Scrocco-Tomasi Model}
\end{figure}

The extra data are indicated by the lines numbered 1 to 7 in the Figure.  These
data are described as follows:
\begin{description}
\item[Line 1. Parameters indicating the level of computation]~\\
(see Miertus, Scrocco, Tomasi, Chem.\ Phys., 55 (1981) 117 and Miertus, Tomasi,
Chem.\ Phys., 65 (1982) 239 for further details)

Layout of data:  A B C D E F

\begin{center}
\begin{tabular}{llll}  \hline
 Datum & Value & Name   &  Description  \\ \hline
    A  & 78.5  & EPS    & Solvent dielectric constant          \\
    B  & 1.0   & DMP    & Self-polarization acceleration factor\\
    C  & 1     & MC     & Self-polarization cycles             \\
    D  & 1     & ICOMP  & Charge compensation                  \\
    E  & 0     & IFIELD & Reaction field calculation on nuclei \\
    F  & 3     & ICAV   & Cavitation energy computation        \\  \hline 
\end{tabular}
\end{center}

\item[Line 2. Parameters related to the cavitation energy]~\\
(This line is needed only if \comp{ICAV $\ne$ 0})

Layout of data: A B C D E F G                                                      

\begin{center}
\begin{tabular}{llll}  \hline
 Datum & Value & Name &  Description  \\  \hline
  A   & 298.15 & TABS &  Absolute temperature (in kelvin) \\
  B   &  18.07 & VMOL &  Solvent molar volume (\aa ngstroms$^3$) \\  
  C   &   2.77 & DMOL &  Solvent molar diameter (\aa ngstroms) \\
  D   & 0.00026 & TCE &  Thermal expansion coefficient (kelvin$^{-1}$) \\ 
  E   & 71.69  & STEN &  Surface tension (dyne$\times$cm$^{-1}$) \\
  F   & 0.657  & DSTEN &  Surface tension derivative \\
  G   & 1.227  & CMF  &  Cavity microscopic coefficient \\ \hline
\end{tabular}
\end{center}

If \comp{ICAV=1} parameters A-D are needed, and Pierotti's cavitation energy
is  computed. 

If \comp{ICAV=2} parameters E-G are needed, and Sinanoglu's cavitation energy
is  computed.  

If \comp{ICAV=3} parameters A-G are needed, and Pierotti's and Sinanoglu's
cavitation energies are computed.

\item[Line 3. Parameters related to the cavity surface]~\\
(see Pacual-Ahuir, Silla, Tomasi, Bonaccorsi, J.\ Comput.\ Chem.\ 77 (1982)
3654 for further details)   

Layout of data: A B C D E F G H

\begin{center}
\begin{tabular}{llll} 
\hline
 Datum & Value & Name   &  Description  \\ \hline
A & 40.0 & OMEGA  &  Sphere overlapping parameter \\
B & 1.5  & RD     &  Solvent molecular radius \\
C & 0.2  & RET    &  Minimal radius to define new spheres \\
D & 0.7  & FRO    &  Numerical factor to define new spheres \\
E & 0.02 & DR     &  Step to compute electric field on cavity surface \\
F &  5   & NDIV   &  Parameter of cavity surface partition \\
G &  3   & NESF   &  Number of spheres \\
H &  1   & ICENT  &  Definition of spheres \\
\hline
\end{tabular}
\end{center}

\item[Lines 4--6. Coordinates and radii of spheres used to build up cavity surface]~\\
\comp{ICLASS} specifies the type of atom.  This information is used  in the
calculation of the van der Walls' energy, determines as the sum of products
between the surface of each atom and a parameter which is  characteristic for
each atom.

\comp{ICLASS} is, in fact, the atomic number of each atom, with the only
exception of hydrogen, for which we distinguish between  hydrogen atoms bound
to a heteroatom (\comp{ICLASS=99}) and those bound to a carbon
(\comp{ICLASS=1}).

If \comp{ICENT=0}, then five parameters are needed for each sphere.

Layout of data: A B C D E                                                            

\begin{center}
\begin{tabular}{llll}
\hline
 Datum & Value & Name   &  Description \\ \hline
    A  & n/a   & XE     &  X-coordinate \\
    B  & n/a   & YE     &  X-coordinate \\
    C  & n/a   & ZE     &  X-coordinate \\
    D  & n/a   & RE     &  Radius of sphere \\
    E  & n/a   & ICLASS &  Type of atom \\ \hline
\end{tabular}
\end{center}

If \comp{ICENT$\ne$0}, then three parameters are needed for each sphere.  The
spheres are defined as being centered on each atom.

Layout of data: A B C
 
\begin{center}
\begin{tabular}{lcccll} 
\hline 
 Datum & \multicolumn{3}{c}{Values} & Name   &  Description \\ \hline
   A  & 1 & 2 & 3            & NC1    &  Atom Number \\
   B  & 1.75 & 1.0 & 1.0     & RE     &  Radius of sphere \\
   C  & 8 & 99 & 99          &ICLASS  &  Type of atom \\ \hline
\end{tabular}
\end{center}
                                                                      
\item[Line 7. Factor used to scale the MEP in the deorthogonal procedure]~\\
Layout of data:    A 

\begin{center}
\begin{tabular}{llll}  
\hline 
 Datum & Value & Name   &  Description  \\ \hline
    A  & 1.0 & FACTOR  &  Factor used to scale the MEP \\ \hline
\end{tabular} 
\end{center}
\end{description}

