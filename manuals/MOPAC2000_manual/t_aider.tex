\subsection{Using  {\em ab initio} derivatives}
\index{ab initio@{{\em ab initio}}!derivatives}
\index{Derivatives!{\em ab initio}}
\comp{AIDER} will allow gradients to be defined for a system.  MOPAC  will
calculate  gradients, as usual, and will then use the supplied gradients to
form an error function.  This error function is  (supplied gradients $-$ 
initial  calculated  gradients),  which is then added to the computed
gradients, so that for the initial  SCF,  the  apparent  gradients  will equal
the supplied gradients.

A typical data-set using \comp{AIDER} is shown in Figure~\ref{aider}.

\begin{figure}
\begin{makeimage}
\end{makeimage}
\index{Cyclohexane}
\index{ab initio@{{\em ab initio}}!geometry}
\index{Data!for cyclohexane}
\compresstable
\begin{verbatim}
 PM3 AIDER AIGOUT GNORM=0.01
 Cyclohexane

   X
   X   1 1.0
   C   1 CX     2 CXX
   C   1 CX     2 CXX     3 120.000000
   C   1 CX     2 CXX     3 120.000000
   X   1 1.0    2 90.0    3   0.000000
   X   1 1.0    6 90.0    2 180.000000
   C   1 CX     7 CXX     3 180.000000
   C   1 CX     7 CXX     3  60.000000
   C   1 CX     7 CXX     3 -60.000000
   H   3 H1C    1 H1CX    2   0.000000
   H   4 H1C    1 H1CX    2   0.000000
   H   5 H1C    1 H1CX    2   0.000000
   H   8 H1C    1 H1CX    2 180.000000
   H   9 H1C    1 H1CX    2 180.000000
   H  10 H1C    1 H1CX    2 180.000000
   H   3 H2C    1 H2CX    2 180.000000
   H   4 H2C    1 H2CX    2 180.000000
   H   5 H2C    1 H2CX    2 180.000000
   H   8 H2C    1 H2CX    2   0.000000
   H   9 H2C    1 H2CX    2   0.000000
   H  10 H2C    1 H2CX    2   0.000000

   CX     1.46613
   H1C    1.10826
   H2C    1.10684
   CXX   80.83255
   H1CX 103.17316
   H2CX 150.96100

   AIDER
     0.0000
    13.7589 -1.7383
    13.7589 -1.7383  0.0000
    13.7589 -1.7383  0.0000
     0.0000  0.0000  0.0000
     0.0000  0.0000  0.0000
    13.7589 -1.7383  0.0000
    13.7589 -1.7383  0.0000
    13.7589 -1.7383  0.0000
   -17.8599 -2.1083  0.0000
   -17.8599 -2.1083  0.0000
   -17.8599 -2.1083  0.0000
   -17.8599 -2.1083  0.0000
   -17.8599 -2.1083  0.0000
   -17.8599 -2.1083  0.0000
   -17.5612 -0.6001  0.0000
   -17.5612 -0.6001  0.0000
   -17.5612 -0.6001  0.0000
   -17.5612 -0.6001  0.0000
   -17.5612 -0.6001  0.0000
\end{verbatim}
\caption{\label{aider} Example of the use of \comp{AIDER}}
\end{figure}
 
Each  supplied  gradient  goes  with  the  corresponding   internal
coordinate.   In  the  example  given,  the  gradients came from a 3-21G
calculation on the geometry shown.  Symmetry will be taken into  account
automatically. Gaussian  prints  out  gradients in atomic units; these need to
be converted into kcal/mol/\AA ngstrom or kcal/mol/radian for MOPAC to use. 
The resulting geometry from the MOPAC run will be nearer to the optimized 3-21G
geometry than  if  the  normal  geometry  optimizers  in Gaussian had been
used.\index{Coordinates!Gaussian!example}
