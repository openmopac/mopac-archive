\section{A Note on Thermochemistry}\index{Thermochemistry|(}
\label{tsuneo}
\begin{center}
by
\ \\
Tsuneo Hirano\index{Hirano@{\bf Hirano, Tsuneo}}\\
Department of Chemistry  \\
Faculty of Science  \\
Ochanomizu University  \\
2-1-1 Otsuka  \\
Bunkyo-ku  \\
Tokyo 112  \\
Japan
\end{center}

\subsection{Basic Physical Constants}
All basic physical constants should be taken from: ``Quantities, Units and
Symbols in Physical Chemistry,'' Blackwell Scientific Publications Ltd, Oxford
OX2 0EL, UK, 1987 (IUPAC, based on CODATA of ICSU, 1986~\cite{codata}).  pp.\
81--82. A summary of these constants can be found in
Table~\ref{codata}
\begin{latexonly}
(p.~\pageref{codata})
\end{latexonly}.

Some derived quantities, which will be used in this section only are:

Moment of inertia: $I$ 1 amu \AA ngstrom$^2$ =
$1.660 540 \times 10^{-40}$ g cm$^2$.

Rotational constants: $A$, $B$, and $C$ (e.g.\ $A = h/(8\pi^2I)$)\\
With $I$ in amu \AA ngstroms$^2$ then:
$A$ (in MHz)  = $5.053 791 \times 10^5 / I$ \\
$A$ (in cm$^{-1}$) = $5.053 791 \times 10^5/ cI = 16.857 63 / I$


\subsection{Thermochemistry from {\em ab initio} MO methods}
Ab initio MO methods provide total energies, $E_{\rm eq}$,  as  the  sum  of
electronic   and   nuclear-nuclear  repulsion  energies  for  molecules,
isolated in vacuum, without vibration at 0  K. \index{ab initio@{{\em ab
initio}}!total energies}
\begin{equation}
E_{\rm eq} = E_{\rm el} + E_{\rm nuclear-nuclear} \nonumber
\end{equation}
From the 0 K potential surface and using  the  harmonic  oscillator
\index{Harmonic oscillator} approximation,  we can calculate the vibrational
frequencies, $\nu_i$, of the normal modes of vibration.  Using these, we can
calculate  vibrational, rotational   and   translational   contributions  to
the  thermodynamic quantities such as the partition function and heat capacity
which  arise \index{Partition function|ff}\index{Heat capacity|ff}\index{C$_p$}
from heating the system from 0 to T K.

$Q$: partition function, $E$: energy, $S$: entropy,
\index{Energy}\index{Entropy|(}
and $C$: Heat capacity at constant pressure = C$_p$. In {\em ab initio}
calculations,\index{Ab initio!C$_v$} the heat capacity calculated is C$_v$.
The relationship between C$_p$ and C$_v$ (in cal.degree$^{-1}$.mol$^{-1}$) is:
$$
C_p=C_v+R=C_v+1.987.
$$

\subsubsection{Vibrational terms}
\begin{equation}
Q_{\rm vib} = \prod_i{\frac{1}{[1 - \exp(-h\nu_i/kT)]}} \label{eq2}
\end{equation}
$E_{\rm vib}$, for a molecule at the temperature $T$ as:
\begin{equation}
E_{\rm vib} = \sum_i\left\{\frac{h\nu_i}{2} +
\frac{h\nu_i\exp(-h\nu_i/kT)}{[1 - \exp(-h\nu_i/kT)]}\right\}  \label{eq3} \nonumber
\end{equation}
where $h$ is Planck's constant, $\nu_i$ the $i$--th normal  vibration
frequency, and $k$  the Boltzmann constant.  For 1 mole of molecules, $E_{\rm
vib}$  should be multiplied by the Avogadro number $N_a = R/k$.  Thus:
\begin{equation}
E_{\rm vib} = N_a  \sum_i\left\{ \frac{h\nu_i}{2}
 + \frac{h\nu_i\exp(-h\nu_i/kT)}{[1-\exp(-h\nu_i/kT)]}\right\}  \label{eq4}
\end{equation}

Note that the first term  in equation~(\ref{eq4})  is   the  zero-point
vibration energy.   Hence,  the second term  in eq.~(\ref{eq4}) is the
additional vibrational contribution due to the temperature increase from 0 K to
T K.  Namely,
\begin{eqnarray}
E_{\rm vib}                & = & E_{\rm zero} + E_{\rm vib}(T)
\label{eq5}\nonumber\\
E_{\rm zero}               & = & N_a \sum_i {\frac{h\nu_i}{2}} \nonumber\label{eq6}\\
E_{\rm vib}( T) & = & N_a  \sum_i
{\frac{h\nu_i\exp(-h\nu_i/kT)}{[1 - \exp(-h\nu_i/kT)]}} \label{eq7}
\end{eqnarray}
The value of $E_{\rm vib}$ from GAUSSIAN 82 and 86 includes
$E_{\rm zero}$ as defined by Eqs.~(\ref{eq4},\ref{eq7}).
\begin{eqnarray}
S_{\rm vib} & = & R \sum_i\left\{
\frac{(h\nu_i/kT)\exp(-h\nu_i/kT)}{[1 - \exp(-h\nu_ii/kT)]}
- \ln[1 - \exp(-h\nu_i/kT)]\right\} \nonumber\label{eq8}\\
C_{\rm vib} & = & R \sum_i\left\{
\frac{(h\nu_i/kT)^2 \exp(-h\nu_i/kT)}{[1 - \exp(-h\nu_i/kT)]^2}\right\}
\nonumber\label{eq9}
\end{eqnarray}

At temperature $T>0$ K, a molecule rotates about  the  x,  y,  and z-axes  and
translates  in  x,  y,  and  z-directions.  By assuming the
\index{Equipartition of energy} equipartition of energy, energies for rotation
and translation,  $E_{\rm rot}$ and $E_{\rm tr}$, are calculated.

\subsubsection{Rotational terms}
\index{Moments of inertia}
$\sigma$ is symmetry number (Examples of symmetry numbers are shown
in Table~\ref{rot}). $I$ is moment of inertia.
$I_A$, $I_B$, and $I_C$ are moments of inertia about A, B, and C axes.
\begin{table}
\caption{\label{rot} Table of Symmetry Numbers }
\begin{center}
\begin{tabular}{lllcrclllcrclrr}
\hline
C$_1$&C$_I $    &C$_S $:&&   1&&D$_2$&D$_{2d}$&D$_{2h}$:&&4 &&  C$_{\infty v}$:& 1  \\
C$_2$&C$_{2v}$&C$_{2h}$:&&  2 &&D$_3$&D$_{3d}$&D$_{3h}$:&&6 &&  D$_{\infty h}$:&
2  \\
C$_3$&C$_{3v}$&C$_{3h}$:&&  3 &&D$_4$&D$_{4d}$&D$_{4h}$:&&8 && T, T$_h$ T$_d$: &12  \\
C$_4$&C$_{4v}$&C$_{4h}$:&&  4 &&D$_5$&D$_{5d}$&D$_{5h}$:&&10  &&  O, O$_h $:   &24  \\
C$_5$&C$_{5v}$&C$_{5h}$:&&  5 &&D$_6$&D$_{6d}$&D$_{6h}$:&&12  &&  I, I$_h$:  &24
\\
C$_6$&C$_{6v}$&C$_{6h}$:&&  6 &&D$_7$&D$_{7d}$&D$_{7h}$:&&14  && S$_4$:&2
\\
C$_7$&C$_{7v}$&C$_{7h}$:&&  7 &&D$_8$&D$_{8d}$&D$_{8h}$:&&16  && S$_6$: &3 \\
C$_8$&C$_{8v}$&C$_{8h}$:&&  8 &&      &       &         && &   &  S$_8$: &4 \\\hline
\end{tabular}
\end{center}
\end{table}

\paragraph*{Linear molecule}
\begin{eqnarray}
Q_{\rm rot} & = & \frac{8\pi^2 IkT}{\sigma h^2} \nonumber\label{eq10} \\
E_{\rm rot} & = & (2/2)RT  \nonumber\label{eq11} \\
S_{\rm rot} & = &  R \ln \left[ \frac{8\pi^2 IkT}{\sigma h^2} \right] + R
\nonumber\label{eq12} \\
            & = & R \ln I + R \ln T - R \ln\sigma - 4.349 203  \nonumber
\end{eqnarray}
where $ -4.349 203 = R \ln[8\times 10^{-16} \pi^2 k/(N_a h^2)] + R$.
\begin{equation}
C_{\rm rot} = (2/2)R    \nonumber\label{eq14}
\end{equation}

\paragraph*{Non-linear molecule}
\index{Non-linear molecules}
\begin{eqnarray}
 Q_{\rm rot} & = & \left( \frac{\sqrt{\pi}}{\sigma}\right)
 \left[ \frac{8\pi^2kT}{h^2}\right]^{3/2} \sqrt{I_AI_BI_C} \nonumber\\
& = & \left( \frac{\sqrt{\pi}}{\sigma}\right)
\left[ \left( \frac{8\pi^2cI_A}{h} \right)
       \left( \frac{8\pi^2cI_B}{h} \right)
       \left( \frac{8\pi^2cI_C}{h} \right) \right]^{1/2}
\left(\frac{kT}{hc}\right)^{3/2} \nonumber\label{eq15} \\
 E_{\rm rot} & = & (3/2)RT \nonumber\label{eq16} \\
 S_{\rm rot} & = & \frac{R}{2} \ln\left\{
 \left(\frac{\pi}{\sqrt{\sigma}}\right) \left(\frac{8\pi^2cI_A}{h}\right)
 \left(\frac{8\pi^2cI_B}{h}\right)      \left(\frac{8\pi^2cI_C}{h}\right)
 \left(\frac{kT}{hc}\right)^3\right\} + (3/2)R  \nonumber\label{eq17} \\
& = & (R/2) \ln{(I_A I_B I_C)} + (3/2) R\ln{T}  - R \ln{\sigma} - 5.3863921
\nonumber
\end{eqnarray}

Here, $-5.386 3921$ is calculated as:
$$ R \ln\left\{\frac{1}{h^3} \left(\frac{10^{-16}}{N_a}\right)^{3/2}
\sqrt{(3\times 2^9 \times \pi^7 \times k)}\right\} + (3/2)R.
$$

\begin{equation}
C_{\rm rot} = (3/2)R \nonumber\label{eq18}
\end{equation}

\subsubsection{Translational terms}
$M$ is Molecular weight.
\begin{eqnarray}
Q_{\rm tra} & = & \left( \frac{\sqrt{2\pi MkT/N_a}}{h}\right)^3
\nonumber\label{eq19} \\
E_{\rm tra} & = & (3/2)RT                            \nonumber\label{eq20} \\
S_{\rm tra} & = & R\left\{ \frac{5}{2} +
\frac{3}{2}\ln\left(\frac{2\pi k}{h^2}\right) + \ln k +
\frac{3}{2}\ln\left(\frac{M}{N_a}\right) + \frac{5}{2}\ln T - \ln p \right\}
\nonumber\label{eq21} \\
            & = & (5/2)R\ln T + (3/2)R\ln M - R\ln p - 2.31482 \nonumber\label{eq22}\\
C_{\rm tra} & = & (5/2)R                              \nonumber\label{eq23}
\end{eqnarray}
or  $H_{\rm tra} = (5/2)RT$  due to the $pV$ term (cf.  $H = U + pV$).
The internal energy $U$ at $T$ is:
\begin{equation}
U = E_{\rm eq} + [E_{\rm vib} + E_{\rm rot} + E_{\rm tra}] \nonumber\label{eq24}
\end{equation}
or
\begin{equation}
U = E_{\rm eq} + [(E_{\rm zero} + E_{\rm vib}(T))
  + E_{\rm rot} + E_{\rm tra}]  \label{eq25}
\end{equation}
Enthalpy $H$ for one mole of gas is defined as
\begin{equation}
H = U + pV     \label{eq26}
\end{equation}
Assumption of an ideal gas (i.e.,  $pV = RT$) leads to
\begin{equation}
H = U + pV = U + RT  \label{eq27}
\end{equation}
Thus, \mi{Gibbs free energy} $G$ can be calculated as:
\begin{equation}
           G = H - T S(T) \label{eq28}
\end{equation}

\subsection{Thermochemistry in MOPAC}
\index{$\Delta H_f$|ff} \index{Heat of Formation|ff}
It should be noted that M.O. parameters for MINDO/3,  MNDO,  AM1, PM3, and
MNDO-$d$  are optimized so as to reproduce the experimental  heat of formation
(i.e., standard enthalpy of formation or the enthalpy change to  form  a mole
of  compound  at  25$^\circ$C from its elements in their standard state) as
well as observed geometries (mostly at 25$^\circ$C), and  not to reproduce the
$E_{\rm eq}$ and equilibrium geometry at 0 K.

In  this  sense, $E_{SCF}$  (defined  as  Heat  of  formation, $\Delta H_f$),
force constants,  normal  vibration  frequencies, etc.\  are  all related to
the values at 25$^\circ$C, not to 0 K.  Therefore, the $E_{\rm zero}$
calculated in FORCE is not the true $E_{\rm zero}$. Its use as $E_{\rm zero}$
should be made at your own risk, bearing in mind the situation discussed above.

Since $E_{SCF}$ is standard enthalpy of formation (at 25$^\circ$C):
\begin{equation}
E_{SCF} = E_{\rm eq} + E_{\rm zero} +
E_{\rm vib}(298.15) + E_{\rm rot} + E_{\rm tra} + pV
 + \sum\left[ - E_{\rm elec}({\rm atom}) + \Delta H_f({\rm atom})\right].
\label{eq29}
\end{equation}
To avoid the complication arising from the definition of $E_{SCF}$, within
\index{Standard  Enthalpy  of Formation}
the  thermodynamics  calculation  the  Standard  Enthalpy  of Formation,
$\Delta H$, is calculated by
\begin{equation}
\Delta H = E_{SCF} + (H_T - H_{298}). \label{eq30}
\end{equation}

Here, $E_{SCF}$ is the heat of formation (at 25$^\circ$C)  given in the output
list, and $H_T$ and $H_{298}$ are the enthalpy contributions for the increase
of the temperature from 0 K to $T$ and 298.15, respectively.  In other words,
the enthalpy of formation is corrected for the difference in temperature from
298.15~K to $T$.

There is a problem in that $H_T$ is the heat of formation at $T$ relative to
the heat of formation of the elements in their standard state at 298K. This
involves mixing standard and not standard terms. There is no easy way to get
 the correct value for $H_T$, but for rough work $H_T$ is useful. For more correct
 work, calculate $\Delta H$ for the elements in their standard state at $T$, and use
 these $\Delta H$'s to get the $\Delta H$ for the compound you're working with (or use
 tables from the literature).

 The method of calculation for $T$ and $H_{298}$ will  be
given below.

In MOPAC, the variables defined below are used:
\begin{equation}
C_1 = \frac{hc}{kT}.   \nonumber\label{eq31}
\end{equation}
The wavenumber, $\omega_i$, in cm$^{-1}$:
\begin{equation}
\nu_i = \omega_i c    \nonumber\label{eq32}
\end{equation}
\begin{equation}
E_{\rm \omega_i} = \exp( -h\nu_i/kT) = \exp(-\omega_i hc/kT) = \exp(-\omega_i C_1) \nonumber\label{eq33}
\end{equation}
The rotational constants $A$, $B$, and $C$ in cm$^{-1}$:
\begin{equation}
A = \frac{h}{(8\pi^2 cI_A)}   \nonumber\label{eq34}
\end{equation}

\index{S} Energy and Enthalpy in cal/mol, and Entropy in cal/mol/K. Thus, eqs.\
(\ref{eq2}--\ref{eq28}) can be written as follows:

\subsubsection{Vibration}
\begin{eqnarray}
Q_{\rm vib} & = & \prod_i \frac{1 }{ (1 - E_{\rm \omega_i})}  \nonumber\label{eq35}\\
E_0         & = & \frac{0.5 N_a hc}{4.184 \times 10^7}\sum_i {\omega_i}
\nonumber\label{eq36}\\
            & = & 1.429 572 \sum_i {\omega_i}    \nonumber\label{eq37}\\
E_{\rm vib}(T) & = &
N_a h c\sum_i{\frac{\omega_i E_{\rm \omega_i}}{1 - E_{\rm \omega_i}}}
= (R/k) h c\sum_i{\frac{W_ iE_{\rm \omega_i}}{1 - E_{\rm \omega_i}}}     \nonumber\label{eq38}\\
S_{\rm vib} & = & R (hc/kT)
\sum_i\left\{\frac{\omega_i E_{\rm \omega_i}}{(1 - E_{\rm \omega_i})}\right\}
                - R\sum_i {\ln (1 - E_{\rm \omega_i})} \nonumber \\
& = & R C_1\sum_i\left\{\frac{\omega_i E_{\rm \omega_i}}{(1 - E_{\rm \omega_i})}\right\}
- R\sum_i {\ln(1 - E_{\rm \omega_i})}            \nonumber\label{eq39}\\
C_{\rm vib} & = & R (hc/kT)^2
\sum_i\left\{\frac{\omega_i^2 E_{\rm \omega_i}}{(1- E_{\rm \omega_i})^2} \right\} \nonumber \\
& = & R C_1^2
\sum_i\left\{\frac{\omega_i^2 E_{\rm \omega_i}}{(1- E_{\rm \omega_i})^2}\right\}
\nonumber\label{eq40}
\end{eqnarray}

\subsubsection{Rotation}

\paragraph*{Linear molecule}
\begin{eqnarray}
Q_{\rm rot} & = & (1/\sigma) (1/{\rm \AA}) (kT/hc) = \frac{1}{\sigma A C_1}
\nonumber\label{eq41}\\
E_{\rm rot} & = & (2/2)RT   \nonumber\label{eq42} \\
S_{\rm rot} & = & R\ln\left(\frac{kT}{\sigma hc{\rm \AA}}\right) + R
= R\ln \left(\frac{1}{\sigma {\rm \AA} C_1}\right) + R
= R\ln\left(\frac{kT}{\sigma hc{\rm \AA}}\right) + R \nonumber\label{eq43}\\
C_{\rm rot} & = & (2/2)R       \nonumber\label{eq44}
\end{eqnarray}

\paragraph*{Non-linear molecule}
\begin{eqnarray}
Q_{\rm rot} & = & \frac{1}{\sigma}
\left[\frac{\pi}{(A B C C_1^3)}\right]^{1/2} \nonumber\label{eq45}\\
E_{\rm rot} & = & (3/2)RT        \nonumber\label{eq46} \\
S_{\rm rot} & = & \frac{R}{2}\ln\left\{\frac{\pi}{\sigma^2ABC}
\left(\frac{kT}{hc}\right)^3 \right\} + (3/2)R \nonumber \\
& = & 0.5R { 3\ln(kT/hc) - 2\ln\sigma +\ln\left(\frac{\pi}{A B C}\right) + 3}
\nonumber\label{eq47} \\
& = & 0.5R { -3\ln C_1 - 2\ln\sigma + \ln\left(\frac{\pi}{A B C}\right) + 3}
\nonumber \\
C_{\rm rot} & = & (3/2)R     \nonumber\label{eq48}
\end{eqnarray}

\subsubsection{Translation}
\begin{eqnarray}
Q_{\rm tra} & = &
   \left( \frac{\sqrt{2\pi MkT/N_a}}{h} \right)^3
=  \left( \frac{\sqrt{1.660540\times^{-24}\times 2\pi MkT}}{h} \right)^3
\nonumber\label{eq49} \\
E_{\rm tra} & = &  (3/2)RT                    \nonumber\label{eq50} \\
H_{\rm tra} & = &  (3/2)RT + pV = (5/2)RT \; {\rm cf.}
\;pV = RT  \nonumber\label{eq51}\\
S_{\rm tra} & = & (R/2) [ 5\ln T + 3\ln M ] - 2.31482
\;{\rm cf.}\; p = 1 {\rm atm} \nonumber \\
& = & 0.993608 [ 5\ln T + 3\ln M] - 2.31482    \nonumber\label{eq52}
\end{eqnarray}

In MOPAC:
\begin{equation}
H _{\rm vib} = E_{\rm vib}(T)     \nonumber\label{eq53}
\end{equation}

(Note: $E_{\rm zero}$ is {\em not\/} included in $H_{\rm vib}$;
$\omega_i$ is not derived from force-constants at 0 K)
and for $T$:
\begin{equation}
H_T   = [H_{\rm vib} + H_{\rm rot} + H_{\rm tra}]   \nonumber\label{eq54}
\end{equation}
while for $T=298.15$~K:
\begin{equation}
H_{298} = [H_{\rm vib} + H_{\rm rot} + H_{\rm tra}]  \nonumber\label{eq55}
\end{equation}

Note that $H_T$ (and $H_{298}$) is equivalent to:
\begin{equation}
(E_{\rm vib} - E_{\rm zero}) + E_{\rm rot} + (E_{\rm tra} + pV) \nonumber\label{eq56}
\end{equation}
except that  the  normal  frequencies  are  those  obtained  from  force
constants at 25$^\circ$C, or at least not at 0 K.

Thus, Standard Enthalpy of Formation, $\Delta H$,  can  be  calculated
according to Eqs.~(\ref{eq25},\ref{eq26}) and (\ref{eq29}),
as shown in Eq.~(\ref{eq30});
\begin{equation}
\Delta H = E_{SCF} + (H_T - H_{298})   \nonumber\label{eq57}
\end{equation}
Note that $E_{\rm zero}$ is already counted in $E_{SCF}$,
see Eq.~(\ref{eq29}).

\index{Standard Internal Energy of Formation}
By using Eq.~(\ref{eq27}), Standard Internal Energy of Formation, $\Delta U$,
can be calculated as:
\begin{equation}
\Delta U = \Delta H - R(T - 298.15) \nonumber\label{eq58}
\end{equation}


Standard Gibbs Free-Energy of Formation, $\Delta G$, can be calculated
\index{Standard Gibbs Free-Energy of Formation}
by  taking  the difference from that for the isomer or that at different
temperature:
\begin{equation}
\Delta G = [\Delta H - TS] \;(\mbox{for the state under consideration})
- [\Delta H - TS]\; (\mbox{for reference state})  \nonumber\label{eq59}
\end{equation}

Taking the difference is necessary  to  cancel  the  unknown  values  of
standard entropy of formation for the constituent elements.

\index{Entropy|)}\index{Thermochemistry|)}
