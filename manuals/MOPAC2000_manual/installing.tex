\chapter{Installing MOPAC}
\section{Overview}
\index{MOPAC!installing}\index{Installing MOPAC}
MOPAC is distributed in compressed form.  The top copy is designed 
for use with the following scalar compilers:
\begin{enumerate}
\item Pentium II/Pro running Linux with Portland Group compiler.
\item The same, but using GNU g77 compiler.
\item DEC Alpha workstation with UNIX 4.0 and Fortran 5.0.
\item SGI R5000 Indy running IRIX 6.2 and Fortran 7.1.
\item SGI R10000 PowerChallenge running IRIX 6.2 and Fortran 7.2.
\item Sun UltraSPARC with Fortran 4.2.
\item Generic Sun SPARC with Fortran 4.2.
\item Older Suns with Sun 2.0.1 compiler.
\item HAL workstations with Fujitsu compiler.
\item IBM workstations.  
\item Fujitsu VX/VPP300/VPP700 series. 
\end{enumerate}
and parallel compilers:
\begin{enumerate}
\item Fujitsu AP3000 using Sun Fortran 77 compiler.
\item Fujitsu AP3000 using Fujitsu Fortran 90 compiler.
\item Fujitsu VX/VPP300/VPP700 running MPI 1.1.
\item Multiprocessor SGI PowerChallenge running SGI's MPI.
\item Sun Ultra Enterprise 10000 running MPICH.
\end{enumerate}

To get MOPAC up and running, carry out the following instructions:
\begin{enumerate}
\item Obtain a copy of MOPAC 2000, by E-mail to 
\htmladdnormallink{\comp{mwada@se.fujitsu.co.jp}}{mailto:mwada@se.fujitsu.co.jp}.
or from the web-site 
\htmladdnormallink{\comp{http://www.winmopac.com}}{http://www.winmopac.com}.
\item Create a directory to hold MOPAC (\comp{mkdir mopac}).
\item Move into the MOPAC directory (\comp{cd mopac}).
\item Copy the distribution file into this directory.
\item If the file is compressed, uncompress it.  Compressed files that end in
``.Z'' should be uncompressed with \comp{uncompress}, files that end in ``.gz''
should be \comp{gunzip}'d.
\item The distribution file is stored as UNIX ``tar'' form, so read the file
using \comp{tar -xvf $<$file$>$}
\item Go into the \comp{./src} directory.
\item  Before the Makefile can be run, it needs to know what set of compiler
options to use. These sets of options are stored in files in the sub-directory 
\comp{MAKE\_INCLUDE}.  Select the appropriate set and note the name of the
file.
\item Make this name known to the Makefile by issuing the command:
\begin{verbatim}
setenv MOPAC_PLATFORM <file>
\end{verbatim}
in which \comp{$<$file$>$} is the name of the file, after removing the prefix
``md.''.
For example, to select the options in the file \comp{md.Sun\_SPARC}, issue the command:
\begin{verbatim}
setenv MOPAC_PLATFORM Sun\_SPARC
\end{verbatim}
\item Run the Makefile by issuing the command \comp{make}.
\end{enumerate}

\section{\comp{sizes.h}}\index{sizes.h|(} \label{sizes}\index{Memory!requirements}
The file \comp{sizes.h} determines how big a system MOPAC can run.  
The current maximum size of MOPAC, in terms of atoms, can be
found in the output file near the start, at the end of the fourth line
of stars.  The end of that line would look something like this:
\begin{verbatim}
******************020000
\end{verbatim}
This means that the program was configured to run systems of up to 20,000 atoms.
 
Whenever you modify \comp{sizes.h}, re-run \comp{make}.

When \comp{make} is run, it compiles  all the FORTRAN files, using  the  array 
sizes given  in \comp{sizes.h}:  these should be modified  before \comp{make}
is run. If, for whatever reason, \comp{sizes.h} needs to be changed, then 
\comp{make} should  be  re-run, as modules compiled with different
\comp{sizes.h} will be incompatible.

\index{NUMATM}\index{NUMATM} The parameters within \comp{sizes.h} that the 
user  can  modify  are \comp{NUMATM},  \comp{MAXTIM} and \comp{MAXDMP}.   
\comp{NUMATM} is assigned a value equal to the largest number of atoms that a
MOPAC  job  is  expected  to run. Note, however, that because MOPAC now uses
dynamic memory allocation, the size of the program depends mainly on the size
of the system being run, and only to a very small degree on the value of
\comp{NUMATM}.  Because of this, the value of \comp{NUMATM} can be quite large,
for example, 100--20,000.

\comp{MAXTIM} is the default maximum time in seconds a job is allowed to  run 
before  either completion  or a restart file being written.  \comp{MAXDMP} is
the default time in seconds for the automatic writing of  the  restart 
files.   If  your computer  is  very  reliable,  and disk space is at a
premium, you might want to set \comp{MAXDMP} as \comp{MAXDMP=999999}.

\index{SYBYL}\index{ISYBYL}\index{NMECI}\index{NPULAY}\index{MESP} If SYBYL
output is wanted, set \comp{ISYBYL} to 1,  otherwise  set  it  to zero.

If you want, \comp{MESP} can be varied. 

The quantity \comp{ISAFE} is used to decide whether memory or reliability is
the more important.  If \comp{ISAFE} is set to 1, then a larger amount of 
memory will be needed, but the reliability of MOPAC is increased.  If the
special SCF convergers are needed, and \comp{ISAFE=1}, then they will be used. 
If \comp{ISAFE=0}, then MOPAC will not allocate memory for these special SCF
convergers.  Whichever choice is made, the other choice can be selected at
run-time by use of keywords. Thus if \comp{ISAFE=0} is selected, then the
run-time default is to minimize memory usage.  If keyword \comp{SAFE} is used,
then the other option will be used for that job.  See also keywords \comp{SAFE}
and \comp{UNSAFE}

\section{\comp{meci.h}}\index{meci.h} \label{mecih}
The maximum values of the variables used in a Multi-Electron Configuration
Interaction calculation are stored in \comp{meci.h}.  These variables are
described in the file. Normally, the values supplied are sufficient for most
work.  However, if you want, \comp{NMECI}  can be changed.  Setting it to 1
will save  some space, but will prevent all C.I.\ calculations except simple
radicals. A useful value for \comp{NMECI} is 4.  This will allow simple excited
states to be run.

\section{\comp{symtry.h}}\index{symtry.h} \label{symtryh}
This file holds the maximum number of symmetry functions allowed. This value
should be sufficiently large to allow any system to be run (the default value
would allow the fullerene C$_{1500}$ to be run, using all possible symmetry
operations.)

\section{Compiling MOPAC}
A recommended sequence of operations to get MOPAC  up  and  running
would be:
\index{mopac.csh}
\begin{enumerate}
\item Read through the \comp{README} and \comp{make} files to  familiarize 
yourself  with what is being done.
\item Edit the file \comp{mopac.csh} so that the address of \comp{mopac.exe} is
correct.
\index{make!use of}
\item Compile MOPAC (\comp{make}).  This operation takes a few minutes, and 
should be run ``on-line''.
\index{mopac.exe}

When everything is successfully compiled,  the  object  files  will then  be 
assembled into an executable image called \comp{mopac.exe}.  Once the image
exists, there is no reason to keep the object files, and if  space is at a
premium these can be deleted at this time (\comp{make clean}).

If you need to make any  changes  to  any  of  the  files, or to any of the
INCLUDE'd files (\comp{sizes.h}, \comp{meci.h},  or \comp{symtry.h}) re-run
\comp{make}.
\end{enumerate}

In order for users to have access to  MOPAC, the file \comp{mopac.csh} must be
placed in a directory in the user's \comp{path}. One possible location would be
\comp{/usr/local/bin}.

\index{sizes.h|)}

\section{Running MOPAC}\index{MOPAC!running}
To run MOPAC, enter the command \comp{mopac} 
                  
You will receive the message {\sc  What file?} to which the reply should be the
actual data-file name. For example, \comp{port}, the file is assumed to end in
\comp{.dat}, e.g.\ \comp{port.dat}.


To familiarize yourself with the system, the  following  operations
might be useful.
\begin{enumerate}
\item Run the (supplied) test molecules, and  verify  that  MOPAC  is producing
``acceptable'' results.
\item Make some simple modifications to  the  datafiles  supplied  in order to
test your understanding of the data format
\item  When satisfied that MOPAC is working, and that data  files  can be made,
begin production runs.
\end{enumerate}
\index{Restarts}
   
Restarts should be user transparent.  If MOPAC does make any restart files, do
not change them (it would be hard  to  do  anyhow,  as they're  in  machine 
code), as they will be used when you run a RESTART job.  The files used by
MOPAC are given in Table~\ref{allfiles}.

\section{The  shut command}
\index{SHUT}
If, for whatever reason, a run needs to be stopped prematurely, the command
\comp{shut $<$jobname$>$} can be issued. This  will execute a small
command-language file, which copies the data-file to  form  a  new  file called
\comp{$<$filename$>$.end}.

The next time MOPAC  calls  function  SECOND,  the  presence  of  a readable
file called \comp{$<$filename$>$.end}, is checked for, and if it exists,  the 
apparent  elapsed  CPU  time  is increased  by  10,000,000  seconds,  and  a 
warning  message issued.  No further action is taken until the elapsed time 
is  checked  to  see  if enough  time remains to do another cycle.  Since an
apparently very long time has been used, there is not enough time left to do 
another  cycle, and the restart files are generated and the run stopped.

The \comp{shut} command is completely machine-independent.

\section{Size of MOPAC}
MOPAC now uses dynamic allocation of memory for all arrays that depend on the
size of the system. This means that the size of MOPAC is no longer a static
quantity. Instead, the size of the MOPAC executable image will depend on the
system being run. 

Because the amount of memory required for the static arrays is small, the value
of \comp{NUMATM} should be set as large as practical.  For a  machine with 10
Mb, \comp{NUMATM} could be set to 200. For larger machines, set \comp{NUMATM}
to the largest value that is  likely to be encountered (20,000, for example).

\begin{table}
\caption{\label{sizesc} Size of Source Code for MOPAC}
\begin{center}
\begin{tabular}{rrcrcr} \hline
Version & \multicolumn{5}{c}{Number of lines in program} \\
\hline
 5.00 & 22,084 & = & 17,718 code & + & 4,366 comment.  \\
 6.00 & 30,815 & = & 23,910 code & + & 6,905 comment.  \\
93.00 & 47,676 & = & 35,841 code & + & 11,835 comment.  \\
\mopacversion .00 & 84,148 & = & 64,649 code & + & 19,499 comment.  \\
\hline
\end{tabular}
\end{center}
\end{table}


\section{Porting MOPAC to other platforms}\label{porting}
\index{MOPAC!porting}\index{Porting MOPAC}
\index{MOPAC!version numbers}\index{Version numbers}
\index{SUN}
\index{SPARC-2}
MOPAC is designed to be run on several different platforms. This set is
obviously not exhaustive, and it might be necessary to ported the program to
other machines. After such a port has been done, the new  program  should  be 
given  a name in accordance with the following convention:

\begin{itemize}
\item MOPAC~\mopacversion .00 is reserved for the original MOPAC.
\item MOPAC~\mopacversion .$nm$, $n$ = 1-9 is reserved for vendors who port MOPAC
to new platforms. The first ``port'' should be called MOPAC~\mopacversion .10. 
If the vendor makes other changes, the names MOPAC~\mopacversion .20 -
MOPAC~\mopacversion .90 may be used.
\item MOPAC~\mopacversion .$nm$, $m$ = 1-9 are reserved for user's in-house use.
\end{itemize}

By this convention, users will know immediately which MOPAC they are running. 
Thus MOPAC~\mopacversion .23 would be the third in-house modification of a
vendor's second version of MOPAC~\mopacversion . 
\index{Validating MOPAC}\index{MOPAC!validating}
 
To  validate  the  new  program,  the data-set \comp{port.dat} should be run
(see Chapter~\ref{examples}). If  all  tests are passed, within the tolerances
given in the tests, then the new program can be called a valid version of
MOPAC. \ Insofar  as is practical, the mode of submission of a MOPAC job should
be preserved, e.g.,

\begin{verbatim}
(prompt) MOPAC <data-set> [<queue-options>...]
\end{verbatim}
 
Any changes which do not violate  the FORTRAN--77  conventions,  and which 
users  believe  would  be  generally desirable, can be sent to the author.

The main difficulty in modifying the program so that it runs on other platforms
is likely to be in the channel assignments.  In an attempt to make this easier,
all channel assignments are defined in subroutine GETDAT.  MOPAC uses the UNIX
instructions \comp{iargc} and \comp{getarg}.  These read in the number of
arguments supplied with the MOPAC command, and if this is greater than zero, it
then reads in the name of the job. This is stored in JOBNAM.

Other possible trouble-spots are (a) the CPU timer (in \comp{second.F}), and
the date  function \comp{fdate} (in \comp{readmo.F} and \comp{writmn.F}); (b)
the \comp{RESTART} command; (c) the INCLUDE files \comp{sizes.h},
\comp{meci.h}, and \comp{symmetry.h}.  Most compilers now support the INCLUDE
statement, in one form or another, so this should not be too much of a problem.

The following is a complete list of known machine-specific code in MOPAC:

\begin{description}
\item[\comp{MALLOC}]~\\
When the \comp{DYNAMIC\_MEMORY} option is used, the dynamic memory of MOPAC is 
assigned by \comp{MALLOC} at the start of a run. If your machine supports
\comp{MALLOC} or other call to allocate memory, then the dynamic version can be
used.  If not, then use the CPP command for static memory allocation
(\comp{STATIC\_MEMORY}).

The function \comp{iargc} and subroutine \comp{getarg} are used to allow data
in  the form of an argument to be passed into the program.  The argument
actually  used is the name of the data-set, less the \comp{.dat} suffix.  This
name is then used to determine all the file-names which are used within MOPAC.

In order to make porting easier, all channel numbers are contained in an array
\comp{IFILES}, and  assigned in GETDAT. The default order is that
\comp{IFILES($i$)=$i$}.  The default assignments are shown in
Table~\ref{allfiles}.

\begin{table}\index{Channel assignments}
\caption{\label{allfiles}Files used by MOPAC}
\begin{center}
\begin{tabular}{cllll}
\hline
Channel & Name                      & Input? & Output? & Contents \\ \hline
2       & \comp{$<$filename$>$.dat} & Yes    &         & Data-set          \\
4       & \comp{$<$filename$>$.end} & Yes    &         & Shut command \\
5       & SCRATCH                   &        &         & Internal working \\
6       & \comp{$<$filename$>$.out} &        & Yes     & Results file\\
9       & \comp{$<$filename$>$.res} & Yes    & Yes     & Restart file \\
10      & \comp{$<$filename$>$.den} & Yes    & Yes     & Density file \\
11      & \comp{$<$filename$>$.log} &        & Yes     & Log file \\
12      & \comp{$<$filename$>$.arc} &        & Yes     & Archive file \\
13      & \comp{$<$filename$>$.gpt} &        & Yes     & Graphics file\\
14      & User-defined              & Yes    &         & New parameters\\
15      & \comp{$<$filename$>$.esr} &        & Yes     & ESP Restart\\
16      & \comp{$<$filename$>$.syb} &        & Yes     & SYBYL file \\
18      & \comp{$<$filename$>$.brz} &        & Yes     & Brillouin Zone \\
20      & \comp{$<$filename$>$.ump} &        & Yes     & Data for grid\\
21      & \comp{$<$filename$>$.esp} &        & Yes     & ESP data\\
22      & SCRATCH                   &        &         & Used by GEOUTG\\
\hline
\end{tabular}
\end{center}
\end{table}

All the data are read in on channel 2, and immediately written to a SCRATCH
channel, 5.  For the rest of the job, data are read off channel 5 as if it were
the ordinary input channel. By doing this, the data-set is used only  at the
very start of a job, and is released immediately afterwards.

If, for whatever reason, a channel number must be changed, then the change
should be made in GETDAT, after the \comp{C====\ldots} line, and should be of
form:
\begin{verbatim}
IFILES(5)=8
\end{verbatim}
This change would make the SCRATCH file, which holds the input data,  channel
8 instead of the default channel 5.  Of course, if the channel numbers are
to be changed, care should be taken to avoid channel numbers which are
already in use.

\item[Character Conversion]~\\
Conversion from lower to upper case or upper to lower case occurs several times
in MOPAC.  Because FORTRAN-77 does not recognize lower case letters, these
conversions may cause some difficulty.

\item[\comp{SECOND}]~\\
SECOND returns the CPU time elapsed since an arbitrary zero of time. Three
mechanisms are provided for determining CPU time.  One of these should  be
suitable for new compilers.

SECOND also checks a file with the  name \comp{$<$filename$>$.end}.  If this
file is of non-zero length, the elapsed CPU time is increased by 10,000,000
seconds. This normally causes a premature shutdown due to elapsed time being
exceeded. This is a useful feature, and should be implemented in all MOPAC
programs.

\item[\comp{INCLUDE}]~\\
The small files \comp{sizes.h}, \comp{meci.h}, \comp{mozyme\_mpi.h}, and
\comp{symtry.h}, are used to set various parameters within MOPAC. These files
are included in many subroutines. INCLUDE is not standard FORTRAN, but appears
to be supported by most compilers in one form or another.

\item[The MOPAC command]~\\
To run MOPAC, a user issues the command \comp{mopac $<$filename$>$}.  The verb
\comp{mopac} causes a shell-script, \comp{mopac.csh}, to be executed.  This
program, \comp{mopac.csh}, is written in SUN UNIX, but should be easily
convertible to other flavors of UNIX.  As with SECOND, \comp{mopac.csh} should
be converted, as it saves users a lot of time.
\end{description}
