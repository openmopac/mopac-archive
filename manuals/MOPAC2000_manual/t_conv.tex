\subsection{Convergence in SCF Calculation}
\index{SCF!convergence techniques}
A  brief  description  of  the  convergence  techniques   used   in subroutine
ITER follows.

ITER, the  SCF  calculation,  employs  six  methods  to  achieve  a
self-consistent field.  In order of usage, these are:

\begin{enumerate}
\index{Fock matrix}
\item Intrinsic convergence by virtue of the way the  calculation  is carried 
out.   Thus  a trial Fock gives rise to a trial density matrix, which in turn
is used to generate a better Fock matrix.

This is normally convergent, but many exceptions  are  known.   The main
situations when the intrinsic convergence does not work are:
\begin{enumerate}
\item A bad starting density  matrix.   This  normally  occurs when the default
starting density matrix is used.  This is a very crude approximation, and is
only  used  to  get  the  calculation started.   A  large  charge  is generated
on an atom in the first iteration,  the  second   iteration  
overcompensates,   and   an oscillation is generated.

\item The equations are only very slowly convergent.  This can be  due  to  a 
long-lived  oscillation  or to a slow transfer of charge.
\end{enumerate}

\index{Oscillations!damping|ff}
\item Oscillation damping.  If, on any two consecutive iterations,  a density 
matrix  element  changes  by  more  than 0.05, then the density matrix element
is set equal to the old element shifted by  0.05  in  the direction  of  the
calculated element.  Thus, if on iterations 3 and 4 a certain density matrix
element was 0.55 and 0.78, respectively, then the element  would  be set to
0.60 (= 0.55 + 0.05) on iteration 4.  The density matrix from iteration 4 would
then be used in the  construction  of  the next  Fock  matrix.   The arrays
which hold the old density matrices are not filled until after iteration 2. 
For this reason they are  not  used in the damping before iteration 3.

\item Three-point interpolation of the  density  matrix.   Subroutine CNVG
monitors the number of iterations, and if this is exactly divisible by three,
and certain other conditions relating to the density  matrices are  satisfied, 
a  three-point interpolation is performed.  This is the default converger, 
and  is  very  effective  with  normally  convergent calculations.    It 
fails  in  certain  systems,  usually  those  where significant charge build-up
is present.

\item Energy-level shift technique (the \comp{SHIFT} technique).  The  virtual
M.O.\  energy  levels are  normally  shifted  to more positive energy.  This
has the effect of damping oscillations, and intrinsically divergent equations
can often be changed   to  intrinsically  convergent  form.   With 
slowly-convergent systems the virtual M.O.\  energy levels can be moved to a
more  negative value.

The precise value of the shift used depends on the behavior of  the iteration
energy.  If it is dropping, then the HOMO-LUMO gap is reduced; if the iteration
energy rises, the gap is increased rapidly.

\index{DIIS}\index{Pulay's DIIS}
\item Pulay's method.  If  requested,  when  the  largest  change  in density 
matrix elements on two consecutive iterations has dropped below \index{CNVG|ff}
0.1, then routine CNVG is abandoned in  favor  of  a  multi-Fock  matrix
interpolation.   This  relies  on  the fact that the eigenvectors of the
density and Fock matrices are identical at self-consistency, so  [P.F] =  (P.F
$-$ F.P) = 0 at  SCF.  The extent to which this condition does not occur is a
measure of  the  deviance  from  self-consistency.   Pulay's  Direct Inversion
of the Iterative Subspace (DIIS) method  uses  this relationship to calculate
that linear combination of Fock matrices which minimize  [P.F].   This  new 
Fock  matrix  is  then  used  in  the  SCF calculation.

Under certain circumstances, Pulay's method  can  cause  very  slow
convergence,   but   sometimes   it   is  the  only  way  to  achieve  a
self-consistent field.  At other times the procedure  gives  a  ten-fold
increase  in  speed,  so care must be exercised in its use.  (started by the
keyword \comp{PULAY})

\item The Camp-King converger.  If  all  else  fails,  the  Camp-King
\index{Camp-King converger} converger  is  just about guaranteed to work every
time.  However, it is time-consuming, and therefore should only be started as a
last resort.

It  evaluates  that  linear  combination   of   old   and   current
eigenvectors  which  minimize the total energy.  One of its strengths is that
systems which  otherwise  oscillate  due  to  charge  surges,  e.g.\ CHO--H, 
the C--H distance being very large, will converge using this very sophisticated
converger.
\end{enumerate}

\subsubsection{Causes of failure to achieve an SCF}
\index{SCF!failure to achieve}
In a system where a biradical can form, such as ethane  decomposing into  
two   CH$_3$  units,  the  normal  RHF  procedure  can  fail  to  go
\index{Biradicaloid character}\index{BIRADICAL}\index{UHF}\index{TRIPLET}
self-consistent.  If the system has marked biradicaloid character,  then
\comp{BIRADICAL}  or \comp{UHF} and \comp{TRIPLET} can often  prove
successful.  These options rely on the assumption that two unpaired  electrons 
can  represent  the open shell part of the wave-function.

\subsection{Use of C.I.\ in Reaction Path Calculations}
\begin{figure}
\begin{makeimage}
\end{makeimage}
\begin{center}
\includegraphics{hof}
\end{center}
\caption{\label{cich2o} Effect of C.I.\ on $\Delta H_f$ in a bond-breaking
reaction (for the reaction CH$_2$O $\rightarrow$ CHO + H)}
\end{figure}

\begin{figure}
\begin{makeimage}
\end{makeimage}
\begin{center}
\includegraphics{charges}
\end{center}
\caption{\label{cich2o2} Effect of C.I.\ on the charge  in a bond-breaking
reaction (for the reaction CH$_2$O $\rightarrow$ CHO + H)}
\end{figure}

Although closed-shell methods are suitable for normal systems, when a reaction
occurs such that a bond makes or breaks, then configuration interaction
can help in the description of the system.

Consider CH$_2$O,  with  the  interatomic  distance  between carbon and  one of
the hydrogen atoms being   steadily increased.    At   first  the  covalent 
bond  will  be  strong,  and  a self-consistent field is readily  obtained.  
Gradually  the  bond  will become  more  ionic,  and  if configuration
interaction is not used, a highly strained system will result.  This exotic
system will still have a large C--H bond order, despite the fact that the C--H
distance is very large.

To a degree,  configuration interaction can correct this picture. When
\comp{C.I.=2} is used, Figure~\ref{cich2o} and \ref{cich2o2}, a more realistic
description of the dissociation is obtained.  Now the  leaving hydrogen atom
becomes neutral as the distance increases, and the energy becomes constant at
large distances.   A C--H bond in formaldehyde is being stretched.  The effect
of C.I.\ is to make the dissociated state more realistic.  Without C.I., the
energy rises continuously, and the charge on the departing hydrogen atom
becomes unrealistic.
