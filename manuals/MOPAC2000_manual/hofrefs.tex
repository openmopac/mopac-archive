\chapter{Reference Heats of formation}
\section*{Test MNDO, PM3 and AM1 compounds}
In order to verify that MOPAC is working correctly, a large  number of  tests
need to be done.   It  is  obviously impractical  to ask users to test all the
options in MOPAC.   However, users must be able to verify the basic working of
MOPAC, and to do this  the  following  tests for the elements have been
provided.

Each element can be tested by making up a data-file using estimated geometries
and running that file using MOPAC.  The optimized geometries should give rise
to heats of formation as shown in Table~\ref{hofr}.   Any difference greater
than 0.1 kcal/mole indicates a serious error  in the program.\index{Heat of
Formation!molecular standards} \index{$\Delta H_f$!molecular standards}

\subsection*{Caveats}
\begin{enumerate}
\item Geometry definitions must be correct.
\item Heats of formation  may  be  too  high  for  certain compounds.   This is
due to a poor starting geometry trapping the system in an excited  state.
(Affects ICl at times)
\end{enumerate}

\begin{table}
\caption{\label{hofr} Reference Heats of Formation}
\begin{center}
\begin{tabular}{llrrrrr}
\hline
Element & Test Compound & \multicolumn{5}{c}{Heat of Formation} \\
\cline{3-7} & & \multicolumn{1}{c}{MINDO/3} & \multicolumn{1}{c}{MNDO} &
\multicolumn{1}{c}{AM1} & \multicolumn{1}{c}{PM3} & \multicolumn{1}{c}{MNDO-$d$} \\
\hline
Hydrogen   & CH$_4$    &      -6.3  & -12.0&  -8.8 &-13.0  \\
Lithium    & LiH       &            & +23.2&      &  \\
Beryllium  & BeO       &            & +38.6&       &+53.0   \\
Boron      & BF$_3$    &    -270.2  &-261.0&-272.1*&  \\
Carbon     & CH$_4$    &      -6.3  & -12.0&  -8.8 &-13.0  \\
Nitrogen   & NH$_3$    &      -9.1  &  -6.4&  -7.3 & -3.1  \\
Oxygen     & CO$_2$    &     -95.7  & -75.1& -79.9 &-85.1  \\
Fluorine   & CF$_4$    &    -223.9  &-214.2&-225.8 &225.1  \\
Magnesium  & MgF$_2$   &            &      &       &160.7  \\
Aluminium  & AlF       &            & -83.6& -77.9 &-50.1 &-65.7 \\
Silicon    & SiH       &     +82.9  & +90.2& +84.5 &+94.6 & +89.3 \\
Phosphorus & PH$_3$    &      +2.5  &  +3.9& +10.2 & +0.2& +5.0 \\
Sulfur     & H$_2$S    &      -2.6  &  +3.8&  +1.2 & -0.9& +0.01  \\
Chlorine   & HCl       &     -21.1  & -15.3& -24.6 &-20.5  \\
Zinc       & ZnMe$_2$  &            & +19.9& +19.8 &  8.2  \\
Gallium    & GaCl$_3$  &            &      &       &-79.7  \\
Germanium  & GeF       &            & -16.4& -19.7 & -3.3  \\
Arsenic    & AsH$_3$   &            &      & +3.0  &+12.7  \\
Selenium   & SeCl$_2$  &            &      & -42.2 &-38.0  \\
Bromine    & HBr       &            &  +3.6& -10.5 & +5.3  & +2.8 \\
Cadmium    & CdCl$_2$  &            &      &       &-48.6  \\
Indium     & InCl$_3$  &            &      &       &-72.8  \\
Tin        & SnF       &            & -20.4&       &-17.5  \\
Antimony   & SbCl$_3$  &            &      & -74.8 &-72.5  \\
Tellurium  & TeH$_2$   &            &      & +12.6 &+23.8  \\
Iodine     & ICl       &            &  -6.7& -4.6  &+10.8 & +0.5 \\
Mercury    & HgCl$_2$  &            & -36.9&-44.8  &-32.7  \\
Thallium   & TlCl      &            &      &       &-13.4  \\
Lead       & PbF       &            & -22.6&       &-21.0  \\
Bismuth    & BiCl$_3$  &            &      &       &-42.6  \\
\hline
\end{tabular} \\
* Not an exhaustive test of AM1 boron.
\end{center}
\end{table}
