\section{Force Constants}\index{Force constants}\index{Internal!coordinate!force constants}\label{fc}
Internal coordinate force constants, $f_{ic}$, can be derived from the
Cartesian coordinates and the Cartesian force constant matrix by use of:
$$
f_{ic}(l) = \sum_j\sum_k
\frac{d({\rm c.c.}_j)}{d({\rm i.c.}_l)}
\frac{d^2E}{d({\rm c.c.}_j)d({\rm c.c.}_k)}
\frac{d({\rm c.c.}_k)}{d({\rm i.c.}_l)}.
$$
As with the gradients, the calculation of $\frac{d({\rm c.c.}_j)}{d({\rm
i.c.}_l)}$ is quite simple, and in done in routine JCARIN.

During the testing of this function, a minor fault in the conventional
force calculation was revealed.  To reduce any error introduced by
finite arithmetic, the Hessian matrix is symmetrized before the
vibrational frequencies and normal coordinates are calculated.  This is
done by \hyperref[pageref]{operating on the Hessian matrix}{ as
described on p.~}{}{sym_force}.

In addition to the requirement that the symmetry of the Hessian should be the
same as that of the nuclear coordinates, a second requirement is that the
diagonal elements of the Hessian should be equal to the negative of the sum of
the off-diagonal elements, that is, that:
$$
F_{ii} = -\sum_{j\ne i}F_{ij}.
$$
During the testing of the internal force constants, very small variations in
the force constants were found where no variation was expected.  This was
traced back to a failure of the above expression.  To correct this, the
diagonal terms of the force constant matrix were modified.  This resulted in a
perfect equivalence of equivalent force constants.  An incidental benefit would
be that the associated error in the calculated frequencies would be eliminated.
