\section{Geometry optimization}\index{Geometry!BFGS optimizer}
The default geometry optimizer in MOPAC uses Baker's EigenFollowing method.  If
this is {\em not} wanted, for example, if there is a need to reduce memory
demands, then the Broyden Fletcher Goldfarb Shanno method can be used.

The most common use of MOPAC is for geometry optimization. This involves
starting with an approximation to the desired geometry and, by calculating the
forces acting on the system, changing the geometry so as to lower the total
energy. The objective of geometry optimization is to achieve a structure in
which all the atoms are at equilibrium, that is, one in which the forces acting
on every atom are very small, and in which the second derivatives are
everywhere positive.  Such a geometry is called a ground state stationary
point.

\input{t_ef}
\subsection{The BFGS function optimizer}
The alternative heat of formation minimization routine in  MOPAC is a modified 
Broyden~\cite{bfgs1}-Fletcher~\cite{bfgs2}-Goldfarb~\cite{bfgs3}-Shanno~\cite{bfgs4}
or BFGS method. Minor changes were made necessary by the presence of phenomena
peculiar to chemical systems.

Starting with a user-supplied geometry $x_o$, MOPAC computes an estimate to the
inverse Hessian $H_o$. The geometry optimization proceeds by
$$
x_{k+1} = x_k+\alpha d_k,
$$
where
$$
d_k=H\,g_k ,
$$
and each element of $H$ is defined by
$$
H_{k+1}=H_k-\frac{H\ y_k\ p_k^t + p_ky_k^tH}{S}+\frac{Q(p_k\ p_k^t)}{S},
$$
where  
$$
Q=1+\frac{y_k^t\ H\ y_k}{p_k^t\ y_k},
$$
and $g_k$ is the gradient vector on step $k$.

\index{Hessian!in BFGS optimizer} Although this expression for the update of
the Hessian matrix looks very complicated, the operation can be summarized as
follows:

The initial Hessian matrix used in geometry optimization is chosen as a
diagonal matrix, with the diagonal elements determined by a simple formula
based on the gradients at two geometries.  As the optimization proceeds, the
gradients at each point are used to improve the Hessian.  In particular, the
off-diagonal elements are assigned based on the old elements and the current
gradients.

Two different methods are used to calculate the displacement of $x$ in the
direction $d$. During the initial stages of geometry optimization, a line
search is used. This proceeds as follows:

\index{Line search|(}\index{NOTHIEL} The geometry is displaced by $(\alpha/4)d$
and the energy evaluated via an SCF calculation. If this energy is lower than
the original value, then a second step of the same size is made. If it is
higher, then a step of -$(\alpha/4)d$ is made. The energy is then re-evaluated.
Given the three energies, a prediction is made as to the value of $\alpha$
which will yield the  minimum  value  of the energy in the direction $d$. Of
course, the size of the steps are constrained so that the system would not
suddenly become unrealistic (e.g., break bonds, superimpose atoms, etc.).
Similarly, the contingency in which the energy versus $\alpha$ function is
inverse parabolic is considered, as are rarely-encountered curves, e.g., almost
perfectly linear regressions. By default, Thiel's FSTMIN \index{Thiel@{\bf
Thiel, Walter}} technique is used~\cite{fstmin}.  This uses gradient
information from the starting point of the search, and the calculated $\Delta
H_f$, to decide when to end the line search.  If \comp{NOTHIEL} is specified,
the older line-search is used, in which case the search is stopped when the
drop in energy on any step becomes less than 5\% of the total drop or 0.5
kcal/mol, whichever is smaller.

An  important modification has  been made to the BFGS routine.  For the
line-search, Thiel's FSTMIN technique is used. This  modification make the
algorithm run faster most of the time.  However, one unfortunate result of
these changes is that there is no guarantee that as  the cycles increase, the
energy will drop monotonically.  If the calculation does not converge on a
stationary point, then re-run the job with \comp{NOTHIEL}.

As the geometry converges on a local minimum, the prediction of the search
direction becomes less accurate. There are many reasons for this. For example,
the finite precision of the SCF calculation may lead to errors in the density
matrix, or finite step sizes in the derivative calculation (if analytical
derivatives are not used) may result in errors in the derivatives. For whatever
reason, the gradient norm and energy minimum may not coincide. The difference
is typically less than 0.00001 kcal/mol and less than 0.05 units of gradient
norm. 

Normally, the initial guess to $H$, the inverse Hessian, is the unit matrix.
However, in chemical systems where the second derivatives are very large, use
of the unit matrix would result in large changes in the geometry. Thus a
slightly elongated bond length could, in the first step, change from 1.6\AA\ 
to -6.5\AA  . To prevent this catastrophe, the initial geometry is perturbed by
a small amount, thus
$$
   x_1=x_0+0.01\times {\rm sign}(g_0),
$$
from which a trial inverse Hessian can be constructed:
$$
  H_1(i,i)=0.01\times {\rm sign}(g_0(i))/y_1(i).
$$
A negative value for $H_1(i,i)$ would lead to difficulties
within the BFGS optimization. To avoid this, $H_1(i,i)$ is set
to $0.06/{\rm abs}(g(i))$ whenever sign$(g_0(i))/y_1(i)$ is negative.

As the optimization proceeds, the inverse Hessian matrix becomes more accurate.
However, as the geometry steadily changes, the inverse Hessian will contain
information which does not reflect the current point. This can lead to the
predicted search direction vector making an angle of more than $90^{\circ}$
with the gradient vector. In other words, the search direction vector may point
uphill in energy. To guard against this, the inverse Hessian is re-initialized
whenever the cosine of the angle between the search direction and the gradient
vector drops below 0.05.

Originally the Davidson-Fletcher-Powell technique was used, but in rare
instances it failed to work satisfactorily. The BFGS formula appears to work as
well as or better than the DFP method most of the time. In the infrequent case
when the DFP is more efficient, the increase in efficiency of the DFP can
usually be traced to a fortuitous choice of a search direction. Small changes
in starting conditions can destroy this accidental increased efficiency and
make the BFGS method appear more efficient. A keyword, \comp{DFP}, is provided
to allow the DFP optimizer to be used.




\subsection{Optimization of one unknown}
If a system has exactly one coordinate to be optimized, then obviously one
line-search will optimize the geometry.  Because of this, the geometry
optimization is done a little differently.  Given the initial geometry, the
$\Delta H_f$ is calculated, and the line-search started. Unlike the normal
line-search, however, the search is not stopped when the minimum is almost
reached, instead, the minimum is located with quite high precision.  After the
line-search is complete, the gradients are not \index{GRADIENTS} calculated
(unless requested by \comp{GRADIENTS}). Instead, it is assumed that the
gradient is small, and the results are output.  This saves some time. However,
if \comp{GRADIENTS} is {\em not} present, and the geometry is not at a
stationary point (because other coordinates are not optimized), then the
warning message that the geometry is not at a stationary point will not be
printed.

\index{Line search|)}

\subsection{Considerations in Geometry Optimization}
The default settings in MOPAC are designed to allow most systems to be
optimized in an efficient way.  Quite often, however, problems arise.  The
following  notes are intended as background material for use when things go
wrong.

\subsubsection{Overriding the default options}
In the EigenFollowing geometry optimization method, the geometry is changed on
each cycle; if the $\Delta H_f$ decreases, the cycle is completed.  If it does
not drop, the step-size is reduced, and the $\Delta H_f$ recalculated.  Only
when the $\Delta H_f$ decreases, compared to the previous cycle, is the current
cycle considered to be successful.  During the calculation, the confidence
level or trust radius is continuously checked.  If this becomes too small, the
calculation will be stopped.  This can readily happen if (a) the geometry was
already almost  optimized; (b) a reaction path or grid calculation is being
performed; (c) if the geometry is in internal coordinates and ``big rings'' are
involved; or (d) if the  gradients are not correctly calculated (in a
complicated C.I., for example).

For cases (a) and (b), add \comp{LET} and \comp{DDMIN=0}.  In case (c) use
either mixed coordinates or entirely Cartesian coordinates. Case (d) is
difficult---if nothing else works, add \comp{NOANCI}; this will always cause
the derivatives to be correctly calculated, but will also use a lot of time.

Adding \comp{LET} and \comp{DDMIN=0} is often very effective, particularly when
reaction paths are being calculated.  The first geometry optimization might
take more cycles, but the resulting Hessian matrix is better tempered, and
subsequent steps are generally more efficient.

\subsubsection{Locating Transition States}\index{Locating transition states}
\index{Transition states!locating}\index{Narcissistic reactions}
Unlike optimizing ground states, locating transition states involves deciding on 
an efficient strategy.  In general, there are three stages in locating 
transition states:
\begin{enumerate}
\item Generating a geometry in the region of the transition state.
\item Refining the transition state geometry.
\item Characterizing the transition state.
\end{enumerate}
Of these three, the first is by far the most difficult.  The following 
approaches are suggested as potential strategies for generating a geometry in 
the region of the transition state.

{\bf For narcissistic reactions (reactions in which the reactants and 
products are the same, e.g.\ the inversion of ammonia.}
\begin{itemize}
\item Use  geometry constrains, e.g.\ \comp{SYMMETRY}, to lock the geometry in
the symmetry of the potential transition state.
\item Minimize the $\Delta H_f$.
\item Verify that the system is a transition state.  If it has more than one 
negative force constant, use another method.
\end{itemize}

{\bf For a bond making-bond breaking reaction (e.g., an S$_{N^2}$ reaction)}
\begin{itemize}
\item  Use \comp{SYMMETRY} to set the two bonds equal.  If does {\em not} 
matter that the bonds are of different type. For example, to locate the
transition  state for Br$^-$ reacting with CH$_4$ to give CH$_3$Br, the C--Br
and C--H bonds  would be set equal.
\item  Optimize the geometry, to minimize the $\Delta H_f$.  Any geometry
optimizer could be used, but of course the default optimizer should be tried
first.
\item Remove the symmetry constraint, and locate the transition state using
\comp{TS}.
At this point, the main geometric change is to adjust the two bond lengths
involved in the reaction.
\end{itemize}

{\bf For barriers to rotation, inversion, or other simple reaction that 
does not involve making or breaking bonds}
\begin{itemize}
\item Optimize the starting geometry.
\item Optimize the final geometry.
\item Identify the coordinate that corresponds to the reaction. This is likely
to be an angle or a dihedral.
\item Starting with the higher energy geometry, use a \hyperref[pageref]{path option}{ (see 
p.~}{)}{rpaths} to drive the reaction in the direction of the other
geometry.   Use about 20 points, and go about half way to the other
geometry---the transition state is likely to be between the higher energy
geometry and the half-way point.
\item From the output, locate the highest energy point---this will be near to
the transition state.
\item Starting with the geometry of the highest energy point, repeat the path 
calculation.  Use smaller steps (0.1 times the previous step is usually OK),
and again do 20 points.
\item Inspect the reaction gradient.  It should drop as the transition state is
approached.  If it does, then use \comp{TS} to refine the transition state.
\end{itemize}

{\bf For bond making or bond breaking reactions}
\begin{itemize}
\item Identify the reaction coordinate (the bond that makes or breaks)
\item Use a \hyperref[pageref]{path calculation}{ (see 
p.~}{)}{rpaths} to drive the reaction.
\item The geometry of the highest point on the reaction path should then be 
used to start a \comp{TS} calculation.
\end{itemize}

{\bf For complicated reactions (e.g.\ Diels Alder)}
For these systems, the \comp{SADDLE} calculation is a suitable method.
\begin{itemize}
\item Optimize the reactant geometry.
\item Using the same atoms in the same sequence, optimize the product geometry.
\item Run the \comp{SADDLE} calculation.
\item If the calculation ends because ``both reactants and products are on
 the same side of the transition state,'' use two of the geometries to set
up a new \comp{SADDLE} calculation.  Use a smaller \comp{BAR=$n.nn$}, e.g.,
\comp{BAR=0.03}, and re-run the calculation.  If CPU time is not important,
run the original data set with \comp{BAR=0.03}.
\item Use the final geometry, or the highest energy geometry, if the 
\comp{SADDLE} does not run to completion, as the starting point for a 
\comp{TS} calculation.
\end{itemize}

