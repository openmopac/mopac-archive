\chapter{Program}
The logic within MOPAC is best understood by use of flow-diagrams.

There are two main sequences, geometric and electronic.  These join only  at
one  common  subroutine COMPFG.  It is possible, therefore, to understand the
geometric or electronic sections  in  isolation,  without having studied the
other section.

\section{Main geometric sequence}
\index{MOPAC! geometric structure}
The main geometric sequence in MOPAC is given in Figure~\ref{picgeo}.
\begin{figure}
\begin{makeimage}
\end{makeimage}
\begin{center}
\hspace*{-1.0in}
%    ()(n1,n2) n1 = start horizontal, distance left
%              n2 = start vertical, distance down
%  -4,2 = 4 inches in and 2 inches down
\setlength{\unitlength}{1in}
\framebox{
\begin{picture}(6.0,4.0)(0.0,0.0)
\put(2.6,0.2){\framebox(0.8,0.4){COMPFG}}
\put(3.0,0.6){\line(0,1){0.2}}
\put(0.5,0.8){\line(1,0){4.7}}
\put(0.5,0.8){\line(0,1){0.2}}
\put(1.1,0.8){\line(0,1){0.8}}
\put(1.7,0.8){\line(0,1){0.2}}
\put(2.3,0.8){\line(0,1){0.8}}
\put(2.6,0.8){\line(0,1){1.4}}
\put(3.1,0.8){\line(0,1){0.8}}
\put(3.9,0.8){\line(0,1){0.2}}
\put(4.4,0.8){\line(0,1){0.8}}
\put(5.2,0.8){\line(0,1){0.8}}
\put(0.1,1.0){\framebox(0.8,0.4){SEARCH}}
\put(0.5,1.6){\framebox(0.8,0.4){SIGMA}}
\put(1.3,1.0){\framebox(0.8,0.4){LOCMIN}}
\put(1.7,1.6){\framebox(0.8,0.4){NLLSQ}}
\put(2.1,2.2){\framebox(0.8,0.4){FORCE}}
\put(2.7,1.6){\framebox(0.8,0.4){FMAT}}
\put(3.4,1.0){\framebox(0.8,0.4){LINMIN}}
\put(3.8,1.6){\framebox(0.8,0.4){FLEPO}}
\put(3.2,2.4){\framebox(0.8,0.4){PATHK}}
\put(4.2,2.4){\framebox(0.8,0.4){PATHS}}
\put(5.2,2.4){\framebox(0.8,0.4){REACT1}}
\put(4.8,1.6){\framebox(0.8,0.4){EF}}
\put(0.7,1.4){\line(0,1){0.2}}
\put(1.9,1.4){\line(0,1){0.2}}
\put(4.0,1.4){\line(0,1){0.2}}
\put(2.6,0.8){\line(0,1){1.4}}
\put(2.3,2.0){\line(0,1){0.2}}
\put(2.8,2.0){\line(0,1){0.2}}
\put(3.0,2.2){\line(1,0){2.6}}
\put(3.6,2.2){\line(0,1){0.2}}
\put(4.2,2.0){\line(0,1){0.2}}
\put(4.6,2.2){\line(0,1){0.2}}
\put(5.6,2.2){\line(0,1){0.2}}
\put(5.2,2.2){\line(0,-1){0.2}}
\put(2.9,2.3){\line(1,0){0.1}}
\put(3.0,2.2){\line(0,1){0.1}}
\put(0.9,2.0){\line(0,1){1.0}}
\put(1.9,2.0){\line(0,1){1.0}}
\put(2.6,2.6){\line(0,1){0.4}}
\put(3.1,2.2){\line(0,1){0.8}}
\put(3.6,2.8){\line(0,1){0.2}}
\put(4.6,2.8){\line(0,1){0.2}}
\put(5.6,2.8){\line(0,1){0.2}}
\put(0.9,3.0){\line(1,0){4.7}}
\put(3.0,3.0){\line(0,1){0.2}}
\put(2.6,3.2){\framebox(0.8,0.4){MAIN}}
\put(1.0,2.4){\framebox(0.8,0.4){IRC/DRC}}
\put(1.4,2.8){\line(0,1){0.2}}
\put(1.4,1.5){\line(0,1){0.9}}
\put(1.1,1.5){\line(1,0){0.3}}
\end{picture}
}

\end{center}
\caption{\label{picgeo}Diagram of Main Geometric Sequence in MOPAC}
\end{figure}

\section{Main electronic flow}
\index{MOPAC!electronic structure}
The main conventional electronic sequence in MOPAC is given in
Figure~\ref{picelec}.
\begin{figure}
\begin{makeimage}
\end{makeimage}
\begin{center}
\hspace*{-0.5in}
%    ()(n1,n2) n1 = start horizontal, distance left
%              n2 = start vertical, distance down
%  -4,2 = 4 inches in and 2 inches down
\setlength{\unitlength}{1in}
\framebox{
\begin{picture}(6.0,5.0)(0.0,-0.0)
\put(2.6,4.4){\framebox(0.8,0.4){COMPFG}}
\put(3.0,4.2){\line(0,1){0.2}}
\put(0.6,4.2){\line(1,0){5.0}}
\put(0.6,4.0){\line(0,1){0.2}}
\put(1.6,4.0){\line(0,1){0.2}}
\put(2.6,4.0){\line(0,1){0.2}}
\put(3.6,4.0){\line(0,1){0.2}}
\put(4.6,2.6){\line(0,1){1.6}}
\put(5.6,4.0){\line(0,1){0.2}}
\put(0.2,3.6){\framebox(0.8,0.4){HCORE}}
\put(1.2,3.6){\framebox(0.8,0.4){GMETRY}}
\put(2.2,3.6){\framebox(0.8,0.4){DERIV}}
\put(3.2,3.6){\framebox(0.8,0.4){SYMTRY}}
\put(5.2,3.6){\framebox(0.8,0.4){MECIP}}
\put(2.6,3.4){\line(0,1){0.2}}
\put(1.6,3.4){\line(1,0){2.0}}
\put(1.6,3.2){\line(0,1){0.2}}
\put(2.6,3.2){\line(0,1){0.2}}
\put(3.6,3.2){\line(0,1){0.2}}
\put(1.2,2.8){\framebox(0.8,0.4){DCART}}
\put(2.2,2.8){\framebox(0.8,0.4){DERNVO}}
\put(3.2,2.8){\framebox(0.8,0.4){DERITR}}
\put(1.6,2.6){\line(0,1){0.2}}
%\put(2.7,2.6){\line(0,1){0.2}}
\put(3.8,2.6){\line(0,1){0.2}}
\put(3.8,2.6){\line(1,0){0.8}}
\put(4.2,2.4){\line(0,1){0.2}}
\put(3.8,2.0){\framebox(0.8,0.4){ITER}}
\put(4.2,1.8){\line(0,1){0.2}}
\put(1.2,2.2){\framebox(0.8,0.4){DHC}}
\put(1.4,2.0){\line(0,1){0.2}}
\put(0.6,3.6){\line(0,-1){1.6}}
\put(0.5,2.0){\line(1,0){1.0}}
\put(0.5,1.8){\line(0,1){0.2}}
\put(0.1,1.4){\framebox(0.8,0.4){ROTATE}}
\put(0.5,1.2){\line(0,1){0.2}}
\put(0.1,0.8){\framebox(0.8,0.4){REPP}}
\put(1.5,1.8){\line(0,1){0.2}}
\put(1.1,1.4){\framebox(0.8,0.4){H1ELEC}}
\put(1.5,1.2){\line(0,1){0.2}}
\put(1.1,0.8){\framebox(0.8,0.4){DIAT}}
\put(1.3,0.6){\line(0,1){0.2}}
\put(0.6,0.2){\framebox(0.8,0.4){COE}}
\put(1.7,0.6){\line(0,1){0.2}}
\put(1.6,0.2){\framebox(0.8,0.4){SS}}
\put(4.6,3.6){\line(0,-1){1.0}}
\put(2.6,1.8){\line(1,0){3.0}}
\put(2.6,1.6){\line(0,1){0.2}}
\put(3.6,1.6){\line(0,1){0.2}}
\put(4.6,1.6){\line(0,1){0.2}}
\put(5.6,1.6){\line(0,1){0.2}}
\put(2.2,1.2){\framebox(0.8,0.4){FOCK1}}
\put(3.2,1.2){\framebox(0.8,0.4){FOCK2}}
\put(4.2,1.2){\framebox(0.8,0.4){HQRII}}
\put(5.2,1.2){\framebox(0.8,0.4){DIAG}}
\put(3.1,1.0){\line(0,1){0.8}}
\put(4.1,1.0){\line(0,1){0.8}}
\put(5.1,1.0){\line(0,1){0.8}}
\put(2.7,0.6){\framebox(0.8,0.4){MECI}}
\put(3.7,0.6){\framebox(0.8,0.4){INTERP}}
\put(4.7,0.6){\framebox(0.8,0.4){DENSIT}}
\end{picture}
}

\end{center}
\caption{\label{picelec}Diagram of Main Conventional Electronic Sequence in
MOPAC}
\end{figure}

The LMO method is similar to the conventional method.  The main changes are:
\begin{itemize}
\item \comp{HQRII} and \comp{DIAG} are replaced by \comp{DIAGG}
(the annihilation operation).
\item \comp{MECI},  \comp{MECIP}, etc, and \comp{INTERP} are not available.
\item Extra subroutines, used in creating the initial LMOs, are present.
\end{itemize}

\section{Control within MOPAC}
\index{MOPAC!programming policy}
Almost all the control information is passed {\em via} the  single  datum
``KEYWRD'',  a  string  of 241 characters, which is read in at the start
of the job.

Each subroutine is made independent, as far as  possible,  even  at the
expense  of  extra code or calculation.  Thus, for example, the SCF
\index{SCF!criterion}\index{ITER}\index{DERIV} criterion is set in  subroutine
ITER,  and  nowhere  else.   Similarly, subroutine  DERIV  has  exclusive
control  of  the  step  size  in  the finite-difference calculation of the
energy derivatives.  If the default values  are  to  be reset, then the new
value is supplied in KEYWRD, and \index{KEYWRD}\index{READA}\index{INDEX}
extracted {\em via} INDEX and READA.  The flow of control is  decided  by
the presence of various keywords in KEYWRD.

When a subroutine is called, it assumes that all data required  for its
operation  are  available  in  either  modules or arguments. Normally no
check is made as to the validity of the data received.   All data  are
``owned'' by one, and only one, subroutine.  Ownership means the \index{STATE}
implied permission and ability to change the data.  Thus  STATE  ``owns'' the
number  of  atomic orbitals, in that it calculates this number, and
\index{NORBS} stores it in the variable NORBS.  Many subroutines use NORBS,
but  none of  them  is  allowed  to  change it.  For obvious reasons no
exceptions should be made to this rule.   To  illustrate  the  usefulness  of
this convention,  consider the eigenvectors, \comp{C} and \comp{CBETA}.  These
are owned by ITER.  Before ITER is called, \comp{C} and \comp{CBETA} are not
calculated, after ITER has  been called \comp{C} and \comp{CBETA} are known, so
any subroutine  which needs to use the eigenvectors can do so in the certain
knowledge that they exist.

Any variables which are only  used  within  a  subroutine  are  not passed
outside the subroutine unless an overriding reason exists.  This
\index{PULAY}\index{CNVG} is found in PULAY and CNVG, among  others  where
arrays  used  to  hold spin-dependent  data  are used, and these cannot
conveniently be defined within the subroutines.  In these  examples,  the
relevant  arrays  are ``owned'' by ITER.

A general subroutine, of which ITER  is  a  good  example,  handles three
kinds of data:  First, data which the subroutine is going to work on, for
example  the  one  and  two  electron  matrices;  second,  data necessary  to
manipulate  the  first set of data, such as the number of atomic orbitals;
third, the calculated quantities, here  the  electronic energy, and the density
and Fock matrices.

Reference data are entered into a subroutine by way of modules.
\index{Reference data}  This is to emphasize their peripheral role.
Thus the number of orbitals, while essential to ITER, is not central to the
task it has  to perform, and is passed through a module.

Data the subroutine is going to work on are passed via the argument list.  Thus
the one and two electron matrices, which are the main reason for ITER's
existence, are entered as two of the four arguments.  As ITER does  not  own
these  matrices it can use them but may not change their contents.  The other
argument is EE, the electronic energy.  EE is owned by ITER even though it
first appears before ITER is called.

\section{Arrays.}
\subsection{Array Specification.}\index{Array specification}

In MOPAC, most arrays are created dynamically. The main exceptions are a few
module arrays: \comp{GEO}, \comp{COORD}, \comp{LABELS},  \comp{NAT},
\comp{NA}, \comp{NB}, and \comp{NC}.  These arrays are only used at the start
of the run, and once the job is fully running, they are abandoned.  These
arrays have the same names as dynamic arrays, but, because they are used only
at the very start of the run, there is no ambiguity.   The size of these arrays
is defined by \comp{NUMATM} in the included file \comp{sizes.F90}.   This
quantity, \comp{NUMATM} is equivalent to the MOPAC~93 parameters  \comp{MAXHEV}
and \comp{MAXLIT}.  By default, MOPAC can run systems of up to 20,000 atoms.
Although this is large, the small number of static arrays means that the memory
required for a small system is still small---less than 8  megabytes.

Most of the memory is assigned dynamically.  Once the data are read in---using
the static arrays---the dynamic memory arrays are created as needed.



\section{Names and Storage of Variables in MOPAC}

Most subroutines in MOPAC contain the module
\comp{common\_systm}. This module contains variables
that are central to the transmission of information throughout
the program.
%\subsection*{\comp{NCORE(100)}}
Almost all of the arrays that depend on the size of the system  are assigned
dynamically.  At the start of the calculation, two large blocks of memory are
reserved. One block, \comp{MCORE}, will be used to hold all the integer arrays,
and the other, \comp{CCORE},  is used to hold all the real arrays. The starting
addresses of the arrays are given in \comp{NCORE}. Thus, for example, the
first dynamic real array assigned is \comp{GEO}, and the corresponding starting
address can be found in \comp{NCORE(3)}.  The order in which the arrays are
stored in \comp{NCORE} is not important, however, each array is associated with
a unique address in \comp{NCORE}.  A full list of the 100 addresses in 
\comp{NCORE} is as follows:

\begin{table}
\caption{\label{asl} Array Storage Locations in MOPAC}
\begin{center}
\begin{tabular}{lllllllll} \hline
MOZYME&MOPAC&No.&MOZYME&MOPAC&No.&&MOPAC&No.\\\hline
Special &  Special &  1 & \comp{TXTATM} & \comp{TXTATM} & 34 & & \comp{NAR} & 67\\
Special &  Special &  2 & \comp{ISORT} & \comp{TOM} & 35 & & \comp{DIRVEC} & 68\\
\comp{GEO} &  \comp{GEO} &  3 & & & 36 & & \comp{BH} & 69\\
\comp{NA,NB,NC} & \comp{NA,NB,NC} &  4 & & & 37 & & \comp{IATSP} & 70\\
\comp{XPARAM} &  \comp{XPARAM} & 5 & \comp{EIGF/EIGV} & \comp{EIGS} & 38 & & \comp{NN} & 71\\ 
\comp{LOC} & \comp{LOC} & 6 & & \comp{EIGB} & 39 & & \comp{QDEN} & 72\\
\comp{COORD} & \comp{COORD} & 7 & \comp{IOPT} & & 40 & & \comp{CH} & 73\\
\comp{LABELS} & \comp{LABELS} & 8 & \comp{JOPT} & \comp{JOPT} & 41 & & \comp{NSETF} & 74\\
\comp{NAT} &  \comp{NAT} &  9 & \comp{OLDXYZ} & & 42 & & \comp{TOM} & 75\\
\comp{NFIRST/NLAST} &  \comp{NFIRST/NLAST} & 10 & \comp{DXYZ} & \comp{DXYZ} & 43 & & & 76\\
\comp{USPD} &  \comp{USPD} & 11 & \comp{PARTP} & PA & 44 & & & 77\\
\comp{PSPD} &  \comp{PSPD} & 12 & \comp{PARTH} & \comp{PB} & 45 & & & 78\\
\comp{IORBS} &  \comp{IORBS} & 13 & \comp{PARTF} & \comp{FB} & 46 & & & 79\\
\comp{IJBO} &  \comp{VECTCI} & 14 & \comp{KOPT} & \comp{KOPT} & 47 & & & 80\\
\comp{H} & \comp{H} & 15 & & & 48 & & \comp{AMAT} & 81\\
\comp{W} & \comp{W} & 16 & & \comp{HESINV} & 49 & & \comp{BMAT} & 82\\
\comp{P} & \comp{P} & 17 & & \comp{JELEM} & 50 & & \comp{CMAT} & 83\\
\comp{F} & \comp{F} & 18 & & \comp{ATMASS} & 51 & & \comp{COSURF} & 84\\
\comp{NCE} & \comp{CONF} & 19 & & \comp{REACT} & 52 & & \comp{ISUDE} & 85\\
\comp{NCF} & & 20 & & \comp{C0} & 53 & & \comp{SUDE} & 86\\
\comp{NCOCC} & \comp{WK} & 21 & & \comp{NC} & 54 & & \comp{PHINET} & 87\\
\comp{NCVIR} & \comp{HQ} & 22 & & \comp{INTERP} & 55 & & \comp{QSCNET} & 88\\
\comp{GRAD} & \comp{GRAD} & 23 & & \comp{ERRFN} & 56 & & \comp{QDENET} & 89\\
\comp{NNCF} & \comp{NSET} & 24 & & \comp{AIDER} & 57 & & \comp{IPIDEN} & 90\\
\comp{NNCE} & & 25 & & \comp{ALPARM} & 58 & & \comp{GDEN} & 91\\
\comp{ICOCC} & \comp{XY} & 26 & & & 59 & & \comp{QSCAT} & 92\\
\comp{ICVIR} & \comp{CIMAT} & 27 & & \comp{NAMO} & 60 & & \comp{IDENET}  & 93\\
\comp{COCC} & \comp{C} & 28 & & \comp{JNDEX}  & 61 & & \comp{ARAT} & 94\\
\comp{CVIR} & \comp{CB} & 29 & & \comp{IPO} & 62 & & \comp{WK} & 95\\
\comp{NBOND} & \comp{DIJKL} & 30 & & \comp{DXYZR} & 63 & & & 96\\
\comp{IFACT/I1FACT} & \comp{IFACT/I1FACT} & 31 & & \comp{XPAREF} & 64 & & Special & 97\\
\comp{IBONDS} & \comp{NPERMA} & 32 & & \comp{PROFIL} & 65 & & Special& 98\\
\comp{NFMO} & \comp{NPERMB} & 33 & & \comp{SRAD} & 66 & & Special& 99\\
 & & & & & & & Special& 100\\
\hline
\end{tabular}
\end{center}
\end{table}

\begin{enumerate}
\item {\bf Start of Real:}  Not an array address, but the starting address of
the first unused element of \comp{CCORE}.  Any new permanent real array would
be given the starting address \comp{NCORE(1)}. At the start of the calculation,
\comp{NCORE(1)=1}.
\item {\bf Start of Integer:}  Not an array address, but the starting address
of the first unused element of \comp{MCORE}.  Any new permanent integer array
would be given the starting address \comp{NCORE(2)}. At the start of the
calculation, \comp{NCORE(2)=1}.
\item \comp{GEO:} The internal coordinates. (3$*$NATOMS)
\item \comp{NA,NB,NC:} The connectivity indices.  Set in \comp{COPY1}. If an
atom position is defined in Cartesian coordinates, then the corresponding
NA($i$), NB($i$), and NC($i$) would all be zero. (3$*$NATOMS)
\item \comp{XPARAM:} The parameters to be optimized. (3$*$NATOMS)
\item \comp{LOC:} An array of size 2 by 3$\times$\comp{NATOMS}, \comp{LOC} is
the list of atom coordinates which are to be optimized.  Each pair of elements
in \comp{LOC} represents the atom  and the coordinate (bond length, angle or
dihedral, or $x$, $y$, or $z$). If all three coordinates of an atom are to be
optimized, then six array elements in \comp{LOC} are used.   Set in
\comp{COPY2}. In  \comp{XPARAM($n$)}, the associated atom would be
\comp{LOC(1,n)}, and the associated coordinate would be \comp{LOC(2,n}).
(6$*$NATOMS)
\item \comp{COORD:} A two dimensional array holding the Cartesian coordinates
of all real atoms, stored as $x$, $y$, $z$ for each atom, in turn.  The order
of the atoms is the same as that in the supplied data set. Units: \AA ngstroms.
Although there are only 3$*$NUMAT of these, during the conversion from
\comp{GEO}, extra storage is  needed to hold the Cartesian coordinates of the
dummy atoms and translation vectors. (3$*$NATOMS)
\item \comp{LABELS:} A one-dimensional array of labels of all atoms, real  and
dummy.  If there are no dummy atoms, this array is the same as the \comp{NAT} 
array.  Set in \comp{COPY1}, \comp{FMAT}, \comp{FORCE}, \comp{POLAR},
\comp{REACT1} and \comp{GEOCHK}. These are integers in the range 1 to 107.
(3$*$NATOMS)
\item \comp{NAT:} The labels of all real atoms. NAT($i$) holds the atomic
number of the $i$'th real atom. Set in \comp{MOLDAT} and \comp{RESEQ}. (NUMAT)
\item \comp{NFIRST, NLAST:} The starting and stopping atomic orbital indices
for all real atoms.  If the first three atoms were C, H, and +, then the values
of (\comp{NFIRST, NLAST}) would be (1,4), (5,5), and (6,5).  Note that the
number of atomic orbitals on an atom $i$ is given by  NLAST($i$)-NFIRST($i$)+1.
(NUMAT)
\item \comp{USPD:} The initial values of the one-electron one-center energies
for all atoms.  (NORBS)
\item \comp{PSPD:} The initial atomic orbital occupancies.  (NORBS)
\item \comp{IORBS:} The number of atomic orbitals on each atom. IORBS($i$) is
the same as NLAST($i$)-NFIRST($i$)+1. (NUMAT)
\item 
\begin{description}
\item[(MOZYME) \comp{IJBO:}] A two-dimensional array giving the starting 
addresses of every calculated atom-pair in \comp{H}, \comp{F}, \comp{P},
\comp{PARTH}, \comp{PARTF}, and  \comp{PARTP}. If atom-pair $i,j$ is
calculated, the starting address is defined as  IJBO($i,j$)+1.  For the first
atom, atom 1, the starting address is IJBO(1,1)+1 = 1. If an atom-pair is not
calculated, the associated array element in \comp{IJBO} is set to \comp{--2} if
electrostatic polarization terms are to be used (see \hyperref{\comp{CUTOF2} and
\comp{CUTOF1}}{, p.~}{}{cutoff}), or to \comp{--1} is only simple
electrostatic terms are to be considered.

\comp{IJBO} is set in subroutine \comp{GETLIM} only.  Once set, \comp{IJBO} is
never changed.
\item[\comp{VECTCI:}] The state vectors (eigenvectors of the C.I. matrix) for
the root requested. (5*LAB)
\end{description}

\item \comp{H:} The one-electron integrals.  \index{H Matrix!structure} The way
in which the array elements are stored in conventional and LMO methods is
different.  In conventional work, the density (\comp{P}), one-electron
(\comp{H}), and Fock (\comp{F}) matrices are stored in packed, lower-half,
triangular form, as shown in Fig~\ref{plhtm}.
\begin{figure}
\begin{makeimage}
\end{makeimage}
\begin{verbatim}
       
            Conventional Method                   LMO Method
      
           s px py pz    s px py pz               s px py pz    s px py pz

       s   1                                   s  1
      px   2  3                               px  2  3
      py   4  5  6                            py  4  5  6
      pz   7  8  9 10                         pz  7  8  9 10

       s  11 12 13 14   15                     s 11 12 13 14   27
      px  16 17 18 19   20 21                 px 15 16 17 18   28 29
      py  22 23 24 25   26 27 28              py 19 20 21 22   30 31 32
      pz  29 30 31 32   33 34 35 36           pz 23 24 25 26   33 34 35 36
\end{verbatim}
\begin{center}
Order of occurrence of array elements for two atoms, each with a $s-p^3$ basis
set.
\end{center}
\caption{\label{plhtm} Order of Storage of Matrix Elements}
\end{figure}

In conventional work, the address of an array element involving atomic orbitals
$\lambda$ and $\sigma$ can be determined from the order in which they occur in
the basis set for the system.  If $\lambda$ is the $i$'th atomic orbital, and
$\sigma$ is the $j$'th, with $i < j$, then the address of the array element is
given by:
$$
\lambda,\sigma  = P((j(j+1))/2+i).
$$

In LMO work, the order in which the array elements in \comp{P},  \comp{H}, and
\comp{F} are stored is the same, but is different from that used in
conventional work. The starting address of the array elements involving atoms
$A$ and $B$ is stored in the two-dimensional array \comp{IJBO}. If  $\lambda$
is the $i$'th atomic orbital on atom $A$ and  $\sigma$ is the $j$'th atomic
orbital on atom $B$, and $A$ occurs before $B$, then the address, $\lambda
\sigma$, of the array element is:
$$
\lambda,\sigma = P(IJBO(A,B)+(i-1)\times N_a+j),
$$
in which $N_a$ is the number of atomic orbitals on atom $A$.

If both $\lambda$ and $\sigma$ are on the same atom, then the address is given
by:
$$
\lambda,\sigma = P(IJBO(A,B)+(i(i-1))/2+j).
$$
Note that the first element of any diatomic or monatomic interaction is given
by \comp{IJBO(A,B)}+1.  As a result of this, the first entry in the 
\comp{IJBO} array, \comp{IJBO(1,1)} is zero.

The reason that elements in arrays in LMO methods are stored differently arises
from the fact that not all diatomic interactions are considered. If two atoms
are sufficiently far apart, then the corresponding array elements are not
considered.  This is indicated in the \comp{IJBO} array by the presence of a
negative number.  By ignoring such terms, a large saving can be made in the
amount of memory needed. (MPACK)

\item \comp{W:} The two-electron integrals. 
\begin{description}
\item[(MOPAC) \comp{W:}]  A one-dimensional array of the two-electron
integrals. The number of these integrals (the size of \comp{W}) is $LM61^2$.
\item[(MOZYME) \comp{W:}]\label{n16} A one-dimensional array of the
two-electron integrals. The number of these integrals (the size of \comp{W}) is
given as the sum of the following quantities:

\begin{itemize}
\item 2025 for each pair of atoms with $d$-orbitals calculated in \comp{H}.
\item 450 for each $d$-orbital - heavy atom pair  calculated in \comp{H}.
\item 45 for each $d$-orbital - light atom pair  calculated in \comp{H}.
\item 100 for each pair of heavy atoms calculated in \comp{H}.
\item 10 for each heavy-light pair of atoms calculated in \comp{H}.
\item 1 for each pair of light atoms calculated in \comp{H}.
\item 7 for each pair of heavy or $d$-orbital atoms which are {\em not} 
calculated in \comp{H}, and which do have polarization functions.
\item 4 for each pair of atoms, one heavy or $d$-orbital and one light,  which 
are {\em not} calculated in \comp{H}, and which do have polarization functions.
\item 1 for each pair of atoms which are {\em not} calculated in \comp{H}, and
which do not have polarization functions.
\end{itemize}

%what's missing here???
eV. Set in \comp{OUTER2} and \comp{ROTATE} only, and not modified during the
SCF calculation. (N2ELEC)
\end{description}
\item \comp{P:} The density matrix. (MPACK)
\item \comp{F:} The Fock matrix. (MPACK)
\item 
\begin{description}
\item[(MOPAC) \comp{CONF:}] The state eigenvectors.  This is the
orthonormal set of linear combination of configurations that form
the states. They are constructed in \comp{MECI}. 
\item[(MOZYME) \comp{NCE:}]
The number of atoms involved in the virtual LMOs.
(NVIR)
\end{description}
\item {\bf (MOZYME) \comp{NCF:}}  The number of atoms involved in the occupied
LMOs. (NOCC)
\item 
\begin{description}
\item[(MOPAC) \comp{WK:}] The two-electron two-center exchange integrals used
in polymer and other extended solids work.  This array is not used in
calculations on molecules.  (N2ELEC) 
\item[(MOZYME) \comp{NCOCC:}] The starting addresses of the atomic orbital
coefficients in the occupied LMOs.  The starting address for LMO $i$ is
NCOCC($i$)+1. (NOCC)
\end{description}
\item 
\begin{description}
\item[(MOPAC) \comp{HQ:}] Used by the Tomasi solvation method. (MPACK)
\item[(MOZYME) \comp{NCVIR:}] The starting addresses of the atomic orbital
coefficients in the virtual LMOs The starting address for LMO $i$ is
NCVIR($i$)+1. (NVIR)
\end{description}
\item \comp{GRAD:} The gradients of \comp{XPARAM}, that is, the derivative
of the heat of formation, in kcal.mol$^{-1}$, with respect to the coordinate
XPARAM($i$) in \AA ngstroms or radians. (3$*$NATOMS)
\item 
\begin{description}
\item[(MOPAC) \comp{NSET:}] Used by the COSMO method, \comp{NSET} is the set
of points on a surface associated with each atom.
\item[(MOZYME) \comp{NNCF:}]  The starting addresses of the atom-lists in 
\comp{ICOCC}.  For the first atom in occupied LMO $i$, the  address in
\comp{ICOCC} is NNCF($i$)+1. (NOCC)
\end{description}
\item {\bf (MOZYME) \comp{NNCE:}}  The starting addresses of the atom-lists in
\comp{ICVIR}.  For the first atom in virtual LMO $i$, the  address in
\comp{ICVIR} is NNCE($i$)+1. (NVIR)
\item 
\begin{description}
\item[(MOPAC) \comp{XY:}] The two-electron integrals over molecular orbitals.
Used in the C.I., $XY(i,j,k,l)$ is the integral involving electron 1 in M.O.s
$\psi_i$ and $\psi_j$ interacting with electron 2 in M.O.s $\psi_k$ and
$\psi_l$. (NMOS$^4$)
\item[(MOZYME) \comp{ICOCC:}] The set of atoms in each occupied M.O.  All the
occupied LMOs are represented in \comp{ICOCC}. All the atom numbers of any
given LMO are contiguous.  Between each LMO is an unused space which can be
used if the LMO expands in \comp{DIAGG2}. Set in \comp{MLMO} or
\comp{PINOUT}, and modified in \comp{TIDY} and \comp{DIAGG2}.

The first atom in LMO $i$ is at address ICOCC($j$), $j$=NCF($i$)+1, and there
are NNCF($i$) atoms in that LMO.
\end{description}
\item 
\begin{description}
\item[(MOPAC) \comp{CIMAT:}] The configuration interaction matrix used 
in \comp{MECI}. (At least (LAB*(LAB+1))/2)
\item[(MOZYME) \comp{ICVIR:}] The set of atoms in each virtual M.O.  All the
virtual  LMOs are represented in \comp{ICVIR}. All the atom numbers of any
given LMO are contiguous.  Between each LMO is an unused space which can be
used if the LMO expands in \comp{DIAGG2}. Set in \comp{MLMO} or \comp{PINOUT},
and modified in \comp{TIDY} and \comp{DIAGG2}.

The first atom in LMO $i$ is at address ICVIR($j$), $j$=NCE($i$)+1, and there
are NNCE($i$) atoms in that LMO.
\end{description}
\item 
\begin{description}
\item[(MOPAC) \comp{C:}] The set of eigenvectors or M.O.s.  In  UHF
calculations, this would be the $\alpha$ set. (NORBS$^2$)
\item[\bf (MOZYME) \comp{COCC:}] The atomic orbital coefficients for the  set
of occupied LMOs.  All the coefficients of any given LMO are contiguous. 
Between each LMO is an unused space which can be used if the LMO expands in
\comp{DIAGG2}.

The contents of \comp{COCC} are modified in \comp{DIAGG2}, \comp{MAKVEC},
\comp{MLMO}, \comp{PINOUT}, and \comp{TIDY}, only.

For LMO $i$, the first coefficient is in COCC($j$), $j$=NCOCC($i$)+1.  The
number of coefficients depends on the atoms in the LMO, see \comp{ICOCC}.
\end{description}
\item 
\begin{description}
\item[(MOPAC) \comp{CB:}] In UHF work, the set of $\beta$ eigenvectors.
(NORBS$^2$)
\item[(MOZYME) \comp{CVIR:}] The atomic orbital coefficients for the  set of
virtual LMOs.  All the virtual  LMOs are represented in \comp{CVIR}.  All the
coefficients of any given LMO are contiguous.  Between each LMO is an unused
space which can be used if the LMO expands in \comp{DIAGG2}.

The contents of \comp{CVIR} are modified in \comp{DIAGG2}, \comp{MAKVEC},
\comp{MLMO}, \comp{PINOUT}, and \comp{TIDY}, only. For LMO $i$, the first
coefficient is in CVIR($j$), $j$=NVIR($i$)+1.  The number of coefficients
depends on the atoms in the LMO, see \comp{ICVIR}.
\end{description}
\item 
\begin{description}
\item[(MOPAC) \comp{DIJKL:}] Derivatives of two-electron integrals 
over M.O.s with respect to geometry.  Used by \comp{DERNVO}.
\item[(MOZYME) \comp{NBOND:}] The number of atoms bonded to each atom. (NUMAT)
\end{description}
\item \comp{IFACT/I1FACT:} FACT($i$)=$(i*(i-1))/2$, I1FACT($i$)=$(i*(i+1))/2$.
The array \comp{I1FACT} shares the same storage as \comp{IFACT}, but has a
starting address in \comp{MCORE} one more than \comp{IFACT}.  That is, the
starting address of \comp{IFACT} is $NCORE(31)$ and the starting address of
\comp{I1FACT} is  $NCORE(31)+1$.
\item 
\begin{description}
\item[(MOPAC) \comp{NPERMA:}] The $\alpha$ molecular orbital occupancy for the 
microstates in a C.I.\ calculation. (At least LAB) 
\item[(MOZYME) \comp{IBONDS:}] The atom numbers of the atoms attached to any
given atom.  For atom $i$, the number of attached atoms is NBONDS($i$), and
the  attached atoms are IBONDS(1:NBONDS($i$),$i$).
\end{description}
\item 
\begin{description}
\item[(MOPAC) \comp{NPERMB:}] The $\beta$ molecular orbital occupancy for the 
microstates in a C.I.\ calculation. (At least LAB) 
\item[(MOZYME) \comp{NFMO:}] This array is used in the annihilation routine 
\comp{DIAGG1}, only. \comp{NFMO} holds the number of filled M.O.s that each
virtual M.O.\ interacts with significantly. (NORBS)
\end{description}
\item \comp{TXTATM:} A short description of each atom.  This description can be
made in two ways.  In the data-set, a short description can be given in
parenthesis after each atom symbol.  For example ``H(on a COO)''.  The
description can be up to 13 characters long.

Alternatively, a useful description can be generated by use of \comp{RESIDUES}
or \comp{RESEQ}, and consists of 13 letters: `$nnnnn$~$RES$f$mmm$', where
`$nnnnn$' is the number of the atom, `$RES$' is the three-letter abbreviation
for the residue, `f' is a space or an asterisk, and `$mmm$' is the residue
number.  Explicit atom charges can also be specified in an atom label.  Thus a
cation would be specified by the presence of a `+' symbol in an atom label, and
an anion by the presence of a `--' symbol. Because both \comp{RESIDUES} and
\comp{RESEQ} create new labels, these keywords should not be used when explicit
charges are present.  Instead, the labels should be generated in one run, then
the explicit charges assigned.  Thereafter, \comp{RESIDUES} should not be used.

When printed, the description is enclosed in parentheses, for example
`(~3674~GLY~231)'.  Set in \comp{GETGEO} and \comp{NAMES}. See also
\hyperref[pageref]{``\comp{RESIDUES}''}{ on p.~}{}{res}. \comp{TXTATM}
is a character string of length 16 bytes, and is  contained in
\comp{CCORE}. (NATOMS)
\item 
\begin{description}
\item[(MOPAC) \comp{TOM:}] An array used by the Tomasi method.
\item[(MOZYME) \comp{ISORT:}] The order in which LMOs increase in energy. In
\comp{VALUES}, the energies of the LMOs are calculated.  The LMO number of the
lowest energy M.O.\ is then put in ISORT(1), then the number of the next higher
M.O.\ is put in ISORT(2), etc. (NORBS)
\end{description}
\addtocounter{enumi}{2} 
\item 
\begin{description}
\item[(MOPAC) \comp{EIGS:}] The eigenvalues of the M.O.s.  If UHF, then the
$\alpha$ eigenvalues. (NORBS)
\item[(MOZYME) \comp{EIGF/EIGV:}] The energy levels of the occupied and virtual
sets of LMOs. (NORBS)
\end{description}
\item \comp{EIGB:} In UHF work, the eigenvalues of the $\beta$ M.O.s (NORBS)
\item \comp{IOPT:} When \comp{RESIDUES} is specified, IOPT($i$) contains the
residue number to which atom $i$ belongs. If atom $i$ is a backbone atom, then
IOPT($i$) holds the {\em negative} of the residue number.  (In a residue, the
sequence of atoms is -NH-CH(R)-CO-, the backbone atoms are -NH-CH-CO- and the
side-chain atoms are those in the `R' group. The hydrogen atom at the -NH$_2$
end is considered part of backbone of residue 1, and the -OH group at the -COOH
end is considered part of the  backbone of the last residue.) Set in
\comp{PICOPT} only. (NUMAT)
\item \comp{JOPT:} The atoms used in a partial geometry optimization.  By
default, JOPT($i$)=$i$.  If \comp{RAPID} is used, then only those atoms that
move are in \comp{JOPT}.   This set is defined as the set of atoms to be
optimized plus all the atoms attached to the atoms that are to be optimized. 
Thus if CH$_3$-NH$_2$ were to be calculated, and the coordinate of the nitrogen
were to be  optimized, then \comp{JOPT} would contain all the atoms except the
hydrogens on the CH$_3$. If the atoms are in the order C, H, H, H, N, H, and H,
the \comp{JOPT} would be: 1, 5, 6, 7. The number of atoms defined in
\comp{JOPT} is \comp{NUMRED}. In the first SCF, all atoms are used. 
\comp{JOPT} only affects the subsequent SCFs.  \comp{JOPT} interacts with the
SCF in \comp{TIDY}. Set in \comp{PICOPT}.
\item \comp{OLDXYZ:} In \comp{RAPID} calculations, \comp{OLDXYZ} holds the
reference Cartesian coordinate gradients. The name should be read as `Old
DXYZ', not `Old XYZ'.   These are the gradients due to all  atoms {\em except}
those involved in the geometry optimization. ($3*$NUMAT)
\item \comp{DXYZ:} A two-dimensional array of derivatives of the energy with
respect to displacement in the $x$, $y$, and $z$ directions for each real
atom.  Units: kcal/mol/\AA ngstrom. \comp{DXYZ} is set in \comp{DCART}.
Except for systems with translational vectors, the size of \comp{DXYZ} is
$3\times $NUMAT.
\item 
\begin{description}
\item[(MOPAC) \comp{PA:}] The $\alpha$ density matrix. (MPACK)
\item[(MOZYME) \comp{PARTP:}] The partial density matrix due to all atoms that
do not move in a \comp{RAPID} calculation. (MPACK)
\end{description}
\item
\begin{description}
\item[(MOPAC) \comp{PB:}] In a UHF calculation, the $\beta$ density  matrix.
(MPACK)
\item[(MOZYME) \comp{PARTH:}] The partial one-electron matrix due to all atoms
that do not move in a \comp{RAPID} calculation.  (MPACK)
\end{description}
\item
\begin{description}
\item[(MOPAC) \comp{FB:}] In a UHF calculation, the Fock matrix. (MPACK)
\item[(MOZYME) \comp{PARTF:}] The partial Fock matrix due to all atoms that do
not move in a \comp{RAPID} calculation.  (MPACK)
\end{description}
\item \comp{KOPT:} In a \comp{RAPID} calculation,  \comp{KOPT} is the set of
atoms which are found in the LMOs used in an SCF. If the LMOs for all the atoms
are used in the SCF, then \comp{KOPT} is the full set.  If only the LMOs for a
few atoms are used (the number of atoms in \comp{JOPT} is small), then
\comp{KOPT} is the set of atoms in that sub-set of LMOs.  Note that the set in
\comp{KOPT} is larger than that in \comp{JOPT}, unless the set in \comp{JOPT}
is full.   Set in \comp{SETUPK}. (NUMAT)
\addtocounter{enumi}{1}
\item \comp{HESINV:} The inverse Hessian matrix in the BFGS method, and the
Hessian matrix in Baker's EigenFollowing \comp{EF} method. ((NVAR$*$(NVAR+1))/2)
\item \comp{JELEM:} In calculating symmetry properties, $JELEM(i,j)$ holds the
atom-number of the atom that $j$ would be moved to under operation 
$i$. ($20*$NUMAT)
\item \comp{ATMASS:} The atomic masses of the atoms.  Set by default, and can
be reset by the input data. (NUMAT)
\item \comp{REACT:} The values of the reaction coordinates in a reaction path
calculation.
\item \comp{C0:} An array used by the Tomasi method.
\item \comp{NC:} An array used by the Tomasi method.
\item \comp{INTERP:} A set of arrays used by the Camp-King SCF converger. 
These arrays can be eliminated by use of \comp{UNSAFE}. ($5*$NORBS$^2$)
\item \comp{ERRFN:} In RHF open-shell calculations, \comp{ERRFN} 
holds the difference
between the ``exact" (\comp{DERNVO}) derivatives and those calculated using 
\comp{DCART} only.  These differences are used in the geometry optimization, by
allowing the derivatives to be calculated using \comp{DCART}, and then applying
a correction.
\item \comp{AIDER:} {\em Ab initio} derivatives.  These are read in from the
data set in an attempt to improve the geometry.
\item \comp{ALPARM:}  In reaction path calculations, \comp{ALPARM} holds the
last three geometries.  These are used to extrapolate to the next geometry. 
($3*$NVAR)
\addtocounter{enumi}{1}
\item \comp{NAMO:} The symmetry labels of the eigenvectors (M.O.s, 
vibrations, or states). 
\item \comp{JNDEX:} The quantum numbers for the symmetry labels (The first
occurrence of symmetry label $X$ would have quantum number 1, the second: 2,
and so on.)
\item  \comp{IPO:} In a FORCE calculation, $IPO(i,j)$  holds the number of the
atom that atom $i$ would be moved to under operation $j$. ($120*$NUMAT)
\item \comp{DXYZR:} In RHF open-shell gradient calculations, \comp{DXYZR} holds
the gradient components due to the frozen core.
\item \comp{XPAREF:} Similar to \comp{XPARAM}.  \comp{XPAREF} is used to
temporarily store the geometric variables.
\item \comp{PROFIL:} The values of all heats of formation in a reaction
path calculation, in which the path coordinates are defined using
\comp{STEP=n.nn} and \comp{POINTS=m}.
\item \comp{SRAD:}  Used by the COSMO method. (NUMAT)
\addtocounter{enumi}{1}
\item \comp{DIRVEC:} Used by the COSMO method. (3246=3*1082)
\item \comp{BH:} Used by the COSMO method.  (LENABC)
\item \comp{IATSP:} Used by the COSMO method.  (LENABC+1)
\item \comp{NN:} Used by the COSMO method. ($3\times $NUMAT)
\item \comp{QDEN:} Used by the COSMO method. (LN61)
\item \comp{CH:} In electrostatic calculations, \comp{CH} hold atomic
charges. (NUMAT)
\item \comp{VECTCI:} In C.I. calculations, \comp{VECTCI} hold the state
vectors that would be used in calculating derivatives. (5 State Vectors)
\item \comp{TOM:} Used in the Tomasi method.
\addtocounter{enumi}{5}
\item \comp{AMAT:} Used by the COSMO method. (LENABC$^2$/2)
\item \comp{BMAT:} Used by the COSMO method. (LENABC$\times$ LM61)
\item \comp{CMAT:} Used by the COSMO method. ((LM61$\times$ (LM61+1))/2)
\item \comp{COSURF:} Used by the COSMO method. (LENABC+1)
\addtocounter{enumi}{12}
\item {\bf Reserved:} The starting address of a new temporary array is
placed in $NCORE(97)$.  This address must be used immediately, otherwise
it might be overwritten when a new temporary array is created.
\item {\bf Start of Integer:} The highest address in \comp{MCORE} available
for use in making temporary integer arrays. Any new temporary integer arrays
would be placed at the end of \comp{MCORE}.  At the start of the calculation,
$NCORE(98)$ is equal to the size of \comp{MCORE}; as temporary arrays are
created, $NCORE(98)$ decreases.
\item {\bf Start of Real:} The highest address in \comp{CCORE} available
for use in making temporary real arrays. Any new temporary real arrays
would be placed at the end of \comp{CCORE}.  At the start of the calculation,
$NCORE(99)$ is equal to the size of \comp{CCORE}; as temporary arrays are
created, $NCORE(99)$ decreases.
\item {\bf Not used}
\end{enumerate}

\subsection*{NASIZE(100)}
An array of size 100, \htmlref{\comp{NASIZE}}{nasize} holds the sizes of all the permanent arrays
that depend on the system being calculated. 
\begin{latexonly}
See p.~\pageref{nasize} for more detail.
\end{latexonly}

\subsection*{\comp{IKST(100)}}
Because MOPAC can calculate many quantities, such as the number of atoms, heat
of formation, etc., a method has to exist to allow these quantities to be
transmitted throughout the program. This is accomplished by two arrays, \comp{IKST}
and \comp{RKST}.  Each element of these
arrays is reserved for use as a storage site for a scalar.  \comp{IKST} holds
integer scalars and \comp{RKST} holds real scalars.  Table~\ref{cuim}
summarizes these quantities.   At the start of a calculation, every element of
\comp{IKST} is set to -99999, and every element of \comp{RKST} is set to
-99999.99. As a result of this, if an element of these arrays is examined, it
is immediately obvious whether or not it has been set.  Thus, until the number
of atoms is known, IKST(2) (NUMAT), is -99999.

\begin{table}
\caption{\label{cuim} Constants used in MOPAC}
\begin{center}
\begin{tabular}{lllllllll} \hline
\comp{RKST} & \comp{IKST} & No. & \comp{RKST} & \comp{IKST} & No. & \comp{RKST} & \comp{IKST} & No.\\
\hline
\comp{ESCF} & \comp{NATOMS} & 1 & \comp{EOUTER} & \comp{ICOMPF} & 34 & & & 67\\
\comp{ENUCLR} & \comp{NUMAT} &  2 & \comp{THRESH} & \comp{ISAFE} & 35 & & & 68\\
\comp{EE} & \comp{NELECS} &  3 & \comp{RSOLV} & \comp{LIMSCF} & 36 & & & 69\\
\comp{ATHEAT} & \comp{NALPHA} &  4 & & \comp{NA1} & 37 & & & 70\\
\comp{FRACT} & \comp{NBETA} &  5 & \comp{CORHYB} & \comp{IGROUP} & 38 & & & 71\\
\comp{EMIN} & \comp{NCLOSE} &  6 & \comp{ESCFL} & \comp{NCLASS} & 39 & & & 72\\
\comp{COSINE} & \comp{NOPEN} &  7 & \comp{VOLUME} & \comp{NIRRED} & 40 & & & 73\\
\comp{GNORM} & \comp{NORBS} &  8 & \comp{FEPSI} & \comp{NSPA} & 41 & & & 74\\
\comp{TIME0} & \comp{ISCF} &  9 & \comp{RDS} & \comp{NPS} & 42 & & & 75\\
\comp{EZQ} & \comp{NSCF} & 10 & \comp{DISEX2} & \comp{NPS2} & 43 & & & 76\\
\comp{EEQ} & \comp{ID} & 11 & \comp{CUTOF1} & \comp{IOLDCV} & 44 & & & 77\\
\comp{AREA} & \comp{NVAR} & 12 & \comp{CUTOF2} & \comp{NSTATE} & 45 & & & 78\\
\comp{EDIEL} & \comp{MOLNUM} & 13 & \comp{CUTOFS} & \comp{NUMPTS} & 46 & & & 79\\
\comp{GVW} & \comp{LATOM} & 14 & \comp{CUTOFP} & & 47 & & & 80\\
\comp{GVWS} & \comp{LPARAM} & 15 & \comp{FNSQ} & \comp{LENABC} & 48 & & & 81\\
\comp{ELC1} & \comp{ITRY} & 16 & \comp{CLOWER} & \comp{MESP} & 49 & & & 82\\
\comp{TLEFT} & \comp{ITYPE} & 17 & \comp{CUPPER} & \comp{NUMRED} & 50 & & & 83\\
\comp{TDUMP} & \comp{MFLAG} & 18 & \comp{DLM} & \comp{NORRED} & 51 & & & 84\\
\comp{ELECT} & \comp{ITERQ} & 19 & \comp{XNORM} & \comp{NELRED} & 52 & & & 85\\
\comp{STEP} & \comp{IFLEPO} & 20 & \comp{DIFE} & \comp{NRES} & 53 & & & 86\\
\comp{TVEC(1,1)} & \comp{MPACK} & 21 & \comp{PRESSURE} & \comp{NUMCAL} & 54 & & & 87\\
\comp{TVEC(2,1)} & \comp{N2ELEC} & 22 & & \comp{NIP} & 55 & & & 88\\
\comp{TVEC(3,1)} & \comp{NMOS} & 23 & & \comp{IDIAGG} & 56 & & & 89\\
\comp{TVEC(1,2)} & \comp{LAB} & 24 & & \comp{MDRCRS} & 57 & & & 90\\
\comp{TVEC(2,2)} & \comp{NELEC} & 25 & & & 58 & & & 91\\
\comp{TVEC(3,2)} & & 26 & & & 59 & & & 92\\
\comp{TVEC(1,3)} & & 27 & & \comp{IPAD2} & 60 & & & 93\\
\comp{TVEC(2,3)} & & 28 & & \comp{IPAD4} & 61 & & & 94\\
\comp{TVEC(3,3)} & \comp{MAXTXT} & 29 & & \comp{NL1} & 62 & & & 95\\
\comp{RJKAB1} & \comp{LAST} & 30 & & \comp{NL2} & 63 & & \comp{ICROS} & 96\\
\comp{WTMOL} & \comp{NDEP} & 31 & & \comp{NL3} & 64 & & \comp{MORB} & 97\\
\comp{VERSON} & \comp{LABSIZ} & 32 & & & 65 & & \comp{MOP/MOZ} & 98\\
\comp{TOTIME} & \comp{LM61} & 33 & & & 66 & & & 99\\
 & & & & & & & \comp{DERIV} & 100\\
\hline
\end{tabular}
\end{center}
\end{table}

\begin{enumerate}
\item \comp{NATOMS} The number of atoms supplied by the data set,
including dummy atoms and translational vectors.  \comp{NATOMS} is set in the
subroutine that
reads in the geometry (\comp{GETGEO}, \comp{GETGEG}, or \comp{GETPDB}).

With one exception, once \comp{NATOMS} is determined, it is not changed.
In \comp{FORCE}, only real atoms are used, therefore, for convenience,
the geometry is re-defined so that any dummy atoms are deleted.
\item \comp{NUMAT} The number of real atoms.  Here, ``real'' atoms are
the sparkles (100\% ionized atoms) and normal atoms.  The difference
between \comp{NATOMS} and \comp{NUMAT} is the number of dummy atoms plus
any translational vectors.
\item\comp{NELECS} The total number of valence electrons in the system.  This
is calculated by adding up the number of valence electrons on each atom, and
subtracting the net charge on the system.
\item\comp{NALPHA} In a \comp{UHF} calculation, \comp{NALPHA} is the number
of $\alpha$ electrons in the system.  By default, the number of $\alpha$
electrons is the same as the number of $\beta$ electrons (for closed-shell
systems), or one more than the number of $\beta$ electrons (for odd-electron
systems). If a spin is defined (\comp{TRIPLET}, \comp{QUARTET}, etc.), then the
number of $\alpha$ electrons is two to four more than the number of $\beta$
electrons. For example, if \comp{QUINTET} is specified, then \comp{NALPHA}
would be four more than \comp{NBETA}. The value of \comp{NALPHA} is set in
\comp{STATE}.

In an RHF calculation, \comp{NALPHA}=0.
\item\comp{NBETA} In a \comp{UHF} calculation, \comp{NBETA} is the number
of $\beta$ electrons in the system. In an RHF calculation, \comp{NBETA}=0.
The value of \comp{NBETA} is set in
\comp{STATE}.


\item\comp{NCLOSE} In an RHF calculation, \comp{NCLOSE} is the number of
doubly-occupied M.O.s. In a UHF calculation, \comp{NCLOSE}=0.
The value of \comp{NCLOSE} is set in  \comp{STATE}.

\item\comp{NOPEN} In an RHF calculation, \comp{NOPEN} is the number of
the highest M.O. that would be fractionally occupied in the SCF calculation.
The ``fraction'' can range from 0.0 to 2.0.  For an isolated carbon atom, the
fractional occupancy (two $p$ electrons in three atomic orbitals) would be
0.6667, and \comp{NCLOSE} would be 1, and \comp{NOPEN} would be 4.  By default,
the fractional occupancy of the SOMO in odd-electron
systems is 1.0. In a UHF calculation, \comp{NOPEN}=0.
The value of \comp{NCLOSE} is set in  \comp{STATE}.
\item\comp{NORBS} The number of atomic orbitals in a system.  Each hydrogen atom
contributes one atomic orbital, every other atom contributes four, except atoms
with $d$-orbitals, which contribute nine, and sparkles, which do not
contribute any.  \comp{NORBS} is set in \comp{STATE}.
\item\comp{ISCF} At the end of a calculation, \comp{ISCF=1} if an SCF exists,
otherwise \comp{ISCF=2}.  Set in \comp{ISITSC}.
\item \comp{NSCF} The number of SCF calculations done.  Each time an SCF
calculation is done, \comp{NSCF} is incremented by one.
\item \comp{ID} The number of dimensions for polymeric systems.  For molecules,
\comp{ID=0}; for polymers, \comp{ID=1}; for layer systems, \comp{ID=2}; and for
solids, \comp{ID=3}.  Set in \comp{GMETRY}.
\item \comp{NVAR} The number of geometric variables.  Normally, this is the
number of coordinates flagged by a ``1'' in the input data set.  In a \comp{FORCE}
calculation, or one that implies a \comp{FORCE} calculation, such as an IRC,
then \comp{NVAR=3*NUMAT}.
\item \comp{MOLNUM}  The number of the system.  Each new molecule run is given
a new number, with the first molecule having \comp{MOLNUM=1}.  If several
geometric operations are done on one system, they all share the
same \comp{MOLNUM}, however, if \comp{OLDGEO} is used, then the system
is considered to be ``new''. The only exception is when an IRC is run, in which
case the value of \comp{MOLNUM} is increased between the \comp{FORCE}
and the \comp{IRC} calculations.  This is needed, because the system is
re-oriented before the \comp{IRC} is run.
\item \comp{LATOM} In a reaction coordinate, \comp{LATOM} is the atom number
of the atom that moves.  This coordinate is indicated in the input data set
by the ``-1'' flag.  The atom can be real or dummy. Set in \comp{READMO}.
See also \comp{LPARAM}.
\item \comp{LPARAM} In a reaction coordinate, \comp{LPARAM} is the coordinate
(first, second, or third) of the atom that moves.  Set in \comp{READMO}. See
also \comp{LATOM}.
\item \comp{ITRY}  The number of iterations allowed in attempting to generate
an SCF.  By default, \comp{ITRY=200}, but this can be changed by use of
\comp{ITRY=n}.
\item \comp{ITYPE} The type of calculation  to be run.
If an MNDO calculation, \comp{ITYPE=4}; AM1, \comp{ITYPE=2}; PM3,
\comp{ITYPE=3}, MINDO/3, \comp{ITYPE=4}; MNDO-$d$, \comp{ITYPE=5}.
\item \comp{MFLAG} Used by the Tomasi method.
\item \comp{ITERQ} Used by the Tomasi method
\item \comp{IFLEPO} A flag to indicate how the geometry optimization
terminated.  Initialized in \comp{RMOPAC} and normally set in
\comp{EF}, \comp{FLEPO}, \comp{NLLSQ}, and \comp{POWSQ}.  Set in \comp{ITER}
when the SCF fails.  The various meanings of IFLEPO are defined in \comp{DATA}
statements in \comp{WRITMN}.
\item \comp{MPACK} The number of array elements in the density, one-electron,
and Fock matrices.  In conventional work, this is  \comp{(NORBS*(NORBS+1))/2}
and is set in \comp{STATE},
in MOZYME work, \comp{MPACK} is determined from the number of atoms that
interact with each other.  For large systems, \comp{MPACK} is normally
much smaller than the value in conventional work.
\item \comp{N2ELEC} The number of two-electron integrals.  The number
of integrals (0,1,16, or 81) contributed by an atom depends on the
number of orbitals (0, 1, 4, or 9) on that atom.  The NDDO
approximation indicates that for atom pairs, the number of integrals is
$(n(n+1))*(m(m+1))/4)$, where $n$ and $m$ are the number of orbitals on
each atom.  In MOZYME, \htmlref{the number is reduced when two atoms
are far apart}{m2el}.
\begin{latexonly}
See p.~\pageref{m2el} for details.
\end{latexonly}
\item \comp{NMOS} The number of M.O.s involved in the active space in a C.I.
calculation.  Set in \comp{STATE} only.
\item \comp{LAB} The number of microstates or configurations included in the
C.I.\ calculation.  Set in \comp{MECI} only.
\item \comp{NELEC} The number of electrons involved in the C.I. calculation.
Set in \comp{MECI} only.
\addtocounter{enumi}{3}
\item \comp{MAXTXT} Maximum number of characters in the description of an
atom.  (An atom label can include a short description of up to 13 letters,
for example ``\verb:N(  634 ASN  46):)''
\item \comp{LAST} \comp{LAST=0} if the current SCF is {\em not} the last SCF. \
Only on the last SCF, when \comp{LAST=1}, are the symmetries of the M.O.s
determined.  If the system is RHF open shell, then in \comp{WRITMN}, \comp{LAST}
is set to 3, and \comp{MECI} is called.  This prints the symmetries of the
states.
\item \comp{NDEP} The number of symmetry-dependent coordinates.  Set in
\comp{GETGEG}, \comp{GETSYM}, and \comp{MAKSYM}.
\item \comp{LABSIZ} The number of states to be calculated in a C.I. (usually
less than \comp{LAB}, until the last C.I., when all states are calculated.)
\item\comp{LM61} The number of matrix elements in the one-center part of the
density, one-electron, or Fock matrices.  Each hydrogen contributes 1 element,
each $s-p$ atom contributes 10 ($(4*(4+1))/2$), and each $s-p-d$ atom
contributes 45 elements ($(9*(9+1))/2$).
\item\comp{ICOMPF} Used by the Tomasi method.
\item\comp{ISAFE} A flag to indicate whether some large arrays can be
ignored.  If \comp{ISAFE=1} then these arrays are used.  If \comp{ISAFE=0}, then
the amount of memory used is reduced, but some functionalities, such as
the Camp-King converger, will not work.  Normally, this is not important.
\comp{ISAFE} is set at compile time, and can be re-set at run time by
\comp{SAFE} or \comp{UNSAFE}.
\item\comp{LIMSCF} In the SCF, if \comp{LIMSCF=1}, then the SCF can be stopped
before the usual criteria are satisfied, if the energy is obviously lower
or obviously higher.  This saves time, in that the geometry is still not
near to the stationary point.  \comp{LIMSCF=0} if the usual SCF criteria must
be met. This is necessary in \comp{FORCE} and transition state calculations.
\item \comp{NA1}  A flag to indicate the nature of the coordinates.
If \comp{NA1=1} then all coordinates are Cartesian.  Default: \comp{NA1=0}.
\item\comp{IGROUP} The number of the point-group of the system.
\item \comp{NCLASS} The number of classes necessary to define the point-group.
For $I_h$, for example, four operations (four classes) are necessary: $E$,
$C_5$, $C_3$, and the inversion operation.
\item \comp{NIRRED} The number of irreducible representations in the point-group.
\item \comp{NSPA} Used by the COSMO method.
\item \comp{NPS} Used by the COSMO method.
\item \comp{NPS2} Used by the COSMO method.
\addtocounter{enumi}{1}
\item \comp{NSTATE} The number of states used in describing the electronic
state of a system.  Normally 1, the number can increase when degenerate
states are involved.  Thus in CH$_4^+$, without Jahn-Teller distortion,
the value of \comp{NSTATE} would be 3 (representing the $^2T_u$ state).
Note: For degenerate open shell systems, \comp{OPEN(n,m)} will normally be
needed.
\item \comp{NUMPTS} Used by the COSMO method.
\addtocounter{enumi}{1}
\item \comp{LENABC} Estimated maximum number of surface points in the
COSMO method.
\item \comp{MESP} Size of the ESP array in the electrostatic potential
calculation, set in  ``sizes.h''.
\item \comp{NUMRED} The number of atoms involved in the SCF when the \comp{RAPID}
option is used.  The number of atoms that move plus the number of atoms attached
to them.
\item \comp{NORRED} The number of LMOs involved in the SCF when the \comp{RAPID}
option is used.
\item \comp{NELRED} The number of electrons involved in the SCF when the
\comp{RAPID} option is used.
\item \comp{NRES} The number of residues identified in a protein when
\comp{RESIDUES} or \comp{RESEQ} is specified.
\item \comp{NUMCAL} For a given calculation (input geometry), the number
of the type of calculation.  For example, an initial SCF (\comp{NUMCAL=1}),
followed by a \comp{FORCE} calculation (\comp{NUMCAL=2}).
\addtocounter{enumi}{1}
\item \comp{IDIAGG} Used in the annihilation operation in MOZYME, \comp{IDIAGG}
increments every time \comp{DIAGG} is called.
\item \comp{MDRCRS} Controls whether the restart file in a DRC or IRC run should
be formatted or unformatted.  Set in \comp{MOPAC} from \comp{sizes.h}.
If \comp{MDRCRS=1}, then a formatted restart file is used.  This can be read
easily, but uses more space and is less precise than unformatted (\comp{MDRCRS=0}.
\addtocounter{enumi}{2}
\item \comp{IPAD2} The estimated maximum average number of atoms in a LMO.
The default value is usually sufficient, but it can be reset by \comp{NLMO=nnn}.
\item \comp{IPAD4} The estimated maximum average number of atomic orbitals
in a LMO.  Set as a function of \comp{IPAD2}.
\addtocounter{enumi}{34}
\item \comp{ICROS} The type of calculation: 1 if an intersystem crossing,
0 otherwise.
\item \comp{MORB} The maximum number of orbitals on any atom: 1, 4, or 9.
\item \comp{unnamed} The type of method used in solving the SCF equations.
0 indicates conventional (MOPAC type), 1 indicates LMO (MOZYME type).
\addtocounter{enumi}{1}
\item \comp{unnamed} The type of derivative calculation: 1 = MOZYME -1 =
remove contributions to derivative arising from moving atoms; 2 = MOZYME 0 =
calculate MOZYME derivatives {\em de novo}; 3 = MOZYME +1 = add in contributions
to derivatives arising from moving atoms; 4 = MOPAC RHF; 5 = MOPAC UHF.
\end{enumerate}

\subsection*{\comp{RKST(100)}}
\begin{enumerate}
\item \comp{ESCF} The calculated heat of formation in kcal.mol$^{-1}$.
\item \comp{ENUCLR} The nuclear energy in eV.
\addtocounter{enumi}{1}
\item \comp{ATHEAT} The reference heat of atomization plus ionization, in eV.
\item \comp{FRACT} The fractional orbital occupancy, in electrons.  Range:
0.0 to 2.0.
\item \comp{EMIN} Within any geometry optimization calculation, the lowest
heat of formation calculated.  Used in deciding when to exit the SCF.
\item \comp{COSINE} The angle between the current and previous gradient
vector.  Range: -1.0 to +1.0.
\item \comp{GNORM} The scalar of the gradient vector. \comp{GNORM=MOD(GRAD)}.
\item \comp{TIME0} The internal zero of time for determining CPU usage.
\item \comp{EZQ} The nuclear - solvent interaction term in the Tomasi model.
\item \comp{EEQ} The electronic - solvent interaction term in the Tomasi model.
\item \comp{AREA} The surface area of a molecule, in \AA $^2$, calculated by COSMO.
\addtocounter{enumi}{1}
\item \comp{GVW} Used by the Tomasi method.
\item \comp{GVWS} Used by the Tomasi method.
\item \comp{ELC1} Used by the Tomasi method.
\item \comp{TLEFT} The allowed amount of CPU time left for the calculation.
Default: one hour.
\item \comp{TDUMP} The interval between checkpoints (dumps of the calculation
that can be used by \comp{RESTART}).  Default: one hour.
\item \comp{ELECT} The electronic energy in eV.
\item \comp{STEP} The step size in a reaction path.
\item \comp{TVEC} A three by three matrix representing the translation vector.
The first vector is \comp{TVEC(1,1):TVEC(3,1)}.
\addtocounter{enumi}{8}
\item \comp{RJKAB1} The two-electron integral over M.O.s used in
calculating the correction to the energy in the half-electron approximation.
\item \comp{WTMOL} The molecular weight in AMU.  Calculated in \comp{MOPAC}.
\item \comp{VERSON} The version number of MOPAC, e.g., \mopacversion .0.
\item \comp{TOTIME} The total time, in seconds, for the calculation,
including time used in previous runs, if \comp{RESTART} is used.
\item \comp{EOUTER} Nuclear energy in eV due to atom pairs not present in
\comp{P}, \comp{H}, or \comp{F}. Calculated in \comp{HCORZ}, and
used by \comp{WRITMN} only.
\item \comp{THRESH} In MOZYME, the threshold value for intensity of a
LMO on an atom.  An atom with intensity below \comp{THRESH} would either
(a) not be added to a LMO (in \comp{DIAGG2}), or (b) would be deleted from the
 LMO (in \comp{TIDY}).
\addtocounter{enumi}{2}
\item \comp{CORHYB} The correction to the MOZYME energy due to the use of the
dipole approximation. (for interactions arising from atoms separated
by more than \comp{CUTOF2} but less than \comp{CUTOF1}, that is, atoms pairs
not present in \comp{P}, \comp{H}, or \comp{F}.
\item \comp{ESCFL} In an intersystem crossing, \comp{ESCFL} is the $\Delta H_f$
 of the lower state.  Note \mbox{\comp{ESCFL} $<$ \comp{ESCF}}.
\item \comp{VOLUME} The volume of the system in \AA $^3$, calculated by COSMO.
\item \comp{FEPSI} Used by COSMO
\item \comp{RDS}  Used by COSMO.
\item \comp{DISEX2} Used by COSMO.  \comp{DISEX2} is needed in both \comp{INITSN}
and \comp{CONSTS}.
\item \comp{CUTOF1} The cutoff distance for dipolar electrostatic terms.  Beyond
\comp{CUTOF1}, only point-charge terms are used.  Default: 30 \AA ngstroms.
\item \comp{CUTOF2} The cutoff distance for NDDO two-electron, two-center terms.
Beyond \comp{CUTOF2}, only dipole and point charge terms are used.
\item \comp{CUTOFS} The distance beyond which overlap integrals are {\em not}
calculated.  By default, \comp{CUTOFS=7\AA }.
\item \comp{CUTOFP} In polymers, all electrostatic terms involving
atoms separated
by more than \comp{CUTOFP} are calculated as if they were separated by
\item \comp{CUTOFP}  Needed in order to ensure that equivalent atoms are correctly
predicted.
\item \comp{FNSQ}
\item \comp{CLOWER}  In a polymer calculation, \comp{CLOWER} marks the onset,
in \AA ngstroms, of the truncation function for electrostatic interactions.
Set in function \comp{TRUNK}
\item \comp{CUPPER}  In a polymer calculation, \comp{CUPPER} marks the upper bound,
in \AA ngstroms, of the truncation function for electrostatic interactions.
Set in function \comp{TRUNK}
\item \comp{DLM}  Used by intersystem crossing.
\item \comp{XNORM}  Used by intersystem crossing.
\item \comp{DIFE}  In an intersystem crossing, \comp{DIFE} is the square
of the difference in energy of the two states involved.
\item \comp{PRESSURE} In a solid-state calculation, \comp{PRESSURE} is the stress
(in kcal/mol) that the system is under.
\end{enumerate}

\subsection*{Files used in MOPAC}
All input-output channels in MOPAC are assigned numbers.  To allow the values
of these numbers to be easily changed, the module \comp{common\_systm} contains
symbolics for the various channels used.  These symbolics are set in \comp{GETDAT}
and can only be changed by editing this subroutine.

\begin{table}
\caption{\label{inout} Channels Used in MOPAC}
\begin{center}
\begin{tabular}{cclllc} \hline
    Input/output & Symbol  & Name & Type & Formatted? \\
                Channel No.  \\
\hline
     2   &        & $<$filename$>$.dat  & Input            &  Yes \\
     3   & \comp{ISCR}   & Scratch                    & Internal working  &  No  \\
     4   & \comp{ISETUP} &     SETUP                  & Setup             &  Yes \\
    25   & \comp{IR}     &    Scratch                 & Input (during run)& Yes \\
    26   & \comp{IW}     & \comp{$<$filename$>$.out}  & Output            & Yes \\
     9   & \comp{IRES}   & \comp{$<$filename$>$.res}  & Restart           & No  \\
    10   & \comp{IDEN}   & \comp{$<$filename$>$.den}  & Density           & No  \\
    11   & \comp{ILOG}   & \comp{$<$filename$>$.log}  & Log               & Yes \\
    12   & \comp{IARC}   & \comp{$<$filename$>$.arc}  & Archive           & Yes \\
    13   & \comp{IGPT}   & \comp{$<$filename$>$.gpt}  & Graphics          & No  \\
    14   & \comp{IEXT}   & User defined               & New parameters    & Yes  \\
    15   & \comp{IESR}   & \comp{$<$filename$>$.esr}  & ESP Restart       & No  \\
    16   & \comp{ISYB}   & \comp{$<$filename$>$.syb}  & SYBYL             & Yes \\
    17   & \comp{ISOL}   & \comp{$<$filename$>$.sol}  & SOL map in MEP    & No  \\
    18   & \comp{IBRZ}   & \comp{$<$filename$>$.brz}  & Brillouin Zone    & No  \\
    17   & \comp{IPDB}   & \comp{$<$filename$>$.pdb}  & PDB file          & No  \\
    20   & \comp{IESR}   & \comp{$<$filename$>$.ump}  & Grid map          & Yes \\
    21   & \comp{IESP}   & \comp{$<$filename$>$.esp}  & Electrostatic map & Yes \\
    22   & \comp{IS}     & \comp{$<$filename$>$.mep}  & Tomasi MEP map    & Yes \\
\hline
\end{tabular}
\end{center}
\end{table}

