\subsection{Sparkles}\label{sparkles}
\index{Sparkles!description}
Four extra ``elements'' have been put into  MOPAC.   These  represent pure 
ionic  charges,  roughly  equivalent  to  the  following  chemical entities:

\index{+!sparkle}\index{++}\index{$-$}\index{$--$}
\begin{center}
\begin{tabular}{cl}
    Chemical Symbol &        Equivalent to  \\
\hline
+    & Tetramethyl ammonium radical \\
     &  Potassium atom or Cesium atom.  \\ 
++   & Barium atom.  \\
$-$  & Borohydride radical, Halogen, or  Nitrate radical  \\
$-\! -$ & Sulfate, oxalate.
\end{tabular} 
\end{center}

For  the  purposes  of  discussion  these   entities   are   called
`sparkles':  the name arises from consideration of their behavior.

\subsection*{Behavior of sparkles in MOPAC}
Sparkles have the following properties:
\begin{enumerate}
\item Their nuclear charge is integer, and is $+1$, $+2$, $-1$,  or  $-2$;
there  are  an  equivalent  number  of  electrons  to  maintain
electroneutrality, $+1$, $+2$, $-1$, and $-2$ respectively.  For example, a 
`+'  sparkle  consists  of  a  unipositive  nucleus  and  an electron.  The
electron is donated  to  the  quantum  mechanics calculation.

\item  They all have an  ionic  radius  of  $0.7$~\AA.   Any  two sparkles  of 
opposite  sign  will  form  an  ion-pair  with  a interatomic separation of
$1.4$~\AA.

\item They have a zero heat  of  atomization,  no  orbitals,  and  no
ionization potential.

\item The associated one-center two-electron integral, G$_{ss}$ is 27.21 for
all sparkles.  Because of this, the monopole-monopole interaction, $AM$, is set
to 1.0.  This is different to that in earlier (before MOPAC 6) MOPACs where
the  value of $AM$ was set to 0.5D0.
\end{enumerate}

They can be regarded as unpolarizable ions of diameter $1.4$\AA.   They do 
not  contribute  to  the  orbital count, and cannot accept or donate electrons.

Since they appear as uncharged species  which  immediately  ionize, attention 
should  be  given  to  the  charge  on the whole system.  For example, if the
alkaline metal salt of formic acid was run, the  formula would be: HCOO+ where
`+' is the unipositive sparkle.   The charge on the system would then be zero.

A water molecule polarized by a positive  sparkle  would  have  the formula
H$_2$O$^+$, and the charge on the system would be +1.

At first sight, a sparkle would appear to be  too  ionic  to  be  a point
charge and would combine with the first charge of opposite sign it encountered.

This representation is faulty, and a better description would be of an  ion, 
of diameter $1.4$\AA, and the charge delocalized over its surface.
Computationally, a sparkle is an integer  charge  at  the  center  of  a
repulsion  sphere  having the exponential form $\exp(-\alpha r)$.   The
hardness of the sphere is such that other atoms or sparkles can approach within
about $2$\AA\ quite easily, but only with great difficulty come closer than
$1.4$\AA.

\subsection*{Uses of Sparkles}
\begin{enumerate}
\item They can be used as counterions, e.g.\  for acid anions  or  for
cations.   Thus,  if  the ionic form of an acid is wanted, then the moieties
H$\cdot$X, H$\cdot -$, and $+\cdot$X could be examined.

\item Two sparkles of equal and opposite sign can form a  dipole  for
\index{Dipole!made from sparkles} mimicking solvation effects.  Thus water
could be surrounded by six dipoles to simulate the solvent cage.  A dipole of
value  D can  be made by using the two sparkles + and $-$, or using ++ and {\bf
$ --$}.  If + and $-$ are used, the inter-sparkle separation would be
$D/4.803$\AA.  If {\bf $ ++$} and {\bf $ --$} are used, the separation would be
$D/9.606$\AA.  If the inter-sparkle separation is  less than $1.0$\AA\  (a
situation that cannot occur naturally) then the energy due to the dipole on its
own is subtracted from the total energy.
\end{enumerate}
