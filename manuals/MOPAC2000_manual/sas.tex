\subsection{Solvent Accessible Surface}\index{SAS|ff}
\index{Solvent Accessible Surface|ff}
The solvent accessible surface is a continuous surface of the molecule which
can be reached by the center of charge of a solvent molecule.  The calculation 
of the SAS is carried out as follows:

\begin{itemize} 
\item Each atom is assigned a van der Waals' radius.  VdW radii used in COSMO
are given in Table~\ref{vdw}.
\begin{table}
\caption{\label{vdw} Van der Waals radii (\AA ) used in COSMO}
\begin{center}
\begin{tabular}{llllllllllllll}
\hline
 I  & R    & II & R    &III& R    &IV & R    & V & R    &VI & R    &VII& R \\
\hline
 H  & 1.08 & \\
 Li & 1.80 &    &      &   &      & C & 1.53 & N & 1.48 & O & 1.36 & F & 1.30\\
 Na & 2.30 &    &      & Al& 2.05 & Si& 2.10 & P & 1.75 & S & 1.70 & Cl& 1.65\\
 K  & 2.80 & Ca & 2.75 &   &      &   &      &   &      &   &      &Br & 1.80\\
    &      &    &      &   &      &   &      &   &      &   &      &I  &  2.05\\
\hline
\end{tabular}
\end{center}
\end{table}

\item To each radius is added a distance equal to the radius of the solvent. By
default, this is 1.0\AA, but may be changed by the user using 
\comp{RSOLV=$n.nn$}. This gives the distance from the nucleus to the center of
a solvent molecule.
\item A set of points is generated on this surface. These points produce a
basic grid.
\item All points which are inside the surface of any other atom are excluded.
\item The remaining points are moved towards the center of the atom.  The
distance moved is equal to the distance of the center of charge of the solvent
molecule from the center of the solvent molecule.  By default, this distance is
set to \comp{RSOLV}, but may be set explicitly by keyword \comp{RSOLV=$n.nn$}.
\item Each of the remaining points represents a small area of the solvent
accessible surface.  The total SAS is calculated from the number of points.
\end{itemize}

From this definition of the SAS we see that the SAS of each atom is a  surface
of radius equal to the van der Waals' radius plus the radius of the solvent
molecule minus the distance of the center of charge of the solvent molecule to
the center of the solvent molecule.  In other words, the radius is the VdW
radius plus the distance from the surface of the solvent molecule to the center
of charge of the solvent molecule.  By default, this extra distance is zero.
Only that part of the atom surface which can be touched by the solvent molecule
is used.  This means that only those atoms on the surface of the molecule can
contribute to the SAS.  Of those atoms  that are on the surface of the molecule
there will be parts of the surface which cannot be reached by the solvent
because the solvent molecule is too bulky.
