\subsection{Considerations in Geometry Optimization}
The default settings in MOPAC are designed to allow most systems to be
optimized in an efficient way.  Quite often, however, problems arise.  The
following  notes are intended as background material for use when things go
wrong.

\subsubsection{Overriding the default options}
In the EigenFollowing geometry optimization method, the geometry is changed on
each cycle; if the $\Delta H_f$ decreases, the cycle is completed.  If it does
not drop, the step-size is reduced, and the $\Delta H_f$ recalculated.  Only
when the $\Delta H_f$ decreases, compared to the previous cycle, is the current
cycle considered to be successful.  During the calculation, the confidence
level or trust radius is continuously checked.  If this becomes too small, the
calculation will be stopped.  This can readily happen if (a) the geometry was
already almost  optimized; (b) a reaction path or grid calculation is being
performed; (c) if the geometry is in internal coordinates and ``big rings'' are
involved; or (d) if the  gradients are not correctly calculated (in a
complicated C.I., for example).

For cases (a) and (b), add \comp{LET} and \comp{DDMIN=0}.  In case (c) use
either mixed coordinates or entirely Cartesian coordinates. Case (d) is
difficult---if nothing else works, add \comp{NOANCI}; this will always cause
the derivatives to be correctly calculated, but will also use a lot of time.

Adding \comp{LET} and \comp{DDMIN=0} is often very effective, particularly when
reaction paths are being calculated.  The first geometry optimization might
take more cycles, but the resulting Hessian matrix is better tempered, and
subsequent steps are generally more efficient.

\subsubsection{Locating Transition States}\index{Locating transition states}
\index{Transition states!locating}\index{Narcissistic reactions}
Unlike optimizing ground states, locating transition states involves deciding on 
an efficient strategy.  In general, there are three stages in locating 
transition states:
\begin{enumerate}
\item Generating a geometry in the region of the transition state.
\item Refining the transition state geometry.
\item Characterizing the transition state.
\end{enumerate}
Of these three, the first is by far the most difficult.  The following 
approaches are suggested as potential strategies for generating a geometry in 
the region of the transition state.

{\bf For narcissistic reactions (reactions in which the reactants and 
products are the same, e.g.\ the inversion of ammonia.}
\begin{itemize}
\item Use  geometry constrains, e.g.\ \comp{SYMMETRY}, to lock the geometry in
the symmetry of the potential transition state.
\item Minimize the $\Delta H_f$.
\item Verify that the system is a transition state.  If it has more than one 
negative force constant, use another method.
\end{itemize}

{\bf For a bond making-bond breaking reaction (e.g., an S$_{N^2}$ reaction)}
\begin{itemize}
\item  Use \comp{SYMMETRY} to set the two bonds equal.  If does {\em not} 
matter that the bonds are of different type. For example, to locate the
transition  state for Br$^-$ reacting with CH$_4$ to give CH$_3$Br, the C--Br
and C--H bonds  would be set equal.
\item  Optimize the geometry, to minimize the $\Delta H_f$.  Any geometry
optimizer could be used, but of course the default optimizer should be tried
first.
\item Remove the symmetry constraint, and locate the transition state using
\comp{TS}.
At this point, the main geometric change is to adjust the two bond lengths
involved in the reaction.
\end{itemize}

{\bf For barriers to rotation, inversion, or other simple reaction that 
does not involve making or breaking bonds}
\begin{itemize}
\item Optimize the starting geometry.
\item Optimize the final geometry.
\item Identify the coordinate that corresponds to the reaction. This is likely
to be an angle or a dihedral.
\item Starting with the higher energy geometry, use a \hyperref[pageref]{path option}{ (see 
p.~}{)}{rpaths} to drive the reaction in the direction of the other
geometry.   Use about 20 points, and go about half way to the other
geometry---the transition state is likely to be between the higher energy
geometry and the half-way point.
\item From the output, locate the highest energy point---this will be near to
the transition state.
\item Starting with the geometry of the highest energy point, repeat the path 
calculation.  Use smaller steps (0.1 times the previous step is usually OK),
and again do 20 points.
\item Inspect the reaction gradient.  It should drop as the transition state is
approached.  If it does, then use \comp{TS} to refine the transition state.
\end{itemize}

{\bf For bond making or bond breaking reactions}
\begin{itemize}
\item Identify the reaction coordinate (the bond that makes or breaks)
\item Use a \hyperref[pageref]{path calculation}{ (see 
p.~}{)}{rpaths} to drive the reaction.
\item The geometry of the highest point on the reaction path should then be 
used to start a \comp{TS} calculation.
\end{itemize}

{\bf For complicated reactions (e.g.\ Diels Alder)}
For these systems, the \comp{SADDLE} calculation is a suitable method.
\begin{itemize}
\item Optimize the reactant geometry.
\item Using the same atoms in the same sequence, optimize the product geometry.
\item Run the \comp{SADDLE} calculation.
\item If the calculation ends because ``both reactants and products are on
 the same side of the transition state,'' use two of the geometries to set
up a new \comp{SADDLE} calculation.  Use a smaller \comp{BAR=$n.nn$}, e.g.,
\comp{BAR=0.03}, and re-run the calculation.  If CPU time is not important,
run the original data set with \comp{BAR=0.03}.
\item Use the final geometry, or the highest energy geometry, if the 
\comp{SADDLE} does not run to completion, as the starting point for a 
\comp{TS} calculation.
\end{itemize}
