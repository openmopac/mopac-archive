\chapter{Program MAKPOL}
\index{MAKPOL!description}
Construction of data sets for multiple unit cells is  tedious.  By automating
this process, only the fundamental unit  cell need be specified, the other unit
cells being generated  by translation.  In addition, since the geometry of all 
fundamental unit cells in a polymer are identical, symmetry  conditions can be
imposed on each unit cell.  This has two  advantages: first, the calculation
runs significantly faster, and  \index{Periodic boundary conditions} second,
periodic boundary conditions are imposed not only on the  whole cluster, but
also on each fundamental unit cell.  

A data set for MAKPOL is similar to a data set for MOPAC.  It consists of one
or more lines of keywords, followed by comments (if needed), then a geometry in
Cartesian or internal coordinates.  The name of the input data set should be of
form `\comp{make~$<\!$filename$\!>$.dat}', in which case the resulting data set
will be `\comp{$<\!$filename$\!>$.dat}'.


MAKPOL uses the unit cell supplied by the user to generate all the unit cells
used in the cluster unit cells. \label{makpol:mers} The number of unit cells in
the cluster is determined by the keyword \comp{MERS($n3$,$n2$,$n1$)}.  If the
system is a layer structure then ``$n3$,'' is omitted. If the system is a
polymer, then ``$n3$,$n2$,'' is omitted. The order in which cells are generated
is as follows:
\begin{sloppypar}
Cell(0,0,0), Cell(0,0,1), Cell(0,0,2), \ldots Cell(0,0,N1-1),
Cell(0,1,0), Cell(0,1,1), \ldots Cell(0,N2-1,N1-1), Cell(1,0,0),
\ldots Cell(N3-1,N2-1,N1-1).
\end{sloppypar}

The geometry of the extended system is first calculated in Cartesian
coordinates. There are two ways the atoms can be arranged.  The default order
is as follows: all atoms in the first unit cell, all atoms in the second unit
cell, etc.  An alternative is to have atom one in all the unit cells then atom
two in all  unit cells, then atom three, etc.  The choice of which order to use
depends on the purpose for which the cluster data set will be used.

The whole system is then converted into internal coordinates, and the 
translation vectors added.   If wanted, symmetry conditions can be imposed -
this is useful when the geometry is to be optimized.

\section{Keywords used by MAKPOL}
Any of the keywords used by MOPAC can be used in a MAKPOL data set, but, only a
few keywords will be used by MAKPOL: the rest will be ignored.  Keywords that
are used in  MAKPOL are given below.

\begin{description}
\item[\comp{BCC}]~\\
When \comp{BCC} is added (Body Centered Cubic), all odd unit cells are omitted.
An odd unit cell is one for which the addition of the cell indices results in
an odd number, thus (0,0,0) and (1,1,0) would be allowed, but (1,1,1) and
(2,1,0) would not.  Diamond is an example of a BCC solid.\label{makpol:bcc}

\item[\comp{DEBUG}]~\\
Print diagnostic data on how MAKPOL is working. This is not a particularly
useful keyword---do not use it for routine work. 

\item[\comp{LET}]~\\ 
By default, when \comp{SORT} is {\em not} used, all atoms in the fundamental
unit cell will be placed inside  the boundaries of the unit cell.  Any atoms
that were outside the boundaries will  be translated inside the unit cell. 
This is necessary for the band structure calculation (in BZ). If you do not
want the atoms to be moved, add \comp{LET}.


\item[\comp{MERS=($n3$,$n2$,$n1$)}]~\\
The number of unit cells in each  direction is defined by $n3$, $n2$,  and
$n1$.  The total number of unit cells generated will be $n3\times n2\times n1$,
or, if \comp{BCC} is used,  $(n3\times n2\times n1)/2$.

\item[\comp{SORT}]~\\
The default order of atoms in a cluster is as follows: all atoms in the first
unit cell, all atoms in the second unit cell, etc.  An alternative order can be
requested by adding \comp{SORT}.  In this case, the order of occurrence is:
atom one in all the unit cells, atom two in all unit cells, atom three, etc. 
The choice of which order to use depends on the purpose for which the cluster
data set will be used.

For geometry optimizations, \comp{SORT} is useful, particularly when used with 
\comp{SYMMETRY}.  

For band structure calculations, \comp{SORT} should {\em not} be used---these
calculations require the default order of atoms.  To emphasize this point, if
\comp{SORT} is specified, then the keyword \comp{MERS} will be deleted from the
resulting data set.

\item[\comp{SYMMETRY}]~\\ 
This keyword has two distinct functions.  When \comp{SYMMETRY} is specified,
then any symmetry data present in the input data set will be used.  This is
useful for reducing the number of variables that need to be changed on going
from one system to another (for example, on going from diamond to cubic boron
nitride). If no symmetry data are present, then this function will not be used.

The second function of \comp{SYMMETRY} is to impose symmetry conditions on the
resulting cluster data set.  Only four symmetry conditions are recognized: 
bond-lengths that are equal, bond angles that are equal, equal dihedral angles,
and dihedral angles that are the negative of reference dihedrals.

For high symmetry systems, the cluster data set should be edited to increase
the symmetry.  For example, in diamond, all angles and dihedrals are symmetry
defined, so all the optimization flags for these variables should be set to
zero, and all symmetry functions involving angles and dihedrals should be
deleted. In addition, all ``bond lengths'' in diamond are simple multiples of
the fundamental C-C distance, so extra symmetry functions (involving function
19) should be added.  If this is done correctly, then only one geometric
variable will be left.
\end{description}

A test-example of the MAKPOL input-data for diamond is given in
Figure~\ref{makpolc}.
\begin{figure}
\begin{makeimage}
\end{makeimage}
\index{Data!for diamond}
\begin{verbatim}
 bcc mers=(4,4,4)
 Diamond, 64 atoms

 c
 xx  0.785065  1
 c   0.785065  0   180       0     0 0     1
 xx  1.0000000 0    54.73561 0     0 0     1 2 3
 xx  1.0000000 0    54.73561 0   120 0     1 2 4
 xx  1.0000000 0    54.73561 0   240 0     1 2 4
 TV  1.8130299 1     0.0     0     0 0     1 4 2
 TV  1.8130299 1     0.0     0     0 0     1 5 2
 TV  1.8130299 1     0.0     0     0 0     1 6 2
\end{verbatim}
\caption{\label{makpolc} Data set for Diamond for MAKPOL}
\end{figure}
