% NOTE: this section had lots of unprintable (non-ASCII) characters in it
% which were screwing things up royally.   I got about 2/3rds through it
% before realising that they should be quotes.   At least some should be.
% Some others could be anything.   Might be worth checking this
% against the ``master copy" and putting the buggers back. CSP 1-5-99

%\subsection{Overview}
%MOPAC has been developed over the past 15 years with the objective of being a
%tool for the study of chemical systems.  For such a tool to be useful, it must be
%fast, accurate, versatile and easy to use.  These objectives have guided the
%development of MOPAC, so that today the program has used successfully in
%many research and teaching projects.

\subsection{Validation}
Before potential users start working with MOPAC, they must be reasonably sure
that the program works as described.  The testing of a new MOPAC is quite
lengthy.  The basic tests verify that all the functions work. This involves
running many hundreds of systems, many of which are highly exotic.  Once all
tests of this type are successful, pairs of functions are tested.  For example,
two  functions in MOPAC are (1) the ability to apply an external electric
field, and  (2) to optimize the geometry of a molecule.  By combining these
functions, the  geometry of a molecular system can be optimized in the presence
of an external  electric field.  Systems that would otherwise not interact can
be made to weakly  bond together under the influence of an external field.
With the wide range of  possible functions, testing all pairs of functions is
not practical; however, most  reasonable pairs of functions have been tested to
ensure that they work correctly.

The number of possible combinations of three or more functions (an example of
a three function calculation would be to calculate the path of an implanting
ion  as it approaches a solid surface while it is accelerating under the
influence of a  potential gradient).  Several of these have been tested.  There
is no reason to  assume that all such combinations of functions will work, but
at the end of  extensive testing, all calculations attempted did work.

\subsubsection*{Algorithmic validation}
A powerful device to validate a program is by examination of the
computer code involved.  Reading the program is not practical:
MOPAC 2000 is over 80,000 lines long. Instead various software
validation tools can be used.  The more important of these are:
\begin{itemize}
\item FTNCHEK:  Checks conformance with the ANSI FORTRAN 77 Standard.
MOPAC has known exceptions to the standard, but all of these have been
independently checked.  Exceptions include the use of machine-specific
code for determining the current date and the amount of CPU time used.

\item IEEE: Some machines support testing for conformance with the Institute of
Electrical and Electronics Engineering standards (IEEE standard 754), for
validity of arithmetic operations.  Again, with a few known and tested
exceptions, MOPAC conforms with this standard.

\item Some machines allow the initial state of otherwise undefined numbers to be
defined.  By setting all undefined numbers to NaNs (Not a Number), any
assumption of initial values is easily detected.

\item The validation tests were carried out with array bounds checking.  Of
course, any errors detected during this test were corrected.

\item Only one top copy of MOPAC exists.  This copy is designed to run on
many different platforms.  Because of this, any errors detected on one
platform can be corrected, and the correction is then automatically
propagated to all platforms.
\end{itemize}

\subsubsection*{The Beta Test}

MOPAC can be run on a wide range of platforms, from Windows 95, 98, and  NT,
through workstations such as the RS6000 and SGI UNIX machines, to  parallel
supercomputers, such as the Fujitsu VPP and AP3000.  After porting,  the
program is sent to several different research groups around the world.  These
groups have volunteered to test MOPAC for use in their own on-house research
programs.  These tests are real in the sense that the testers are interested
in  performing serious research, and any limitations found would be the cause
of  severe irritation.  In practice, very few bugs are uncovered by this test,
but all  such bugs are corrected before the final release.

\subsubsection*{Usage}

Over 50,000 copies of MOPAC have been requested.  This large installed base
means that bugs and limitations are rapidly found and reported.  Where
practical, these bugs and limitations are corrected, and consequently the
program  has become quite robust.  Of course, with any major new release, such
as  MOPAC 2000, new bugs are likely to be introduced.  Most of these, however,
will be detected and corrected rapidly.

\subsubsection*{Support}

Although MOPAC is routinely maintained in the status of no known bugs,
problems do occur.  If a user suspects that there is a bug in the program,
assistance can be requested by telephone, E-mail, or FAX.  Suspected bugs are
given a high priority, and most are resolved within one working day.  In the
case  of genuine bugs, a bug-fix or patch is provided to the user.

\subsection{Geometry Optimization}

\subsubsection*{General}

One of the commonest operations in computational chemistry is optimizing the
geometry.  This involves modifying the geometry until the energy is a
minimum.   At that point, the net forces acting on every atom vanish.  Over the
years,  various methods for optimizing geometries have been developed.  One of
these,  Baker's EigenFollowing procedure, has proven to be very robust, and
this is  now used as the default.  If, for any reason, the EF method is not
wanted, other  methods are available.

\subsubsection*{Constrained by definition}

When only part of a system is to be optimized, options are provided to
constrain  other parts of the system.  For example, a bond length, angle, or
dihedral for one  or more atoms can be defined as fixed at some initial value,
or the $x$ or $y$  or $z$ coordinate can be defined as unchanging.

\subsubsection*{Constrained by symmetry}

Some systems have symmetry that can be used to accelerate geometric
operations.  Thus in Fullerene, C$_{60}$, all atoms are in the same
environment, and  there are exactly two unique bond-lengths.  The geometry
optimization  calculation can be reduced from order 174 (3$\times$60-6) to
order two by use of  symmetry.  Symmetry can also be used in unsymmetric
systems.  For example,  in determining the transition state for the S$_{N^2}$
reaction  Br$^-$ + CH$_4$ $\rightarrow$ CH$_3$Br + H$^-$,  the C--Br and C--H
distances can be set equal, and the geometry optimized.  After  releasing the
symmetry constraint, the transition state geometry is determined in  one step
by use of Baker's EigenFollowing method.

\subsubsection*{Energy Minimization}

The degree to which geometries can be optimized can be varied according to the
user's need: from the default, which is the best compromise between precision
and speed, to high precision for publication work, to low precision for rapid
screening.  Unconditionally, the degree of optimization can be displayed,
although the default is for this information to be suppressed unless problems are
encountered.

Unless special action is taken, only local minima are located.  Thus optimization
of the geometry of dimethyl ether would not yield ethanol, nor would
optimization of {\em gauche} butane yield {\em trans} butane.
\subsubsection*{Transition State location}

Several methods are provided for refining transition states.  One, Baker's
EigenFollowing, is usually sufficient for most systems; the others are provided
for the rare instances when the default TS does not work.

\subsection{Methods}

\subsubsection*{MNDO-$d$}

The full MNDO-$d$ method is supported.  Currently, parameters are available for
Al, Si, P, S, Cl, Br, and I only, although new parameters will be added as they
become available.

\subsubsection*{NDDO Methods}

Three NDDO methods are supported: MNDO, AM1, and PM3.  Parameters for  PM3 are
available for the following elements: H, Li, Be, B, C, N, O, F,  Mg, Al,  Si,
P, S, Cl, Zn, Ga, Ge, As, Se, Br, Cd, In, Sn,  Sb, Te, I, Hg, Tl, Pb, Bi.  For
all  methods, 100\% ionic charges are provided; these mimic Group I and II
ions.  A  unique type of atom that has a single valence, and which takes on
the  electronegativity  of whatever atom it is attached to is provided.  This
atom, called Cb for Capped Bond, is designed for use where dangling bonds
might  be created, for example, when a fragment of a system is being studied.

\subsubsection*{MINDO/3}

The MINDO/3 method is available for most functions.  However, it is likely that
MINDO/3 will not continue to be supported.

\subsubsection*{New Parameters}

As new parameter sets become available, a mechanism must exist to allow these
parameters to be used by MOPAC.  The simplest way is to convert the
parameters into a small data set, and then to make this data set known to
MOPAC, by use of a keyword.  Only the unique parameters need be defined in
the data set, any derived parameters are computed internally.

\subsection{SCF Procedures}

\subsubsection*{Restricted Hartree Fock}

The default self-consistent method is restricted Hartree-Fock.  This allows both
closed shell and open shell systems.  For open shell systems, errors due to the
half electron approximation are automatically corrected.

\subsubsection*{Unrestricted Hartree Fock}

At user's choice, unrestricted Hartree-Fock SCF calculations can be run.  In these,
the $\alpha$ and $\beta$  spin molecular orbitals have different spatial forms.
When UHF is
used, the expectation values of the S and S$^2$ operators are printed. In
addition to printing the $\alpha$ and $\beta$ spin molecular orbitals, the associated $\alpha$ and $\beta$
spin density matrices can be printed, as can the $\alpha - \beta$ spin
density matrix.

\subsubsection*{SCF-CI}

The C.I.\ in MOPAC is sufficiently versatile to allow almost any closed or open
shell system, in any state, to be calculated.  Unpaired spin densities can be
generated.  Geometry optimization can be performed on both ground and excited
states, including degenerate states (both with and without Jahn Teller
effects).   This can be contrasted to the MCSCF method, which is not supported,
in which  the SCF is performed using several configurations.

\subsection{Derived Properties}

\subsubsection*{Bond orders}

Several types of bond orders can be displayed.  The simplest is the Wyberg
indices, which mirror the simple ideas of single, double, and triple bonds.
More  complicated representations include Mulliken populations,
delocalizability,  superdelocalizability, free valence, and spin density.

\subsubsection*{Charges}

By default, the net charge on each atom is printed.  These are the Coulson
charges, although if requested, Mulliken charges can also be printed.

\subsubsection*{Dipole moment}

By default, the dipole vector (calculated from atomic charges and the lone
pairs)  is calculated and printed, along with the net dipole moment, in Debye.

\subsubsection*{Static (zero frequency) Polarizability}

The polarizability of a system is a measure of the response of the electron
density distribution to a static electric field .  This can be calculated two
ways,  by the application of electric fields (the static method) and by direct
analysis of  the wavefunction (the sum over states method).  For large
systems, the  application of electric fields method is faster, although the sum
over states  method is more precise for all systems.   Because of the limited
precision of the  static method, a way has to be provided to allow the user to
determine the  precision.  Static polarizability (and first and second
non-linear optical  responses) can be determined from the changing dipole or
from the changing  heat of formation.  By printing the results of the
calculation based on both dipole  and $\Delta H_f$, side by side, the user can
get an estimate of the precision.

\subsubsection*{Frequency dependent Non-Linear Optics}

In addition to the polarizability, the following non-linear optical properties can
be calculated: first and second order properties ($\beta$ and $\gamma$),
Electric Field Second
Harmonic Generation (EFISH), optical rectification, electrooptic Pockel's
Effect, second and third harmonic generation. Where appropriate, averages are
printed.

\subsection{Graphics}

Although MOPAC does not have any graphics, it has the capability of  generating
data for use by graphics packages.  This data includes information on  the
molecular orbitals, the overlap matrix, and the orbital exponents, etc.
Graphics packages can then use the data to generate contour maps of M.O.s,
density, density differences, etc.

\subsection{Types of Species}

\subsubsection*{Atoms}

Simple calculations on atoms yield conventional heats of formation.  These are
the heats of atomization of the elements.  Most atoms have open shells in the
ground state, and these can be allowed for by use of appropriate keywords.  By
default, the electronic state of the system will be printed.  Excited states
can be  readily calculated, and the transition dipole for photoexcitation can
be printed.   If, as is common, the originating state (usually the ground
state) or the  terminating state (usually an excited state) is degenerate, the
degeneracy is taken  into account in the calculation of transition dipole.  The
conventional selection  rules can be derived by an analysis of the states and
transition dipoles.

\subsubsection*{Molecules}

The commonest type of system calculated consists of neutral, polyatomic
molecules.  These can be simple, closed shell species, such as benzene, to
radicals, e.g.\ nitric oxide, to zwitterions, to multiple open shell radicals,
such as NH$_{4}\cdot$, to excited electronic states, e.g.\ $n\pi^{*}$ pyridine.

\subsubsection*{Ions}

Calculations can be performed on ionized species, both isolated and with
counterions, in both gas phase and solvated.  All degrees of ionization are
allowed.

\subsubsection*{Polymers}

Regular polymer  systems for which periodic boundary conditions can be
imposed can be calculated.  The time for such calculations is about 30\%
greater than for a discrete molecule of the same size as the unit cell used.
Geometries, including unit cell length, can be optimized.

\subsubsection*{Effect of Stretching}

In polymers, a reaction can be set up, in which the translation vector
distance  can be steadily increased.  The effect of this is to steadily
increase the distance  between the repeat units of the polymer.  Initially, the
energy would not change  significantly, as any conformational flexibility in
the polymer is used up, but  once that is done, the energy of the system (a
measure of the stress) would rise  parabolically with increased strain.  From
an analysis of this parabola, and given  the density of the polymer, the
Young's modulus can readily be calculated.  If  the strain is increased without
limit, a point would be reached at which the  weakest bond in the polymer would
break, and the stress would immediately  drop to zero.  From this, and the
experimental density, the tensile strength can be  calculated.

\subsubsection*{Vibrations (Phonon Spectrum)}

As with molecules, the vibrational spectrum of polymers can be calculated.
However, unlike molecular normal modes, which have quantized vibrational
frequencies, polymers give rise to bands of frequencies, defined by the
associated wave-vector $k$ in the Brillouin zone. At $k$=0, four of these bands have
zero frequency, corresponding to the four trivial vibrations of a polymer.

\subsubsection*{Defect polymers (solitons)}

Excitations can occur in some polymers, to give rise to ion pairs, isolated
electronic excited states, and other electronic defects.  These can be modeled
using oligomers.  A simple excited state would be in polyacetylene, for
example, in which one carbon atom was singly bonded or doubly bonded to two
other carbon atoms.  The effect of an applied electric field to induce hopping
of solitons can be modeled.

\subsubsection*{Layer Systems }

By using the Born-von K\'{a}rm\'{a}n periodic boundary conditions, layer
systems can  be modeled.  Geometry optimization, including optimizing the unit
cell  dimensions, can be carried out.

\subsubsection*{Solids}

By using the Born-von K\'{a}rm\'{a}n periodic boundary conditions, regular
solid  systems can be modeled.  Geometry optimization, including optimizing the
unit  cell dimensions, can be carried out.

\subsubsection*{Compressivity}

For cubic systems, the compressivity of the crystal can be deduced by
calculating the $\Delta H_f$  in a reaction in which the unit cell dimension
is  systematically varied.

\subsubsection*{Electronic Structure (Brillouin Zone)}

A utility program for analyzing the electronic structure of polymers, layer
systems, and solids is provided.  This uses output from MOPAC, the  space-group
symmetry operations, and interactive user input to  generate the Little  groups
for points in $k$-space, band structures, and cross-sections  through
$k$-space  for selected bands.  The first step in this analysis is to
symmetrize the energy  matrix for the solid.  After this is done, the resulting
structures in $k$-space are  fully symmetry adapted.

\subsection{Giant Molecules}

\subsubsection*{Linear Scaling}

A new method for solving the SCF equations is implemented.  In this method,
the time required for calculating the SCF increases linearly with the size of
the  system.  The memory demand is also considerably reduced (for the larger
systems run, by 97 to 98 percent of what would be needed conventionally). This
allows systems of many thousands of atoms to be calculated rapidly.

\subsubsection*{Lewis Structures}

The Lewis structures for most systems can be printed.  A limitation is that
systems involving atoms with more than four bonds cannot be handled, although
in practice, this can partly be circumvented by breaking bonds. Systems as
complicated as  fullerene, TCNQ, SF$_6$, cystein zwitterion, and benzene, can
be  described.

\subsubsection*{Determining Net Charge}

Determining the net change on a large system, such as a protein, can be
tedious.   The simplest way would involve examination of the structure, and
identifying  each ionized site.  This task can be performed rapidly by running
a simple  calculation.  If a MOZYME calculation is run, then any ionized sites,
and the net  charge on the system, will be printed.  It is not necessary to
perform an SCF  calculation, so this operation is very rapid.

Frequently, the geometry supplied contains errors.  If any of these errors
would  prevent a meaningful calculation being run, then a brief description of
the errors  detected is printed.  This speeds up the preparation of data sets.

\subsubsection*{Residue Sequence}

If requested, the residue sequence, along with information on the formal
charges  on each residue, can be printed.  Both the one letter and three
letter  abbreviations are given.  Non-standard residues can be accommodated by
use of  keywords.

\subsubsection*{PDB input/output}

Molecular geometries can be read in using Brookhaven Protein Data Bank
format.  Regardless of the input format, geometries can be output in PDB
format, under user control.  There are several commonly used minor variations
on the PDB format definition,  most of which are accommodated in the input,
and standard PDB format is used in the output.

\subsubsection*{Charge on Residues}

The net charge on residues is of interest in protein chemistry.  This quantity
is  printed for each residue, and is broken down into the contribution from
the  backbone atoms and the side chain atoms of each residue.

\subsubsection*{Description of Electronic Structure}

The standard description of the electronic structure of giant molecules is given,
although, because the default format would produce a very voluminous output,
the format is changed.  Normal descriptors include: Atomic charge, valence,
bond orders $\sigma - \pi$ decomposition, and dipole.

\subsubsection*{Partial Geometry Optimization}

When localized molecular orbitals are used, the M.O.s associated with each
atom can readily be identified.  This means that when a partial geometry
optimization is run, only the LMOs involved in the moving atoms need be used.
A consequence of this is that such calculations run much faster.  For large
systems, in which only a few atoms move, the largest increase in speed over
conventional methods was a factor of 13,000 times.

\subsection{Symmetry theory}

Symmetry is very important in chemistry, and, in recognition of this, MOPAC
makes extensive use of symmetry theory.  Most of the symmetry theory is done
automatically, without any user action being necessary.  The simplest
operation  is the recognition of the point-group of a molecule.  All
non-magnetic single  point groups up to order 7 and most of the groups up to
order 8 are recognized.   This set includes the three infinite groups (R$_3$,
the order of the sphere,  C$_{\infty v}$, and  D$_{\infty h}$) for all
chemically realizable representations, and the seven  cubic point-groups.
There is an ambiguity in the definition of some irreducible  representations of
some groups, such as $b_1$ and $b_2$ in C$_{2v}$.   This ambiguity is  resolved
using the conventional definition for orienting molecules.  To allow for  the
fact that molecular geometries might not be completely precise, a certain
tolerance is built in to the test for point-groups.  Sometimes, a user might
wish  to use a sub-group of the full point-group.  This can be done by using
\comp{NOREOR}, or, if symmetry theory is not wanted at all \comp{NOSYM} can be
specified.

\subsubsection*{Symmetry Labels }

Symmetry labels are automatically assigned to molecular orbitals, vibrations,
and electronic states.  These labels are of the form $nR$, where $R$ is the
irreducible representation and $n$ is the $n$th occurrence of that
representation.  For each eigenfunction, the symmetry label is unique, and for
this and other reasons, the symmetry labels are true quantum numbers.  This is
particularly evident in electronic states, where the symmetry label includes
information on the spin state.

\subsubsection*{Symmetrizing vibrations}

To facilitate vibrational analysis of high-symmetry systems, symmetry theory is
used to symmetrize the force matrix.  This has the side effect of reducing any
errors introduced by finite precision mathematics.  The resulting normal modes
are completely symmetry adapted, and subsequent analysis of these modes is
made much easier.

\subsubsection*{Accelerating Calculation of Vibrations}

For molecules that have symmetry, symmetry theory is automatically used to
accelerate the construction of the force matrix.  This reduces the time needed
for  the calculation of normal modes.  For C$_{60}$, the increase in speed is a
factor of  about 40.

\subsection{Electric Fields}

The effect of applied external electric fields can be modeled.  The fields are
uniform, and the orientation and intensity of the field is under user control.
The  applied fields cannot be used in the translation directions of infinite
systems.

\subsection{Electrostatic Potential}

Four electrostatic potential methods are available, of which two can generate
data for use by graphics programs. These are the Wang-Ford Parametric
Electrostatic Potential (PMEP) and the Merz-Bessler ESP methods.  The PMEP
method gives results similar to those from ab initio 6-31G calculations, but
is  limited in the range of atoms allowed.  In contrast, the ESP method is
quite  general.

\subsection{Solvent effects}

Two solvent models are provided: the COSMO technique and the MST or  Tomasi
method.  In the COSMO method, excited state systems in solution can  be
modeled, and the geometries of solvated systems optimized.  The Tomasi  method
includes cavitation and surface tension effects.  Both methods allow the  user
to define the properties of the solvent.  In the Tomasi method, keywords  can
be used to define specific solvents, i.e., water, chloroform, and carbon
tetrachloride.

\subsection{Electronic Excited states}

MOPAC contains an extensive configuration interaction package.  This allows a
wide range of open and closed shell ground and excited state phenomena to be
modeled.  Examples of the types of systems that can be modeled include:
methane, stabilized by mixing in excited states; oxygen, with an open shell
ground state; methane cation, with and without Jahn Teller effects, and high
spin  systems (up to nonet).  Keywords are provided for the commonest types of
C.I.\  (single electron excitation, single plus double, single plus paired
double, single,  double, and triple excitations).  All other types of C.I.\ can
be defined explicitly,  under keyword control.

Internal checks are automatically carried out to ensure that the calculations
do  not violate any theoretical rules, although these constraints can be
relaxed, at  user discretion.  State spin and symmetries are automatically
assigned.

The effect of C.I.\ on electron density distributions can be modeled, both in
ground and excited states, and for ground state and vibrational states.

\subsection{Intersystem Crossing}

In some reactions, the electronic state of the system can change.  The
geometry  at which this change occurs can be modeled.  This geometry can be
defined as  the minimum energy geometry for two degenerate states. This
technique is  likely to be of use in photochemical research, for example in the
photographic  industry.

\subsection{Vibrations}

The normal modes of vibration of a stationary system can be calculated.  For
ground states, this consists of the $3N-6$ or $3N-5$ non-zero modes, while for
simple transition states, the $3N-7$ real modes and 1 imaginary mode are
reported.   If desired, the force constants for the system can be printed.

\subsubsection*{Description of Vibrations}

Because of the complexity of normal modes, an analysis of these modes is
printed.  This analysis allows the nature of the normal mode (i.e., X-Y
stretch,  A-B-C bend, etc) to be rapidly described.  The normal coordinates are
printed. However these are of limited use because they are velocity vectors and
do not  indicate the energy carried by each atom.  An additional display shows
the effect  of mass-weighting the normal modes; this gives an alternative view
of the  molecular vibrations.

\subsubsection*{Effective Mass and Travel}

To the degree to which normal modes can be described as a simple harmonic
oscillator, the effective mass involved can be calculated.  For a homonuclear
diatomic, this is half the atomic mass; if one of the atoms is extremely
massive,  the mass is approximately that of the other atom.

From quantum theory, the energy of vibration is quantized.  Therefore, given a
knowledge of the force constant for the vibration, and the effective mass, the
excursion distance (in mass weighted space) can readily be calculated.  By
default, when normal modes are calculated, these two quantities are printed.

\subsubsection*{Transition Dipole}

The relative intensity of an infra-red active band is a simple function of the
transition dipole.  This quantity is printed whenever the vibrational analysis
is  printed.

\subsubsection*{Internal Coordinate Force Constants}

Although not an observable, the internal coordinate force constants are often
of  interest.  These quantities are printed at the end of a normal coordinate
calculation.

\subsubsection*{Isotopic Substitution}

Although calculating the force matrix in a normal coordinate analysis is often
lengthy, the final stage, mass weighting and generating normal modes, is very
rapid.  To allow different isotopic masses to be used in a single normal
coordinate analysis, the option exists to save the force matrix.  The isotopic
masses can then be changed, and the old force matrix used again.  Such
calculations are very rapid.

\subsubsection*{Trivial Modes, projecting out}

A minor increase in precision is achieved by projecting out the six trivial
modes.   These can subsequently be printed out.  The order of the modes is then
defined  as $x$, then $y$, then $z$ translation, followed by the three
rotations, about the moments  of inertia.

\subsection{Thermodynamics}

Various thermodynamic quantities (Partition function, enthalpy, heat capacity,
and entropy) can be calculated for any temperature, or range of temperatures.
These quantities can be decomposed into vibrational, rotational, internal, and
translational contributions.  The effect of changing temperature on the
$\Delta H_f$ can be monitored.

\subsection{Molecular Dynamics}

\subsubsection*{Conservation of Energy}

The time evolution of a system can be investigated.  The starting point can be
either a stationary point (optimized geometry or a transition state) or a
non-stationary point, and the initial velocity vector can  be zero, or
determined by one  of the normal modes, or an arbitrary (user supplied)
vector.  At each step, the  position of each atom is modified by (a) the forces
acting on it and its isotopic  mass, (b) the velocity vector of the atom, and
(c) information on the  acceleration, and rate of acceleration, of the atom.
During the course of the  molecular dynamics, the total energy (kinetic plus
potential) is constant  as the  system moves down a potential energy surface,
the velocity increases, and vice  versa.

\subsubsection*{Simulated Annealing}

Although the default is to conserve energy, the option exists to allow kinetic
energy to be reduced, with the rate of reduction being a function of time, by
specifying a half-life in femtoseconds.  The effect of this option is to
simulate  cooling of the system, analogous to the molecular mechanics method
of  simulated annealing.

\subsubsection*{Simulated heating}

No constraint is placed on the sign of the half-life, consequently, a negative half
life can be used.  This simulates heating of the system.  Invariably, unlimited
heating results in atomization of any compound.

By an appropriate choice of keywords, any combination of heating and cooling,
including conserving energy, can be modeled in a single run.

\subsubsection*{Sampling}

Sampling of the state of the system can be done several ways: each point
calculated can be printed, or constant steps in time, energy, or position can be
chosen.  At each point printed, the amount of information printed is determined
by keywords, from a single line giving simple energetics, to the energetics plus
geometry plus velocity vector.

\subsubsection*{Simulated vibrations}

The easiest way to calculate vibrational frequencies is to calculate the force
matrix and perform a normal mode analysis.  While this is acceptable most of
the time, in some systems the normal modes are sufficiently
non-simple-harmonic that these vibrational frequencies are unacceptable.

An alternative is to set the system in motion, using the normal modes
calculated  conventionally, and to determine directly the period of vibration.
For well  behaved systems, this is almost exactly the same as that given from
the force  matrix, verifying the internal consistency of the various methods.

For very simple systems, e.g.\ diatomics, a second alternative is to plot the
energy coordinate graph, and determine the vibrational period from either
the  energies or the gradients.

\subsubsection*{Intrinsic Reaction Coordinate}

The time-independent behavior of a system can be modeled using the IRC
option.  In this, the atoms in a system move in response to the forces acting
on  them, moderated by their isotopic masses, however, at each step all
kinetic  energy is annihilated.  The effect is to produce a time-independent
trajectory, the  steepest decent from the starting geometry to a stationary
point.

\subsection{Reaction Paths}

Where a definable reaction coordinate can be identified, this can be used to
drive  a chemical reaction.  For example, in a bond-breaking bond-making
reaction,  the bond being made can be used as the reaction coordinate.  Several
options are  provided for specifying reaction paths, the three most common of
which are: (a) to supply  the various values of the reaction coordinate as
extra data (at each point on the  reaction path, the gradient of the path is
calculated); (b) to define a  step size and  number of point to be calculated
(at the end of the calculation, the  $\Delta H_f$  are  printed in a form
suitable for plotting); and (c) to give two step sizes and  numbers of points
in two directions.  This option is useful in mapping out a  potential energy
surfaces.

\subsection{Saddle-Point Location}

The geometry of the transition state is not always easy to define. The option
exists to allow the reactants and the  products to be defined, and to allow the
saddle or transition state between these  two systems to be calculated. Recent
changes in this function have made it more  robust, so that now the failure
rate in locating transition states is very low. (In  practice, transition
states have been located for all reactions investigated.)

\subsection{Data checking}

In practice, almost all the errors that occur are a result of user error.  To
allow  for this, extensive data checking is done.  Where possible, corrective
action is  automatically taken; if this is not possible, the run is stopped
before much time  is used, and an error message printed.  Most error messages
are self-explanatory.

\subsection{Restarts}

For various reasons, a job might stop before the calculation is complete.  To
allow for this, intermediate results are output in a form that can be used to
restart  the calculation.  Restarts are a useful way to change the course of a
calculation,  thus a MOZYME job could be stopped and restarted as a MOPAC
job.   Likewise, an AM1 job could be restarted as a MNDO-$d$ calculation.

\subsection{Portability}

The program is written in almost pure FORTRAN-77.  Minor exceptions from  the
ANSI standard are necessary in order to allow CPU time and current date to  be
used.  MOPAC 2000 currently runs on 16 different platforms.  Extension of  this
set to still more platforms is expected to be straightforward.

\subsection{Program Structure}

The source code of MOPAC is designed for ease of reading and modification.
Although there are many structures in MOPAC, once these are learned,
navigating within the program is relatively easy.  Every occurrence of the
common variables has the same name, and the distinction between geometric
operations and the electronics is complete and absolute.  One common block
holds information on all the variables and arrays, and control within the
program  is kept as simple as possible.

MOPAC is available as source code, so that researchers can both know what is
in the program and make changes to add new functionalities.

\subsection{Dynamic Memory Allocation}

Almost all the arrays in MOPAC are fully dynamic.  At the start of a run, the
number of atoms in the system is not known, therefore some static arrays are
needed.  However, once the calculation begins, all the data in the static
arrays is  transferred to dynamic arrays.

Only those arrays needed by MOPAC are created, and those that are created are
dimensioned to the smallest size possible.  As a result, there is little
wasted  memory, and many jobs that could otherwise not be run will now work.
