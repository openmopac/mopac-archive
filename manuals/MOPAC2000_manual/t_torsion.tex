\section{Torsion or Dihedral Angle Coherency}\label{coherency}
\index{Dihedral angles!coherency}\index{Chirality}\index{Enantiomers}
MOPAC  calculations  do  not   distinguish   between   enantiomers,
consequently  the  sign of the dihedrals can be multiplied by $-1$ and the
calculations will be unaffected.  However, if chirality is important,  a user
should be aware of the sign convention used.

The dihedral angle convention used in  MOPAC  is  that  defined  by Klyne and
Prelog~\cite{klyne}. \index{klyne@{\bf Klyne and Prelog}}  In this convention,
four atoms, AXYB, with a dihedral angle of 90 degrees, will have atom  B
rotated  by 90 degrees clockwise relative to A when X and Y are lined up in the
direction of sight, X being nearer to the eye.  In  their  words, ``To
distinguish between enantiomeric types the angle $\tau$ is considered as
positive when it is measured clockwise from the front  substituent  A to  
the   rear   substituent  B,  and  negative  when  it  is  measured
anticlockwise.'' The alternative  convention  was  used  in  programs which
preceded MOPAC.
