\section{COSMO (Conductor-like Screening Model)}\index{COSMO|ff}
\index{Klamt@{\bf Klamt, Andreas}}
\index{Solvation model!COSMO}
Based on materials provided by
\begin{center}
Dr Andreas Klamt \\
Burscheider Strasse 524 \\
D-51381 Leverkusen \\
Germany \\
\end{center}

Unlike the Self-Consistent Reaction Field model~\cite{scrf}, the  {\bf
Co}nductor-like {\bf S}creening {\bf Mo}del (COSMO) is a  continuum
approach~\cite{cosmo} which, while more  complicated, is computationally quite
efficient.  The expression for the total screening energy is simple enough to
allow the first derivatives of the energy with respect to atomic  coordinates
to be easily evaluated.

The COSMO procedure generates a conducting polygonal surface around the system
(ion or molecule), at the van der Waals' distance.  By introducing a
$\varepsilon$-dependent correction factor,
$$
f(\varepsilon)=\frac{(\varepsilon-1)}{(\varepsilon+\frac{1}{2})}, 
$$
into the expressions for the screening energy and its gradient, the theory can
be extended to finite dielectric constants with only a small error.

The accuracy of the method can be judged by how well it reproduces known
quantities, such as the heat of solution in water (water has a dielectric
constant of 78.4 at 25$^{\circ}$C), Table~\ref{cosmo_tab}.   Here, the keywords
used were:

\comp{NSPA=60 GRADIENTS 1SCF  EPS=78.4 AM1 CHARGE=1}

From the Table we see that the glycine zwitterion becomes the stable form in
water, while the neutral species is the stable gas-phase form.

(After the COSMO paper was published, improvements in the method made the
results shown in Table~\ref{cosmo_tab} invalid.  However, the general
conclusion that the method is of useful accuracy is still true.)

The COSMO method is easy to use, and the derivative calculation is of
sufficient precision to allow gradients of 0.1 to be readily achieved.

\begin{table}
\caption{\label{cosmo_tab} Calculated and Observed Hydration Energies}
\begin{center}
\begin{tabular}{llrrrr}
\hline
Compound   &   Method  &  \multicolumn{2}{c}{$\Delta H_f$ (kcal/mol)} & \multicolumn{2}{c}{Hydration} \\
           &           & gas phase & solution phase & $\Delta H$(calc.)  & Enthalpy(exp.)~\dag  \\
\hline
NH$_4^+$ & AM1 & 150.6 & 59.5 & 91.1 & 88.0 \\
N(Me)$_4^+$  & AM1 & 157.1 & 101.1 & 56.0 & 59.9 \\
N(Et)$_4^+$  & AM1 & 132.1 &  84.2 & 47.9 & 57.0 \\
Glycine \\
neutral & AM1 & -101.6 & -117.3 & 15.7 & $--$ \\
zwitterion & AM1 & -59.2 & -125.6 & 66.4 & $--$ \\
\hline
\end{tabular} \\
\dag : Y. Nagano, M. Sakiyama, T. Fujiwara, Y. Kondo, J. Phys. Chem., {\bf 92},
5823 (1988).
\end{center}
\end{table}

\subsection{A Walk Through COSMO}
To explain the COSMO program, it is best to do a walk through the program first.
%(FORTRAN symbols will be written in {\bf BOLD}, and algebraic quantities
%written in $math$ font.)

\subsection*{\comp{INITSV} (INITialize SolVation)}
This subroutine is called by the main program, if the \comp{EPS=$nn.n$} keyword
is set.  Here, all initializations for the COSMO calculation are done.

\begin{itemize}
\item In the \comp{DATA} statement, the van der Waals (VDW) radii are set.
\item The dielectric constant $\epsilon$ (\comp{EPSI}) is read in from 
\comp{KEYWRD}, and transformed to the dielectric factor \comp{FEPSI} = 
($\epsilon$-1)/($\epsilon$+0.5).
\item The number of interatomic density matrix elements \comp{NDEN} is 
calculated from \comp{NORBS} and \comp{NUMAT}:

\comp{NDEN=3$\times$NORBS-2$\times$NUMAT}

\item The solvent radius, \comp{RSOLV}, for the construction of the SAS is
read in off  \comp{RSOLV=$n.nn$} (default: 1.0 \AA ngstroms).
\item \comp{DISEX} is set (default = 2).  \comp{DISEX} controls the  distance
within which  the interaction of two segments is calculated accurately using
the basic grid.
\item The solvation radius \comp{SRAD} is set for each atom.  Be careful:  the
distance of the \comp{SAS} will be \comp{SRAD-RDS}.
\item \comp{NSPA} (Number of Segments Per Atom) is set (default = 42).
\comp{NSPA} is the number of segments created on a full VDW sphere.
\item To guarantee the most homogeneous initial segment distribution, it is
best to create one of the ``magic numbers'' (see \comp{DVFILL}) of segments.
Therefore the next smallest ``magic number'' compared to \comp{NSPA}  is
evaluated and the corresponding set of direction vectors is stored in
\comp{DIRSM}.

For hydrogens the number of segments can safely be reduced by a factor of three
or four, since the potential is rather homogeneous on hydrogen spheres. This is
done and the corresponding direction vectors are stored in \comp{DIRSM}.

\item By the equation:
$$
\comp{NSPA}\times \pi\times \comp{RSEG}^2 = 4\pi R^2
$$
the mean segment radius can easily be found to be 
$$
\comp{RSEG}=2R/\sqrt{\comp{NSPA}}.
$$

With a mean VDW-radius of 1.5 \AA ngstroms, the mean radius is 
$R=1.5+\comp{RSOLV}-\comp{RDS}$.
Thus \comp{DISEX2} is set to the squared mean distance of 
two segments multiplied by
\comp{DISEX}.
\end{itemize}

\subsection*{\comp{DVFILL} (Direction Vector FILLing)}
This routine constructs  a homogeneous set of points on a unit sphere
(direction vectors).  It starts with  12 face-centers on a regular
dodecahedron, i.e.\ the 12 corners of a regular icosahedron. These form 20
regular triangles.  If the centers of these 20 triangles are added to the 12
initial directions, a new set of triangles is created, which, by a Wigner-Seitz
construction on the unit sphere corresponds to 32 faces, 12 pentagons, and 20
hexagons.  Thus we have constructed a soccerball.  By iterative addition of the
triangle centers, any number $N$ of homogeneously distributed points can be
generated, which can be written in the form $N=3^i\times 10+2$,  with $i$ being
an integer ($N=12, 32, 92, 272, \ldots$).

But there is a second way of generating a finer mesh of regular triangles from
a cruder one.  Instead of additional centers, the midpoints of the edges can be
added. This roughly corresponds to an increase in the number of directions by a
factor  of four, instead of three.  By combining these two procedures, a number
$N=3^i\times 4^j\times 10+2$ of directions can be created.  These  are the
allowed values of \comp{NPPA} in the \comp{DVFILL} routine (``magic numbers'').
The smallest are 12, 32, 42, 92, 122, 162, \ldots . The default for the
construction of the fine grid is $1082 = 3^3\times 4\times 10+2$.

\subsection*{\comp{COSCAV/COSCAN} (COSmo CAVity)}
This routine constructs the SAS and calculates and inverts the surface charge
interaction matrix $A$. 

Then the new transformation matrix is built from the local coordinate system 
defined by the nearest neighbors.  The transformation to a local coordinate 
system guarantees a higher stability of the segmentation during geometry
optimization.  Then the basic grid of direction vectors is transformed 
according to the new transformation matrix.

\subsection*{\comp{MKBMAT}}
This routine calculates the $B$-matrix, i.e., the density-surface charge
interaction matrix.
\begin{itemize}
\item First the array \comp{COSURF} is filled with the explicit segment center
coordinates instead of directions only.

\item Then in a loop over all atoms, all segments for all density matrix
elements of each atom the Coulomb interaction of the density with the segment
charge is calculated.  With $r$ being the atom-segment distance vector, the
various elements are:
\begin{center}
\begin{tabular}{ll}\hline
$ss$ & $r^{-1}$ \\
$sp_k$ & $\frac{DD(i)r_k}{r^3}$ \\
$p_kp_k$ & $r^{-1}+QQ(i)^2(\frac{3r_k^2}{r^5}-r^{-3})$ \\
$p_kp_l$ & $6QQ(i)^2\frac{r_kr_l}{r^5}$ \\ 
\hline
\end{tabular}
\end{center}
  
Here, $DD(i)$ and $QQ(i)$ are the atomic dipole and quadrupole lengths, 
respectively, as coded in MOPAC.
\item The $B$-matrix is stored in \comp{BMAT}.
\end{itemize}


\subsection*{\comp{DIEGRD} (DIElectric GRaDient)}
The dielectric part of the gradient is calculated and added to \comp{DXYZ}.
There are two dielectric gradient contributions:
$$
d\ E_{die}/d\ R_k^{\alpha} = Q(\frac{d}{d\ R_k^{\alpha}}B)G -
 \frac{1}{2}G(\frac{d}{d\ R_k^{\alpha}}A)G
$$
with $G=A^{-1}BQ$ being the vector of charges on the segments.
\begin{itemize}
\item First the \comp{COSURF} transformation is done.
\item Then \comp{Q} and \comp{QS} are calculated. (If you are interested in 
a visualization of the screening charges, you may write out \comp{QS} and 
\comp{COSURF} at this point.)
\item Then the `second part' of the gradient is calculated.
\item In the next part the first part is calculated.  Here the gradient of the
density-segment-charge interactions have to be evaluated.

\item Finally, \comp{COSURF} is re-transformed.
\end{itemize}

\subsection{COSMO Keywords}
\index{Dielectric energy}
\begin{description}
\item[\comp{EPS}=$n.n$] Defines the dielectric constant of the solvent.   This
keyword triggers the whole COSMO.
\item[\comp{NSPA}=$nn$] Controls the number of segments, default = 42.
\item[\comp{DISEX}=$n.n$] Controls the radius, up to which the segment-segment
interactions are evaluated on the basis of the basic grid points.  Default =
2.0.  For accurate calculations or very high  dielectric
energies\footnote{\samepage The dielectric energy is the energy of
stabilization arising from the  interaction of the charges in the solute with
the induced charges on the  solvent accessible surface plus the electrostatic
energy due to the charges on the SAS interacting with each other.}, (e.g.\
ions) larger values may be preferable.  The calculation time may increase  as  
\comp{DISEX$^2$}  (until all interactions are calculated accurately).
\item[\comp{RSOLV}=$n.n$] Effective VDW radius of the solvent molecule. 
Default = 1.0\AA .
\end{description}
\subsection{Solvent Accessible Surface}\index{SAS|ff}
\index{Solvent Accessible Surface|ff}
The solvent accessible surface is a continuous surface of the molecule which
can be reached by the center of charge of a solvent molecule.  The calculation 
of the SAS is carried out as follows:

\begin{itemize} 
\item Each atom is assigned a van der Waals' radius.  VdW radii used in COSMO
are given in Table~\ref{vdw}.
\begin{table}
\caption{\label{vdw} Van der Waals radii (\AA ) used in COSMO}
\begin{center}
\begin{tabular}{llllllllllllll}
\hline
 I  & R    & II & R    &III& R    &IV & R    & V & R    &VI & R    &VII& R \\
\hline
 H  & 1.08 & \\
 Li & 1.80 &    &      &   &      & C & 1.53 & N & 1.48 & O & 1.36 & F & 1.30\\
 Na & 2.30 &    &      & Al& 2.05 & Si& 2.10 & P & 1.75 & S & 1.70 & Cl& 1.65\\
 K  & 2.80 & Ca & 2.75 &   &      &   &      &   &      &   &      &Br & 1.80\\
    &      &    &      &   &      &   &      &   &      &   &      &I  &  2.05\\
\hline
\end{tabular}
\end{center}
\end{table}

\item To each radius is added a distance equal to the radius of the solvent. By
default, this is 1.0\AA, but may be changed by the user using 
\comp{RSOLV=$n.nn$}. This gives the distance from the nucleus to the center of
a solvent molecule.
\item A set of points is generated on this surface. These points produce a
basic grid.
\item All points which are inside the surface of any other atom are excluded.
\item The remaining points are moved towards the center of the atom.  The
distance moved is equal to the distance of the center of charge of the solvent
molecule from the center of the solvent molecule.  By default, this distance is
set to \comp{RSOLV}, but may be set explicitly by keyword \comp{RSOLV=$n.nn$}.
\item Each of the remaining points represents a small area of the solvent
accessible surface.  The total SAS is calculated from the number of points.
\end{itemize}

From this definition of the SAS we see that the SAS of each atom is a  surface
of radius equal to the van der Waals' radius plus the radius of the solvent
molecule minus the distance of the center of charge of the solvent molecule to
the center of the solvent molecule.  In other words, the radius is the VdW
radius plus the distance from the surface of the solvent molecule to the center
of charge of the solvent molecule.  By default, this extra distance is zero.
Only that part of the atom surface which can be touched by the solvent molecule
is used.  This means that only those atoms on the surface of the molecule can
contribute to the SAS.  Of those atoms  that are on the surface of the molecule
there will be parts of the surface which cannot be reached by the solvent
because the solvent molecule is too bulky.


\subsubsection{Some hints on the use of COSMO}
\begin{itemize}
\item \comp{1SCF} calculations run in general without problems. On  gas-phase
geometries they give useful solvation energies for  neutral rigid molecules.
\item For geometry optimization Eigenvector Following has proved to be most
efficient in combination with COSMO. Gradient norms of about 1\% of the
dielectric energy should be reachable, in many cases even less. Nevertheless,
don't use a too small \comp{GNORM} criterion, since the calculation may have
convergence problems.
\item Keep in mind that energy differences of about 1\% of the dielectric
energy may arise due to small differences in the segmentatation.
\item Dr Klamt does not recommend the use of COSMO in \comp{FORCE} calculations
at the present time.
\item \comp{UHF} calculations should run without additional problems.
\item \comp{C.I.} calculations can now be done, and C.I.\ gradients are now 
valid; this has been the result of recent work by Dr Klamt.
\end{itemize}
