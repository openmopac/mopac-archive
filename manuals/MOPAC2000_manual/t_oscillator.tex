\subsection{Atom Transition Moments}\index{Transition!dipole moments}\index{Polarizability!atomic transition}
\index{Dipole!transition}\index{Exponents!transition dipole}
\index{Polarizability}\label{oscil}
\index{Oscillator}
A system can go from the ground state to an excited  state as the result of the
absorption of a photon.  The probability of this happening, $\kappa$, is
given\footnote{\samepage Wilson Decius and Cross,  ``Molecular Vibrations'', p
163, McGraw-Hill (1955)} in terms of the  oscillator integral:
\begin{eqnarray}
<\!\Psi_{0}|\stackrel{\rightharpoonup}{r}|\Psi_{*}\!> \label{os1},
\end{eqnarray}
by
$$
\kappa = \frac{8\pi^3}{3ch}\nu_{n'n''}(N_{n'}-N_{n''})
<\!\Psi_{0}|\stackrel{\rightharpoonup}{r}|\Psi_{*}\!>^2 .
$$
For electronic photoexcitations, $\Psi_A$ are state functions:
$$
\Psi_A = \sum_ic_i\Psi_i,
$$
and the $\Psi_i$ are microstates; see p.~\pageref{sd} for a
definition of microstates.
\subsubsection*{Some Mathematical tools}
In order to evaluate \ref{os1}, a property of integrals of the type:
$$
<\!\psi_{i}|\stackrel{\rightharpoonup}{r}|\psi_{j}\!>
$$
will be used several times.  This property is:
$$
<\!\psi_{i}|\stackrel{\rightharpoonup}{r}|\psi_{i}\!> = 0.
$$
From this, it follows that, if
$$
<\!\psi_{j}|\stackrel{\rightharpoonup}{r}|\psi_{i}\!> \neq 0,
$$
then
$$
<\!\psi_{i}|\stackrel{\rightharpoonup}{r}|\psi_{j}\!> =
-<\!\psi_{j}|\stackrel{\rightharpoonup}{r}|\psi_{i}\!> .
$$
To prove this relationship, consider the integral
$$
<\!(\psi_i+\psi_j)|\stackrel{\rightharpoonup}{r}|(\psi_i+\psi_j)\!>.
$$
Obviously, this integral has a value of zero, therefore
$$
<\!\psi_i|\stackrel{\rightharpoonup}{r}|\psi_i\!>  +
<\!\psi_j|\stackrel{\rightharpoonup}{r}|\psi_i\!>  +
<\!\psi_i|\stackrel{\rightharpoonup}{r}|\psi_j\!>  +
<\!\psi_j|\stackrel{\rightharpoonup}{r}|\psi_j\!>  =0.
$$
In this expression, the first and fourth terms are obviously zero, therefore
$$
<\!\psi_j|\stackrel{\rightharpoonup}{r}|\psi_i\!>  =
-<\!\psi_i|\stackrel{\rightharpoonup}{r}|\psi_j\!> .
$$

\subsubsection{Evaluation of Transition Dipole}

\begin{table}
\caption{\label{transx} ``$x$" Transition Integrals}
\begin{center}
\begin{tabular}{l|ccccccccc} \hline
& $s$  &  $p_x$  &  $p_y$  &  $p_z$  &  $d_{x^2-y^2}$  & $d_{xz}$  &
$d_{z^2}$  &  $d_{yz}$  &  $d_{xy}$ \\ \hline
$s$ & X$_A$\\
$p_x$ & sp & X$_A$\\
$p_y$  & 0 & 0 & X$_A$ \\
$p_z$  & 0 & 0 & 0 & X$_A$\\
$d_{x^2-y^2}$ & 0 & pd & 0 & 0 & X$_A$\\
$d_{xz}$      & 0 & 0 & 0 & pd & 0 & X$_A$\\
$d_{z^2}$     & 0 & -$\frac{1}{\sqrt{3}}$pd & 0 & 0 & 0 & 0 & X$_A$\\
$d_{yz}$      & 0 & 0 & 0 & 0 & 0 & 0 & 0 & X$_A$\\
$d_{xy}$      & 0 & 0 & pd & 0 & 0 & 0 & 0 & 0 & X$_A$\\  \hline


\end{tabular}\\
\hspace{-0.3in}Note: X$_A$ = $<\! \phi_{\lambda}|\stackrel{\rightharpoonup}{x}|\phi_{\lambda}\! >$;
sp = $<\! ns|\stackrel{\rightharpoonup}{r}|np\! >$; pd = $<\! np|\stackrel{\rightharpoonup}{r}|nd\! >$ (see below).
\end{center}
\end{table}

The probability, $B_{0\rightarrow *}$, that a photon will be absorbed by a system that has a
ground state $\Psi_0$ and an excited state $\Psi_*$ separated by an energy $\epsilon$ when
irradiated by an energy density $\rho_{\epsilon}$
is given by
$$
B_{0\rightarrow *} = \frac{2\pi}{3\hbar^2}|R_{0*}|^2\rho_{\epsilon},
$$
in which
$$
|R_{0*}|^2 = |X_{0*}|^2 + |Y_{0*}|^2 + |Z_{0*}|^2.
$$

$X_{0*}$ is the matrix element for the $x$ component of the dipole moment:
$$
X_{0*} = \int \Psi_0|e\sum_j \stackrel{\rightharpoonup}{x_j} |\Psi_* d\tau.
$$

Evaluation of this integral requires evaluating the effect of the operators  $\stackrel{\rightharpoonup}{x_j}$,
$\stackrel{\rightharpoonup}{y_j}$, and $\stackrel{\rightharpoonup}{z_j}$
acting on an atomic orbital.  Tables \ref{transx}, \ref{transy}, and \ref{transz} show
the integrals of the type $<\! \phi_{\lambda}
 |\stackrel{\rightharpoonup}{r_j}|\phi_{\sigma}\!>$, where $\phi_{\lambda}$ and $\phi_{\sigma}$ are
 pairs of atomic orbitals.


The integral
 $<\! \phi_{\lambda}|\stackrel{\rightharpoonup}{r}|\phi_{\lambda}\! >$ is simply the appropriate Cartesian coordinate,
that is, the $x$, $y$, or $z$ coordinate of the atom that $\phi_{\lambda}$ is on.

\begin{table}
\caption{\label{transy} ``$y$" Transition Integrals}
\begin{center}
\begin{tabular}{l|ccccccccc} \hline
& $s$  &  $p_x$  &  $p_y$  &  $p_z$  &  $d_{x^2-y^2}$  & $d_{xz}$  &
$d_{z^2}$  &  $d_{yz}$  &  $d_{xy}$ \\ \hline
$s$ & Y$_A$\\
$p_x$ & 0 & Y$_A$\\
$p_y$  & sp & 0 & Y$_A$ \\
$p_z$  & 0 & 0 & 0 & Y$_A$\\
$d_{x^2-y^2}$ & 0 & 0 & -pd & 0 & Y$_A$\\
$d_{xz}$      & 0 & 0 & 0 & 0 & 0 & Y$_A$\\
$d_{z^2}$     & 0 & 0 & -$\frac{1}{\sqrt{3}}$pd & 0 & 0 & 0 & Y$_A$\\
$d_{yz}$      & 0 & 0 & 0 & pd & 0 & 0 & 0 & Y$_A$\\
$d_{xy}$      & 0 & pd & 0 & 0 & 0 & 0 & 0 & 0 & Y$_A$\\  \hline


\end{tabular}\\
\end{center}
\end{table}


 $<\! ns|\stackrel{\rightharpoonup}{r}|np\! >$
 and  $<\! np|\stackrel{\rightharpoonup}{r}|nd\! >$
 can be evaluated using the following expressions:

$$
<\! ns|\stackrel{\rightharpoonup}{r}|np\! > = a_0\frac{(2n+1).2^{2n+1}.(\xi_s\xi_p)^{n+1/2}}{\sqrt{3}(\xi_s+\xi_p)^{2n+2}}
$$

$$
<np|\stackrel{\rightharpoonup}{r}|nd>=a_0\frac{(n_p+n_d+1)!.2^{n_p+n_d+1}.\xi_p^{n_p+1/2}.\xi_d^{n_d+1/2}}
{\sqrt{5}(\xi_p+\xi_d)^{n_p+n_d+2}.\sqrt{(2n_p)!(2n_d)!}},
$$
where $ns$, $np$, and $nd$ are $s$, $p$, and $d$ quantum numbers, respectively.
For the $sp$ transition, $n=ns=np$.
The Slater orbital exponents, $\xi_s$, $\xi_p$, and $\xi_d$, are usually
given in atomic units, that is, in
inverse Bohr, therefore they must converted to \AA ngstroms before use, hence the
presence of the $a_0=0.529$ in these expressions.

All integrals of the type used here are in \AA ngstroms, therefore the units of the integral
of the dipole operator on a M.O. is also in \AA ngstroms:
$$
<\! \psi_i|\stackrel{\rightharpoonup}{r}|\psi_j\! > =\sum_{\lambda}\sum_{\sigma}c_{\lambda i}c_{\sigma j}
<\! \phi_{\lambda}\stackrel{\rightharpoonup}{r}\phi_{\sigma} \!>,
$$
 Once the value of the integral is known,
the phase to be used must be determined. The simplest way to achieve this
is to reverse the sign of the oscillator whenever the second M.O.\ has a
higher index than the first.

\begin{table}
\caption{\label{transz} ``$z$" Transition Integrals}
\begin{center}

\begin{tabular}{l|ccccccccc} \hline
& $s$  &  $p_x$  &  $p_y$  &  $p_z$  &  $d_{x^2-y^2}$  & $d_{xz}$  &
$d_{z^2}$  &  $d_{yz}$  &  $d_{xy}$ \\ \hline
$s$ & Z$_A$\\
$p_x$ & 0 & Z$_A$\\
$p_y$  & 0 & 0 & Z$_A$ \\
$p_z$  & sp & 0 & 0 & Z$_A$\\
$d_{x^2-y^2}$ & 0 & 0 & 0 & 0 & Z$_A$\\
$d_{xz}$      & 0 & pd & 0 & 0 & 0 & Z$_A$\\
$d_{z^2}$     & 0 & 0 & 0 & $\frac{2}{\sqrt{3}}$pd & 0 & 0 & Z$_A$\\
$d_{yz}$      & 0 & 0 & pd & 0 & 0 & 0 & 0 & Z$_A$\\
$d_{xy}$      & 0 & 0 & 0 & 0 & 0 & 0 & 0 & 0 & Z$_A$\\  \hline
\end{tabular}\\
\end{center}
\end{table}


  Evaluation of the integrals over microstates is straightforward,
in that all integrals are zero, unless the number of differences between the microstates is exactly two,
in which case the integral is equal to that of the two M.O.s involved, times a phase factor.  That is,
for each pair of microstates that are identical, except for $\psi_i$
in $\Psi_a$ and  $\psi_j$ in $\Psi_b$, the integral is:.

$$
<\! \Psi_a|\stackrel{\rightharpoonup}{r}|\Psi_b\! > =<\! \psi_i|\stackrel{\rightharpoonup}{r}|\psi_j\! >*(-1)^n,
$$
where $n$ is the number of permutations necessary to move $\psi_i$ in microstate $\Psi_a$ to the position
occupied by $\psi_j$ in microstate $\Psi_b$. This is
similar to the `b' option on page~\pageref{b}.  As with
the molecular orbitals, the oscillators for microstates change sign
when the order of the microstates is reversed.  The simplest way to
achieve this is to use the same device that was used with the M.O.s;
that is, to reverse the sign of the oscillator whenever the second
microstate has a higher index than the first.


For completeness, the sign of $R_{0*}$ should be reversed if $k>l$, but
since only the modulus is used, this operation does not need to be done.


Finally, the state transition dipole can be calculated from:
$$
<\! \Psi_A|\stackrel{\rightharpoonup}{r}|\Psi_B\! > =\sum_a\sum_bc_{A a}c_{B b}<\! \Psi_a\stackrel{\rightharpoonup}{r}\Psi_b \!>,
$$

Although the transition dipole is normally regarded as involving the ground and an excited state, it is
possible to calculate the transition between two excited states.  The initial state is, by default, the
ground state, however if \comp{ROOT=n} $n\neq 1$, or any other keyword that specifies
a state other than the ground state, then the initial state will be an excited state.

For degenerate states, the transition dipole is the sum over all states involved.
