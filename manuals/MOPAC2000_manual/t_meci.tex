\section{Multi-Electron Configuration Interaction}\label{meci}
\index{MECI|(}\index{Configuration interaction|(}
\index{C.I.!description} For some systems a single determinant is insufficient
to describe the electronic wave function. For example, square
\index{Cyclobutadiene!need for C.I.}\index{Ethylene, twisted!need for C.I.}
cyclobutadiene and twisted ethylene require at least two configurations to
describe their ground states. More than one configuration is also needed if an
excited state is required---the RHF SCF converges on a ground state or, if
\index{Half-electron} half-electron methods are used, on a mixture of states,
while the excited state involves a different configuration. \index{Radicals|ff}
Radicals also present a difficulty at the RHF level in that the SCF
wavefunction corresponds to an equal mixture of the two doublets, with a
corresponding error in the total energy. In order to correct for this error,
MOPAC contains   a    very   large   Multi-Electron   Configuration
Interaction  calculation,  MECI~\cite{meci} (pronounced ``me-sigh'')   which,
in addition to automatically correcting ``half-electron'' energies, allows
almost any configuration interaction calculation to be performed.  Because of
its complexity, two distinct  levels  of  input are supported; the default
values will be of use to the novice while an expert has available  an
exhaustive  set  of keywords from which a specific C.I.\ can be tailored.

MECI is a completely general C.I. The resulting states \index{Space
quantization}\index{Spin!quantization} are space and spin-quantized, there is
no restriction on total spin, the starting wavefunction can be closed or open
shell, and both even and odd electron systems are allowed, although for
simplicity in describing the method,  the starting configuration is assumed to
be closed-shell.

\subsection{Starting electronic configuration}
As MECI requires the space parts of the $\alpha$ and $\beta$ molecular orbitals
to be identical, only RHF wavefunctions are used. However, this is not a severe
restriction in  that  any  starting configuration will be supported.  Examples
of starting configurations are shown in Table~\ref{conf_meci}.

\begin{table}
\caption{\label{conf_meci}Examples of SCF configurations used in MECI}
\index{SCF! configurations}
\begin{center}
\begin{tabular}{lcc} \hline
    System    &          KeyWords used  &    Starting Configuration\\ \hline
   Methane            &     none      &        2.00 2.00 2.00 2.00  \\
   Methyl Radical     &     none      &       2.00 2.00 2.00 1.00  \\
   Twisted Ethylene   &   \comp{OPEN(2,2)}   &       2.00 2.00 1.00 1.00  \\
Twisted Ethylene Cation& \comp{OPEN(1,2)}    &     2.00 2.00 0.50 0.50  \\
   Methane Cation     &    \comp{CHARGE=1 OPEN(5,3)}&2.00 1.67 1.67 1.67  \\ \hline
\end{tabular}
\end{center}
\end{table}

\index{TRIPLET}\index{OPEN} Choice of starting configuration is  important.
For  example,  if twisted  ethylene,  a ground-state triplet, is not defined
using  \comp{OPEN(2,2)}, then  the  closed-shell  ground-state  structure
will  be calculated.   Obviously,  this configuration is a legitimate
microstate, but from the symmetry of the system a better choice would be  to
define \index{Degenerate M.O.s|ff} one electron in each of the two formally
degenerate $\pi$-type M.O.s.

Each configuration which can be generated in a molecule may be represented by
a single Slater determinant; this is called a microstate. The final states will
be linear combinations of these microstates. In general, microstates will not
be eigenfunctions of the total spin operator, but will be mixtures of different
spin states.

The initial configuration used to generate the SCF is arbitrary; for
half-electron systems it will not even correspond to a microstate, each M.O.\
having a fractional \index{Fractional!M.O.\ occupancy} electron occupancy. Even
if the starting wavefunction is a closed shell it would still correspond to
only one of a large number of microstates to be used in the MECI.\ As a result,
before the MECI is started all electronic terms arising from the electrons in
the initial configuration, which will be used by MECI, are removed. The
starting wavefunction will thus consist of a low-lying doubly occupied set of
M.O.s and a high-lying empty set of M.O.s, neither of which will be involved in
the MECI, and in between a small set of M.O.s from which the electrons have
been removed. This set of M.O.s will be involved in the MECI.

\subsection{Microstates} \index{Slater determinant|( }\index{Microstates|ff}\label{sd}
Microstates, which are normally represented by a Slater determinant, are
normally written as an antisymmetrized product of $p\  \alpha$- and $q\
\beta$-electrons:
$$
\hspace*{-0.5in} \Psi_g=[(p+q)!]^{-\frac{1}{2}}\sum_P(-1)^PP[\psi_1(1)\alpha(1)\ \psi_2(2)\alpha(2)\
\ldots \psi_p(p)\alpha(p)\ \psi_1(p+1)\beta(p+1)\ \ldots \psi_q(p+q)\beta(p+q)],
$$
where $[(p+q)!]^{-\frac{1}{2}}$ is the normalization constant, $P$ is an
operator which permutes the electron coordinates, and $(-1)^P$ assumes the
values --1 or +1 for odd and even permutations respectively.  A more compact
and useful notation for representing a  general microstate is:
$$
\Psi_j = \frac{1}{\sqrt{N!}}\sum_{P=1}^{N!}(-1)^PP(\prod_{k=1}^N\psi_k^j)
$$
where $\Psi_j$ is any microstate consisting of $N$ electrons.  Given the full
set of M.O.s, a subset of these is used in the microstate.  This subset is
defined by the M.O.s $\psi_k^j$, $k$=1,$N$.  Each microstate will consist of a
different set of M.O.s from the full set.

Rather than having all the $\alpha$ electrons appearing first in a microstate,
it is more convenient to order the one electron wavefunctions in the order in
which their indices occur in the full set of M.O.s.  If both $\alpha$ and
$\beta$ M.O.s of the same index occur, then $\alpha$ precedes $\beta$, thus:
$$
\hspace*{-0.2in}\Psi_g=[(p+q)!]^{-\frac{1}{2}}\sum_P(-1)^PP[\psi_1(1)\alpha(1)\ \psi_1(2)\beta(2)\
\psi_2(3)\alpha(3)\ \psi_2(4)\beta(4)\ \ldots  \psi_j(p+q)\beta(p+q)]
$$

This numbering scheme follows  the  Aufbau  principle, in that the  order  of
\index{Aufbau principle} filling is in order of energy.  This point is
critically  important  in deciding  the  sign of matrix elements.  For a 5
M.O.\  system, then, the order of filling is:
$$      (1)(\bar{1})(2)(\bar{2})(3)(\bar{3})(4)(\bar{4})(5)(\bar{5}) $$
A triplet state arising from two microstates, each with a component
of spin = 0, will thus be the positive combination:
$$        (\bar{1})(2)   +    (1)(\bar{2}). $$
 This standard sign convention was chosen in order to allow the signs of the
microstate coefficients  to  conform  to those resulting from the spin
step-down operator.

Only those M.O.s involved in the MECI are of interest,
thus from the full set of M.O.s, filled and empty
$$
\left[
\begin{array}{lll}
\psi_1(1)\alpha(1)&\psi_2(3)\alpha(3)&\ldots\\
\psi_1(2)\beta(2)&\psi_2(4)\beta(4)&\ldots\\
\end{array}
\right]
$$
the ground-state configuration (assumed to be closed shell for simplicity) can
be represented by
$$
\left[
\begin{array}{llllllllllllll}
1&1&1&1&\ldots&1&1&1&0&0&0&\ldots&0&0\\
1&1&1&1&\ldots&1&1&1&0&0&0&\ldots&0&0\\
\end{array}
\right]
$$
where a 1 represents a spin molecular orbital occupied by one electron and 0
represents an empty M.O.

\index{Active space!in C.I.|ff} The M.O.s involved in the C.I.\ are called the
``active space''. For convenience, the index of the M.O.\ at the lower bound of
the active space will be called ``B'', and the index of the M.O.\ at the upper
bound of the active space will be called ``A''. All M.O.s below the active
space can be considered as part of the core while those above it are empty and
can likewise be ignored. We can thus focus our attention on the M.O.s in the
active space. Most of the time, MECI calculations will  involve  between  1
and  5 M.O.s,  so  a system such as pyridine, with 15 filled levels and 29
M.O.s, would \index{Pyridine} use M.O.s 13--17 in a large C.I.

For convenience, microstates will be expressed as a sum of molecular orbital
occupancies, so that:
$$
\Psi_p=\sum_{i=B}^A(O_i^{\alpha p} + O_i^{\beta  p} ).
$$
For example, if the ground state configuration $\Psi_g$ is closed shell, then
the occupancy of the M.O.s would be
$$
O^{\alpha g} = O^{\beta  g} = |1,\ldots,1,0,\ldots,0|
$$

Microstates are particular electron configurations.  Examples of
microstates involving  5  electrons  in  5  levels are given in Table~\ref{m55}.

\begin{table}
\begin{center}
\caption{\label{m55}  Microstates for 5 electrons in 5 M.O.s}
\begin{tabular}{cccrcccc}\\ \hline
& \multicolumn{2}{c}{Electron Configuration}& & &  \multicolumn{2}{c}{Electron Configuration}& \\
\cline{2-3} \cline{6-7}
  &     Alpha   &   Beta    & M$_S$  &    &   Alpha    &  Beta  & M$_S$    \\
M.O.  &   1 2 3 4 5 &1 2 3 4 5  &       & M.O. &  1 2 3 4 5 &1 2 3 4 5\\ \hline

1 &   1,1,1,0,0 &1,1,0,0,0  & 1/2   &  4 &  1,1,1,1,1 &0,0,0,0,0 &   5/2\\
2 &   1,1,0,0,0 &1,1,1,0,0  &-1/2   &  5 &  1,1,0,1,0 &1,1,0,0,0 &   1/2\\
3 &   1,1,1,0,0 &0,0,0,1,1  & 1/2   &  6 &  1,1,0,1,0 &1,0,1,0,0 &   1/2 \\ \hline
\end{tabular}
\end{center}
\end{table}
\index{Slater determinant|)}

\subsubsection{Permutations}
For  5  electrons  in  5   M.O.s   there   are   252   microstates
($10!/(5!5!)$),  but as states of different spin do not mix, we can use a
smaller  number.   If  doublet  states  are  needed, then   100   states
($5!/(2!3!)(5!/3!2!$)  are  needed.   If  only  quartet  states  are of
\index{Quartet states} interest, then 25 states ($5!/(1!4!)(5!/4!1!$) are
needed  and  if  the \index{Sextet states} sextet state is required, then only
one state is calculated.

In  the  microstates  listed,   state   1   is   the   ground-state
configuration.   This can be written as (2,2,1,0,0), meaning that M.O.s 1 and 2
are doubly occupied, M.O.\  3 is  singly  occupied  by  an  alpha electron, and
M.O.s 4 and 5 are empty.  Microstate 1 has a component of \index{Kramer's
degeneracy} spin of 1/2, and is a pure doublet.  By Kramer's
degeneracy---sometimes called time-inversion symmetry---microstate 2 is also a
doublet, and has a spin of 1/2 and a component of spin of $-1/2$.

Microstate 3, while it has a component of spin of  1/2,  is  not  a doublet,
but  is  in  fact  a  component  of a doublet, a quartet and a sextet.  The
coefficients of these states can  be  calculated  from  Wigner's symbol, also
called the \index{Clebsch-Gordon 3-J symbol}\index{Wigner's symbol}\index{3-J
symbol} Clebsch-Gordon  3-J  symbol\footnote{\samepage The symbol is of form
\begin{eqnarray}
\hspace*{-0.5in}<j_1j_2m_1m_2|j_1j_2jm>&=&
\left \{\frac{(j+m)!(j-m)!(j_1-m_1)!(j_2-m_2)!(j_1+j_2-j)!(2j+1)}
{(j_1+m_1)!(j_2+m_2)!(j_1-j_2+j)!(j_2-j_1+j)!(j_1+j_2+j+1)!}\right \}^{\frac{1}{2}}
\nonumber \\
&&\delta(m,m_1+m_2)\sum_r(-1)^{j_1+r-m_1}\frac{(j_1+m_1+r)!(j_2+j-r-m_1)!}
{r!(j-m-r)!(j_1-m_1-r)!(j_2-j+m_1+r)!} \nonumber
\end{eqnarray}
where the summation is over all values of r such that all factorials occurring
are of non-negative integers (0!=1). See~\cite{griffith}. To use the symbol,
the coefficient of momentum $(j,m)$ due to two momenta $(j_1,m_1)$ and
$(j_2,m_2)$ is
$<j_1j_2m_1m_2|j_1j_2jm>$
}.
Thus, the coefficient in the doublet is $\sqrt{1/2}$ ($j_1=3/2,\; m_1=3/2,
\;j_2=1, \;m_2=-1, \;j=1/2$), in the quartet is $\sqrt{4/10}$ ($j_1=3/2,\;
m_1=3/2, \;j_2=1, \;m_2=-1, \;j=3/2$), and in the sextet,  $\sqrt{1/10}$
($j_1=3/2,\; m_1=3/2, \;j_2=1, \;m_2=-1, \;j=5/2$).

Microstate 4 is a pure sextet.  If all 100 microstates of component of  spin
=  1/2  were used in a C.I., one of the resulting states would have the same
energy as the state resulting from microstate 4.

Microstate 5 is an excited doublet, and microstate 6 is an  excited state of
the system, but not a pure spin-state.

By default, if $n$ M.O.s are included in the MECI, then all possible
microstates which give rise to a component of spin = 0 for even electron
systems, or 1/2 for odd electron systems, will be used.

\index{Microstates!number used in C.I.}\label{nmeci}
\begin{table}
\caption{\label{setmic}Sets of Microstates for Various MECI Calculations}
\begin{center}
\begin{tabular}{|rccr|ccr|}
\hline
&\multicolumn{2}{c}{Odd Electron Systems} &  & \multicolumn{2}{c}{Even Electron Systems}&\\ \hline
        &    Alpha   Beta &&No. of   &    Alpha   Beta &&No. of\\
        &                 &&Configs. &                 &&Configs.\\\hline
 C.I.=1&(1,1) $\times$ (0,1) &=&  1    &     (1,1) $\times$ (1,1)&=&   1  \\
      2&(1,2) $\times$ (0,2) &=&  2    &     (1,2) $\times$ (1,2)&=&   4  \\
      3&(2,3) $\times$ (1,3) &=&  9    &     (2,3) $\times$ (2,3)&=&   9  \\
      4&(2,4) $\times$ (1,4) &=& 24    &     (2,4) $\times$ (2,4)&=&  36  \\
      5&(3,5) $\times$ (2,5) &=&100    &     (3,5) $\times$ (3,5)&=& 100  \\\hline
\end{tabular}\\
($n$,$m$) means $n$ electrons in $m$ M.O.s.
\end{center}
\end{table}

\index{MECI!Increasing number of states} MOPAC is configured to  allow a
maximum of \comp{MAXCI} states, where \comp{MAXCI} is defined in the file
\comp{meci.h}.  If more states are needed (see Table~\ref{setmic}), then
\comp{MAXCI} in  \comp{meci.h} should be modified. Of course, if \comp{MAXCI}
is changed, MOPAC should be recompiled.

If \comp{CIS}, \comp{CISD}, or \comp{CISDT} are specified, then the number of
microstates is defined by \comp{C.I.=$k$} and the keyword.   The number of
microstates is a function of $k$.  Let $n$ and $m$ be integers, such that:
$$
n=\frac{k}{2}
$$
$$
m=\frac{k+1}{2}
$$
If $k$ is odd, then round down to the next lower integer.  Then the number
of microstates $n_{CIS}$, $n_{CISD}$, and $n_{CISDT}$, for even-electron
systems is:
$$
\begin{array}{lcll}
n_{CIS}&=& &2nm \nonumber \\
n_{CISD}&=& 1 + &2nm + (nm)^2 + \frac{n!m!}{2(n-2)!(m-2)!}\nonumber  \\
n_{CISDT} &=& 1 + &2nm + (nm)^2 + \frac{n!m!}{2(n-2)!(m-2)!} +
\frac{n!m!}{18(n-3)(m-3)}+\frac{nm\times n!m!}{2(n-2)!(m-2)!}\nonumber
\end{array}
$$

Note that when \comp{CIS} is used, the ground state is {\em not} included in
the list of microstates. Values for the more important $k$ are given in
Table~\ref{micsdt}.
\begin{table}
\caption{\label{micsdt}Number of Microstates for CIS, CISD, and CISDT}
\begin{center}
\begin{tabular}{|crcrcrc|}
\hline
C.I.=$k$ & CIS &  & CISD &  & CISDT & \\ \hline
   1     &  0  &    &  1   &    &   1   &    \\
   2     &  2  &    &  4   &    &   4   &    \\
   3     &  4  &    &  9   &    &   9   &    \\
   4     &  8  &    &  27  &    &  35   &    \\
   5     & 12  &    &  55  &    &  91   &    \\
   6     & 18  &    &  118 &    & 282   &    \\
   7     & 24  &    &  205 &    & 635   &    \\
   8     & 32  &    &  361 &    & 1545  &    \\
\hline
\end{tabular}\\
(for even electron systems only)
\end{center}
\end{table}
\subsubsection{Energy of microstates}
The electronic energy, $E_r$, of any microstate $\Psi_r$ is the sum
\index{Electronic energy!of microstates}
on the one and two-electron energies:
$$
E_r = \sum_i^pH_{ii}+\sum_i^qH_{ii}+\frac{1}{2}\sum_{ij}^p(J_{ij}-K_{ij})
+\frac{1}{2}\sum_{ij}^q(J_{ij}-K_{ij}) +\sum_i^p\sum_j^qJ_{ij}
$$
where $H_{ii}$ is the one-electron energy of M.O.\ $\psi_i$,
$J_{ii}$ is the two-electron Coulomb integral\\
$<\psi_i\psi_i|\psi_j\psi_j>$,
and $K_{ii}$ is the two-electron exchange integral  $<\psi_i\psi_j|\psi_i\psi_j>$.

In this section it is more convenient to express it in terms
of molecular orbital occupancies:
$$
E_r = \sum_{i=B}^AH_{ii}(O_i^{\alpha r}+ O_i^{\beta r})
+\sum_{ij=B}^A(\frac{1}{2}(J_{ij}-K_{ij})
(O_i^{\alpha r}O_j^{\alpha r}+ O_i^{\beta r}O_j^{\beta r})
+J_{ij}O_i^{\alpha r}O_j^{\beta r})
$$
Similarly, the orbital energies can be written
\index{Orbital!energies in C.I.}
$$
\epsilon_{ii}^{\alpha r} = H_{ii}+\sum_j^p(J_{ij}-K_{ij})+\sum_j^qJ_{ij}
$$
or, in terms of orbital occupancies
$$
\epsilon_{ii}^{\alpha r} = H_{ii}+\sum_{j=B}^A(J_{ij}-K_{ij})O_j^{\alpha r}
+\sum_{j=B}^AJ_{ij}O_j^{\beta r}.
$$
\subsubsection{Zero of energy used in MECI}
The energy of the system after all the electronic terms arising from the
electrons of the M.O.s involved in the starting configuration are removed
is a useful quantity. Removal of these terms lowers the orbital
energies thus:
$$
\epsilon_{ii}^+ = \epsilon_{ii} -\sum_{j=B}^A(J_{ij}-K_{ij})O_j^g.
$$
The arbitrary zero of energy in a MECI calculation is the  starting ground
state, without any correction for errors introduced by the use of fractional
occupancies.  In order to calculate the energy of the various configurations,
the  energy  of  the  vacuum  state  (i.e.,  the  state resulting from removal
of the electrons used in the C.I.)  needs  to  be evaluated.  This energy is
given by:
$$
GSE=E_g^+ = - \sum_{i=B}^A2\epsilon_{ii}^+O_i^g+J_{ii}(O_i^g)^2+
\sum_{i=B}^A\sum_{j=B}^{i-1}2(2J_{ij}-K_{ij})O_i^gO_j^g
$$
\index{GSE@{$GSE$, used in MECI}}
(Within the MECI routine, $GSE$ refers to $E_g^+$.)

By redefining the system so that those filled M.O.s which are not used in the
MECI are considered part of an \index{Unpolarizable core!used in C.I.}
unpolarizable core, the new energy levels $\epsilon_i^+$ can be identified with
the one-electron energies $H_{ii}$ and the total electronic energy $E_r$ of any
microstate is set equal to the sum of the energy of the electrons considered in
the microstate plus $E_g^+$.

\subsection{Construction of secular determinant}
\index{Secular determinant!in MECI}
Microstates can be generated by permuting available electrons among the
available levels. Elements of the C.I.\ matrix are then defined by
$$
<\Psi_a|H|\Psi_b>
$$
Evaluation of these matrix elements is difficult. Each microstate is a Slater
determinant, and the Hamiltonian operator involves all electrons in the system.
Fortunately, most matrix elements are zero because of the orthogonality of the
M.O.s. Only the non-zero elements need be evaluated; three types of interaction
are possible:
\begin{enumerate}
\item $\Psi_a=\Psi_b$. Since the two wavefunctions are the same,
this corresponds to the energy of a microstate. As the
electronic energy of the closed shell is common to all
configurations considered in the C.I., it is sufficient to
add on to $E_g^+$ the energy terms which are specific to the
microstate, thus
\begin{eqnarray}
<\Psi_a|H|\Psi_b>& =& E_g^+ +\sum_{i=B}^A\left(\epsilon_{ii}+\sum_{j=B}^A(J_{ij}-K_{ij})
O_j^{\alpha p} \right )O_i^{\alpha p} \nonumber  \\&&
+\sum_{i=B}^A\left(\epsilon_{ii}+\sum_{j=B}^A(J_{ij}-K_{ij}) O_j^{\beta p}
\right )O_i^{\beta p} + \sum_{i=B}^A\sum_{j=B}^AJ_{ij}O_i^{\alpha p}
O_j^{\beta p}. \nonumber
\end{eqnarray}
\item \label{b} Except for $\psi_i$ in $\Psi_a$ and  $\psi_j$ in $\Psi_b$;
$\Psi_a = \Psi_b$. Assuming $\psi_i$  and $\psi_j$  to be $\alpha$-spin the
interaction energy is
$$
<\Psi_a|H|\Psi_b> = (-1)^W(\epsilon_{ij}^++\sum_{k=B}^A(<ij|kk>-<ik|jk>)O_k^{\alpha a}
+<ij|kk>O_k^{\beta a}).
$$
This presents a problem. Unlike $\epsilon_{ii}^+$, which has already been
defined, there is no easy way to calculate $\epsilon_{ij}^+$. Rather than
undertake this calculation, use can be made of the fact that, for the starting
configuration:
$$
\epsilon_{ij} = <\psi_i|H|\psi_j> = H_{ij} +
\sum_{k=B}^A(<ij|kk>-<ik|jk>)O_k^{\alpha g}
+<ij|kk>O_k^{\beta g}
$$
or
$$
\epsilon_{ij} = \epsilon_{ij}^+ + \sum_{k=B}^A(<ij|kk>-<ik|jk>)O_k^{\alpha g}
+<ij|kk>O_k^{\beta g} .
$$
$\epsilon_{ij}$ corresponds to an off-diagonal term in the Fock matrix,
which at self-consistency is, by definition, zero.
Therefore:
$$
\epsilon_{ij}^+ = - \sum_{k=B}^A(<ij|kk>-<ik|jk>)O_k^{\alpha g}
+<ij|kk>O_k^{\beta g}  ,
$$
which can be substituted directly into the expression for
$<\Psi_a|H|\Psi_b>$ to give
$$
<\Psi_a|H|\Psi_b> = (-1)^W\sum_{k=B}^A(<ij|kk>-<ik|jk>)(O_k^{\alpha a}-O_k^{\alpha g})
+<ij|kk>(O_k^{\beta a} -O_k^{\beta g} ).
$$
All that remains is to determine the phase factor. One
of the microstates is permuted until the two unmatched M.O.s
occupy the same position. The number of permutations needed
to do this when the two M.O.s are of $\alpha$ spin is
$$
W=\sum_{k=i+1}^{j-1}(O_k^{\alpha p}-O_k^{\beta p}),
$$
assuming $j > i$; otherwise:
$$
W=O_j^{\alpha p} +\sum_{k=i+1}^{j-1}(O_k^{\alpha p}-O_k^{\beta p}).
$$

\item Except for $\psi_i$ and $\psi_j$ in $\Psi_a$ and $\psi_k$ and
$\psi_l$ in $\Psi_b$; $\Psi_a = \Psi_b$.
 Two situations exist: (a) when all four M.O.s are of
the same spin; and (b) when two are of each spin. Thus,
\begin{enumerate}
\item All four M.O.s are of the same spin. The
interaction energy is
$$
<\Psi_a|H|\Psi_b> = (-1)^W[<ik|jl>-<il|jk>],
$$
in which the phase factor is:
$$
W=\sum_{m=i+1}^{j-1}(O_m^{\alpha a}-O_m^{\beta a})
+ \sum_{m=k+1}^{l-1}(O_m^{\alpha a}-O_m^{\beta a})+O_i^{\beta a} + O_k^{\beta a},
$$
if the four M.O.s are of $\alpha$ spin; otherwise,
$$
W=\sum_{m=i+1}^{j-1}(O_m^{\alpha a}-O_m^{\beta a})
+ \sum_{m=k+1}^{l-1}(O_m^{\alpha a}-O_m^{\beta a})+O_j^{\beta a} + O_l^{\beta a}.
$$
\item Two M.O.s are of each spin. In this case there is
no exchange integral, therefore the interaction energy is
$$
<\Psi_a|H|\Psi_b> = (-1)^W<ik|jl>
$$
and the phase factor is:
$$
W=\sum_{m=k}^{i}(O_m^{\alpha a}-O_m^{\beta a})
+ \sum_{m=j}^{l}(O_m^{\alpha a}-O_m^{\beta a}).
$$
If $i>k$, then $W=W+O_k^{\alpha a}+O_i^{\beta a}$,
if $j>l$, then $W=W+O_k^{\alpha b}+O_i^{\beta b}$,
finally, if $i>k$ and $j>l$ or  $i<k$ and $j<l$, then $W=W+1$.

All other matrix elements are zero. The completed secular determinant is then
diagonalized. This yields the \index{State!vectors}\index{State!energies} state
vectors and state energies, relative to the starting configuration. In turn,
the state vectors can be used to generate  spin density (at the RHF level) for
pure spin states. If the density matrix for the state is of interest, such as
in the calculation of transition dipoles for vibrational modes of excited or
open shell systems, or for other use, the perturbed density matrix is automatically  reconstructed.
\index{Density matrix!reconstruction in MECI}
\end{enumerate}
\end{enumerate}
\subsection{Atom Transition Moments}\index{Transition!dipole moments}\index{Polarizability!atomic transition}
\index{Dipole!transition}\index{Exponents!transition dipole}
\index{Polarizability}\label{oscil}
\index{Oscillator}
A system can go from the ground state to an excited  state as the result of the
absorption of a photon.  The probability of this happening, $\kappa$, is
given\footnote{\samepage Wilson Decius and Cross,  ``Molecular Vibrations'', p
163, McGraw-Hill (1955)} in terms of the  oscillator integral:
\begin{eqnarray}
<\!\Psi_{0}|\stackrel{\rightharpoonup}{r}|\Psi_{*}\!> \label{os1},
\end{eqnarray}
by
$$
\kappa = \frac{8\pi^3}{3ch}\nu_{n'n''}(N_{n'}-N_{n''})
<\!\Psi_{0}|\stackrel{\rightharpoonup}{r}|\Psi_{*}\!>^2 .
$$
For electronic photoexcitations, $\Psi_A$ are state functions:
$$
\Psi_A = \sum_ic_i\Psi_i,
$$
and the $\Psi_i$ are microstates; see p.~\pageref{sd} for a
definition of microstates.
\subsubsection*{Some Mathematical tools}
In order to evaluate \ref{os1}, a property of integrals of the type:
$$
<\!\psi_{i}|\stackrel{\rightharpoonup}{r}|\psi_{j}\!>
$$
will be used several times.  This property is:
$$
<\!\psi_{i}|\stackrel{\rightharpoonup}{r}|\psi_{i}\!> = 0.
$$
From this, it follows that, if
$$
<\!\psi_{j}|\stackrel{\rightharpoonup}{r}|\psi_{i}\!> \neq 0,
$$
then
$$
<\!\psi_{i}|\stackrel{\rightharpoonup}{r}|\psi_{j}\!> =
-<\!\psi_{j}|\stackrel{\rightharpoonup}{r}|\psi_{i}\!> .
$$
To prove this relationship, consider the integral
$$
<\!(\psi_i+\psi_j)|\stackrel{\rightharpoonup}{r}|(\psi_i+\psi_j)\!>.
$$
Obviously, this integral has a value of zero, therefore
$$
<\!\psi_i|\stackrel{\rightharpoonup}{r}|\psi_i\!>  +
<\!\psi_j|\stackrel{\rightharpoonup}{r}|\psi_i\!>  +
<\!\psi_i|\stackrel{\rightharpoonup}{r}|\psi_j\!>  +
<\!\psi_j|\stackrel{\rightharpoonup}{r}|\psi_j\!>  =0.
$$
In this expression, the first and fourth terms are obviously zero, therefore
$$
<\!\psi_j|\stackrel{\rightharpoonup}{r}|\psi_i\!>  =
-<\!\psi_i|\stackrel{\rightharpoonup}{r}|\psi_j\!> .
$$

\subsubsection{Evaluation of Transition Dipole}

\begin{table}
\caption{\label{transx} ``$x$" Transition Integrals}
\begin{center}
\begin{tabular}{l|ccccccccc} \hline
& $s$  &  $p_x$  &  $p_y$  &  $p_z$  &  $d_{x^2-y^2}$  & $d_{xz}$  &
$d_{z^2}$  &  $d_{yz}$  &  $d_{xy}$ \\ \hline
$s$ & X$_A$\\
$p_x$ & sp & X$_A$\\
$p_y$  & 0 & 0 & X$_A$ \\
$p_z$  & 0 & 0 & 0 & X$_A$\\
$d_{x^2-y^2}$ & 0 & pd & 0 & 0 & X$_A$\\
$d_{xz}$      & 0 & 0 & 0 & pd & 0 & X$_A$\\
$d_{z^2}$     & 0 & -$\frac{1}{\sqrt{3}}$pd & 0 & 0 & 0 & 0 & X$_A$\\
$d_{yz}$      & 0 & 0 & 0 & 0 & 0 & 0 & 0 & X$_A$\\
$d_{xy}$      & 0 & 0 & pd & 0 & 0 & 0 & 0 & 0 & X$_A$\\  \hline


\end{tabular}\\
\hspace{-0.3in}Note: X$_A$ = $<\! \phi_{\lambda}|\stackrel{\rightharpoonup}{x}|\phi_{\lambda}\! >$;
sp = $<\! ns|\stackrel{\rightharpoonup}{r}|np\! >$; pd = $<\! np|\stackrel{\rightharpoonup}{r}|nd\! >$ (see below).
\end{center}
\end{table}

The probability, $B_{0\rightarrow *}$, that a photon will be absorbed by a system that has a
ground state $\Psi_0$ and an excited state $\Psi_*$ separated by an energy $\epsilon$ when
irradiated by an energy density $\rho_{\epsilon}$
is given by
$$
B_{0\rightarrow *} = \frac{2\pi}{3\hbar^2}|R_{0*}|^2\rho_{\epsilon},
$$
in which
$$
|R_{0*}|^2 = |X_{0*}|^2 + |Y_{0*}|^2 + |Z_{0*}|^2.
$$

$X_{0*}$ is the matrix element for the $x$ component of the dipole moment:
$$
X_{0*} = \int \Psi_0|e\sum_j \stackrel{\rightharpoonup}{x_j} |\Psi_* d\tau.
$$

Evaluation of this integral requires evaluating the effect of the operators  $\stackrel{\rightharpoonup}{x_j}$,
$\stackrel{\rightharpoonup}{y_j}$, and $\stackrel{\rightharpoonup}{z_j}$
acting on an atomic orbital.  Tables \ref{transx}, \ref{transy}, and \ref{transz} show
the integrals of the type $<\! \phi_{\lambda}
 |\stackrel{\rightharpoonup}{r_j}|\phi_{\sigma}\!>$, where $\phi_{\lambda}$ and $\phi_{\sigma}$ are
 pairs of atomic orbitals.


The integral
 $<\! \phi_{\lambda}|\stackrel{\rightharpoonup}{r}|\phi_{\lambda}\! >$ is simply the appropriate Cartesian coordinate,
that is, the $x$, $y$, or $z$ coordinate of the atom that $\phi_{\lambda}$ is on.

\begin{table}
\caption{\label{transy} ``$y$" Transition Integrals}
\begin{center}
\begin{tabular}{l|ccccccccc} \hline
& $s$  &  $p_x$  &  $p_y$  &  $p_z$  &  $d_{x^2-y^2}$  & $d_{xz}$  &
$d_{z^2}$  &  $d_{yz}$  &  $d_{xy}$ \\ \hline
$s$ & Y$_A$\\
$p_x$ & 0 & Y$_A$\\
$p_y$  & sp & 0 & Y$_A$ \\
$p_z$  & 0 & 0 & 0 & Y$_A$\\
$d_{x^2-y^2}$ & 0 & 0 & -pd & 0 & Y$_A$\\
$d_{xz}$      & 0 & 0 & 0 & 0 & 0 & Y$_A$\\
$d_{z^2}$     & 0 & 0 & -$\frac{1}{\sqrt{3}}$pd & 0 & 0 & 0 & Y$_A$\\
$d_{yz}$      & 0 & 0 & 0 & pd & 0 & 0 & 0 & Y$_A$\\
$d_{xy}$      & 0 & pd & 0 & 0 & 0 & 0 & 0 & 0 & Y$_A$\\  \hline


\end{tabular}\\
\end{center}
\end{table}


 $<\! ns|\stackrel{\rightharpoonup}{r}|np\! >$
 and  $<\! np|\stackrel{\rightharpoonup}{r}|nd\! >$
 can be evaluated using the following expressions:

$$
<\! ns|\stackrel{\rightharpoonup}{r}|np\! > = a_0\frac{(2n+1).2^{2n+1}.(\xi_s\xi_p)^{n+1/2}}{\sqrt{3}(\xi_s+\xi_p)^{2n+2}}
$$

$$
<np|\stackrel{\rightharpoonup}{r}|nd>=a_0\frac{(n_p+n_d+1)!.2^{n_p+n_d+1}.\xi_p^{n_p+1/2}.\xi_d^{n_d+1/2}}
{\sqrt{5}(\xi_p+\xi_d)^{n_p+n_d+2}.\sqrt{(2n_p)!(2n_d)!}},
$$
where $ns$, $np$, and $nd$ are $s$, $p$, and $d$ quantum numbers, respectively.
For the $sp$ transition, $n=ns=np$.
The Slater orbital exponents, $\xi_s$, $\xi_p$, and $\xi_d$, are usually
given in atomic units, that is, in
inverse Bohr, therefore they must converted to \AA ngstroms before use, hence the
presence of the $a_0=0.529$ in these expressions.

All integrals of the type used here are in \AA ngstroms, therefore the units of the integral
of the dipole operator on a M.O. is also in \AA ngstroms:
$$
<\! \psi_i|\stackrel{\rightharpoonup}{r}|\psi_j\! > =\sum_{\lambda}\sum_{\sigma}c_{\lambda i}c_{\sigma j}
<\! \phi_{\lambda}\stackrel{\rightharpoonup}{r}\phi_{\sigma} \!>,
$$
 Once the value of the integral is known,
the phase to be used must be determined. The simplest way to achieve this
is to reverse the sign of the oscillator whenever the second M.O.\ has a
higher index than the first.

\begin{table}
\caption{\label{transz} ``$z$" Transition Integrals}
\begin{center}

\begin{tabular}{l|ccccccccc} \hline
& $s$  &  $p_x$  &  $p_y$  &  $p_z$  &  $d_{x^2-y^2}$  & $d_{xz}$  &
$d_{z^2}$  &  $d_{yz}$  &  $d_{xy}$ \\ \hline
$s$ & Z$_A$\\
$p_x$ & 0 & Z$_A$\\
$p_y$  & 0 & 0 & Z$_A$ \\
$p_z$  & sp & 0 & 0 & Z$_A$\\
$d_{x^2-y^2}$ & 0 & 0 & 0 & 0 & Z$_A$\\
$d_{xz}$      & 0 & pd & 0 & 0 & 0 & Z$_A$\\
$d_{z^2}$     & 0 & 0 & 0 & $\frac{2}{\sqrt{3}}$pd & 0 & 0 & Z$_A$\\
$d_{yz}$      & 0 & 0 & pd & 0 & 0 & 0 & 0 & Z$_A$\\
$d_{xy}$      & 0 & 0 & 0 & 0 & 0 & 0 & 0 & 0 & Z$_A$\\  \hline
\end{tabular}\\
\end{center}
\end{table}


  Evaluation of the integrals over microstates is straightforward,
in that all integrals are zero, unless the number of differences between the microstates is exactly two,
in which case the integral is equal to that of the two M.O.s involved, times a phase factor.  That is,
for each pair of microstates that are identical, except for $\psi_i$
in $\Psi_a$ and  $\psi_j$ in $\Psi_b$, the integral is:.

$$
<\! \Psi_a|\stackrel{\rightharpoonup}{r}|\Psi_b\! > =<\! \psi_i|\stackrel{\rightharpoonup}{r}|\psi_j\! >*(-1)^n,
$$
where $n$ is the number of permutations necessary to move $\psi_i$ in microstate $\Psi_a$ to the position
occupied by $\psi_j$ in microstate $\Psi_b$. This is
similar to the `b' option on page~\pageref{b}.  As with
the molecular orbitals, the oscillators for microstates change sign
when the order of the microstates is reversed.  The simplest way to
achieve this is to use the same device that was used with the M.O.s;
that is, to reverse the sign of the oscillator whenever the second
microstate has a higher index than the first.


For completeness, the sign of $R_{0*}$ should be reversed if $k>l$, but
since only the modulus is used, this operation does not need to be done.


Finally, the state transition dipole can be calculated from:
$$
<\! \Psi_A|\stackrel{\rightharpoonup}{r}|\Psi_B\! > =\sum_a\sum_bc_{A a}c_{B b}<\! \Psi_a\stackrel{\rightharpoonup}{r}\Psi_b \!>,
$$

Although the transition dipole is normally regarded as involving the ground and an excited state, it is
possible to calculate the transition between two excited states.  The initial state is, by default, the
ground state, however if \comp{ROOT=n} $n\neq 1$, or any other keyword that specifies
a state other than the ground state, then the initial state will be an excited state.

For degenerate states, the transition dipole is the sum over all states involved.

\subsection{States arising from various calculations}
Each MECI calculation invoked by use of the keyword C.I.=$n$ normally
\index{Spin!quantization|ff} gives  rise to states of quantized spins.  When
C.I.\ is used without any other modifying keywords, the states shown in
Table~\ref{sq} will be obtained. These numbers of spin states will be obtained
irrespective  of  the chemical nature of the system.

\begin{table}
\caption{\label{sq} States arising from \comp{C.I.=$n$}}
\begin{center}
\begin{tabular}{crrrrrr} \hline
No.\ of M.O.s & \multicolumn{3}{c}{States Arising from} &
\multicolumn{3}{c}{States Arising from}\\ in MECI &\multicolumn{3}{c}{Odd
Electron Systems}& \multicolumn{3}{c}{Even Electron Systems}\\ \hline
\index{States!arising from C.I.}
 & Doublets &Quartets&Sextets &Singlets& Triplets&Quintets
\\ \hline
1 & 1 &   & &1\\
2& 2 &   &   &3& 1\\
3& 8 & 1 & &6& 3 \\
4&20 & 4 & & 20&15 &1\\
5&75 &24 & 1& 50&  45&  5 \\ \hline
\end{tabular}
\end{center}
\end{table}

\subsection{Spin angular momentum}
State functions are eigenvalues of the $S_z$ and $S^2$ operators. The
derivation of the expectation value of the $S^2$ operator is given in this
section.

The fundamental spin operators have the following \index{Spin!operators}
effects:
$$
\begin{array}{ll}
S_x\alpha = \frac{1}{2}\beta & S_x\beta =\ \  \frac{1}{2}\alpha \\
S_y\alpha = \frac{i}{2}\beta & S_y\beta = -\frac{i}{2}\alpha \\
S_z\alpha = \frac{1}{2}\alpha & S_z\beta = -\frac{1}{2}\beta \\
\end{array}
$$
Using these expressions, various useful identities can be
established:
\index{Shift operators}
$$
\begin{array}{ll}
S^2 = S_x^2+S_y^2+S_z^2 \\
I^+ = (S_x+iS_y);& I^+\beta = \alpha \\
I^- = (S_x-iS_y);& I^-\alpha = \beta
\end{array}
$$
\begin{eqnarray}
S_x^2+S_y^2&=&(I^+I^-)+i(S_xS_y-S_yS_x)\nonumber \\
           &=&(I^-I^+)+i(S_yS_x-S_xS_y)\nonumber \\
           &=&\frac{1}{2}(I^+I^- + I^-I^+)\nonumber
\end{eqnarray}
and finally $i(S_yS_x - S_x S_y ) = S_z$.

For any microstate $\Psi$, the expectation value of the $S^2$ operator is
given by
$$
<S^2>=<\Psi|S_z^2+S_y^2+S_x^2|\Psi>.
$$
The first part of this expression is obvious, {\em vis}:
$$
<\Psi|S_z^2|\Psi> = \frac{1}{4}(N^{\alpha}+N^{\beta}).
$$
However, the effect of $S_y^2+S_x^2$ is not so simple. By making use of the
fact that the operators involve two electrons, a large number of integrals
resulting from the expansion of the Slater determinants can be readily
eliminated. The only integrals which are not zero due to the orthogonality of
the eigenvectors, i.e., those which may be finite due to the spin operators,
are
$$
<\Psi|S_y^2+S_x^2|\Psi>  = 2\sum_{i<j}[<\psi_i\psi_i|S_y^2+S_x^2|\psi_j\psi_j>-
<\psi_i\psi_j|S_y^2+S_x^2|\psi_i\psi_j>].
$$
Using the relationships already defined, this expression
simplifies~\cite{delaat} as follows:
$$
S_1S_2=S_{1z}S_{2z}+\frac{1}{2}(I_1^+I_2^-+I_1^-I_2^+)
$$
$$
<\Psi|S^2|\Psi> = 2\sum_{i<j}[\frac{1}{4}(2\delta(m_{s_i}m_{s_j}
-1-\frac{1}{2}(1-\delta(m_{s_i}m_{s_j}))<\psi_i\psi_j>^2]
$$
 or,
$$
<\Psi|S^2|\Psi> =\frac{3(p+q)}{4}+\frac{p(p-1)}{2}+\frac{q(q-1)}{2}
-\frac{(p+q)(p+q-1)}{4}-\sum_{ij}^{pq}<\psi_i\psi_j>^2.
$$
Recall that $p$ is the number of $\alpha$ electrons, and $q$, the number of
$\beta$ electrons.  This expression simplifies to yield
$$
<\Psi|S^2|\Psi> =\frac{1}{2}(p+q)+\frac{1}{4}(p-q)^2-
\sum_i^p\sum_j^q<\psi_i\psi_j>^2.
$$
For the general case, in which the state function $\Phi$, is a
\index{State!function|ff}
linear combination of microstates, the expectation value of
S is more complicated:
$$
<\Phi_k|S^2|\Phi_k> = \sum_i\sum_jC_{ik}C_{jk}<\Psi_i|S^2|\Psi_j> .
$$
As with the construction of the C.I.\ matrix, the elements of this expression
can be divided into a small number of different types:
\begin{enumerate}
\item $\Psi_a=\Psi_b$: Since the two wavefunctions are the same, this
corresponds to the expectation value of a microstate, and has already been
derived.

\item Except for $\psi_i$ in $\Psi_a$ and $\psi_j$ in $\Psi_b$; $\Psi_a
=\Psi_b$:  Assuming $\psi_i$ and $\psi_j$ to have alpha-spin the expectation
value is
$$
<\Psi_a|S_y^2+S_x^2|\Psi_b>=
 (-1)^W\sum_{k=B}^A(<ij|kk>-<ik|jk>)O_k^{\alpha a}
+<ij|kk>O_k^{\beta a}.
$$
The effect of the spin operator is to change the spin of the electrons but
leave the space part unchanged. All integrals vanish identically due to one or
more of the following identities:
$$
\begin{array}{rclrcl}
<\psi_i\psi_j>&=&0;&<m_im_j>&=&\delta(i,j)\\
<\psi_i\psi_k>&=&\delta(i,k);&<\psi_j\psi_k>&=&\delta(j,k).
\end{array}
$$
Therefore, $<\Psi_a|S^2|\Psi_b>=0$.

\item Except for $\psi_i$ and $\psi_j$ in $\Psi_a$ and $\psi_k$  and $\psi_l$
in $\Psi_b$; $\Psi_a = \Psi_b$. Two situations exist: (a) when all four M.O.s
are of the same spin; and (b) when two are of each spin.

When all four M.O.s have the same spin, the effect of the spin operator is to
reverse the spin of two M.O.s in the ket half of the integral. By spin
orthogonality this results in an integral value of zero.

In the case where two M.O.s are of $\alpha$ spin and two are of $\beta$ spin,
the matrix elements, after elimination of those terms which are zero due to
space orthogonality, are
$$
<\Psi_a|S^2|\Psi_b> = (-1)^W(<\psi_i\psi_k|S^2|\psi_j\psi_l>-
<\psi_i\psi_l|S^2|\psi_j\psi_k>)
$$
The effect of $S^2$ on $\psi_k$ and $\psi_l$ is to reverse the spin of these
functions; this gives
$$
<\Psi_a|S^2|\Psi_b> = (-1)^W(<\psi_i\psi_k'><\psi_j\psi_l'>-
<\psi_i\psi_l'><\psi_j\psi_k'>) ,
$$
where $\psi'$ has the opposite spin to that of $\psi$.

Thus, only if $\psi_i$ and $\psi_j$ are spatially identical with $\psi_k$ and
$\psi_l$ will $<\Psi_a|S^2|\Psi_b >$ be non-zero. The phase-factor W is such
that if $i=k$ and $j=l$ then W=--1, and if $i=l$ and $j=k$ then W=1; for all
other cases the matrix element is zero, so the phase of W is irrelevant. For
these two cases, the matrix element is $ <\Psi_a|S^2|\Psi_b>=1$ if
$(I^++I^-)(\psi_i+\psi_j)=(\psi_k+\psi_l)$, otherwise $<\Psi_a|S^2|\Psi_b> =
0$.

\item If more than two differences exist, $<\Psi_a|S^2|\Psi_b> = 0$.
\end{enumerate}

\subsubsection{Calculation of spin-states}
In order to calculate the spin-state, the expectation value  of  $S^2$ is
calculated.
\begin{eqnarray*}
<\Phi_k|S^2|\Phi_k> & = & S(S+1) = S_z^2 + 2 I^+I^-   \\
   & = & \frac{1}{2}(p+q) - \\
   &&\sum_i
\left\{C_{ik}C_{ik}
\left((1/4)(N^{\alpha}_i-N^{\beta}_i)^2
+ \sum_l O^{\alpha}_{li} O^{\beta}_{li}\right)
+ \sum_j2
\left[C_{ik}C_{jk} [\delta(\Psi_i,(I^+I^-)\Psi_j) ]\right]
\right\}
\end{eqnarray*}
where  $C_{ik}$  is the coefficient of microstate $\Psi_i$ in State $\Phi_k$,
$N^{\alpha}_i$ is the number of alpha electrons in microstate $\Psi_i$,
$N^{\beta}_i $ is the number of beta electrons in microstate $\Psi_i$,
$O^{\alpha}_{lk}$ is the occupancy of alpha M.O.\ $l$ in microstate $\Psi_k$,
$O^{\beta}_{lk}$ is the occupancy of beta M.O.\ $l$ in microstate $\Psi_k$,
$I^+$ is the spin shift up or step up operator,  and  $I^- $ is the spin shift
down or step down operator.

The spin state is calculated from:
$$
S = (1/2) [\sqrt{(1+4 S^2)} - 1 ]
$$
In practice, $S$  is  calculated  to  be  exactly  integer,  or  half
integer.   That  is,  there is insignificant error due to approximations used.
This does not mean, however, that the method  is  accurate.   The spin
calculation  is  completely precise, in the group theoretic sense, but the
accuracy of the calculation is limited by the Hamiltonian  used, a
space-dependent function.

\subsection{Choice of State to be Optimized}\label{cos}
MECI can calculate a large number of states of various total  spin. Two
schemes are provided to allow a given state to be selected.  First,
\comp{ROOT=$n$} will, when used on its own, select the $n$'th  state,
irrespective of  its  total  spin.  By default, $n$=1.  If \comp{ROOT=$n$} is
used in conjunction with a keyword from the set \comp{SINGLET}, \comp{
DOUBLET}, \comp{ TRIPLET},  \comp{ QUARTET}, \comp{ QUINTET}, \comp{SEXTET},
\comp{SEPTET},  \comp{OCTET}, or \comp{NONET}, then  the  $n$'th  root of that
spin-state  will be used.  For example, \comp{ROOT=4} and \comp{SINGLET} will
select the 4th singlet state.  If there are  two  triplet  states  below the
fourth singlet state then this will mean that the sixth state will be selected.

Sometimes the energy required to form an excited state is wanted.  By this
we mean the energy of the excited state relative to the energy of the ground
state, and not the heat of formation of the excited state.  To calculate this
quantity, the keywords \comp{PRECISE, GNORM=0.01, MECI} and
\comp{C.I.=2} should be used.  For
formaldehyde, these keywords would produce the output shown in Figure~\ref{figch2o}.
\begin{figure}
\begin{makeimage}
\end{makeimage}
\caption{\label{figch2o} Energies of Excited States}
\begin{center}
\begin{tabular}{cccccc}         \hline
  State &  \multicolumn{2}{c}{Energy (eV)}     &  Q.N.&  Spin &  Symmetry  \\
        & Absolute  &  Relative  \\  \hline
    1 & -0.0049   &  0.0000    &  1 & Singlet  &  A1  \\
    2 &  \ 2.7109   &  2.7158    &  1 & Triplet  &  A2  \\
    3 &  \ 3.1029   &  3.1078    &  2 & Singlet  &  A2  \\
    4 &  \ 7.8630   &  7.8679    &  2 & Singlet  &  A1  \\
\hline
\end{tabular}
\end{center}
\end{figure}
This output can be read as follows:  The first state (the one at -0.004891eV)
is the new ground state.  C.I.\ will lower the energy of the ground state,
relative to the SCF ground state, and for formaldehyde this extra stabilization
amounts to 0.0049 eV.  The ground state is a singlet, and has A$_1$ symmetry.
The second state is a triplet, with energy 2.7109eV above the SCF energy, or
2.7158eV above the ground state, and has A$_2$ symmetry.  The third and fourth
states are both singlets.

Using the two keywords given, the system would optimize on the ground singlet
state, and the bond orders and density matrix would reflect this.   If the
first excited singlet state were wanted, then the extra keywords \comp{ROOT=2}
and \comp{SINGLET} would also be used.  Alternatively, the single extra keyword
\comp{ROOT=3} could be used.  If the first triplet state were wanted, then
\comp{TRIPLET} or \comp{ROOT=2} (but not both!) could be used.

\subsubsection{Quantum Numbers}\index{Q.N.}\index{Quantum numbers!of states}
When \comp{MECI} is used, the output contains information on the symmetry of
each state.  States of different symmetries are automatically orthogonal, but
states of the same symmetry do not need to be orthogonal. Of course they are
orthogonal, and, to emphasize this fact, an extra symmetry label is added. This
label is, in fact, a quantum number, and is given under the heading ``Q.N.'' in
the output. The first occurrence of a given irreducible representation is given
the Q.N.\ ``1'', the second, ``2'', etc.  By using the Q.N.\ and the symmetry
label, each state can be assigned a unique label.

\subsubsection{Polarizability}\index{Polarizability}
The expectation value of the polarization operator is given under
``POLARIZABILITY.'' This is an approximation to the transition moment
for the absorption or emission of a photon.  One of the two states
involved is the state defined by the keywords.  By default, this is the
ground state, but might be an excited state, for example
\hyperref[pageref]{if \comp{ROOT=2} is used.}{ For a description of
this calculation, see  p.~}{}{oscil}.

\subsubsection{Franck-Condon considerations}\index{Franck-Condon}\label{FC}
This section was written based on discussions with
\begin{center} Victor I. Danilov\\
Department of Quantum Biophysics\\ Academy of Sciences of the Ukraine\\
Kiev 143\\Ukraine\end{center}
The Frank-Condon principle states that electronic transitions take place in
times that are very short compared to the time required for the nuclei to move
significantly.  Because of this, care must be taken to ensure that the
calculations actually do reflect what is wanted.

Examples of various phenomena which can be studied are:
\begin{description}
\item[Photoexcitation]\index{Photoexcitation energy}
If the purpose of a calculation is to predict the energy of photoexcitation,
then the ground-state should first be optimized.  Once this is done, then a
C.I.\ calculation can be carried out using \comp{1SCF}.  With the appropriate
keywords (\comp{MECI C.I.=$n$ } etc.), the energy of photoexcitation to the
various states can be predicted.

A more expensive, but more rigorous, calculation would be to optimize the
geometry using all the C.I.\ keywords.  This is unlikely to change the results
significantly, however.

\item[Fluorescence]\index{Fluorescence}\index{Red-shift}\index{Photoemission}
If the excited state has a sufficiently long lifetime, so that the geometry
can relax, then if the system returns to the ground state by emission of
a photon, the energy of the emitted photon will be less (it will be red-shifted) than
that of the exciting photon.  To do such a calculation, proceed as follows:
\begin{itemize}
\item Optimize the ground-state geometry using all the keywords for the
later steps, but specify the ground state, e.g.\ \comp{C.I.=3  GNORM=0.01 MECI}.
\item Optimize the excited state, e.g.\ \comp{C.I.=3 ROOT=2  GNORM=0.01 MECI}.
\item Calculate the Franck-Condon excitation energy, using the results of the
ground-state calculation only.
\item Calculate the Franck-Condon emission energy, using the results of the
excited state calculation only.
\item If indirect emission energies are wanted, these can be obtained from
the $\Delta H_f$ of the optimized excited and optimized ground-state calculations.
\end{itemize}
In order for fluorescence to occur, the photoemission probability must be quite
large, so only transitions of the same spin are allowed.  For example, if the
ground state is S$_0$, then the fluorescing state would be S$_1$.
\item[Phosphorescence]\index{Phosphorescence}
If the photoemission probability is very low, then the lifetime of the excited
state can be very long (sometimes minutes).  Such states can become populated
by S$_1 \rightarrow $ T$_1$ intersystem crossing.  Of course, the geometry of
the system will relax before the photoemission occurs.
\item[Indirect emission]
If the system relaxes from the excited electronic, ground vibrational state to
the ground electronic, ground vibrational state, then a more complicated
calculation is called for.  The steps of such a calculation are:
\begin{itemize}
\item Optimize the geometry of the excited state.
\item Using the same keywords, except that the ground state is specified,
optimize the geometry of the ground state.
\item Take the difference in $\Delta H_f$ of the optimized excited and optimized
ground-state calculations.
\item Convert this difference into the appropriate units.
\end{itemize}
\item[Excimers]\index{Excimers}
An excimer is a pair of molecules, one of which is in an electronic excited
state.  Such systems are usually stabilized relative to the isolated systems.
Optimization of the geometries of such systems is difficult.  Suggestions on
how to improve this type of calculation would be appreciated.
\end{description}

\subsection{Definition of some C.I.\ Keywords}\label{chadef}
It has been my policy, ever since the first release of MOPAC in 1983, to resist
changing the definition of keywords.  This policy has allowed users to
confidently use a new MOPAC in the belief that old keywords will have their
old, familiar, meaning.  However, an ambiguity was found in certain keywords,
an ambiguity which has, at times, resulted in severe frustration.

Consider the word \comp{TRIPLET}.  This meant (but no longer means) ``Do an SCF
calculation in which the M.O.\ populations are [\ldots,2,2,2,1,1,0,0,\ldots],
then do a C.I.\ on the two half-occupied M.O.s, and select the triplet state.''
This definition meant that twisted ethylene would have the correct symmetry, as
would triplet oxygen.  However, if a user wanted to examine triplet
formaldehyde, and compare it with the ground state, problems arose.  The
keywords \comp{C.I.=2 ROOT=2} would generate the correct energy, but a user
might expect that \comp{TRIPLET} should achieve the same result.  Because of
the definition of \comp{TRIPLET}, the SCF starting configuration was different,
and as a result, the $\Delta H_f$ was also different. Under earlier MOPACs,
there was no way to set up a calculation using the keyword \comp{TRIPLET} and
go SCF on a closed-shell configuration as the precursor to a C.I.\
calculation.


Because of the limitations of the earlier definitions of spin-states
(\comp{TRIPLET}, \comp{QUARTET}, \comp{QUINTET}, \comp{SEXTET}, etc.), these
words were all redefined in 1993,  in MOPAC~93.  In order to reproduce the
earlier keywords, pairs of keywords, such as \comp{TRIPLET} \comp{OPEN(2,2)} or
\comp{SEXTET} \comp{OPEN(5,5)} must now be used.  Spin-states which result from
SCF calculations on ground-state configurations can now be specified by the
following pairs of keywords: \comp{TRIPLET C.I.=2}; \comp{QUARTET C.I.=3};
\comp{QUINTET C.I.=4}; \comp{SEXTET C.I.=5}.

Using these new definitions, spin-states of a system can now be more easily
related.  Consider the various states of formaldehyde (Table~\ref{cich2o1}),
in which all calculations use the ground-state geometry and \comp{1SCF}.
\begin{table}
\caption{\label{cich2o1}Examples of Use of C.I.\ Keywords}
\begin{center}
\begin{tabular}{lr}
\hline
Keywords Used   &  \multicolumn{1}{c}{$\Delta H_f$}   \\ \hline
(No keywords)          & -32.9040 \\
\comp{C.I.=1} & -32.9040 \\
\comp{C.I.=2} & -33.0166\\
\comp{C.I.=3} & -39.7234\\
\comp{C.I.=4} & -39.9665\\
\comp{C.I.=5} & -40.1743\\
\comp{C.I.=2 TRIPLET} &  29.6348\\
\comp{C.I.=3 ROOT=2} &  28.2840\\
\comp{C.I.=3 TRIPLET} &  28.2840\\
\comp{C.I.=3 TRIPLET MS=0} &  28.2840\\
\comp{OPEN(2,2) TRIPLET} &  27.9318\\
\hline
\end{tabular}
\end{center}
\end{table}

Now we see that \comp{C.I.=3} \comp{ROOT=2} and \comp{C.I.=3} \comp{TRIPLET}
do, in fact, give the same result.  The ``old'' MOPAC (pre-1993) result of
using  \comp{TRIPLET} can still be generated by \comp{OPEN(2,2)}
\comp{TRIPLET}.  Note that \comp{C.I.=1} generates the normal $\Delta H_f$ of
CH$_2$O, and that increasing the C.I.\ lowers the energy steadily.

\subsection{Degenerate States}\label{dest}
\index{Jahn-Teller!theorem}
\index{Point-group!dynamic Jahn-Teller}
By the Jahn-Teller theorem, systems with orbital degeneracy will distort so as
to remove the degeneracy.  However, many dynamic Jahn-Teller systems are known
in which the time-average geometry is of the higher point-group.  These systems
are the kind that will be addressed here. \index{Liotard@{\bf Liotard, Daniel}}

The analytical RHF configuration interaction first derivative calculation
developed by Liotard~\cite{analci} has been modified to allow systems with
degenerate states to be run.

Each of the degenerate states is a linear combination of microstates. Each
microstate can be described by a Slater
determinant~\cite{slater_det1,slater_det2}, which represents a specific pattern
of occupancy of molecular orbitals. Each M.O.\ is a linear combination of
Slater atomic orbitals.

\index{States!mixtures of} The whole state is best described by an equal
mixture of the degenerate states of which it is composed.  Note that this is
NOT a combination of states, rather it is a mixture.  An example of a
combination of  states is a state function, composed of a linear combination of
microstates.  In such a combination the phase-factor between microstates is
significant, thus state(1)= $1/\sqrt{2}$(Microstate(a) + Microstate(b)) is
different from state(2) $1/\sqrt{2}$(Microstate(a) - Microstate(b)).  An
example of a mixture of states is the $^2$T$_{2g}$ state of TiF$_6^{3-}$, a
$d^1$ system, in which the best description of the state is an equal mixture of
the three degenerate space components of T$_{2g}$, and an equal mixture of the
two spin components of the Kramer's doublet.  The overall state is thus
1/6($\alpha$(T$_{2g}$(x)+T$_{2g}$(y)+T$_{2g}$(z))+
$\beta$(T$_{2g}$(x)+T$_{2g}$(y)+T$_{2g}$(z)).

If equimixtures are not used, then the Jahn-Teller theorem applies, and the
system would immediately distort so as to remove the degeneracy.  In the case
of TiF$_6^{3-}$, this would result in distortion from O$_h$ to D$_{4h}$
symmetry.


\subsection{Calculation of  Spin Density}
\index{Unpaired spin density|ff}\index{Spin!density, unpaired|ff}
Starting  with  the  state  functions  as  linear  combinations  of
configurations,  the    spin density, corresponding to the alpha spin density
minus the beta spin density, will  be  calculated  for  the first  few
states.   This  calculation  is straightforward for diagonal terms, and only
those terms are used. \index{MECI|)}\index{Configuration interaction|)}
