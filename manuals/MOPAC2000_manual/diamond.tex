\subsubsection{Diamond}\index{Diamond}\index{Data!Diamond!construction of}\index{Solid state!data sets}

The unit cell of diamond consists of two carbon atoms. If symmetry operations were allowed,
 only one atom would be needed, but only translation operations are allowed, so two atoms
 must be used.

Each carbon atom forms four single bonds with other carbon atoms. 
This can be used in defining the translation vectors: the effect 
of a translation vector acting on atom 1 would be to move it to 
one of the atoms attached to atom 2. 
 Using this fact, the data set can easily be made:

First attempt at a data set for MAKPOL. \begin{verbatim}
MERS=(4,4,4)  Diamond
C
C  1.545 1 0 0 0 0 1
Tv 2.523 1 35.26439 10 0 1 2
Tv 2.523 1 35.26439 1 120 1 1 2 3
Tv 2.523 1 35.26439 1 240 1 1 2 3
\end{verbatim}

In this description, the number of unit cells to be used is too
small: 4 by 4 by 4.  Ideally, at least a 6 by 6 by 6 system
 should be use. The smaller number is used here purely for the
 purposes of illustration.



This data set is, however, not ideal, for the following reasons:

The angle between the translation vectors is 60 degrees. This 
means that the unit cell must be very large in order for the 
distance between opposite faces to be at least 10 \AA ngstroms. 
Diamond is a Body-Centered-Cubic lattice. This means that every 
odd cell (e.g. 001,111,012, etc.) is missing. By specifying BCC, 
the angle between the translation vectors can be increased to 90 
degrees. This is the ideal angle to maximize the distance between 
faces. The system has a lot of symmetry. By adding ``SYMMETRY" two 
objectives can be met: First, symmetry can be used in defining the 
Tv. This is not very important in this unit cell, but does make it 
easier to change the bond-length of atom 2, in that any change in 
this distance is automatically made in the Tv. Second, if SYMMETRY 
is present, MAKPOL will automatically add symmetry to the data 
set.

\begin{verbatim}
Data set for MAKPOL. File name: Make_diamond.dat
 SYMMETRY MERS=(4,4,4) BCC
 Diamond, 64 atoms 
C 
C 1.545 1 0 0 0 0 1 
Tv 1.784 0 54.73561 0 0 0 1 2 
Tv 1.784 0 54.73561 0 120 0 1 2 3 
Tv 1.784 0 54.73561 0 240 0 1 2 3 

2 19 1.1547005 3 4 5 
\end{verbatim}

When this data set is run using MAKPOL, the following data set is 
generated: 
\begin{verbatim}
SYMMETRY MERS=(4,4,4) BCC 
Diamond, 64 atoms 

C  0.000000 0   0.000000 0    0.000000 0 
C  1.545000 1   0.000000 0    0.000000 0 1 
C  3.568025 1  54.735610 1    0.000000 0 1 2 
C  1.545000 0 125.264390 1    0.000000 1 3 1 2 
C  2.522974 1  35.264390 1   60.000000 1 1 2 3 
C  1.545000 0 144.735610 1    0.000000 0 5 1 2 
C  2.522974 0 135.000000 1   45.000000 1 3 1 2 
C  1.545000 0 144.735610 0  -90.000000 1 7 3 1 
C  2.522974 0  90.000000 1   90.000000 0 5 1 2 
C  1.545000 0  90.000000 0 -144.735610 1 9 5 1  

(many lines deleted) 

C  1.545000 0  90.000000 0  160.528779 1 61 57 41 
C  2.522974 0  60.000000 0  -35.264390 0 59 43 26
C  1.545000 0  90.000000 0  160.528779 0 63 59 43 
XX 2.522974 0  90.000000 0  180.000000 0 7   3 1 
XX 2.522974 0  90.000000 0 -125.264390 0 29  9 5 
XX 2.522974 0 180.000000 0    0.000000 0 53 41 42 
Tv 7.136049 1   0.000000 0    0.000000 0 1  65 2 
Tv 7.136049 0   0.000000 0    0.000000 0 1  66 2 
Tv 7.136049 0   0.000000 0    0.000000 0 1  67 2 

2 1 4 6 8 10 12 14 16 18 20 22
2 1 24 26 28 30 32 34 36 38 40 42
2 1 44 46 48 50 52 54 56 58 60 62
2 1 64
4 3 6 11 12 15 16 18 33 34 35 36
4 3 39 40 43 44 45 51 52 53 54 55
4 3 56 59 60 67
5 1 7 9 11 13 15 17 19 21 23 25
5 1 27 29 31 33 35 37 39 41 43 45
5 1 47 49 51 53 55 57 59 61 63 65
5 1 66 67
5 2 17
5 14 17

(many lines deleted) 

\end{verbatim}
Before running this data set, more symmetry can be added 
``by hand." Every angle is symmetry defined, and does not need to 
be optimized, therefore every angle and dihedral optimization flag 
can be set to zero. Every angle and dihedral symmetry relation 
defined at the end of the data set can also be deleted. These are 
the lines that start with a number followed by a 2, a 3, or a 14. 
Every distance can be related to the bond-length of atom 2. The 
position of atoms 3 and 5, and the length of the translation 
vector Tv can be defined by distances that are exactly Sqrt(16/3). 
Sqrt(8/3) and Sqrt(64/3) times the C$_2$-C$_1$ 
distance, respectively. These symmetry relations can be defined 
using MOPAC symmetry function 19. This has the form: Defining-atom 
19 multiplier dependent atom(s) When these changes are made to the 
data set, the final data set is produced. This is: 
\begin{verbatim}
 SYMMETRY MERS=(4,4,4) BCC 
 Diamond, 64 atoms

  C    0.000000  0   0.000000  0    0.000000  0
  C    1.545000  1   0.000000  0    0.000000  0     1
  C    3.568025  0  54.735610  0    0.000000  0     1     2
  C    1.545000  0 125.264390  0    0.000000  0     3     1     2
  C    2.522974  0  35.264390  0   60.000000  0     1     2     3
  C    1.545000  0 144.735610  0    0.000000  0     5     1     2
  C    2.522974  0 135.000000  0   45.000000  0     3     1     2
  
  (many lines deleted) 
 
  C    1.545000  0  90.000000  0  160.528779  0    61    57    41
  C    2.522974  0  60.000000  0  -35.264390  0    59    43    26
  C    1.545000  0  90.000000  0  160.528779  0    63    59    43
 XX    2.522974  0  90.000000  0  180.000000  0     7     3     1
 XX    2.522974  0  90.000000  0 -125.264390  0    29     9     5
 XX    2.522974  0 180.000000  0    0.000000  0    53    41    42
 Tv    7.136049  0   0.000000  0    0.000000  0     1    65     2
 Tv    7.136049  0   0.000000  0    0.000000  0     1    66     2
 Tv    7.136049  0   0.000000  0    0.000000  0     1    67     2

   2 19 2.3094  3
   2 19 1.6330  5
   2 19 4.6188 68
   2  1    4    6    8   10   12   14   16   18   20   22
   2  1   24   26   28   30   32   34   36   38   40   42
   2  1   44   46   48   50   52   54   56   58   60   62
   2  1   64
   5  1    7    9   11   13   15   17   19   21   23   25
   5  1   27   29   31   33   35   37   39   41   43   45
   5  1   47   49   51   53   55   57   59   61   63   65
   5  1   66   67
  68  1   69   70
\end{verbatim} 

Is the use of symmetry worth all this effort? Most 
definitely! If symmetry is NOT used, then for this small system of 
64 atoms, 195 parameters would need to be optimized. That is, 
3*64-6 for the 64 atoms plus 9 parameters for the three 
translation vectors. Optimization of a system with 195 unknowns 
would take much longer than for a system with precisely one 
unknown. For a more realistic system, involving 6 by 6 by 6 
primitive unit cells, symmetry would lower the complexity of the 
calculation from 651 unknowns to precisely 1. 
