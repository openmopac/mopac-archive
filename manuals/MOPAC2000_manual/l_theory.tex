\section{Localized Molecular Orbital Theory}\label{lmot}
\index{MOZYME!theory}\index{Localized MOs!theory of}
\subsection{Why Use Localized Molecular Orbitals?}
Using conventional SCF methods, the time required for a SCF calculation rises
as the third or higher power of the size of the system (number of atoms,
orbitals, or electrons).  For semiempirical methods, this places a practical
limit on the number of atoms in a molecule:   about 500--1,000, using a
supercomputer.  

\index{Conventional SCF!N$^3$ dependence} The origin of the $N^3$ dependence
lies in the way conventional molecular orbitals are generated and used.  They
are generated from a Fock matrix  either  by a diagonalization, or, if
approximate M.O.s exist, by a  pseudo-diagonalization.  Both processes are of
type $N^3$.  Once generated, they are used in the construction of a density
matrix.   Because every M.O. extends over every atom, this process is also of
order $N^3$.   Most of the time in a semiempirical calculation is spent in
these two operations, therefore the overall time dependency of conventional
semiempirical calculations rises as $N^3$.

To understand how limiting this dependency is, consider how the time
requirement rises with increasing numbers of atoms.  For the sake of
discussion, assume that the time required for an SCF calculation of an amino
acid is one second.  Table~\ref{t-amino} shows the time required for larger
systems.

\begin{table}
\begin{center}
\caption[Times for One SCF Calculation]{\label{t-amino}Predicted Times Required 
for One SCF Calculation Using Conventional SCF Methods*}
\begin{tabular}{rrl}\\ \hline
No.\ of Residues  &   \multicolumn{2}{c}{Time} \\ \hline
2        &   8.0 & seconds \\
5        &   2.1 & minutes \\
10       &  16.7 & minutes \\
20       &   2.2 & hours   \\
50       &  1.4 & days   \\
100      & 11.6 & days  \\
200      & 3.1 & months \\
500      & 4.0 & years  \\
1,000    & 31.7 & years \\
2,000    & 253.6 & years \\
\hline
\end{tabular} \\
$*$: Using Conventional SCF Methods, \\ and assuming that one
amino acid \\ runs in one second.
\end{center}
\end{table}

Many enzymes containing thousands of residues are known.  It is obvious that
these systems cannot be studied using conventional SCF methods.  By using
localized molecular orbitals, the time dependency can be changed from $N^3$ to
$N^1$.  LMO methods require much more arithmetic manipulation, in that every
array which holds atomic orbital information has one or more associated integer
arrays which specify the atoms involved and their location. For this reason,
LMO methods are much slower than conventional methods for small systems.  
%LMO and conventional methods run at about the same speed for
%systems of about 100 atoms, and for larger systems LMO methods are faster.
%If the $N^3$ to $N^1$ ratio is followed exactly, then a system of 1,000 atoms 
%should run about 100 times faster using LMO methods, and a system of 
%10,000 atoms should run 10,000 times faster.

\subsubsection{What is a Localized Molecular Orbital?}
Unlike a conventional molecular orbital, which extends over all atoms, a
localized molecular orbital is   localized in a small region of the molecule. 
Almost all of a localized M.O.\ can be found on one, two, or, at most, three
atoms. The nature and behavior of the LMO is dictated by these few atoms. Small
amounts of the LMO can be found on the surrounding atoms.  Although the effect
on these small contributions on the nature of the LMO is very small, their
existence is responsible for almost all of the time spend on mathematical
manipulation of the LMOs.

\subsubsection{Energy Considerations.}\index{Energy considerations}
The electronic energy of a system is given by 
\begin{equation}
E_{elect} = \frac{1}{2}\sum_{\lambda}\sum_{\sigma},
P_{\lambda\sigma}(H_{\lambda\sigma}+F_{\lambda\sigma})
\end{equation}
where $P_{\lambda\sigma}$ is the density matrix element connecting atomic
orbitals $\phi_{\lambda}$ and $\phi_{\sigma}$.

In practice, the density matrix elements rapidly become small as the
interatomic distance increases.  Only for atoms which are chemically bonded
together will $P_{\lambda\sigma}$ be large.  As an example, in propanolamine,
the smallest bond order between two chemically bonded atoms is 0.98. Between
any two neighboring but non-chemically bound atoms (that is, atoms  separated
by one atom), the largest bond order is 0.015, and for all other  interactions,
the bond-orders are very small.

Because the electronic energy is determined by the density matrix, the
contribution to the electronic energy arising from density matrix elements
between atoms which  are not near to each other is small, and becomes very
small with increasing  distance.  The contributions from atoms separated by
more than about 10~\AA ngstroms is quite negligible, and can safely be
ignored.  From this it follows that the only density matrix terms which need to
be considered are those arising from atoms separated by less that 10~\AA
ngstroms.  Since the density matrix is constructed from the M.O.s only, then,
in LMO theory, the LMOs need only extend over a distance of about 10~\AA
ngstroms from the center of the LMO.  

\subsubsection{Limitations on Systems.}
The use of localized molecular orbitals to help solve the self-consistent
field equations is limited to what we will call `normal' compounds.  In
order for a system   to be classified as `normal', it must be possible to
sketch the molecular structure using only the following drawing elements:
\index{Compounds!`normal'!definition}

\begin{enumerate}
\item Chemical symbols to represent the atoms.
\item A maximum valency of 4 for any non-hydrogen atom.
\item Lines to represent $\sigma$ bonds.
\item Pairs of parallel lines to represent double bonds, and three
parallel lines to represent triple bonds.
\item Pairs of dots to represent lone-pairs.
\item `+' and `--' signs, to represent charges.
\end{enumerate}

\index{Aromatic rings}\index{Delocalized $\pi$ systems}
\index{pi@$\pi$ systems!delocalized}
This definition allows aromatic rings and delocalized $\pi$ systems to be
present, as these structures can be represented as localized $\pi$ bonds.

\index{Hypervalent systems!allowed}
What is {\em not} allowed are radicals, open-shell systems, electronic
excited states, etc.  Hypervalent systems are allowed, as they can always be
written as non-hypervalent Lewis structures, for example, SF$_6$ can be submitted
as a system of point group $O_h$, in which case the Lewis structure generated
would be represented by SF$_4^{++}$ plus 2F$^-$.

\subsection{Memory Management}\index{Memory!management}
\subsubsection{The Nature of the Problem.}
Conventional semiempirical software is unsuitable for the calculation of large
systems, due to the heavy memory demands made.  For example, in MOPAC~93, the
memory requirement (in Kb) rises as $2249+17.15N+7.395N^2$,  where $N$ is the
number of heavy (non-hydrogen) atoms in the molecule, assuming that the number
of hydrogen and non-hydrogen atoms are equal.  In proteins, the number of
hydrogen and non-hydrogen atoms are roughly equal, so the expression for
MOPAC~93 would be applicable to proteins.  Table~\ref{n2size} illustrates how
rapidly the memory requirement increases for proteins.

\begin{table}
\caption{\label{n2size} MOPAC Memory Requirements for Proteins}
\begin{center}
\begin{tabular}{rrr}
\hline 
$N$   &  No.$^{\dag}$ of &   Memory   \\
      & Residues & Needed (Mb)  \\ \hline
1000  &  65  &  7,414 \\
2000  & 131  & 29,616 \\
3000  & 196  & 66,609 \\
4000  & 262  &118,389 \\
5000  & 327  &184,961 \\
\hline
\end{tabular} \\
\dag : Number of residues is approximate only;\\ the number will vary
from protein to protein
\end{center}
\end{table}

Clearly, calculation of systems of only a few hundred residues is
impractical.   By making approximations, by altering the way arrays are
specified, and by use of localized molecular orbitals, the memory
requirements can be reduced. As a result of these changes, the memory
requirement for MOZYME is considerably less than that for MOPAC.  For
example, a MOPAC calculation of a system of 3,686 atoms
(bacteriorhodopsin)  would require over 100Gb; the same calculation
using MOZYME would require only 408Mb.  This represents less than
0.5\%\ of the memory required by MOPAC.  A more detailed
\htmlref{description of array specification}{newmat} appears
\begin{htmlonly}
elsewhere.
\end{htmlonly}
\begin{latexonly}
on p.~\pageref{newmat}.
\end{latexonly}


\subsubsection{Distance Cutoffs.}\label{cutoff}\index{Cutoff|ff}\label{m2el}
In conventional SCF calculations, all interactions regardless of distance are
calculated.  This is both impractical and unnecessary in LMO work. Impractical,
because the storage required would rapidly become very large: the storage
necessary for merely the two-electron two-center integrals of  a system of $M$
non-hydrogen and $N$ hydrogen atoms  would be $50M(M-1)+10M\times
N+\frac{1}{2}N(N-1)$ array elements. Unnecessary, because many of the integrals
would never be used in the LMO calculation, anyway.  Consider the electronic
energy terms.  These all depend on the  density matrix elements.  If a density
matrix element, $P_{\lambda\sigma}$  is zero, then the associated  energy term
is independent of $H_{\lambda\sigma}$ and $F_{\lambda\sigma}$. Similarly, the
Fock element $F_{\lambda\sigma}$ depends on $P_{\lambda\sigma}$; therefore, if
$P_{\lambda\sigma}$ is zero, then  $F_{\lambda\sigma}$ is independent of the
value of the exchange integral 
$<\!\lambda\sigma|\frac{e^2}{r}|\lambda\sigma\!>$.

\begin{table}
\caption{\label{bo}Bond-Orders in GLY-GLY-GLY-GLY}
\begin{center}
\begin{tabular}{lrlrrr}
\hline
Bond to  &  Order & Bond to & Order\\
\hline
C$_2$& 1.017090  &C$_8$ & 0.000009   \\
C$_3$& 0.015249  &C$_9$ & 0.000001 \\
N$_4$& 0.004989  &N$_{10}$ & 0.000000  \\
C$_5$& 0.001219  &C$_{11}$ & 0.000000  \\
C$_6$& 0.000086  &C$_{12}$ & 0.000000 \\
N$_7$& 0.000031  \\
\hline
\end{tabular}

Bonds are between the terminal nitrogen and the backbone atoms.

GLY-GLY-GLY-GLY has the structure:
$\chem H_2N\!-\!CH_2\!-\!CO\!-\!NH\!-\!CH_2\!-\!CO\!-\!NH\!-\!CH_2\!-\!CO\!-\!NH\!-\!CH_2\!-\!COOH  $.
\end{center}
\end{table}

The bond-order is a measure of the electron density between two atoms.  
Table~\ref{bo} shows the bond-orders between the terminal nitrogen and the
backbone atoms in the tetrapeptide GLY-GLY-GLY-GLY. \label{cutofs}  For atoms
separated by more than 3 - 4 bonds, that is, by more than  5 - 7 \AA ngstroms,
the bond-orders rapidly become very small.  Also, at such distances the overlap
of two atomic orbitals becomes extremely   small.  Since  the one-electron
interactions depend on the overlap 
\begin{equation}
H_{\lambda\sigma} = S_{\lambda\sigma}(\beta_{\lambda}+\beta_{\sigma}),
\end{equation}
where $S_{\lambda\sigma}$ is the atomic orbital overlap, and $\beta_{\lambda}$
and $\beta_{\sigma}$ are atomic orbital parameters, it follows that the
one-electron interactions at distances greater than about 6--7 \AA ngstroms will
also be vanishingly small.

By calculating and storing only the one-electron integrals representing
interactions of less than a given distance, a considerable saving in storage
requirements can be achieved.

Unlike the one-electron integrals, the value of some two-electron two-center 
integrals does not fall off rapidly with increasing distance.  The 100
two-electron  integrals involving two atoms can be divided into five groups, as
shown in Table~\ref{multipoles}.  Some of these integrals, such as the
$<\!pp|pp\!>$ integrals, can be expressed in terms of two multipoles, here a
monopole and a quadrupole; however, for the purpose of this discussion the
effect of the lower multipole---the monopole in this case---dominates.  

\begin{table}
\caption{\label{multipoles} Multipolar Representation of the Two Electron Two Center Integrals}
\begin{center}
\begin{tabular}{lcrcc} \hline
Multipole    & Example & No. & \multicolumn{2}{c}{Distance Dependence} \\
             &         &     & Energy & Gradient \\ \hline
Monopole     & $<\!ss|ss\!>$& 16  & $r^{-1}$& $r^{-2}$\\
Dipole       & $<\!ss|sx\!>$ & 24  & $r^{-2}$& $r^{-3}$\\
Quadrupole   & $<\!ss|xy\!>$ & 33  & $r^{-3}$& $r^{-4}$\\
Octapole     & $<\!sx|xy\!>$ & 18  & $r^{-4}$& $r^{-5}$\\
Hexadecapole & $<\!xy|xy\!>$ &  9  & $r^{-5}$& $r^{-6}$\\
\hline
\end{tabular}
\end{center}
\end{table}

At distances greater than about 5 or 6 \AA ngstroms, the behavior of these
integrals becomes very simple.  Of the 100 integrals, 60 represent quadrupoles 
or higher multipoles and have a negligible value.  The 16 monopole terms --
representing simple electrostatic repulsion -- are all composed of the same
monopole term and various quadrupolar components, and can be set equal. The 24
dipole terms can be expressed as a dipole interacting with either a simple
monopole (an $<\! ss|sx\!>$, for example), or a monopole plus quadrupole,
e.g.,   $<\! xx|sx\!>$. Since the quadrupolar terms can be ignored, the 24
dipolar terms can be represented by 6 simpler terms, of generic form  $<\!
sp|ss\!>$  and  $<\! ss|sp\!>$.  Thus, out of the 100 integrals needed at small
distances, only 7 are needed at larger distances. At still larger distances, 30
\AA ngstroms or more, even the dipolar terms become negligible.  Therefore, at
such large distances only the single monopole term, representing simple
electrostatic repulsion, need be used.

Two-electron two-center integrals must therefore be represented in a different
way from the one-electron two-center integrals.  As with the one-electron
integrals, a cutoff distance can be specified.  For convenience, the
one-electron and two-electron cutoff distances are set equal. As a result, all
one- and two-electron  integrals which represent interactions between atoms
that are separated by less than the cutoff distance would then be treated using
standard NDDO approximations. The simplest electrostatic repulsion 
($<\!ss|ss\!>$) between electrons on atoms  that are separated by more than the
cutoff distance is approximated by the conventional NDDO term 
\begin{equation}
<\!\phi_{\lambda}\phi_{\lambda}|e^2/r_{AB}|\phi_{\sigma}\phi_{\sigma}\!> =
\frac{1}{\left (R_{AB}^2+\frac{1}{2}(
\frac{1}
{<\!\phi_{s}\phi_{s}|e^2/r_{AA}|\phi_{s}\phi_{s}\!>} +
\frac{1}{<\!\phi_{s}\phi_{s}|e^2/r_{BB}|\phi_{s}\phi_{s}\!>})
\right )^{\frac{1}{2}}}.
\end{equation}

This was chosen in order to minimize the discontinuity at  the cutoff
distance.  Similar NDDO approximations were used for the other dipolar terms.

In recognition of this fact, SCF calculations can be simplified considerably 
by including only those NDDO terms which apply to atoms separated by less than
a preset limit, and including only the electrostatic term for interactions
between atoms separated by more than that limit.

\label{cutof1} 
\label{cutof2} 
The various cutoffs used have specific names and defaults.  These are summarized
in Table~\ref{cutoffs}.
\begin{table}
\caption{\label{cutoffs} Distance Cutoffs for Various Types of Interaction}
\begin{center}
\begin{tabular}{|l|l|l|l|r|cc}  \hline
Type &Less than Cutoff & Greater than Cutoff & Cutoff Name & Default \\ \hline
Two electron & All multipoles & Dipole plus Monopole & \comp{CUTOF2} & 12\AA \\
Two electron & Dipole plus Monopole & Monopole & \comp{CUTOF2} & 30\AA \\
One electron & Overlap calculated & Overlap not calculated & \comp{CUTOF2} & 7\AA \\ \hline
\end{tabular}
\end{center}
\end{table}

By default, the cutoff distances are set sufficiently large that any  $\Delta
H_f$ calculated using these values will agree with that obtained by using
conventional methods  within 0.1 kcal/mol. \comp{CUTOF2} can be reduced, under
user control, to save computational time and to reduce the memory demand.  To
avoid serious errors, \comp{CUTOF2} should not be set below about 5~\AA
ngstroms.

Once \comp{CUTOF2} is set, all one-electron integrals involving atoms separated
by less than \comp{CUTOF2} can be evaluated.  To save space, only those
integrals which are evaluated are stored an an array. For the one-electron
integral,  this array is called $H$.  All integrals relating to any pair of
atoms are stored contiguously.  Given the starting address of an atom-pair, the
sequence in which the integrals occur is determined only by the number of
atomic orbitals on each atom. However, the order in which atom pairs are stored
is not so simple.  The only way to find the starting address of any atom pair
is to have that address stored in a new array.  This array is called
\comp{IJBO}, and has the following structure:

\comp{IJBO} is a square array, of size $N$ by $N$, where $N$ is the
number of atoms. The starting address of integral string for atoms 
$A$ and $B$ is stored in array element IJBO(A,B).  If $A \ne B$, then the
starting address is also stored in IJBO(B,A).  All pairs of atoms
separated by more than \comp{CUTOF2} do not have associated integral
strings.  To recognize this, the relevant array elements in \comp{IJBO}
are set to `$-1$', and if the interatomic distance is greater than
\comp{CUTOF1} then the array element is set to `$-2$'. Zero cannot be
used, because of \hyperref[pageref]{the way starting addresses are
defined}{ (see Page~}{)}{starting-addresses}.

To reiterate: All integrals involving atoms, $A$ and $B$,  separated by  less
than \comp{CUTOF2} are stored, and the starting address of the integrals  is
given by IJBO(A,B) and IJBO(B,A).  With the exception of some electrostatic
terms, no integrals involving atoms separated by more than \comp{CUTOF2} are
calculated or stored.  The relevant array element in \comp{IJBO} is set to
`$-1$' or `$-2$'. 


To summarize:

There are three regions around each atom.  The sizes of these regions are
determined by \comp{CUTOF2=$m.mm$} (default: $m.mm$=12) and
\comp{CUTOF1=$n.nn$} (default $n.nn$=30). In the first region, 0.0 to $m.mm$
\AA ngstroms, all  NDDO interactions are used.  Between $m.mm$ and $n.nn$ \AA
ngstroms, only monopolar and dipolar electrostatic terms are used, and beyond
$n.nn$ \AA ngstroms, the only term considered is the monopolar electrostatic
interaction.

\subsection{Lewis Structures.}
\index{Lewis structure! checking}
One of the more difficult operations involved in using localized molecular
orbitals is the generation of a Lewis structure.  While this operation is
almost trivially easy for a competent chemist, setting up the instructions so
that a program can do the same operation has proved to be a daunting task.

In the following section, the steps involved in calculating the Lewis structure
are described.  This description is intended to be definitive, in the sense
that it should allow the Lewis structure for {\em any} compound to be
generated. At the same time, any deficiency in the description should be
reflected in the inability of MOZYME to generate the Lewis structure for
certain systems.

Because of its complexity, the main sequence involved will be given first,
followed by a more detailed explanation of the individual steps.

\subsubsection{Lewis Structure---Main Sequence.}
\begin{enumerate}
\item The connectivity is calculated.  This determines which atom is bonded to
which atom.
\item All atoms that have explicit charges are identified, and the charges
assigned.
\item The $\sigma$ framework is determined.
\item Most of the lone pairs are identified.
\item Open-ended (non-aromatic) $\pi$-bonds are identified.
\item Aromatic $\pi$-bonds are identified.
\item All cations, anions, and any remaining lone pairs are identified.
\end{enumerate}

\subsubsection{Detailed Description of Lewis Structure.}
\begin{description}
\item[Calculation of Connectivity]~\\
Hydrogen atoms are monovalent.  Because of this, they can only bond to one
other atom.  Therefore, the first set of bonds formed are the X--H $\sigma$
bonds.  The criterion used is that each hydrogen atom is connected to the atom
nearest to it, except that a hydrogen atom is not allowed to be bonded to
another hydrogen atom.

The connectivity of all other atoms is determined.  Atoms are considered as
being connected (bonded together) if the interatomic distance is less than
110\% of the sum of their Van der Waals radii.

Any bridging hydrogen bonds are identified.  These usually indicate a faulty
geometry.  If any are present, then the SCF calculation will not be run, unless
\comp{LET} is present.

Any user-defined chemical bonds are identified. This is useful in cases where a
Lewis structure could not otherwise be created.  An example is the simple
system HNO$_3$.

Finally, the number of atoms bonded to each atom is checked.  Conventional
Lewis structures do not allow more than four bonds to each atom (assuming an
$sp^3$ basis set), so if there are more than four atoms bonded to any atom, a
conventional Lewis structure cannot be generated, and the SCF cannot be run. 
Before abandoning the calculation, an attempt is made to reduce the number of
bonds to a hypervalent atom.  First, any bonds from a hypervalent atom to a
halogen are broken.  If this makes the hypervalent atom normal, then the
calculation can proceed.  If that does not work, then bonds to elements of
group VI are broken.  If that still does not make the atom normal-valent, then
the  calculation is stopped.

\item[$\sigma$-framework]~\\
This is the simplest set of bonds to identify.  A $\sigma$ bond exists for
every pair of atoms that are connected.  The number of $\sigma$ bonds is equal
to the number of connections in the system.  For water, this would be 3; for
benzene, 12; and for ethylene, 5.

Each time a bond is formed, the number of available atomic orbitals on the
atoms involved is decremented by 1 and the number of available electrons is
decremented by 1.  For example, before the $\sigma$ framework  of ammonia is
formed, the number of available orbitals on nitrogen is 4 and the number of
electrons is 5. After the $\sigma$ bonds are formed, there is 1 orbital and 2
electrons left.

\item[Lone pairs]~\\
The next set of Lewis elements formed are the lone pairs.  The rule used here
is that if there are more electrons than orbitals on an atom, the extra
electrons are used in the construction of lone pairs.  Each lone pair uses up
two electrons and one orbital.  Thus, one lone pair would be assigned to the
nitrogen in ammonia, two lone pairs would be assigned to oxygen in water, and
three lone pairs would be assigned to chlorine in HCl.

The lone pairs could only assigned after the $\sigma$ bonds were created.  If
they were assigned before the $\sigma$ bonds, then some ionic systems, such as
NH$_4$, could not be represented.

\item[Open-ended $\pi$-bonds]~\\
Once the lone pairs are assigned, the number of unused atomic orbitals on an
atom will be equal to the number of unused electrons on that atom.  These
unused orbitals are then available for forming multiple bonds between atoms. 
If two atoms, that are $\sigma$ bonded together, both have unused orbitals,
then they can form a $\pi$ bond.  

The order in which $\pi$ bonds are generated is important.  If in styrene,  for
example, a $\pi$ bond is assigned to the ring-C$_1$ as in Figure~\ref{styrene},
$A$, then when the remaining double bonds are created, there are two unused
atomic orbitals on atoms that are not bonded together.  In order for a Lewis
structure to be generated, two electrons are put into one of these unused
orbitals,  creating an anionic center, and no electrons are put into the other
orbital, making it a cationic center.

\begin{figure}
\begin{makeimage}
\end{makeimage}
\begin{center}  
\includegraphics{styrene}
\end{center}  
\caption{\label{styrene} Examples of Initial Choice of Double Bond}  
\end{figure} 


A better choice is to identify open-ended $\pi$ systems, and to assign these first,
option $B$ in Figure~\ref{styrene}.  

The order in which the $\pi$ bonds are assigned in an open-ended $\pi$ system is
important.  In the case of a simple conjugated polyene, the order is simple -
the carbon atom that has only one atom $\pi$ bonding to it is identified.
A $\pi$ bond is constructed between the two atoms.  This is repeated until
all $\pi$ bonds in the polyene are identified.

Problems arise in more complicated systems, such as buta-1,3-diyne.  If the
simple rule just described is used, then a zwitterionic cumulated polyene results,
Figure~\ref{butdiyne} $A$,
instead of a diyne, Figure~\ref{butdiyne} $B$.

\begin{figure}
\begin{makeimage}
\end{makeimage}
\begin{center}
\includegraphics{butdiyne}
\end{center}
\caption{\label{butdiyne} Generation of Yne Bond}
\end{figure}

Also, if both ends of the olefinic group are connected to aromatic rings, as
in stilbene, then identification of the olefin group is not obvious.

To allow for this, the following two rules are used:
\begin{enumerate}
\item Where there is the possibility of forming a triple bond, do so.
\item When a delocalized $\pi$ system is opened, the $\pi$ bond formed should
involve atoms that are $\pi$ bonded to exactly two other atoms.  
\end{enumerate}
In the case of stilbene ($\phi$--CH=CH--$\phi$), this prevents a $\pi$
bond forming between the ring and a carbon atom of the olefin.

Whenever a delocalized $\pi$ system is encountered, as soon as a $\pi$ bond
is formed and the delocalization destroyed, then the rest of the $\pi$ system
is treated as a simple conjugated polyene.  This ensures that the maximum
number of $\pi$ bonds is formed, and prevents unconnected $\pi$ bonds
from being created.  Thus benzene would have the three $\pi$ bonds:
C$_1$-C$_2$, C$_3$-C$_4$, and C$_5$-C$_6$, and not the quinoidal C$_1$-C$_2$
and C$_4$-C$_5$ bonds.

To allow compounds that contain several fused delocalized $\pi$-systems, such 
as the higher buckyballs (specifically, C$_{960}$) to be modeled, two extra rules
are needed.  These rules can be regarded as minor qualifications to the earlier
rules:
\begin{enumerate}
\item If any atoms attached to an open-ended $\pi$ system belong to a delocalized
$\pi$ system, then when delocalized $\pi$ systems are opened, the opening is done
using these atoms.
\item If a five-membered $\pi$ ring is attached to another delocalized 
$\pi$ system, then when the $\pi$ system is opened, at least one atom must be
in the other delocalized system.
\end{enumerate}

The effect of the first rule is that as soon as a graphitic network is encountered,
all the $\pi$ bonds in the network are assigned in one pass.  Without this rule,
individual parts of the network could be assigned separately, and at the junctions
of the various domains, the potential for isolated $\pi$ orbitals exists. This would
lead to charges that would cause severe problems with the SCF calculation.

\begin{figure}
\begin{makeimage}
\end{makeimage}
\begin{center}
\includegraphics{ful_fac}
\end{center}
\caption{\label{ful_fac}Kekule Structure for a Facet of Fullerene C$_{1500}$}
\end{figure}

For extended graphite-like systems, two more rules are needed.  These are:
\begin{enumerate}
\item  If a six-membered ring has two $\pi$ bonds already, then
         add a third $\pi$ bond to make it an aromatic ring.
\item If a six-membered ring has one $\pi$ bond, then add another
      $\pi$ bond to the same ring, so that the previous rule can be used.
\end{enumerate}
The effect of these rules is that when a graphitic lattice is encountered, all
the atoms in the lattice will be assigned in such a way as to maximize the
number of aromatic rings.  An example of such a system is provided by the large
icosahedral fullerene C$_{1500}$.  A facet of this system is shown in
Figure~\ref{ful_fac}.

\item[Remaining unused atomic orbitals]~\\
All that remains is to identify any unused orbitals and to assign them to
either the occupied or virtual sets.  The action taken depends mainly on
the group in the periodic table to which the atom belongs, to a lesser degree
on the nature of its environment, and sometimes on the charge on the system.

Each group has different properties. Thus:

\begin{description}
\item[Group I]~\\
The alkali metals are extremely electropositive.  Therefore,
without exception, the unused orbital is assigned to the virtual
set, and the atom is charged unipositive.

\item[Group II]~\\
The alkaline earth elements are very electropositive.  If the
atom does not form any bonds, then the charge is set to +2,
and all orbitals are assigned to the virtual set.

An atom that has formed $N$ bonds will have a charge of 2-$N$.
This can allow the atom to have a negative charge, in which case
the data set should indicate that the atoms attached to it have a positive charge.

\item[Group III]~\\
These elements are electropositive.  The action taken depends on the
number of unused valence electrons:
\begin{description}
\item[1 unused electron]  The atom is unipositive.
\item[2 unused electrons]  The atom is neutral:  the two unused electrons
are used in the formation of a `lone pair'.  
\item[3 unused electrons]  The atom is unipositive.
\end{description}

\item[Group IV]~\\
This is the most complicated group, with the charge on the atom depending on
many factors.  If there are two unused valence electrons, then the atom will be
neutral (a carbene, for example). Otherwise, if the first or second nearest
neighboring atom is of Group~6, then the atom is assigned a negative charge; if
the first or second nearest neighboring atom is of Group~5, then the atom is
assigned a positive charge. If the charge is still not determined, then the
assignment of charge is deferred until all other atoms have been assigned.  At
that point, the charge is assigned as either +1 or --1, depending on the
calculated charge on the system and the charge supplied by the data set.  The
charges are assigned so that the calculated charge equals the supplied charge.

\item[Group V]~\\
If there is one unused valence electron, then the charge assigned to Group V
elements is +1.  If there are 2, then the atom is neutral.

\item[Group VI]~\\
If oxygen  and there are two unused orbitals, then both are filled, and the
charge is --2.  If the atom has one unused orbital, the charge is --1;
otherwise the atom is neutral.

\item[Group VII]~\\
A very electronegative group, the charge is invariably --1.
\end{description}
\end{description}


\subsection{Construction of Starting Localized Molecular Orbitals}
\index{LMOs! construction of starting}
\index{Construction of starting LMOs}
In conventional semiempirical methods, molecular orbitals are created by the
diagonalization of an atomic orbital secular determinant.  Because of the size
of the matrices involved in proteins, this operation is impractical; therefore
the starting orthogonal localized molecular orbitals must be constructed in a
different way.

Monatomic and diatomic LMOs are constructed using all the atomic orbitals in
the system.  Care must be taken to ensure that a realistic set of starting LMOs
are generated.  If the set is not realistic, then, although an SCF could be
calculated, more work would be involved.  The most obvious set of LMOs is that
set corresponding to the classical line-drawing of a molecule: a sigma
framework, lone pairs, $\pi$-bonds, and charges, localized on specific atoms. 
Without loss of rigor, delocalized $\pi$ systems can be represented  as
localized $\pi$ bonds.  The purpose in constructing the mono and di-atomic LMOs
is to have a starting set of LMOs, which can then be used as the basis for a
self-consistent set of LMOs.

The steps involved in constructing the LMOs are as follows:
\begin{enumerate}
\item Hybrid atomic orbitals are constructed for each heavy atom.
\item All sigma bonds are identified.
\item Diatomic $\sigma$ LMOs are constructed.
\item All lone pairs are identified, and monatomic LMOs constructed.
\item All non-cyclic $\pi$ systems are identified, and diatomic $\pi$ LMOs
 bonds constructed.
\item All cyclic $\pi$ systems are identified, and diatomic $\pi$ LMOs
constructed.
\item Any unused atomic orbitals are identified as cationic or anionic centers,
and monatomic LMOs constructed.
\end{enumerate}

Inspection of this sequence shows that the resulting set of LMOs will, in fact,
correspond to the classical line-sketch of the molecular structure.

There are one or two less than obvious details involved in this sequence, which
will now be described.

\subsubsection{Construction of Hybrids}\label{make_hybrid}
\index{Hybrid orbitals!construction of}
\index{Construction of hybrid orbitals}

With the exception of hydrogen, all atoms have an $s-p$ basis set, which must
be mixed to form four hybrid atomic orbitals.  Ideally, the main lobes of the
hybrid orbitals involved in $\sigma$ bonding would point towards the other
atoms involved in those $\sigma$ bonds.  To achieve this, information regarding
the immediate environment of the atom must be used.  As only the 
direction of the neighboring atoms is important, the energy terms relating
to the one-electron integrals between the atom and the $s$ orbital of the
neighboring atoms can be used.

The process of using these energy terms to generate hybrid orbitals involves
two steps.  First, a set of M.O.s which involve the atomic orbitals of an
atom and the $s$ orbitals of its immediate neighbors is constructed by
diagonalizing the interaction matrix.  Consider a carbon atom in 
ethane.  Using the numbering scheme and orientation shown in 
Figure~\ref{c2h6pic}, the complete MNDO one-electron matrix for ethane is 
shown in Table~\ref{c2h6f}.

\begin{figure}
\begin{makeimage}
\end{makeimage}
% \begin{picture}(120,130)
% \setlength{\unitlength}{0.07cm}
% \put(40 ,-20){\includegraphics{c2h6}}
% \put(120,45){
% \begin{picture}(10,10)
% \put(40,-20){\vector(0,1){20}}
% % \put(40,-20){\line(0,1){20}}
% \put(38,2){Y}
% \put(40,-20){\vector(1,0){20}}
% \put(61,-22){X}
% \end{picture}
% }
% \end{picture}
\includegraphics{c2h6}
\caption{\label{c2h6pic} Ethane, Showing Orientation and Numbering System}
\end{figure}

The one-electron matrix is used in the construction of the interaction matrix.
This matrix is of size 4+$n$, where $n$ is the number of neighbors, i.e, $n$ =
1, 2, 3, or 4.  The first four functions are the $s$ and $p$ atomic orbitals
of the atom; the remaining functions are the $s$ orbitals of the neighbors.
The only non-zero terms are those representing the interaction of the 
atom with its neighbors.  Therefore, for the first carbon in ethane, the
interaction matrix would be that shown in Table~\ref{c2h6c1}.

\begin{table}
\caption{\label{c2h6f}One-Electron Integral Matrix for Ethane}
\begin{center}
\begin{tabular}{l|rrrrrrr} \hline
A.O.&C$(1)_s$&C$(1)_{p_x}$&C$(1)_{p_y}$&C$(1)_{p_z}$&C$(2)_s$&C$(2)_{p_x}$&C$(2)_{p_y}$\\
\hline
C$(1)_s$&-126.7891\\
C$(1)_{p_x}$ &  -6.5701 &-114.2822\\
C$(1)_{p_y}$ &   0.0000&  0.0000&-111.8502\\
C$(1)_{p_z}$ &   0.0000&  0.0000&  0.0000&-111.8502\\
C$(2)_s$     &  -5.3977& -4.2826&  0.0000&  0.0000&-126.7891\\
C$(2)_{p_x}$ &   4.2826&  2.4743&  0.0000&  0.0000&  6.5701&-114.2822\\
C$(2)_{p_y}$ &   0.0000&  0.0000& -1.1990&  0.0000&  0.0000&  0.0000 &-111.8502\\C$(2)_{p_z}$ &   0.0000&  0.0000&  0.0000& -1.1990&  0.0000&  0.0000 &  0.0000 \\H$(1)$       &  -1.0116& -0.6430& -0.3460&  0.0000& -5.7834& -1.2013 & -3.0984\\
H$(1)$       &  -1.0116& -0.6430&  0.1730& -0.2996& -5.7834& -1.2013 &  1.5492\\
H$(1)$       &  -1.0116& -0.6430&  0.1730&  0.2996& -5.7834& -1.2013 &  1.5492\\
H$(1)$       &  -5.7834&  1.2013& -1.5492&  2.6833& -1.0116&  0.6430 & -0.1730\\
H$(1)$       &  -5.7834&  1.2013&  3.0984&  0.0000& -1.0116&  0.6430 &  0.3460 \\H$(1)$       &  -5.7834 &  1.2013& -1.5492& -2.6833& -1.0116&  0.6430 & -0.1730\\\rule[-0.0cm]{0cm}{0.6cm}\\
\hline
A.O.&C$(2)_{p_z}$&H$(1)$&H$(2)$&H$(3)$&H$(4)$&H$(5)$&H$(6)$\\
\hline
C$(2)_{p_z}$&-111.8502\\
H$(1)$ &0.0000    & -99.6727\\
H$(1)$ &2.6833    &  -0.9455& -99.6727\\
H$(1)$ &2.6833    &  -0.9455&  -0.9455& -99.6727\\
H$(1)$ &0.2996    &  -0.2444&  -0.0814&  -0.2444& -99.6727\\
H$(1)$ &0.0000    &  -0.0814&  -0.2444&  -0.2444&  -0.9455& -99.6727\\
H$(1)$ &0.2996    &  -0.2444&  -0.2444&  -0.0814&  -0.9455&  -0.9455& -99.6727\\
\hline
\end{tabular}
\end{center}
\end{table}

\begin{table}
\caption{\label{c2h6c1}Interaction Matrix used in Constructing Hybrid Orbitals for C$_1$ in Ethane}
\begin{center}
\begin{tabular}{l|rrrrrrrr} 
\hline
A.O. &   C$(1)_s$ &C$(1)_{p_x}$ & C$(1)_{p_y}$ &  C$(1)_{p_z}$ &  C$(2)_s$ &H$(1)$ &H$(2)$ & H$^3$ \\
\hline
C$(1)_s$ & 0.0000  &  \\
C$_{p_x}$& 0.0000  & 0.0000  &  \\
C$_{p_y}$& 0.0000  & 0.0000  & 0.0000 &      \\
C$_{p_z}$& 0.0000  & 0.0000  & 0.0000 &  0.0000 &   \\
C$(2)_s$ &-5.3977  &-4.2826  & 0.0000 &  0.0000 &  0.0000  \\
H$(1)$   &-5.7834  & 1.2013  &-1.5492 &  2.6833 &  0.0000 &  0.0000  \\
H$(1)$   &-5.7834  & 1.2013  & 3.0984 &  0.0000 &  0.0000 &  0.0000 &  0.0000  \\
H$(1)$   &-5.7834  & 1.2013  &-1.5492 & -2.6833 &  0.0000 &  0.0000 &  0.0000 &  0.0000  \\
\hline
\end{tabular}
\end{center}
\end{table}

\begin{table}
\caption{\label{low4}Eigenvectors of Lowest Energy from Interaction Matrix}
\begin{center}
\begin{tabular}{r|rrrrrrrr}  \hline
$\Psi$ &   C$(1)_s$ &C$(1)_{p_x}$ & C$(1)_{p_y}$ &  C$(1)_{p_z}$ &  
C$(2)_s$ &H$(1)$ &H$(2)$ & H$(3)$ \\
\hline
1&  0.7069& 0.0150& 0.0000& 0.0000& 0.3409& 0.3577& 0.3577& 0.3577 \\
2& -0.0150& 0.7069& 0.0000& 0.0000& 0.6195&-0.1968&-0.1968&-0.1968 \\
3&  0.0000& 0.0000& 0.6112& 0.3556& 0.0000&-0.0019&-0.4990& 0.5009 \\
4&  0.0000& 0.0000&-0.3556& 0.6112& 0.0000&-0.5773& 0.2903& 0.2870 \\
\hline
\end{tabular}
\end{center}
\end{table}

\begin{table}
\caption{\label{low4l}Localized Diatomic Molecular Orbitals for C$_1$ in Ethane}
\begin{center} 
\begin{tabular}{r|rrrrrrrr} \hline
$\Psi$ &   C$(1)_s$ &C$(1)_{p_x}$ & C$(1)_{p_y}$ &  C$(1)_{p_z}$ &  
C$(2)_s$ &H$(1)$ &H$(2)$ & H$(3)$ \\
\hline
1&   0.3277& 0.6266& 0.0000& 0.0000& 0.7071& 0.0000& 0.0000& 0.0000  \\
2&  -0.3617& 0.1892&-0.2887& 0.5000& 0.0000&-0.7071& 0.0000& 0.0000  \\
3&   0.3618&-0.1892& 0.2887& 0.5000& 0.0000& 0.0000& 0.0000& 0.7071  \\
4&   0.3618&-0.1892&-0.5774& 0.0000& 0.0000& 0.0000& 0.7071& 0.0000  \\
\hline
\end{tabular}
\end{center}
\end{table}

\begin{table}
\caption{\label{c1hyb}Hybrid Atomic Orbitals for C$_1$ in Ethane}
\begin{center}  
\begin{tabular}{c|rrrrrrrr}\hline
Hybrid &   C$(1)_s$ &C$(1)_{p_x}$ & C$(1)_{p_y}$ &  C$(1)_{p_z}$  \\
1 & 0.46345 &   0.88612 &   0.00002 &  -0.00004  \\
2 &-0.51159 &   0.26761 &  -0.40825 &   0.70711  \\
3 & 0.51161 &  -0.26756 &   0.40825 &   0.70711  \\
4 & 0.51161 &  -0.26756 &  -0.81650 &   0.00000  \\
\hline
\end{tabular}
\end{center} 
\end{table}

The four eigenvectors of lowest energy (Table~\ref{low4}) are used to form the
hybrid orbitals.  A simple unitary transformation converts these eigenvectors
into localized di-atomic bonds (Table~\ref{low4l}).  This transformation is 
the localization procedure, which is very rapid. Finally, before the hybrid
functions can be used, they must be normalized to unity.  For C$_1$ in ethane,
the final hybrid atomic orbitals are presented in Table~\ref{c1hyb}.

\index{sp$^3$ hybrid atomic orbitals} This simple procedure works not just for
atoms of $sp^3$ hybridization. Consider an $sp^2$ system.  Now, with only three
ligands, the interaction  matrix for the atom will be of order seven.  The
hybrid orbitals are constructed from the lowest four eigenvectors resulting
from diagonalization  of this matrix. As with the $sp^3$ system, the hybrid
orbitals are first localized, then re-normalized.  This gives four hybrid
orbitals, three of which are the $sp^2$ hybrids, with the fourth being a pure
$p$ orbital.

This  procedure is quite general, as can be illustrated by oxygen, first in a
hydroxy and then in a double-bonded oxygen environment.  In a hydroxy
environment an oxygen will have two ligands: so the interaction  matrix will be
of order six.  Two $sp$ hybrids  and two $p$ orbitals will be  created.  Later
on, we will see that the two $p$ orbitals become lone-pairs.

In the case of the double-bonded oxygen, the interaction matrix is of  order
five.  The lowest eigenvector of this matrix corresponds to the oxygen-ligand
$\sigma$-bond, and consists of a mixture of oxygen $s$ and $p$ orbitals.  To
satisfy orthogonality requirements, two other $s-p$ hybrids are also made, so
the overall hybridization is three $sp^2$ orbitals plus one  pure $p$ orbital.
As with the hydroxy oxygen, two of these orbitals will eventually be used to 
create lone pairs. 

When there is only one or two atoms bonded to an atom, the non-bonding hybrid
orbitals generated are ill-defined by a unitary transform.  This ill-definition
is unimportant in that at self-consistency the density matrix is independent of
the starting LMOs.  It is important, however, in that the starting LMOs are
very sensitive to minute numerical differences, of the type encountered on
porting the program to different platforms.

To prevent this, the ill-definition is removed by adding an extra constraint,
subroutine \comp{MINLOC}. If the atom has only one ligand, oxygen in carbonyl,
for example, then the three non-bonding degenerate hybrid orbitals are mixed so
that the $p_x$ and $p_y$ orbitals on one hybrid have zero coefficients, and the
$p_x$ orbital on another hybrid also has a zero coefficient.

If the atom has two ligands, then the two non-bonding degenerate hybrids are
mixed so that $p_x$ orbital on one of the hybrids has a zero coefficient.

Hybrid atomic orbitals form an orthonormal set.  This is a natural consequence
of the fact that the hybrids are obtained by a unitary transform of the
starting orthonormal atomic orbitals.  

\subsubsection{Construction of Starting Localized Molecular Orbitals}
In order to illustrate the procedure for generating starting LMOs, we will use
benzaldehyde as the test example.  The line-sketch of benzaldehyde is shown in
Figure~\ref{c6h5cho}.  Before starting the procedure, let us spend a little
time examining the sketch.  From the empiric formula of  benzaldehyde,
C$_7$H$_6$O, we see that there are ($7\times4+6\times1+1\times6$) = 40 valence 
electrons, and ($7\times4+6\times1+1\times4$) = 38 valence orbitals.  Since
each  diatomic bond involves two electrons and two orbitals, and each lone-pair
involves two electrons and one orbital, we see that there must be 18 diatomic
bonds and two lone pairs. The diatomic bonds can be divided into three sets: 14
$\sigma$ bonds, one  localized $\pi$ bond and three $\pi$ bonds in the aromatic
ring. Every atom contributes both atomic orbitals and electrons to the system. 
These contributions are shown in Table~\ref{contrib}.

\begin{table}
\caption{\label{contrib} Atomic Orbital and Electron Contributions in
Benzaldehyde}
\begin{center}
\begin{tabular}{lcccccccccccccc} \hline
Quantity & C$_1$ & C$_2$ & C$_3$ & C$_4$ & C$_5$ & C$_6$ & H$_2$ & H$_3$ & 
H$_4$ & H$_5$ & H$_6$ & C$_7$ & O$_1$ & H$_7$  \\ \hline
Orbitals  & 4 &  4 & 4 & 4 & 4 & 4 & 1 & 1 &  1 &  1 &  1 &  4 & 4 & 1  \\
Electrons & 4 &  4 & 4 & 4 & 4 & 4 & 1 & 1 &  1 &  1 &  1 &  4 & 6 & 1  \\
\hline
\end{tabular}
\end{center}
\end{table}

\begin{figure}
\begin{makeimage}
\end{makeimage}
% \begin{center}
\begin{picture}(120,150)( -60,-40)
\setlength{\unitlength}{0.03cm}
\thicklines
 \put(00, 00){\line(1,1){20}}
 \put(20, 20){\line(0,1){20}}
 \put(20, 40){\line(-1, 1){20}}
 \put(00, 60){\line(-1,-1){20}}
 \put(-20, 40){\line(0,-1){20}}
 \put(-20, 20){\line( 1,-1){20}}
 \put( 00, 60){\line( 0, 1){18}}
 \put(-04, 86){\line(-1, 1){12}}  %  to H
 \put( 07, 91){\line( 1, 1){12}}  %  to O
 \put( 09, 88){\line( 1, 1){12}}  %  to O
 \put(-07, 80){ C} \put( 05,79){$_7$}
 \put( 17,102){ O} \put( 32,101){$_1$}
 \put( 31,104){\circle*{2}}
 \put( 31,108){\circle*{2}}
 \put( 26,112){\circle*{2}}
 \put( 22,112){\circle*{2}}
 \put(-27,100){ H} \put(-15,99){$_7$}
 \put(00, 05){\line(1,1){16}}
 \put(16, 40){\line(-1,1){15}}
 \put(-17,20){\line(0,1){19}}
 \put(00, 00){\line(0,-1){13}}
 \put(20, 20){\line(1,-1){14}}
 \put(20, 40){\line(1, 1){14}}
 \put(-20, 40){\line(-1,1){12}}
 \put(-20, 20){\line(-1,-1){12}}
 \put(-8,-23){ H} \put( 04,-24){$_4$}
 \put(32,00){ H} \put(44,-01 ){$_3$}
 \put(32,55){ H} \put(44,54  ){$_2$}
 \put(-45,55){ H} \put(-33,54 ){$_6$}
 \put(-45,00){ H} \put(-33,-01){$_5$}
\put(100, 20){\begin{minipage}[b]{3.0in}
\begin{center}
Benzaldehyde - Electronic Structure

\begin{tabular}{lr} \\  \hline
Number of Atomic Orbitals & 38 \\
Number of Electrons       & 40 \\
Number of Diatomic Bonds  & 18 \\
Number of Lone Pairs      & 2  \\ \hline
\end{tabular}
\end{center}
\end{minipage}}
\end{picture}
\end{center}

\begin{center}
\includegraphics{l_c6h5cho}
\end{center}
\caption{\label{c6h5cho} Electronic Structure of Benzaldehyde}
\end{figure}

\subsubsection{Construction of Sigma Framework}
\index{Construction of sigma orbitals}
\index{Sigma Orbitals!construction of}
Because it is the simplest set of bonds to identify,  the $\sigma$ framework 
is constructed first.  $\sigma$ bonds are defined as existing between every
pair of atoms which are chemically bonded together.  In our example, there are
14 $\sigma$ bonds.  Each di-atomic $\sigma$ bond is constructed by identifying
which pair of unused hybrid orbitals on the atoms involved has the largest
interaction energy.  When one of the atoms is a hydrogen, only  the hybrid
orbitals on the other atom need to be examined at this point.

Once the appropriate orbitals have been identified, bonding and antibonding
diatomic LMOs are constructed.   This is accomplished by diagonalizing the
associated two by two secular matrix.  The elements of this matrix are shown in
Figure~\ref{2by2}.
\begin{figure}
\begin{makeimage}
\end{makeimage}
\caption{\label{2by2}Secular Equation used in Constructing Starting LMOs}
\[
C \begin{array}{cc}
<\mathit{Hybrid}_1|H|\mathit{Hybrid}_1>-E_{1,1} & \mathit{Hybrid}_1|H|\mathit{Hybrid}_2> \\
<\mathit{Hybrid}_2|H|\mathit{Hybrid}_1> & \mathit{Hybrid}_2|H|\mathit{Hybrid}_2> -E_{2,2}
\end{array} C^{t} =0
\]
\end{figure}

For every $\sigma$ bond formed, two electrons are used, one from each of the
atoms involved in forming the bond, and two atomic orbitals are used - again,
one from each atom.  After all 14 $\sigma$ bonds are formed, the number of
still unused atomic orbitals and electrons is much reduced.  The remaining
unused atomic orbitals and electrons are shown in Table~\ref{after-sig}. The
oxygen forms only one $\sigma$ bond, so  there are still three unused hybrid
orbitals and five valence electrons left.

\begin{table}
\caption{\label{after-sig} Unused Electrons and Orbitals after Construction of
$\sigma$ Framework}
\begin{tabular}{lcccccccccccccc}\\ \hline
Quantity & C$_1$ & C$_2$ & C$_3$ & C$_4$ & C$_5$ & C$_6$ & H$_2$ & H$_3$ &
H$_4$ & H$_5$ & H$_6$ & C$_7$ & O$_1$ & H$_7$  \\ \hline
Orbitals  & 1 &  1 & 1 & 1 & 1 & 1 & 0 & 0 &  0 &  0 &  0 &  1 & 3 & 0  \\
Electrons & 1 &  1 & 1 & 1 & 1 & 1 & 0 & 0 &  0 &  0 &  0 &  1 & 5 & 0  \\
\hline
\end{tabular}
\end{table}

\subsubsection{Identification of Lone Pairs}
\index{Lone pairs!construction of}
\index{Construction of lone pairs}
After the $\sigma$ framework has been constructed, all lone pairs are
identified.  They could not be identified earlier, because sometimes a lone
pair may be used up in making the $\sigma$ framwork.  An example of this would
be the lone pair on an -\"{N}H$_2$ group, which is used up in forming a bond
when a proton is added to form a -NH$_3^+$ ion.

The maximum number of lone pairs on each atom is given by the difference 
between the number of electrons and the number of orbitals on an atom.   Since
a lone pair consists of two electrons in one atomic orbital, the remaining
atomic orbitals are used up, one at a time, until either all the atomic
orbitals on the atom are accounted for or all the potential lone pairs have
been formed.

In benzaldehyde, only the oxygen atom has the potential to form lone pairs:
five electrons and three atomic orbitals imply a maximum of two lone pairs. The
fact that there are three available atomic orbitals means that both lone pairs
can in fact be constructed.  This uses up two atomic orbitals and four
electrons, to leave the remaining orbital and electron as shown in
Table~\ref{after-lp}.

\begin{table}
\caption{\label{after-lp} Unused Electrons and Orbitals after Construction of
Lone Pairs}
\begin{center}
\begin{tabular}{lcccccccccccccc} \hline
Quantity & C$_1$ & C$_2$ & C$_3$ & C$_4$ & C$_5$ & C$_6$ & H$_2$ & H$_3$ &
H$_4$ & H$_5$ & H$_6$ & C$_7$ & O$_1$ & H$_7$  \\ \hline
Orbitals  & 1 &  1 & 1 & 1 & 1 & 1 & 0 & 0 &  0 &  0 &  0 &  1 & 1 & 0  \\
Electrons & 1 &  1 & 1 & 1 & 1 & 1 & 0 & 0 &  0 &  0 &  0 &  1 & 1 & 0  \\
\hline
\end{tabular}
\end{center}
\end{table}
When a lone pair is constructed, an occupied LMO is formed, but, unlike the
$\sigma$ bonds, no unoccupied LMO is formed at the same time.

\subsubsection{Construction of Non-Cyclic $\pi$ Bonds}
\index{Construction of acyclic $\pi$ bonds}
\index{pi@$\pi$ bonds, acyclic!construction of}
A potential complication exists when the $\pi$ bonds are constructed, in that
it is possible to start constructing $\pi$ bonds in such a way that a
di-radical is formed.  For example, in butadiene, if a $\pi$ bond is
constructed between carbon atoms 2 and 3, then radical sites would be formed at
atoms 1 and 4.  To avoid this, the first set of $\pi$ bonds constructed is that
set in which the way the $\pi$ bond is drawn is unambiguous.  The rule for
assigning  $\pi$ bonds is simple: a $\pi$ bond is constructed if an atom can
make  a $\pi$ bond with exactly one other atom.  

In benzaldehyde, this situation exists only for the carbon and the oxygen of
the -CHO group.  The $\pi$ bonds in the ring cannot be assigned using this
rule, because {\em every} atom in the ring can potentially make a $\pi$ bond
with {\em two} other atoms.  In more complicated systems, repeated application 
of the rule automatically results in the correct assignment of $\pi$ bonds  in
all non-cycle conjugated systems.

As with the $\sigma$ bonds, when a $\pi$ bond is made, two electrons and two
atomic orbitals are used up, and an occupied and an unoccupied LMO are formed. 
After the non-cyclic $\pi$ bonds are formed, the number of available orbitals
and electrons is as shown in Table~\ref{after-pi}.

\begin{table}
\caption{\label{after-pi} Unused Electrons and Orbitals after Construction of 
Non-Cyclic $\pi$ Bonds}
\begin{center}
\begin{tabular}{lcccccccccccccc} \hline
Quantity & C$_1$ & C$_2$ & C$_3$ & C$_4$ & C$_5$ & C$_6$ & H$_2$ & H$_3$ & 
H$_4$ & H$_5$ & H$_6$ & C$_7$ & O$_1$ & H$_6$  \\ \hline 
Orbitals  & 1 &  1 & 1 & 1 & 1 & 1 & 0 & 0 &  0 &  0 &  0 &  0 & 0 & 0  \\ 
Electrons & 1 &  1 & 1 & 1 & 1 & 1 & 0 & 0 &  0 &  0 &  0 &  0 & 0 & 0  \\ 
\hline 
\end{tabular} 
\end{center}
\end{table} 

\subsubsection{Construction of Cyclic $\pi$ Bonds}
\index{Construction of cyclic $\pi$ bonds}
\index{pi@$\pi$ bonds, cyclic!construction of}
At this point, the only remaining bonds are those in cyclic $\pi$ systems. The
sequence in which these are assigned is similar to that for the non-cyclic
$\pi$ bonds.  One of the remaining $\pi$ bonds is assigned arbitrarily, then
the rule for assigning non-cyclic $\pi$ bonds is used until no more $\pi$ bonds
can be made.  This two-step sequence is repeated until all cyclic $\pi$ systems
have been accounted for.

In the case of benzaldehyde, this completes the construction of the occupied
and virtual starting LMOs.  Two more structures are possible in enzymes,
however.  These will be considered next.

\subsubsection{Identification of Ions} 
\index{Ions!identification of}
\index{Identification of ions}

For un-ionized systems, all the atomic orbitals on all atoms are  accounted for
at this point.  For ionized systems, however, one or more atomic orbitals will
still be unused.  At most, there would be one unused atomic orbital on any
given atom.  The only problem left is to decide whether to assign  these unused
orbitals as belonging to the  occupied or unoccupied sets of LMOs.

In proteins, charged sites occur in only a limited number of environments.
Table~\ref{ions} lists the more common species.
\begin{table}
\index{Charged sites in proteins}
\index{Proteins!charged sites in}
\caption{\label{ions} Charged Sites in Proteins}
\begin{center}
\begin{tabular}{ccc} \hline
Neutral  &   Ion      &   Charge \\ \hline
-COOH    &  -COO      &$-$ \\
-NH$_2$  &  -NH$_3$   &$+$ \\
=NH      &  =NH$_2$   &$+$ \\
-SH      &  -S        &$-$ \\
\hline
\end{tabular}
\end{center}
\end{table}
The following general rules can be derived from this Table:
\begin{enumerate}
\item If the immediate environment has an oxygen or a sulfur atom, the
unused atomic orbital is assigned to the occupied set.
\item If the immediate environment has a nitrogen atom,  the
unused atomic orbital is assigned to the unoccupied set. 
\end{enumerate}
In other words, all ions associated with oxygen or sulfur are anions, and
all ions associated with nitrogen are cations.  

As with the lone pairs, LMOs representing ionized sites involve only one
orbital, an occupied LMO in the case of an anion, and an unoccupied LMO for a
cation.

If the calculated number of cationic and anionic sites is different, then the
entire system will have a net charge.  The numerical value of that  charge is
given by subtracting the number of anionic sites from the number of cationic
sites.

\subsubsection{Properties of Starting LMOs}
\index{LMOs!properties of starting}
The starting LMOs have two properties which are essential for the SCF
calculation.  First, since they are constructed from atomic or hybrid atomic
orbitals, either directly or as a linear combination of two hybrids, they form
an orthogonal set.  Because of the way bonding and antibonding  LMOs are
constructed, all occupied LMOs are orthogonal to all unoccupied LMOs. The
second essential property is that every LMO is normalized to unity.

Both properties are essential for the SCF calculation, in that their existence
means that any unitary transform involving any two LMOs (usually one occupied
and one unoccupied LMO) will preserve the orthonormality of the entire set of
occupied plus virtual LMOs.

\subsection{Other Considerations}

The first time a large system is run, there is a high probability that  errors
will exist in the data-set which would make the full calculation useless.  To
save time, a check is run on the system to determine the number of each type of
LMO.    If an atom bonds to more than four neighbors, then the $\sigma$
framework cannot be made, and the run is stopped with a warning message.  

If the calculated net charge is different from the charge specified on the
keyword line, then the assigned number of occupied LMOs will not match the
calculated number of occupied LMOs, i.e., the number of  $\sigma$ and $\pi$
LMOS plus the lone pairs plus the anions. If that happens,  then the job is
stopped, and a warning message printed out.

\subsubsection{Re-Orthogonalization of LMOs}\label{reorth}
During the course of a long calculation, the LMOs will become increasingly
non-orthogonal.  Subroutine \comp{CHECK} ensures that the LMOs remain
normalized; this is a very rapid calculation.  Ensuring that the LMOs are all
orthogonal is not so simple.  Re-orthogonalizing the LMOs is a lengthy
calculation, and is not routinely performed.  However, the option exists to
re-orthogonalize the LMOs, and this can be done either during a \comp{1SCF}
calculation (the preferred way), or periodically during a geometry
optimization.

Re-orthogonalizing consists of taking pairs of LMOs, $\psi_i$ and $\psi_j$, and
forming linear combinations such that the overlap ($<\!\psi_i|\psi_j\!>$) is
zero.  The LMOs involved form the full set, that is, both occupied  plus
virtual sets are used.

The two LMOs can be regarded as unit vectors that are almost at 90$^{\circ}$ to
each other.  Let the difference from  90$^{\circ}$ be $\alpha$.  If the vectors
are each rotated by -$\frac{1}{2}\alpha$, then they will become perfectly
orthogonal. This operation is most conveniently performed using perturbation
theory. Let:
$$
<\!\psi_i|\psi_i\!> \simeq 1 \simeq <\!\psi_j|\psi_j\!>
$$
and 
$$
|<\!\psi_i|\psi_j\!>| = S_{ij} \ll 1,
$$
then
$$
\psi_i' = \psi_i - \frac{1}{2}S_{ij}\psi_j
$$
and 
$$
\psi_j' = \psi_j - \frac{1}{2}S_{ij}\psi_i.
$$

That the new LMOs are orthogonal can readily be demonstrated:
\begin{eqnarray}
<\!\psi_i'|\psi_j'\!> &=&<(\psi_i - \frac{1}{2}S_{ij}\psi_j)|
(\psi_j - \frac{1}{2}S_{ij}\psi_i)\!> \nonumber \\
 &=&<\psi_i|\psi_j>-\frac{1}{2}S_{ij}<\!\psi_i|\psi_i>-
\frac{1}{2}S_{ij}<\!\psi_j|\psi_j\!>+\frac{1}{4}S_{ij}^2<\!\psi_j|\psi_i\!> 
\nonumber \\
&=& S_{ij}-\frac{1}{2}S_{ij}-\frac{1}{2}S_{ij} +\frac{1}{4}S_{ij}^3 \nonumber \\
&=& 0 \nonumber
\end{eqnarray}

The calculation of the overlaps, $S_{ij}$, is most conveniently done for one
LMO, $\phi_i$, with all other LMOs.  Because of this, $\phi_i$ should not be
modified while the re-orthogonalization is done.  In order to avoid modifying
$\phi_i$, the rotation is changed so that $\phi_i$ remains stationary and all
the rotation is incurred by $\phi_j$, thus:
$$
\psi_i' = \psi_i 
$$
and
$$
\psi_j' = \psi_j - S_{ij}\psi_i.
$$

Before the re-orthogonalization, the LMOs are almost orthogonal, and the use of
perturbation theory here is fully justified.


\subsubsection{Localized Molecular Orbitals:  Storage Considerations}
\index{LMOs!storage considerations}
When they are first made, the LMOs are highly compact functions.  Each LMO
consists of one or two atom labels and a set of four to eighteen atomic orbital
coefficients.  However, as soon as the LMOs start to interact,  mixing of the
LMOs will occur, and the size of the LMOs will increase rapidly.  This
expansion must be allowed for.  Later on, further mixing of the LMOs may result
in the contribution to an atom to a certain LMO becoming negligible, in which
case the atom can be deleted from the LMO.    This results in the LMO becoming
smaller.  A mechanism must be constructed to allow the size of the LMOs  to be
changed.  Because the occupied and unoccupied LMO sets are treated in a similar
manner, only the storage of the occupied set will be described. Therefore, for
the rest of this section, all reference to LMOs should be understood as
applying to the occupied  set only.

\label{starting-addresses}

Before going further, we need to spend a little time to understand how LMOs are
stored.   The atomic orbital coefficients of all LMOs are stored in  a single
REAL array, \comp{COCC}.  A second array, \comp{NCOCC}, holds the starting 
location in \comp{COCC} of each LMO.   For convenience, all starting locations
are defined as the location immediately {\em before} the  location of the first
element of an array.  The atom numbers (not the atomic numbers)  used in all
LMOs are stored in an INTEGER array, \comp{ICOCC}, and a second INTEGER  array,
\comp{NNCF}, holds the starting location in \comp{ICOCC} of  each LMO.  
Finally, the  number of atoms in each LMO are stored in an array \comp{NCF}.  
Figure~\ref{store-slmo} illustrates these arrays for methane.

\begin{figure}
\begin{makeimage}
\end{makeimage}
\begin{center}
\setlength{\unitlength}{0.1in}
\begin{picture}(50,30)(0,-24)
 \put(00, 02){\makebox(2.7,2){1}\makebox(2.7,2){2}\makebox(2.7,2){3}
\makebox(2.7,2){4}\makebox(2.7,2){5}\makebox(2.7,2){6}\makebox(2.7,2){7}
\makebox(2.7,2){8}\makebox(2.7,2){9}\makebox(2.7,2){10}\makebox(2.7,2){11}
\makebox(2.7,2){12}\makebox(2.7,2){13}\makebox(2.7,2){14}\makebox(2.7,2){15}
\makebox(2.7,2){16}\makebox(2.7,2){17}\makebox(2.7,2){18}\makebox(2.7,2){19}
\makebox(2.7,2){20}}
 \put(00, 00){\framebox(14,2){$H_1,C_s,C_{px},C_{py},C_{pz}$}}
 \put(14, 00){\framebox(14,2){$C_s,C_{px},C_{py},C_{pz},H_2$}}
 \put(28, 00){\framebox(14,2){$C_s,C_{px},C_{py},C_{pz},H_3$}}
 \put(42, 00){\framebox(14,2){$C_s,C_{px},C_{py},C_{pz},H_4$}}
\put(-6,00){\makebox(4,2){COCC:}}
%
 \put(00,-04){\makebox(2.7,2){1}\makebox(2.7,2){2}\makebox(2.7,2){3}\makebox(2.7,2){4}}
\put(-6,-6){\makebox(4,2){NCOCC:}}
\put(00,-6){\framebox(2.7,2){0}\framebox(2.7,2){5}\framebox(2.7,2){10}\framebox(2.7,2){15} }
%
 \put(00,-10){\makebox(2.7,2){1}\makebox(2.7,2){2}\makebox(2.7,2){3}
\makebox(2.7,2){4}\makebox(2.7,2){5}\makebox(2.7,2){6}\makebox(2.7,2){7}
\makebox(2.7,2){8}}
\put(-6,-12){\makebox(4,2){ICOCC:}}
\put(00,-12){\framebox(2.7,2){1}\framebox(2.7,2){2}\framebox(2.7,2){2}\framebox(2.7,2){3}\framebox(2.7,2){2}\framebox(2.7,2){4}\framebox(2.7,2){2}\framebox(2.7,2){5}}
%
 \put(00,-16){\makebox(2.7,2){1}\makebox(2.7,2){2}\makebox(2.7,2){3}
\makebox(2.7,2){4}}
\put(-6,-18){\makebox(4,2){NNCF:}}
\put(00,-18){\framebox(2.7,2){0}\framebox(2.7,2){2}\framebox(2.7,2){4}\framebox(2.7,2){6} }
%
 \put(00, 02){\makebox(2.7,2){1}\makebox(2.7,2){2}\makebox(2.7,2){3}
\makebox(2.7,2){4}}
\put(-6,-24){\makebox(4,2){NCF:}}
\put(00,-24){\framebox(2.7,2){2}\framebox(2.7,2){2}\framebox(2.7,2){2}\framebox(2.7,2){2} }
   \end{picture}
\end{center}
\caption{\label{store-slmo}Storage of Starting LMOs in Methane}
\end{figure}

During the SCF calculation, the number of atoms in each LMO will increase.   To
allow for this, some space must be left between the end of one  LMO and the
start of the next LMO. The amount of space is determined by dividing all the
unused space by the number of LMOs.  If the size of the vector \comp{COCC} is
40, and  the size of \comp{ICOCC} is 25, then the storage of the LMO vectors,
after adjusting to allow for expansion, is as shown in
Figure~\ref{store-slmoe}.

\begin{figure}
\begin{makeimage}
\end{makeimage}
\begin{center}
\setlength{\unitlength}{0.1in}
\begin{picture}(50,30)(0,-24)
\put(00, 02){\makebox(2.7,2){1}\makebox(2.7,2){2}\makebox(2.7,2){3}
\makebox(2.7,2){4}\makebox(2.7,2){5}\makebox(2.7,2){6}\makebox(2.7,2){7}
\makebox(2.7,2){8}\makebox(2.7,2){9}\makebox(2.7,2){10}\makebox(2.7,2){11}
\makebox(2.7,2){12}\makebox(2.7,2){13}\makebox(2.7,2){14}\makebox(2.7,2){15}
\makebox(2.7,2){16}\makebox(2.7,2){17}\makebox(2.7,2){18}\makebox(2.7,2){19}
\makebox(2.7,2){20}}
\put(00, 00){\framebox(28,2){$H_1,\ C_s,C_{px},C_{py},C_{pz}$}}
\put(28, 00){\framebox(28,2){$C_s,C_{px},C_{py},C_{pz},\ H_2$}}
\put(56, 00){\makebox ( 4,2){\ldots.}}
\put(-6,00){\makebox(4,2){COCC:}}
%
\put(00,-04){\makebox(2.7,2){1}\makebox(2.7,2){2}\makebox(2.7,2){3}\makebox(2.7,2){4}}
\put(-6,-6){\makebox(4,2){NCOCC:}}
\put(00,-6){\framebox(2.7,2){0}\framebox(2.7,2){10}\framebox(2.7,2){20}\framebox(2.7,2){30} }
%
\put(00,-10){\makebox(2.7,2){1}\makebox(2.7,2){2}\makebox(2.7,2){3}\makebox(2.7,2){4}\makebox(2.7,2){5}\makebox(2.7,2){6}\makebox(2.7,2){7}\makebox(2.7,2){8}\makebox(2.7,2){9}\makebox(2.7,2){10}\makebox(2.7,2){11}\makebox(2.7,2){12}\makebox(2.7,2){13}\makebox(2.7,2){14}\makebox(2.7,2){15}\makebox(2.7,2){16}\makebox(2.7,2){17}\makebox(2.7,2){18}\makebox(2.7,2){19}\makebox(2.7,2){20}\makebox(2.7,2){21}\makebox(2.7,2){22}}
\put(-6,-12){\makebox(4,2){ICOCC:}}
\put(00,-12){\framebox(2.7,2){1}\framebox(2.7,2){2}\framebox(2.7,2){ }\framebox(2.7,2){ }\framebox(2.7,2){ }\framebox(2.7,2){ }\framebox(2.7,2){2}\framebox(2.7,2){3}\framebox(2.7,2){ }\framebox(2.7,2){ }\framebox(2.7,2){ }\framebox(2.7,2){ }\framebox(2.7,2){2}\framebox(2.7,2){4}\framebox(2.7,2){ }\framebox(2.7,2){ }\framebox(2.7,2){ }\framebox(2.7,2){ }\framebox(2.7,2){2}\framebox(2.7,2){5}\framebox(2.7,2){ }\framebox(2.7,2){ }\makebox(2.7,2){\ldots}}
%
\put(00,-16){\makebox(2.7,2){1}\makebox(2.7,2){2}\makebox(2.7,2){3}\makebox(2.7,2){4}}
\put(-6,-18){\makebox(4,2){NNCF:}}
\put(00,-18){\framebox(2.7,2){0}\framebox(2.7,2){6}\framebox(2.7,2){12}\framebox(2.7,2){18}}
%
\put(00,-22){\makebox(2.7,2){1}\makebox(2.7,2){2}\makebox(2.7,2){3}\makebox(2.7,2){4}}
\put(-6,-24){\makebox(4,2){NCF:}}
\put(00,-24){\framebox(2.7,2){2}\framebox(2.7,2){2}\framebox(2.7,2){2}\framebox(2.7,2){2} }
\end{picture}
\end{center}
\caption{\label{store-slmoe}Storage of Starting LMOs in Methane, Showing Space Available for Expansion}
\end{figure}

After a few iterations, the number of atoms in each LMO will have increased. In
general, this number will be less than the number of atoms in the system, but
in the example we are using, every LMO will contain all the atoms.  This is
shown in Figure~\ref{store-slmoes}.  For large systems, the number of atoms in
any one LMO will vary.  Very large LMOs may have 300 or more atoms, while small
LMOs may have only 25 atoms.  On average, LMOs will have about 150--200 atoms.

For every array used in describing the occupied LMOs, there exists an 
equivalent  array for the virtual LMOs.  Table~\ref{arrays} gives the names  of
the  arrays for the virtual LMOs.

\begin{figure}
\begin{makeimage}
\end{makeimage}
\setlength{\unitlength}{0.1in}
\begin{center}
\begin{picture}(50,30)(0,-24)
 \put(00, 02){\makebox(2.7,2){1}\makebox(2.7,2){2}\makebox(2.7,2){3}
\makebox(2.7,2){4}\makebox(2.7,2){5}\makebox(2.7,2){6}\makebox(2.7,2){7}
\makebox(2.7,2){8}\makebox(2.7,2){9}\makebox(2.7,2){10}\makebox(2.7,2){11}
\makebox(2.7,2){12}\makebox(2.7,2){13}\makebox(2.7,2){14}\makebox(2.7,2){15}
\makebox(2.7,2){16}\makebox(2.7,2){17}\makebox(2.7,2){18}\makebox(2.7,2){19}
\makebox(2.7,2){20}}
 \put(00, 00){\framebox(28,2){$H_1,\ C_s,C_{px},C_{py},C_{pz},\ H_2,\ H_3,\ H_4$}}
 \put(28, 00){\framebox(28,2){$C_s,C_{px},C_{py},C_{pz},\ H_2,\ H_1,\ H_3,\ H_4$}}
 \put(56, 00){\makebox ( 4,2){\ldots.}}
\put(-6,00){\makebox(4,2){COCC:}}
%
 \put(00,-04){\makebox(2.7,2){1}\makebox(2.7,2){2}\makebox(2.7,2){3}\makebox(2.7,2){4}}
\put(-6,-6){\makebox(4,2){NCOCC:}}
\put(00,-6){\framebox(2.7,2){0}\framebox(2.7,2){10}\framebox(2.7,2){20}\framebox(2.7,2){30} }
%
 \put(00,-10){\makebox(2.7,2){1}\makebox(2.7,2){2}\makebox(2.7,2){3}\makebox(2.7,2){4}\makebox(2.7,2){5}\makebox(2.7,2){6}\makebox(2.7,2){7}\makebox(2.7,2){8}\makebox(2.7,2){9}\makebox(2.7,2){10}\makebox(2.7,2){11}\makebox(2.7,2){12}\makebox(2.7,2){13}\makebox(2.7,2){14}\makebox(2.7,2){15}\makebox(2.7,2){16}\makebox(2.7,2){17}\makebox(2.7,2){18}\makebox(2.7,2){19}\makebox(2.7,2){20}\makebox(2.7,2){21}\makebox(2.7,2){22}}
\put(-6,-12){\makebox(4,2){ICOCC:}}
\put(00,-12){\framebox(2.7,2){1}\framebox(2.7,2){2}\framebox(2.7,2){3}\framebox(2.7,2){4}\framebox(2.7,2){5}\framebox(2.7,2){ }\framebox(2.7,2){2}\framebox(2.7,2){3}\framebox(2.7,2){1}\framebox(2.7,2){4}\framebox(2.7,2){5}\framebox(2.7,2){ }\framebox(2.7,2){2}\framebox(2.7,2){4}\framebox(2.7,2){1}\framebox(2.7,2){3}\framebox(2.7,2){5}\framebox(2.7,2){ }\framebox(2.7,2){2}\framebox(2.7,2){5}\framebox(2.7,2){1}\framebox(2.7,2){3}\makebox(2.7,2){\ldots}}
%
 \put(00,-16){\makebox(2.7,2){1}\makebox(2.7,2){2}\makebox(2.7,2){3}\makebox(2.7,2){4}}
\put(-6,-18){\makebox(4,2){NNCF:}}
\put(00,-18){\framebox(2.7,2){0}\framebox(2.7,2){6}\framebox(2.7,2){12}\framebox(2.7,2){18}}
%
 \put(00,-22){\makebox(2.7,2){1}\makebox(2.7,2){2}\makebox(2.7,2){3}\makebox(2.7,2){4}}
\put(-6,-24){\makebox(4,2){NCF:}}
\put(00,-24){\framebox(2.7,2){5}\framebox(2.7,2){5}\framebox(2.7,2){5}\framebox(2.7,2){5} }
\end{picture}
\end{center}
\caption{\label{store-slmoes}Storage of Iterating LMOs in Methane}
\end{figure}

\begin{table}
\caption{\label{arrays} Names of Arrays Used to Represent LMOs}
\begin{center}
\begin{tabular}{lll} 
\hline
Occupied      &  Unoccupied      &  Contents \\ \hline
NCF           &  NCE             &  Number of Atoms in LMO \\
NNCF          &  NNCE            &  Starting Address of Atoms\\
ICOCC         &  ICVIR           &  Atom Numbers\\
NCOCC         &  NCVIR           &  Starting Address of LMO \\
COCC          &  CVIR            &  LMO Atomic Orbital Coefficients \\ \hline
\end{tabular}
\end{center}
\end{table}

\subsection{Diagonalization}
\index{Diagonalization}
At self-consistency, all matrix elements connecting every occupied LMO with
every unoccupied LMO is zero.  Annihilation of these matrix elements is
performed by the subroutine DIAGG.  The only matrix elements which need to be
considered are those which involve an occupied LMO with an  unoccupied LMO
which shares one or more common atoms.  With increasing size of system, the
number of interactions of this type compared with the product of the number of
occupied times unoccupied LMOs  becomes increasingly small.

Although DIAGG looks quite formidable, the working of this subroutine can be
easily understood in terms of the principles involved in matrix element
annihilation.  Therefore, before going into the details of the LMO matrix
annihilation, the conventional procedure used in MOPAC will be reviewed.

\subsubsection{Conventional Matrix Annihilation}
The raw material DIAGG is supplied with consists of the occupied LMOs and the
Fock matrix over atomic orbitals.  The first step, therefore, is to calculate
the occupied and virtual energy levels, and the potentially non-zero matrix
elements connecting the occupied and virtual sets.

In a system of $N$ atoms, let a molecular orbital, $\psi_i$, be represented by
\begin{equation}
\psi_i = \sum_{A=1}^N\sum_{\lambda\epsilon A}c_{\lambda i}.
\end{equation}
Then the energy of a M.O.\ can be evaluated from
\begin{equation}
\varepsilon_{ii}=\sum_A^N\sum_B^N\sum_{\lambda\epsilon A}\sum_{\sigma\epsilon B}c_{\lambda i}F_{\lambda\sigma}c_{\sigma i}
\end{equation}
and the matrix element representing the energy term between occupied M.O.\
$\psi_i$ and virtual M.O.\ $\psi_j$ would be 
\begin{equation}
\varepsilon_{i j}=\sum_A^N\sum_B^N\sum_{\lambda\epsilon A}\sum_{\sigma\epsilon B}c_{\lambda  i}F_{\lambda \sigma}c_{\sigma  j}.
\end{equation}

Annihilation of a matrix element  is achieved by performing a unitary 
transform on the M.O.s involved 
\begin{equation}
\begin{array}{llrr}
\psi_i^{'}  & = & \alpha\psi_i & + \beta\psi_j \\
\psi_j^{'}  & = &-\beta\psi_i  & +\alpha\psi_j 
\end{array}
\end{equation}
where the rotation angles $\alpha$ and $\beta$ are calculated from the molecular
orbital energy matrix elements
\begin{equation}
\begin{array}{lll}
\alpha &=& \sqrt{\frac{1}{2}(1+D/\sqrt{4\varepsilon_{i j}^2+D^2}} \\
\beta  &=& \phi\sqrt{1-\alpha^2} 
\end{array}
\end{equation}
where $D= \varepsilon_{j j}-\varepsilon_{i i}$ and $\phi$ = 1 if 
$\varepsilon_{i j}$ is negative, $\phi =-1$, otherwise.

By evaluating partial sums, the calculation of  $\varepsilon_{i j}$ can 
be made more efficient. If $\varepsilon_{i j}$ is re-written as
\begin{equation}
\varepsilon_{i j} = \sum_A^N\sum_B^N\sum_{\lambda\epsilon A}c_{\lambda  i}\sum_{\sigma\epsilon B}F_{\lambda \sigma}c_{\sigma  j};
\end{equation}
then the partial sum
\begin{equation}
F_{\lambda}^{'}(j) = \sum_{\sigma\epsilon B}F_{\lambda \sigma}c_{\sigma  j}
\end{equation}
can be used to simplify $\varepsilon_{i j}$:
\begin{equation}
\varepsilon_{i j} = 
\sum_A^N\sum_{\lambda\epsilon A}c_{\lambda  i}F_{\lambda}^{'}(j).
\end{equation}

Because the vector $F^{'}(j)$ is evaluated once, and then used for all $i$, the
calculation of the $\varepsilon_{i j}$ is changed from an $n^3$ process to a
$n^2$ process.  The use of the partial sum $F^{'}(j)$ also speeds up the
evaluation of the virtual M.O.\ energies.  Unfortunately, it does not speed up 
the calculation of the occupied M.O.\ energies.


\subsubsection{Localized Molecular Orbital Matrix Element Evaluation}
It is important in LMO work that only those matrix elements which are
potentially non-zero be evaluated; therefore, modifications need to be made to
the equations just described. For the purpose of the following discussion, the
system being studied should be assumed to be very large, i.e., to contain
thousands of atoms.

To assist in understanding how the modifications are made, it is convenient at
this time to write the occupied LMO as:
\begin{equation}
\psi_i = \sum_{j=1}^{NCF(i)}\sum_{\lambda\epsilon ICOCC(j+NNCF(i))}COCC(\lambda+NCOCC(i))
\end{equation}
and the virtual LMO as:
\begin{equation}
\psi_j = \sum_{l=1}^{NCE(j)}\sum_{\sigma \epsilon ICVIR(l+NNCE(j))}CVIR(\sigma 
+NCVIR(j)).\end{equation}
These expressions also illustrate the relationships of the five arrays which
are used to represent the LMOs.

Then the partial sum, $F^{'}(j)$, can be represented as:
\begin{equation}
F_{\lambda}^{'}(j) = 
\sum_{l=1}^{NCE(j)}\sum_{\sigma \epsilon ICVIR(l+NNCE(j))}CVIR(\sigma +
NCVIR(j))F_{\lambda \sigma}.
\end{equation}

The number of terms in this sum is already much smaller than in conventional
matrix annihilation in that the number of atoms represented in each LMO, 
$NCE(j)$, is much less than the total number of atoms.

Because all Fock matrix elements connecting atoms which are separated by  more
than \comp{CUTOF2} are automatically zero, only those terms which refer to
atoms separated by less that \comp{CUTOF2} need be evaluated.

All LMOs consist of a central part consisting of one to three atoms, which 
accounts for almost all of the wavefunction.  In the regions of an LMO far away
from the center, the contribution of any atom to the wavefunction becomes very
small.  If the Fock matrix elements connecting this distant atom to any other
atom is also very small, then quantities which depend of both of these terms
becomes quite negligible.    To test whether this condition exists, two scalar
quantities need to be calculated.  The first is the contribution of each atom
to the LMO:
\begin{equation}
\rho_A(j) =\sum_{\lambda\epsilon A}\psi_{\lambda j}^2
\end{equation}
and second is the magnitude  of the Fock matrix vector connecting each pair of
atoms:
\begin{equation}
F_{A B} = \sum_{\lambda\epsilon A \sigma\epsilon B}F_{\lambda \sigma}^2.
\end{equation}

During the evaluation of  $F_{\lambda}^{'}(j)$, the quantity $\rho_AF_{A B}$ is
computed. Only if it is above a preset limit are the terms involving atoms $A$
and $B$ used.  

As soon as $F_{\lambda}^{'}(j)$ is calculated, the magnitude of each atom's
contribution to $F_{\lambda}^{'}(j)$ is determined, and stored:
\begin{equation}
F_A(j)=\sum_A\sum_{\lambda\epsilon A}F_{\lambda}^{'}(j)^2.
\end{equation}

Together, these three modifications result in a large reduction in the time
necessary to compute the partial sum $F_{\lambda}^{'}(j)$.

The virtual energy levels are calculated using $F_{\lambda}^{'}(j)$:
\begin{equation}
\varepsilon_{jj}= \sum_{l=1}^{NCE(j)}\sum_{\sigma \epsilon ICVIR(l+NNCE(j))}
CVIR(\sigma +NCVIR(j))F_{\sigma}^{'}(j).
\end{equation}
In the same manner as $\rho_AF_{A B}$ was used in deciding whether the terms
involving atoms $A$ and $B$ should be evaluated in the calculation of 
$F_{\lambda}^{'}(j)$,  the quantity $F_A(j)\rho_A$ is evaluated and used to
decide which terms in the current summation should be used.

Now the calculation of the occupied-virtual matrix elements can be performed. 
The quantity to be calculated is
\begin{equation}
\varepsilon_{i j} =
\sum_{k=1}^{NCF(i)}\sum_{\lambda\epsilon ICOCC(k)+NNCF(i)}COCC(\lambda+NCOCC(i))
F_{\lambda}^{'}(j).
\end{equation}

Because this is a single sum over atoms, not much time is saved by testing to
see if any terms can be omitted.  However, the test is a simple one, and  it
does result in a small increase in speed by evaluating the quantity
$F_A(j)\rho_A(i)$, and comparing it to a preset limit.  Since both $F_A(j)$ and
$\rho_A(i)$ have already been calculated, this test is very rapid.

Calculation of the occupied-virtual matrix elements presents a new problem.
There are a large number of these elements, many of which involve only atoms
which are far from the LMO centers.  To see why this is so, consider two LMOs
which have exactly one atom in common.  For this to happen, the atom in
question must be very far from the centers of both LMOs.  The energy terms
arising from such an atom must, of necessity, be very small. In the interest of
efficiency, all calculations involving such atoms should be ignored. 
Unfortunately, it is not possible to {\it a priori} determine which terms to
include and which to leave out.  Therefore, at the start of the SCF
calculation, all terms must be evaluated.  As soon as  large changes in the
LMOs have stopped, a list can be constructed of those occupied-virtual matrix
elements which need to be considered for annihilation.


\subsubsection{Localized Molecular Orbital Matrix Element Annihilation}
Once the matrix elements have been calculated, annihilation is
relatively straightforward.  As with conventional SCF matrix annihilation,
the operation to be performed is:
\begin{equation}
\begin{array}{llrr}
\psi_i^{'}  & = & \alpha\psi_i & + \beta\psi_j \\
\psi_j^{'}  & = &-\beta\psi_i  & +\alpha\psi_j
\end{array}
\end{equation}
where $\alpha$ and $\beta$ are calculated using the matrix elements just
derived.

Because of the way LMOs are represented (the five arrays for each set, occupied
and virtual), care must be taken to ensure that the two by two rotation is done
correctly.  For the rotation to be performed, each atomic orbital in $\psi_i$
must be matched with the same atomic orbital in $\psi_j$.  Three distinct
situations occur: the same atom might be present in both $\psi_i$ and $\psi_j$;
$\psi_i$ might have an atom which $\psi_j$ does not have; and $\psi_j$ might
have an atom which $\psi_i$ does not have. The procedure for dealing with these
situations is as follows:
\begin{itemize}
\item Both LMOs have the same atom.

$\ \ $Simple rotation of the coefficients is performed:
\begin{equation}
\begin{array}{llrr}
\sum_{\lambda\epsilon A} \psi_i^{'}(\lambda)  & = 
& \alpha\psi_i(\lambda) & + \beta\psi_j(\lambda) \\
\sum_{\lambda\epsilon A}\psi_j^{'}(\lambda)  & = 
&-\beta\psi_i(\lambda)  & +\alpha\psi_j(\lambda)
\end{array}
\end{equation}

\item Occupied LMO $\psi_i$ has atom $A$, virtual LMO $\psi_j$ does not 
have atom $A$.

$\ \ $If the product $\beta\rho_A(i)$ is above a threshold, then the rotation
\begin{equation}
\begin{array}{llrr}
\sum_{\lambda\epsilon A} \psi_i^{'}(\lambda)  & =
& \alpha\psi_i(\lambda)  \\
\sum_{\lambda\epsilon A}\psi_j^{'}(\lambda)  & =
&-\beta\psi_i(\lambda)  
\end{array}
\end{equation}
is performed. If it is less than the threshold, no rotation is done.
The most important effect of this rotation is to place an atom in the virtual
LMO $\psi_j$, an atom which was not in $\psi_j$ before the two by two rotation
was started.  That is, the number of atoms in $\psi_j$ is increased by one.

\item Virtual  LMO $\psi_j$ has atom $A$, occupied LMO $\psi_i$ does not 
have atom $A$.

$\ \ $Again, if the product $\beta\rho_A(i)$ is above a threshold, then the rotation
\begin{equation}
\begin{array}{llrr} 
\sum_{\lambda\epsilon A} \psi_i^{'}(\lambda)  & =
&  \beta\psi_j(\lambda) \\
\sum_{\lambda\epsilon A}\psi_j^{'}(\lambda)  & =
&\alpha\psi_j(\lambda)
\end{array}
\end{equation}
is performed.  The number of atoms in the occupied LMO is increased by one.
\end{itemize}

At the start of the first SCF calculation, each LMO contains at most two atoms,
so the most dramatic effect of matrix element annihilation is to cause the
LMOs to expand so that they involve more atoms.  In the first few iterations,
the number of atoms in each LMO increases rapidly.  After about 10-30 iterations,
the rate of increase becomes very small, as the size of the LMOs becomes stable.

During the first SCF calculation, the number of atoms in certain LMOs may 
increase so rapidly that the gap between LMOs might vanish.  Any further
expansion would then cause vector overwriting to occur.  To avoid this
happening, the annihilation step is modified as follows:
The situation where vector overwriting is about to occur is detected.  The
degree of mixing of the two LMOs is then halved, and the annihilation is attempted
again.  If this procedure does not work, then the degree of mixing 
is halved again.  This
is done repeatedly until the danger of vector overwriting has been removed.
This technique is only used near the end of the SCF calculation; at other
times the degree of mixing is simply set to zero.

\subsection{Reducing the Size of Localized Molecular Orbitals}
If, as a result of matrix element annihilation,  the LMOs are allowed  to
expand so that they include more and more atoms without limit, then  eventually
the efficiency of the calculation would be severely impaired. To prevent this
occurring, a procedure is needed which will allow the size  of LMOs to be
reduced.  

Atoms can be added at an LMO as a result of matrix element annihilation. What
is not so obvious is that  annihilation can also make the contribution of an
atom to the LMO so small that the atom can subsequently be deleted from that
LMO without harm. Of course, if an atom is deleted from an LMO, changes should
be made to the  five arrays representing the LMOs.  The operation of removing
atoms from LMOs and adjusting the  appropriate arrays is done in subroutine
\comp{TIDY}.

This compression is carried out by copying all the information on each atom in
the LMOs.  During the copy operation, any atom whose contribution to an LMO is
insignificant is  not copied.  In addition, unused space at the end of each LMO
is also not copied.  The result of the copy is to produce a set of LMOs in
which only atoms which contribute significantly are present, and with no unused
space between the LMOs. 

Before the LMOs can be used once more, they must be redistributed in their
arrays so that each LMO has some unused space at the end to allow for expansion
during the next matrix annihilation operation.

\subsubsection{Size of LMOs}\label{size_of_lmo}
When a SCF is achieved, the LMOs extend over more atoms than might be
expected.  Each LMO is about 90--99\% on one or two atoms, and if the
surrounding few atoms are included, almost 100\% of the LMO can be accounted
for.   From this, it would appear that the intensity of the LMO would continue
to decrease rapidly with distance from the center.  This is normally not the
case. Instead, LMOs usually have intensity on a large number of atoms,
sometimes several  hundred atoms.  As a result, the calculations take a much
longer time than would otherwise be expected.

Some effort has been expended in trying to find ways of reducing the size of
LMOs. These attempts have not been successful.  The definitive failure was
provided as follows:

Using MOPAC, the LMOs for a large system were generated using \comp{PRECISE}.
The resulting LMOs were as expected, over 90\%\ on two atoms, with most of the
rest of the wavefunction on the nearby atoms.  However, the intensity did not
drop rapidly to zero with increasing distance.  Instead, it held  more or less
constant at about 10$^{-5}$ to 10$^{-8}$ for a large number of atoms before
finally dropping to a negligible value.

This behavior did not change on increasing the precision of the localization.

Because of this result, it was obvious that further localization of the MOZYME 
LMOs would not be useful.


\subsection{Density Matrix Construction}
The general expression for constructing the density matrix from the occupied
set of molecular orbitals is:
\begin{equation}
P_{\lambda \sigma} = 2\sum_{i=1}^{occ}c_{\lambda i}c_{\sigma i}.
\end{equation}

This same basic equation is used in the construction of the density matrix in
LMO theory.  The main difference is that not all density matrix elements are,
or need to be, evaluated.  

The only elements of the density matrix which need to be evaluated are
those relating to atoms which are \hyperref[pageref]{separated by less than
\comp{CUTOF2}}{ (see Page~}{)}{cutoff}.  To understand why, consider how
the density matrix is used in the SCF calculation. The Fock matrix is
constructed from products involving the density matrix, the
one-electron matrix, and the two-electron integrals. All terms which
involve  density matrix elements connecting any two atoms only, involve
one electron integrals and two electron exchange integrals for  the
same two atoms. All one-electron integrals and two-electron exchange
integrals for atoms separated by more than \comp{CUTOF2} are zero.
Therefore, the value of density matrix elements for atoms separated by
more than \comp{CUTOF2} is unimportant.  To save time and to reduce
array size, therefore, these density matrix elements are not
calculated.

Having stated that the construction of the density matrix is the same in
principle as that used in conventional M.O.\ theory, the way in which LMOs are
stored does introduce a technical difference.  The sequence in which the atoms
are represented in LMOs changes from LMO to LMO.  While it is possible for two
LMOs to have the same sequence   (a $\sigma$ and $\pi$ LMO involving the same
two atoms is an example),  in general the LMOs should be considered as
consisting of a random set of atoms, and should be treated as such.

This unpredictable nature of the composition of the LMOs means that the order
in which density matrix elements are calculated is determined by the  LMOs, and
not by the sequence of atoms in the molecule.  

\subsection{Energy Effects of \comp{CUTOF2}}
Energy terms arise from pairs of atoms which are separated by distances greater
than the cutoff distance.  These terms are purely electrostatic in origin:
their magnitude is simply proportional to the net charge, $Q_A$, on the atoms
involved.  Net charge is defined as 
\begin{equation}
Q_A = Z_A-Ne_A
\end{equation}
where $Ne_A$ is the total electron population on atom $A$ 
\begin{equation}
Ne_A=\sum_{\lambda\varepsilon A}P_{\lambda \lambda}.
\end{equation}
From simple electrostatics, the overall effect of these net charges is to
contribute a long-range energy term, $E_{lre}$ to the energy of the system
thusly 
\begin{equation}
 E_{lre} = \frac{1}{2}\sum_A\sum_B\phi_{AB}Q_AQ_B\gamma_{AB},
\end{equation}
where $\phi_{AB}$ \ is 1 \ if $R_{AB}$ \ is greater \ than the
\ cutoff distance but
0 otherwise, and $\gamma_{AB}$ is the $<ss|1/R_{AB}|ss>$ integral.
This simple description must be modified because of the effect of net
charges on the electron density.  Each distant net positive charge will induce
a small, but in general not negligible, stabilizing effect on the electron
distribution, while a distant net negative charge will have the opposite
effect.  This can be expressed formally in terms of the Fock matrix 
\begin{equation}
F_{\lambda\varepsilon A \lambda\varepsilon A} =
F_{\lambda\varepsilon A \lambda\varepsilon A}^{'}-\sum_B\phi_{AB}Q_B\gamma_{AB}.
\end{equation}
The effect on the nuclear term is similar, but opposite in sign:
\begin{equation}
E_{A(nuc)}=E_{A(nuc)}^{'}+\sum_B\phi_{AB}Q_BZ_A\gamma_{AB}.
\end{equation}
The total energy is given by the sum of the electronic plus nuclear energies.
Given that the electronic energy is 
\begin{equation}
E_{ee}=\frac{1}{2}\sum_{\lambda}\sum_{\sigma}P_{\lambda \sigma}
(H_{\lambda \sigma}+F_{\lambda \sigma}),
\end{equation}
the electronic term due to distant nuclei is
\begin{equation}
E_{lre(ee)}=\frac{-1}{2}\sum_A\sum_{\lambda\varepsilon A}P_{\lambda \lambda}\sum_B\phi_{AB}Q_B\gamma_{AB}
\end{equation}
or
\begin{equation}
E_{lre(ee)}=\frac{-1}{2}\sum_ANe_A\sum_B\phi_{AB}Q_B\gamma_{AB}.
\end{equation}
Likewise, given that the nuclear energy is 
\begin{equation}
E_{ne}=\sum_A\sum_{B<A} Z_AZ_B\gamma_{AB},
\end{equation}
the nuclear energy term due to distant nuclei is
\begin{equation}
E_{lre(ne)}=\sum_A\sum_{B<A}\phi_{AB} Q_AZ_B\gamma_{AB}.
\end{equation}
This can be re-written in a more symmetric form as
\begin{equation}
E_{lre(ne)}=\frac{1}{4}\sum_A\sum_{B}\phi_{AB}(Q_AZ_B +Q_BZ_A)\gamma_{AB}.
\end{equation}
Together, the total contribution due to the electronic and nuclear terms
arising from distant atoms is 
\begin{equation}
E_{lre(ee)}+E_{lre(ne)}= \frac{1}{4}\sum_A\sum_{B}\phi_{AB}(Q_A(Z_B-Ne_B)+Q_B(Z_A-Ne_A))\gamma_{AB}.
\end{equation}
Rearranging gives 
\begin{equation}
E_{lre(ee)}+E_{lre(ne)}= \frac{1}{4}\sum_A\sum_{B}\phi_{AB}(Q_AQ_B +Q_BQ_A)\gamma_{AB}
\end{equation}
or
\begin{equation}
E_{lre(ee)}+E_{lre(ne)}= \frac{1}{2}\sum_A\sum_{B}\phi_{AB}Q_AQ_B\gamma_{AB},
\end{equation}
which is identical to the simple equation this discussion began with.

In the 1996 version of MOZYME, this sum was calculated explicitly.  However, 
evaluation of the sum can be avoided by adding into the one-electron matrix
the  electrostatic stabilization terms arising from atoms beyond
\comp{CUTOF2}.  This concisely and effectively takes into account the
quantities just discussed.

Although the point-charge electrostatic effects are important, they are not the
only long-range effect.  The effect of a point-charge on a lone pair of
electrons is also significant.  Consider a lone pair on an atom pointing in the
direction of a positive charge.  Clearly this will lead to a stabilizing
effect. A lone pair pointing in the opposite direction will be destabilized.
Similarly, a lone pair pointing at 90$^{\circ}$ to a charge will be subjected
to a torque.  In order to include these effects in the Hamiltonian some extra
terms are needed.

The size of a lone pair on an atom is represented in the density matrix by the
value of the associated  $s-p$ terms.  For each atom, there are three such
terms: $P_{sp_x}$, $P_{sp_y}$, and $P_{sp_z}$. The effect of  distant charges
on $F_{sp_x}$ is given by 
\begin{equation}
F_{sp_x}' = F_{sp_x}-\sum_{B}Q_B\! <\!ss|sx\!>\!P_{sp_x}
\end{equation}
Similar terms exists for $F_{sp_y}$  and $F_{sp_z}$.

%\section{Timing Considerations}


