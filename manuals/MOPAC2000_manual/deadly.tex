\chapter{Error messages produced by MOPAC}\label{errormessages}
\index{MOPAC!error messages}
\index{Error!messages|(}
\index{Messages|see{Error messages}}
\index{Derivatives!|see{{\em also} Gradient}}
\index{Gradient!|see{{\em also} Derivatives}}
MOPAC produces several hundred messages, all of which are intended to be 
self-explanatory.  However, when an error occurs it is useful to have more
information than is given in the standard messages.

The following alphabetical list gives more complete definitions of the
messages printed.

\begin{description}
\item[\comp{1SCF SPECIFIED WITH PATH. \ldots} (FATAL)]~\\ 
The pair of options, \comp{1SCF} with a path calculation, is not allowed, except
in a \comp{RESTART} calculation.
\index{1SCF SPECIFIED WITH \ldots}
\index{Error message!1SCF SPECIFIED WITH\ldots}

\item[\comp{A SINGLE ATOM HAS NO VIBRATIONAL MODES} (FATAL)]~\\
An attempt has been made to calculate the vibrations of a single atom.
The smallest system that can have vibrations is a diatomic molecule.
\index{Error message!A SINGLE ATOM HAS \ldots}
\index{Error message!SINGLE ATOM HAS \ldots}
\index{A SINGLE ATOM HAS \ldots}
\index{SINGLE ATOM HAS NO VIB\ldots}

\item[\comp{A SYMMETRY FUNCTION IS USED TO DEFINE A NON-EXISTENT ATOM} (FATAL)]~\\
Symmetry functions can only be used in the definition of atoms or dummy atoms.
Check the dependent atom numbers in the symmetry data.
\index{Error message!A SYMMETRY FUN\ldots}
\index{Error message!SYMMETRY FUNCTION \ldots}
\index{A SYMMETRY FUNCTION \ldots}
\index{SYMMETRY FUNCTION IS \ldots}

\item[\comp{ALL CONVERGERS ARE NOW FORCED ON}]~\\
The default SCF convergers have not worked.  A new set of convergers will be tried.
This is often caused by faulty data, so the data should be checked to see if
anything is wrong.  This sometimes happens naturally, particularly with exotic 
systems.
\index{ALL CONVERGERS ARE \ldots}

\item[\comp{AN ATTEMPT WAS MADE TO PRINT\ldots}]~\\
To avoid printing very large matrices in MOZYME, an upper limit (200) 
has been set on the 
size of matrix that can be printed. If the array requested is larger than this
limit, only part of the array will be printed.
\index{AN ATTEMPT WAS MADE \ldots}

\item[\comp{AN UNOPTIMIZABLE GEOMETRIC PARAMETER \ldots}]~\\
When internal coordinates are supplied, six coordinates  cannot  be
optimized.   These  are  the  three coordinates of atom 1, the angle and
dihedral on atom 2 and the dihedral on atom 3.  An attempt has been made to 
optimize  one of these.  This is usually indicative of a typographic error, but
might simply be an oversight.  Either way, the error will  be corrected and the
calculation will not be stopped here.
\index{Error message!AN UNOPTIMIZABLE \ldots}
\index{Error message!UNOPTIMIZABLE \ldots}
\index{AN UNOPTIMIZABLE GEO\ldots}
\index{UNOPTIMIZABLE GEO\ldots}

\item[\comp{ANALYTIC C.I.\ DERIVATIVES DO NOT WORK\ldots} (FATAL)]~\\
The analytical C.I.\ derivative calculation failed.  Add \comp{NOANCI} or 
\comp{UHF} and re-run.
\index{ANALYTIC C.I.\ DERIV\ldots}
\index{Error message!ANALYTIC C.I.\ DERIV\ldots}

\item[\comp{ANALYTICAL DERIVATIVES TOO INACCURATE\ldots} (FATAL)]~\\
The analytical C.I.\ derivative calculation failed.  Add \comp{NOANCI} or 
\comp{UHF} and re-run.
\index{ANALYTICAL DERIVATIVES \ldots}
\index{Error message!ANALYTICAL DERIV\ldots}

\item[\comp{AT LEAST ONE ATOM DEFINED BY CVB\ldots} (FATAL)]~\\
The atoms defined in \comp{CVD} are limited to the real atoms.  If any
dummy atoms are used, these should {\em not} be included in the atom
numbering used.
\index{AT LEAST ONE ATOM \ldots}
\index{Error message!AT LEAST ONE ATOM \ldots}

\item[\comp{AT LEAST ONE ATOM HAS A ZERO MASS} (FATAL)]~\\
In a \comp{FORCE} calculation, the mass of an atom is zero.  To correct
this, \hyperref[pageref]{give the mass explicitly}{, see p.~}{}{atom_mass}.
\index{Error message!AT LEAST ONE ATOM \ldots}
\index{AT LEAST ONE ATOM HAS \ldots}

\item[\comp{At this point, both reactants \ldots}]~\\
In a \comp{SADDLE} calculation, both reactants and products are on the same side
of the transition state.  Options at this point are:
\index{At this point, both reactants \ldots}
\begin{itemize}
\item If it is near to the transition state (the gradient norm has been dropping
for reactants or products, or the ``DISTANCE A-B'' is small, e.g.\ less than 
0.2), refine the geometry using \comp{TS}.
\item If it is not near to the transition state, identify geometries on
both sides of the transition state (these will be separated in the output by
the message ``REACTANTS AND PRODUCTS SWAPPED AROUND''), and use these to start
a new \comp{SADDLE} calculation.  Add \comp{BAR=0.02}.
\item If CPU time is not important, add \comp{BAR=0.02} to the original data set
and re-run.
\end{itemize}

\item[\comp{Atom $nn$ is Cartesian\ldots} (FATAL)]~\\
An attempt has been made to use symmetry function 19 (the bond length is
a multiple of the reference bond length) using an atom whose position is 
defined using Cartesian coordinates.  Correct the error and re-run.
\index{Atom $nn$ is Cartesian\ldots}
\index{Error message!Atom $nn$ is Cartesian\ldots}

\item[\comp{Atom $nn$ is internal\ldots} (FATAL)]~\\
An attempt has been made to use symmetry function 18 (the ``$z$'' coordinate is
set equal to minus the reference ``$y$'' coordinate) using an atom whose
position is defined using internal coordinates.  Correct the error and re-run.
\index{Atom $nn$ is internal\ldots}
\index{Error message!Atom $nn$ is internal\ldots}

\item[\comp{ATOM NUMBER $nn$ IS ILLDEFINED} (FATAL)]~\\
The rules for definition of atom connectivity are:
\index{Error message!ATOM NUMBER \ldots}
\index{ATOM NUMBER $nn$ IS ILL\ldots}
\begin{enumerate}
\item Atom 1 has no connectivity.
\item Atom 2 can be Cartesian or internal.  If internal, it must be defined
with connectivity (1,0,0).
\item Atom 3 can be Cartesian or internal.  If internal, it must be defined 
with connectivity (2,1,0) or (1,2,0). 
\item All other atoms can be Cartesian or internal.  If internal, they
must be defined  in  terms  of  already-defined
atoms:   these  atoms must all be different.  Thus atom 9 might
be connected to atom 5, make an angle with atom 6, and  have  a
dihedral  with  atom  7.  If the dihedral was with atom 5, then
the geometry definition would be faulty.
\end{enumerate}
 
If any of these rules is broken, a fatal error message is  printed,
and the calculation stopped.
 
\item[\comp{ATOMIC MASS OF ATOM $nn$ TOO SMALL} (FATAL)]~\\
In a \comp{DRC} calculation, all atoms must have masses greater than 0.1 times
the mass of a hydrogen atom.  If sparkles are used, these have no default mass.
Atom masses are specified as numbers after the symbol, e.g.\ H1.008 or +12.00 (for
a ``+'' sparkle having a mass of 12.00.)
\index{ATOMIC MASS OF \ldots}
\index{Error message!ATOMIC MASS OF ATOM\ldots}

\item[\comp{ATOMIC NUMBER $nn$ IS NOT AVAILABLE \ldots} (FATAL)]~\\
An element has been used for which parameters  are  not  available. Only  if  a
typographic error has been made can this be rectified.  This check is not
exhaustive, in that even if  the  elements  are  acceptable there  are  some 
combinations  of  elements within \mi{MINDO/3} that are not allowed.  This is a
fatal error message.
\index{Error message!ATOMIC NUMBER \ldots}
\index{ATOMIC NUMBER $nn$ IS \ldots}


\item[\comp{ATOMIC NUMBER OF $nn$ ?}  (FATAL)]~\\
An atom has been specified with a negative or zero  atomic  number. This  is 
normally  caused  by forgetting to specify an atomic number or symbol.  This is
a fatal error message.
\index{Error message!ATOMIC NUMBER \ldots}
\index{ATOMIC NUMBER OF $NN$ ?}

             
\item[\comp{ATOMS  $nn$ AND $nn$ ARE SEPARATED BY nn.nnnn ANGSTROMS}  (FATAL)]~\\
Two genuine atoms (not dummies)  are  separated  by  a  very  small
distance.    This  can  occur  when  a  complicated  geometry  is  being
optimized, in which case the user may wish to  continue.   This  can  be done 
by  using  the  keyword \comp{GEO-OK}. \  More often, however, this message
indicates a mistake, and the calculation is, by default, stopped.
\index{Error message!ATOMS $nn$ AND $nn$ \ldots}
\index{ATOMS $nn$ AND $nn$ ARE SEP\ldots}

\item[\comp{ATTEMPT TO GO DOWNHILL IS UNSUCCESSFUL \ldots}]~\\
A  quite  rare  message,  produced  by   Bartel's gradient\index{Bartel's
method}   norm minimization.  Bartel's method attempts to minimize the gradient
norm by searching the gradient space for a minimum.  Apparently  a  minimum 
has been found, but not recognized as such.  The program has searched in all
$3N-6$ directions, and found no way down, but the criteria for a minimum have 
not been satisfied.  No advice is available for getting round this error.
\index{Error message!ATTEMPTS TO GO \ldots}
\index{ATTEMPT TO GO DOWN\ldots}


\item[\comp{AVAILABLE MEMORY IN GREENS FUNCTION TOO SMALL} (FATAL)]~\\
This error is caused by a program bug.  Please make a report to Dr.~Stewart.
\index{Error message!AVAILABLE MEMORY\ldots}
\index{AVAILABLE MEMORY IN \ldots}

\item[\comp{BOTH GEOMETRIES ARE IDENTICAL} (FATAL)]~\\
The \comp{SADDLE} technique uses two geometries, one for the reactant(s) and
one for the product(s).  These must be different.  Correct fault and re-run.
\index{Error message!BOTH GEOMETRIES ARE\ldots}
\index{BOTH GEOMETRIES ARE \ldots}

\item[\comp{BOTH SYSTEMS ARE ON THE SAME SIDE \ldots}]~\\
A non-fatal message, but still cause for concern.  During a  SADDLE
calculation  the  two  geometries  involved are on opposite sides of the
transition  state.   This  situation  is  verified  at  every  point  by
calculating  the  cosine  of the angle between the two gradient vectors. For as
long as it is negative, then the two geometries are  on  opposite sides  of 
the   TS.  If, however, the cosine becomes positive, then the assumption is
made that one moiety has fallen over the  TS  and  is  now below  the other
geometry.  That is, it is now further from the  TS than the other, temporarily 
fixed,  geometry.   To  correct  this,  identify geometries  corresponding  to 
points  on  each  side  of the  TS.  (Two geometries on the output separated
by  the  message   \comp{``SWAPPING\ldots''})  and make  up  a  new  data-file
using these geometries.  This corresponds to points on the reaction path near
to the  TS. \ Run a new job using  these two geometries, but with 
\comp{BAR}\index{BAR} set to  a third or a quarter of its original value, e.g.\
\comp{BAR=0.05}.  This normally allows the  TS to be located.
\index{Error message!BOTH SYSTEMS ARE \ldots}
\index{BOTH SYSTEMS ARE ON \ldots}


\item[\comp{C.I.\ IS OF SIZE LESS THAN ROOT SPECIFIED} (FATAL)]~\\
The value of $n$ in \comp{ROOT=$n$} is less than the size of the C.I. 

\begin{latexonly}
See p.~\pageref{setmic} for the sizes of various C.I.\ matrices, implied by
\comp{C.I.=$m$}
\end{latexonly}
\begin{htmlonly}
\htmlref{The sizes of various C.I.\ matrices}{setmic}, implied by 
\comp{C.I.=$m$} appear elsewhere
\end{htmlonly}.

\index{Error message!C.I.\ IS OF SIZE LESS \ldots}
\index{C.I.\ IS OF SIZE LESS THAN \ldots}

\item[\comp{C.I.\ NOT ALLOWED WITH UHF} (FATAL)]~\\
There is no UHF configuration  interaction  calculation  in  MOPAC.  Either
remove the keyword that implies C.I.\ or the word UHF.
\index{Error message!C.I.\ NOT ALLOWED \ldots}
\index{C.I.\ NOT ALLOWED WITH UHF}

\item[\comp{CALCULATION ABANDONED AT THIS POINT} (FATAL)]~\\
A particularly annoying message!  In  order  to  define  an  atom's position, 
the  three  atoms  used  in  the  connectivity table must not accidentally fall
into a  straight  line.   This  can  happen  during  a geometry  optimization
or gradient minimization.  If they do, and if the angle made by the atom being
defined is not zero or  180  degrees,  then its  position  becomes 
ill-defined.   This  is  not  desirable, and the calculation will stop in order
to allow corrective action to  be  taken. Note  that  if  the  three  atoms 
are in an exactly straight line, this message will not be triggered.  
\index{Error message!CALCULATION ABAND\ldots} 
\index{CALCULATION ABANDONED \ldots} 


\item[\comp{Cannot open {\em filename}.out!} (FATAL)]~\\
The program cannot open the output file.  Possible causes are (a) the file
already exists, but is owned by another user, (b) the subdirectory is ``read
only'', (c) there is no space left on the partition (unlikely).
\index{Cannot open {\em filename}.out!}
\index{Error message!Cannot open {\em filename}.out!}

              
\item[\comp{Cannot write density matrix to {\em filename}.den} (FATAL)]~\\
The program cannot open the density restart file.  Possible causes are (a)  the
file already exists, but is owned by another user, (b) the subdirectory  is
``read only'', (c) there is no space left on the partition (unlikely).
\index{Cannot write density \ldots}
\index{Error message!Cannot write density matrix\ldots}


\item[\comp{CARTESIAN CALCULATION NOT ALLOWED WITH \ldots} (FATAL)]~\\
\comp{XYZ} is not allowed when geometries are specified using GAUSSIAN format.
To allow \comp{XYZ} to be used, first do a \comp{0SCF} calculation to convert
GAUSSIAN format geometry into the MOPAC format. 
\index{Error message!CARTESIAN CALC\ldots}
\index{CARTESIAN CALCULATION \ldots}


\item[\comp{CHARGE ON ATOM $N$ UNREASONABLE} (FATAL)]~\\
The range of allowed charges for an atom is limited.  Allowed charges are:
Group I: +1; II: +1, +2; III: +1, +2, +3; IV: -1, +1; V: +1; VI: -1, -2; VII:
-1.  Any charges outside these ranges are considered unreasonable.
\index{CHARGE ON ATOM $N$ \ldots}
\index{Error message!CHARGE ON ATOM $N$\ldots}

Correct the geometry and re-run.

\item[\comp{Connectivity of atom $NN$\ldots}]~\\
The rules for definition of atom connectivity are:
\index{Connectivity of atom $NN$\ldots}
\index{Error message!Connectivity of atom $NN$\ldots}
\begin{enumerate}
\item Atom 1 has no connectivity.
\item Atom 2 can be Cartesian or internal.  If internal, it must be defined
with connectivity (1,0,0).
\item Atom 3 can be Cartesian or internal.  If internal, it must be defined
with connectivity (2,1,0) or (1,2,0).
\item All other atoms can be Cartesian or internal.  If internal, they
must be defined  in  terms  of  already-defined
atoms:   these  atoms must all be different.  Thus atom 9 might
be connected to atom 5, make an angle with atom 6, and  have  a
dihedral  with  atom  7.  If the dihedral was with atom 5, then
the geometry definition would be faulty.
\end{enumerate}

If any of these rules is broken, a fatal error message is  printed,
and the calculation stopped.

\item[\comp{COORDINATES MUST BE CARTESIAN} (FATAL)]~\\
In a \comp{DRC} calculation, in which an initial velocity is to be used, the
geometry must be supplied in Cartesian coordinates, in order for the velocity
to be meaningful.  If it is essential that internal  coordinates be used for
the geometry, add \comp{LET}, and re-run.  The velocity vector, however, must
still be in Cartesian coordinates.
\index{Error message!COORDINATES MUST \ldots}
\index{COORDINATES MUST BE \ldots}

\item[\comp{CROSS requires ROOT=2 or higher} (FATAL)]~\\
In an intersystem crossing calculation, the crossing is defined in terms of
the higher energy state.  Because of this, \comp{ROOT=$n$} {\em must} be used,
and $n$ \comp{must} be more than 1.
\index{CROSS requires ROOT=2 \ldots}
\index{Error message!CROSS requires ROOT=2\ldots}

\item[\comp{CUTOF1 WAS SET SMALLER THAN CUTOF2}]~\\
In a MOZYME calculation in which \comp{CUTOF}s were used, \comp{CUTOF1} (the 
transition from dipole plus monopole to monopole only) was set smaller than
the transition from NDDO to dipole plus monopole.  \comp{CUTOF1} would be
automatically increased to a reasonable value and the calculation allowed to
continue.
\index{CUTOF1 WAS SET SMALL\ldots}

\item[\comp{DATA ARE NOT AVAILABLE FOR ELEMENT NO.\ $N$} (FATAL)]~\\

Parameters are not available for the element with atomic number $N$. If
new parameters are available, these can be supplied to MOPAC by use of
\hyperref[pageref]{\comp{EXTERNAL=}}{, see p.~}{}{external}.
\index{DATA ARE NOT AVAIL\ldots}
\index{Error message!DATA ARE NOT AVAIL\ldots}

\item[\comp{Data for TOMASI Model missing or faulty} (FATAL)]~\\
When \comp{TOM} is specified, extra data are required after the geometry
in order to define the solvent, unless \comp{H2O}, \comp{CCL4}, or 
\comp{CHCL3} is used.
\index{Error message!Data for TOMASI Model \ldots}
\index{Data for TOMASI Model miss\ldots}

\item[\comp{DEGENERATE LEVELS DETECTED IN MECI \ldots } (FATAL)]~\\
If only some M.O.s of a degenerate manifold are used in a MECI calculation,
the results will be nonsense.  To prevent such calculations, the message
\comp{DEGENERATE LEVELS \ldots} is printed, and the job stopped.  To continue,
specify \comp{GEO-OK}. \ See Space Quantization and \comp{GEO-OK}.
\index{Error message!DEGENERATE LEVELS \ldots}                              
\index{DEGENERATE LEVELS \ldots}                              

\item[\comp{DENSITY FILE MISSING OR CORRUPT} (FATAL)]~\\
In a run involving  \comp{OLDENS}, the old density matrix,
in $<\!$filename$\!>$.den, is either missing or corrupt.  Either generate
a new $<\!$filename$\!>$.den file using \comp{1SCF} and \comp{DENOUT} or
do not use \comp{OLDENS}.
\index{Error message!DENSITY FILE MISSING \ldots}
\index{DENSITY FILE MISSING \ldots}

\item[\comp{Density Restart File missing or corrupt} (FATAL)]~\\
In a run involving \comp{OLDENS}, the old density matrix,
in $<\!$filename$\!>$.den, is either missing or corrupt.  Either generate
a new $<\!$filename$\!>$.den file using \comp{1SCF} and \comp{DENOUT} or
do not use \comp{OLDENS}.
\index{Error message!Density Restart File missing \ldots}
\index{Density Restart File missing \ldots}

\item[\comp{DIPOLE CONSTRAINTS NOT USED}]~\\
An attempt had been made to run an \comp{ESP} calculation, with dipole 
constraints, on an ionized system.  This is not allowed, the keyword
\comp{DIPOLE} will be ignored and the calculation allowed to proceed.
\index{DIPOLE CONSTRAINTS \ldots}

\item[\comp{DUE TO A PROGRAM BUG, THE FIRST THRE\ldots} (FATAL)]~\\
Due to a problem caused by the definition of internal coordinates,
the first three atoms must not form a straight line.
\index{Error message!DUE TO A PROGRAM\ldots}
\index{DUE TO A PROGRAM \ldots}

\item[\comp{ECHO is not allowed at this point} (FATAL)]~\\
\comp{ECHO} can only be used at the start of a run.  If the run has several
geometries, and at least one is in GAUSSIAN format, then \comp{ECHO} would
cause an infinite loop to be created.  Remove \comp{ECHO} and re-run.
\index{Error message!ECHO is not allowed \ldots}
\index{ECHO is not allowed at this point}

\item[\comp{EIGENVECTOR FOLLOWING IS NOT RECOMMENDED\ldots} (FATAL)]~\\
If internal coordinates are used, the maximum number of variables is 3$N$-6.  If 
Cartesian coordinates are used, up to 3$N$ variables can be used.  If \comp{GEO-OK}
is present, any number of variables can be used.
\index{EIGENVECTOR FOLLOWING \ldots}
\index{Error message!EIGENVECTOR FOLLOW\ldots}

\item[\comp{EITHER ADD `LET' OR \ldots} (FATAL)]~\\
A \comp{FORCE} calculation is {\em only} meaningful if the geometry is
at a stationary point.  Either add \comp{LET}, to run the current geometry,
or refine the geometry and re-run.
\index{EITHER ADD 'LET' OR \ldots}
\index{Error message!EITHER ADD 'LET' \ldots}

\item[\comp{ELEMENT NOT FOUND}     (FATAL)]~\\
When an external file  is  used  to  redefine  MNDO,  AM1, PM3, or MNDO-$d$
parameters, the chemical symbols used must correspond to known elements. Any
that do not will trigger this fatal message.
\index{Error message!ELEMENT NOT FOUND}
\index{ELEMENT NOT FOUND}
\index{MNDO}\index{AM1}\index{PM3}\index{MNDO-$d$}
                           
\item[\comp{ELEMENT NOT RECOGNIZED}     (FATAL)]~\\
When a Gaussian data set is supplied, the chemical symbols used must
correspond  to known elements.  Any that do not will trigger this fatal
message. Correct the data set and re-run.
\index{ELEMENT NOT RECOGNIZED}    
\index{Error message!ELEMENT NOT REC\ldots}    

\item[\comp{ERROR  TOO MANY NEIGHBORS} (FATAL)]~\\
An extraordinarily difficult error to make.  An atom has more than 200
neighboring atoms, in an \comp{ESP} or \comp{PMEP} calculation.
\index{ERROR  TOO MANY \ldots}
\index{Error message!ERROR TOO MANY \ldots}

\item[\comp{ERROR DETECTED DURING READ\ldots} (FATAL)]~\\
In a \comp{PATH} calculation involving a \comp{RESTART}, the \comp{RESTART} file
is faulty---damaged or corrupt.  To correct this,  start over again.
\index{Error message!ERROR DETECTED \ldots}
\index{ERROR DETECTED DURING \ldots}
                    
\item[\comp{ERROR DETECTED IN SUBROUTINE CHECK\ldots} (FATAL)]~\\
In a MOZYME calculation, the LMOs are periodically re-normalized.  This
involves making small adjustments to the LMOs.  If large changes are necessary,
something has gone wrong in the LMO procedure.  Try restarting the job, but do 
{\em not} use \comp{OLDENS}.  If this does not work, there is a problem in the 
program.
\index{ERROR DETECTED IN SUB\ldots}
\index{Error message!ERROR DETECTED IN \ldots}

\item[\comp{ERROR DURING READ AT ATOM NUMBER \ldots} (FATAL)]~\\
Something is wrong with the geometry data.  In order to  help  find the 
error,  the  geometry  already  read in is printed.  The error lies either on
the last  line  of  the  geometry  printed,  or  on  the  next (unprinted)
line.  This is a fatal error.
\index{Error message!ERROR DURING READ \ldots} 
\index{ERROR DURING READ AT \ldots} 


\item[\comp{Error in BLAS} (FATAL)]~\\
This error is caused by a program bug.  Please make a report to Dr.~Stewart.
\index{Error message!Error in BLAS}

\item[\comp{ERROR IN CVB KEYWORD} (FATAL)]~\\
An error is present in the \htmlref{\comp{CVB} keyword}{cvb} used in
defining specific bonds in a Lewis structure.  
\begin{latexonly}
See p.~\pageref{cvb} for details of this keyword.
\end{latexonly}

\item[\comp{Error in EF} (FATAL)]~\\
This error is caused by a program bug.  Please make a report to Dr.~Stewart.
\index{Error message!Error in EF}

\item[\comp{Error in GETGEO} (FATAL)]~\\
This error is caused by a program bug.  Please make a report to Dr.~Stewart.
\index{Error message!Error in GETGEO}

\item[\comp{Error in GETMEM} (FATAL)]~\\
This error is caused by a program bug.  Please make a report to Dr.~Stewart.
\index{Error message!Error in GETMEM}

\item[\comp{Error in GREENF} (FATAL)]~\\
This error is caused by a program bug.  Please make a report to Dr.~Stewart.
\index{Error message!Error in GREENF}

\item[\comp{Error in MOLDAT.} (FATAL)]~\\
This error is caused by a program bug.  Please make a report to Dr.~Stewart.
\index{Error message!Error in MOLDAT.}

\item[\comp{Error in PATHS} (FATAL)]~\\
This error is caused by a program bug.  Please make a report to Dr.~Stewart.
\index{Error message!Error in PATHS}

\item[\comp{Error in PMEP} (FATAL)]~\\
This error is caused by a program bug.  Please make a report to Dr.~Stewart.
\index{Error message!Error in PMEP}

\item[\comp{ERROR IN READ OF FIRST THREE LINES} (FATAL)]~\\
The data-set has a severe error in the first three lines.  This is a very
unusual error, and indicates that the data-set is likely to be severely
in error, or that MOPAC has not been installed correctly.
\index{Error message!ERROR IN READ OF \ldots}
\index{ERROR IN READ OF FIRST \ldots}

\item[\comp{Error in READMO} (FATAL)]~\\
This error is caused by a program bug.  Please make a report to Dr.~Stewart.
\index{Error message!Error in READMO}

\item[\comp{Error in TOM} (FATAL)]~\\
This error is caused by a program bug.  Please make a report to Dr.~Stewart.
\index{Error message!Error in TOM}

\item[\comp{Error in UPDATE} (FATAL)]~\\
This error is caused by a program bug.  Please make a report to Dr.~Stewart.
\index{Error message!Error in UPDATE}

\item[\comp{Error in USAGE} (FATAL)]~\\
This error is caused by a program bug.  Please make a report to Dr.~Stewart.
\index{Error message!Error in USAGE}

\item[\comp{ERROR!! MODE IS LARGER\ldots} (FATAL)]~\\
In a system of $N$ variables, a request has been made to follow the $M$th
mode, where $M>N$.  Correct the data and re-run.
\index{ERROR!! MODE IS LARGER\ldots}
\index{Error message!ERROR!! MODE IS LARGER\ldots}

\item[\comp{ERROR: PERMANENT ARRAY} (FATAL)]~\\
This error can {\em only} be caused by a programming error.  Please contact
Dr~Stewart as soon as possible.
\index{ERROR: PERMANENT ARRAY}
\index{Error message!ERROR: PERM\ldots}

\item[\comp{ERRORS DETECTED IN CONNECTIVITY} (FATAL)]~\\
The \htmlref{connectivity in the MOPAC internal coordinate
Z-matrix}{int} is faulty.  
\begin{latexonly}
For a description of the connectivity rules, see p.~\pageref{int}.
\end{latexonly}
\index{Error message!ERRORS DETECTED \ldots}
\index{ERRORS DETECTED IN CON\ldots}

\item[\comp{EXCITED USED WITH TRIPLET} (FATAL)]~\\
\comp{EXCITED} implies the first singlet excited state.  This cannot be used
if \comp{TRIPLET} is requested.  Correct the data set and re-run.
\index{EXCITED USED WITH TRIPLET}
\index{Error message!EXCITED USED \ldots}

\item[\comp{EXTERNAL PARAMETERS FILE MISSING OR EMPTY} (FATAL)]~\\
\comp{EXTERNAL=$text$} has been specified, but the file $text$ is
either missing or empty.
\index{Error message!EXTERNAL PARA\ldots}
\index{EXTERNAL PARAMETERS \ldots}

\item[\comp{FAILED IN SEARCH, SEARCH CONTINUING}]~\\
Not a fatal error.   The  McIver-Komornicki\index{McIver-Komornicki method}
gradient  minimization involves use of a line-search to find the lowest
gradient.  This message is merely advice.  However, if \comp{SIGMA}
\index{SIGMA}  takes a long time,  consider  doing\index{TS}\index{NLLSQ}
something  else,  such  as  using \comp{TS} or  \comp{NLLSQ}, or refining the
geometry a bit before resubmitting it to \comp{SIGMA}.
\index{Error message!FAILED IN SEARCH \ldots} 


\item[\comp{FAILED TO ACHIEVE SCF. }]~\\
The SCF calculation failed to go to  completion;  an  unwanted  and depressing
message that unfortunately appears every so often.
\index{Error message!FAILED TO ACHIEVE SCF} 
\index{FAILED TO ACHIEVE SCF}

To  date  three  unconditional  convergers  have  appeared  in  the
literature:   the  \comp{SHIFT}\index{SHIFT}  technique,   Pulay's  method, and
the Camp-King converger.  It would not  be  fair  to  the  authors  to 
condemn  their methods.   In  MOPAC  all  sorts  of  weird  and  wonderful 
systems are calculated, systems the authors of  the  convergers  never 
dreamed  of. MOPAC  uses  a  combination  of all three convergers at times. 
Normally only a quadratic damper is used.

If this message appears, suspect first that the  calculation  might be  faulty,
then, if you feel confident, use \comp{PL} to monitor a single SCF. Based  on 
the  SCF  results  either  increase  the  number  of  allowed iterations
(default:  200) or use \comp{PULAY}, or Camp-King, or a mixture.

If nothing works, then consider slackening the SCF criterion.  This will  
allow  heats  of  formation  to  be  calculated  with  reasonable precision,
but the gradients are likely to be imprecise.

                  
\item[\comp{Fatal error in reading from channel 9} (FATAL)]~\\
A fatal error has occurred during an attempted restart of a \comp{DRC} or 
\comp{IRC} calculation.  Likely causes are:
\begin{itemize}
\item The restart file does not exist.
\item The restart file is from a different type of job.
\item The restart file was written in FORMATTED or UNFORMATTED code, and an
attempt was made to read it in in the other code.  This is most likely to
happen if MOPAC has been recompiled between runs.
\end{itemize}

\item[\comp{Fatal error in trying to open RESTART file} (FATAL)]~\\
In a \comp{FORCE} calculation, the \comp{RESTART} file is missing. Remove {\bf
RESTART} and re-run.
\index{Error message!Fatal error in try\ldots}
\index{Fatal error in trying to open \ldots}


\item[\comp{FAULT DETECTED IN INTERNAL COORDINATES} (FATAL)]~\\
The nature of the fault in the internal coordinates is described in the
output file immediately before this message.  Correct fault and re-run.
\index{Error message!FAULT DETECTED IN INT\ldots}
\index{FAULT DETECTED IN INT\ldots}

\item[\comp{FAULT IN READ OF AB INITIO DERIVATIVES} (FATAL)]~\\
When \comp{AIDER} is used, the {\em ab initio} derivatives must be supplied
after the Z-matrix.  Correct fault and re-run.
\index{Error message!FAULT IN READ OF AB \ldots}
\index{FAULT IN READ OF \ldots}

\item[\comp{FAULTY LINE: $text$} (FATAL)]~\\
An error was detected during the read of an \htmlref{\comp{EXTERNAL}
parameter set}{external}. An unrecognized parameter type was used.

\begin{latexonly}
See p.~\pageref{external} for details of how to define parameters.
\end{latexonly}
\index{FAULTY LINE: $text$}
\index{Error message!FAULTY LINE: $text$}

\item[\comp{FILE {\em file}.den is missing} (FATAL)]~\\
An attempt has been made to read in an old density matrix ({\em fime}.den), but
this file apparently does not exist in the subdirectory.  Correct the error and
re-run.
\index{FILE {\em file}.den is missing}
\index{Error message!FILE {\em file}.den is missing}

\item[\comp{First atom must not be H if keyword H2O used} (FATAL)]~\\
In \comp{TOM} calculations where \comp{H2O} is used, the first atom must not be
hydrogen.  Hydrogen atoms are assigned values that depend on the type of atom
they are attached to.  Re-arrange the system so that the positions of hydrogen
atoms are defined using other atoms.
\index{Error message!First atom must not be H \ldots}
\index{First atom must not be H \ldots}


\item[\comp{GAUSSIAN INPUT REQUIRES STAND-ALONE JOB} (FATAL)]~\\
Because of the way Gaussian geometries are recognized, only one such geometry
is permitted in any given run, unless \comp{AIGIN} is used.  To correct this
fault, either add \comp{AIGIN} or break the run into parts, and run each part as
a separate job.
\index{Error message!GAUSSIAN INPUT REQ\ldots}
\index{GAUSSIAN INPUT REQ\ldots}


\item[\comp{Geometry cannot be optimized when TOM is used} (FATAL)]~\\
In its current form, the gradients in \comp{TOM} calculations are incorrect. 
The best strategy here is to optimize the geometry first, possibly using {\bf
COSMO}, then do a single SCF using \comp{TOM}.
\index{Error message!Geometry cannot be \ldots}
\index{Geometry cannot be optimized \ldots}


\item[\comp{GEOMETRY CONTAINS FAULTS\ldots} (FATAL)]~\\
Errors in the geometry were detected when the Lewis structure was constructed
in a MOZYME calculation.  Correct the errors using information in the output,
or add \comp{GEO-OK},  and re-run.
\index{GEOMETRY CONTAINS \ldots}
\index{Error message!GEOMETRY CONTAINS \ldots}

\item[\comp{Geometry in PLATO is unrecognizable!} (FATAL)]~\\
This error occurs when a geometry appears to be cubic, but does not belong  to
any of the cubic point groups.  Check the geometry to verify that it is what is
wanted.  If it is, then add \comp{NOSYM} and re-run; this will prevent the 
symmetry routines being used.
\index{Error message!Geometry in PLATO \ldots}
\index{Geometry in PLATO \ldots}

\item[\comp{Geometry is apparently cubic\ldots} (FATAL)]~\\
A severe error.  The geometry has confused the symmetry recognition
subroutines. Most likely, the geometry is nonsense.  Examine the geometry
printed after this message, and take corrective action.  If the geometry is
correct, add \comp{NOSYM}, to disable the symmetry features.
\index{Geometry is apparently cubic\ldots}
\index{Error message!Geometry is apparently cubic\ldots}

\item[\comp{GEOMETRY IS FAULTY} (FATAL)]~\\
The nature of the fault in the geometry is described in the output file
immediately before this message.  Correct fault and re-run.
\index{Error message!GEOMETRY IS FAULTY}
\index{GEOMETRY IS FAULTY}


\item[\comp{GEOMETRY TOO UNSTABLE FOR EXTRAPOLATION \ldots}]~\\
In a reaction path calculation the initial geometry for a point is calculated
by quadratic extrapolation using the previous three points.
\index{Error message!GEOMETRY TOO UN\ldots}
\index{GEOMETRY TOO UNSTABLE \ldots}

If a quadratic fit is likely to lead to an inferior geometry,  then the 
geometry  of  the  last  point  calculated will be used.  The total effect  is 
to  slow  down  the  calculation,  but  no  user  action  is recommended.

\item[\comp{GNORM HAS BEEN SET TOO LOW\ldots}]~\\
By default, the lowest value for \comp{GNORM} is 0.01.  To override this, add
\comp{LET}.  There is no routine need to reduce the \comp{GNORM} below 0.01, and
if \comp{LET} is used, the geometry optimization procedures are modified.
Because of this, \comp{LET} should not be used routinely.
\index{GNORM HAS BEEN SET \ldots}

\item[\comp{GRADIENT IS TOO LARGE TO ALLOW \ldots} (FATAL)]~\\
Before a FORCE calculation can be performed the gradient norm  must be  so
small that the third and higher order components of energy in the force field
are negligible.  If, in the system  under  examination,  the gradient  norm 
is  too  large,  a warning message will be printed and the calculation
stopped,  unless \comp{LET} has been specified.  In  some  cases  the  {\bf
FORCE} calculation  may be run only to decide if a state is a ground state or a
transition  state,  in  which   case   the   results   have   only   two
interpretations.  Under these circumstances, \comp{LET} may be warranted.
\index{Error message!GRADIENT IS TOO LA\ldots}
\index{GRADIENT IS TOO LARGE \ldots}

\item[\comp{GRADIENT IS VERY LARGE \ldots}]~\\
In a calculation of the thermodynamic properties of the system,  if the 
rotation  and  translation vibrations are non-zero, as would be the case if the
gradient norm was significant, then these `vibrations' would interfere  with 
the  low-lying  genuine  vibrations.   The criteria for THERMO  are  much 
more  stringent  than  for  a  vibrational  frequency calculation,  as  it is
the lowest few genuine vibrations that determine the internal vibrational
energy, entropy, etc.
\index{Error message!GRADIENT IS VERY \ldots}
\index{GRADIENT IS VERY LARGE \ldots}

\item[\comp{GREENS FUNCTION IS NOT ALLOWED  WITH EXTENDED SYSTEMS} (FATAL)]~\\
The program has not been set up to allow Greens function corrections to
polymers and other extended systems.  No correction is possible, this kind of
calculation cannot be run.
\index{Error message!GREENS FUNCTION IS \ldots}
\index{GREENS FUNCTION IS NOT \ldots}

\item[\comp{Hydrogen atom $N$ is\ldots} (FATAL)]~\\
In a MOZYME calculation, hydrogen atom $N$ is unreasonably far from any heavy
atom. Check the geometry and re-run.
\index{Hydrogen atom $N$ is\ldots}
\index{Error message!Hydrogen atom $N$ is\ldots}

\item[\comp{Hydrogen atoms must be defined using bond-lengths when keyword H2O is used} (FATAL)]~\\
In \comp{TOM} calculations where \comp{H2O} is used, hydrogen atoms are  assigned
values that depend on the type of atom they are attached to.  Re-arrange the
system so that the positions of hydrogen atoms are defined in terms of their
bond-lengths.
\index{Error message!Hydrogen atoms must be \ldots}

\item[\comp{HYDROGENS MUST BE BONDED TO NON\ldots} (FATAL)]~\\
In a MOZYME calculation, hydrogen atoms should be bonded to non-hydrogen atoms.
If molecular hydrogen is present, add \comp{LET}, otherwise remove the molecular
hydrogen.
\index{HYDROGENS MUST BE \ldots}
\index{Error message!HYDROGENS MUST BE \ldots}

\item[\comp{If QPMEP is used, then PMEP must also be present} (FATAL)]~\\
\comp{QPMEP} is a keyword that modifies a \comp{PMEP} (Parametric Molecular
Electrostatic Potential) calculation.  On its own, \comp{QPMEP} will do nothing
useful.
\index{Error message!If QPMEP is used, \ldots}
\index{If QPMEP is used, then  \ldots}

\item[\comp{ILLEGAL ATOMIC NUMBER} (FATAL)]~\\
An element has been specified by an atomic number which is  not  in the  range 
1  to  107.   Check the data:  the first datum on one of the lines is faulty. 
Most likely line 4 is faulty.
\index{Error message!ILLEGAL ATOMIC\ldots}
\index{ILLEGAL ATOMIC NUMBER}
                  
\item[\comp{IMPOSSIBLE NUMBER OF CLOSED SHELL ELECTRONS} (FATAL)]~\\
The keywords used imply that the number of closed shells (doubly-occupied
levels) is less than zero!  Correct the error in the data set, and re-run.
\index{Error message!IMPOSSIBLE NUMBER \ldots}
\index{IMPOSSIBLE NUMBER OF \ldots}

\item[\comp{IMPOSSIBLE NUMBER OF OPEN SHELL ELECTRONS} (FATAL)]~\\
The keyword \comp{OPEN($n1$,$n2$)}\index{OPEN} has been used,   but  for  an 
even-electron system  $n1$  was  specified  as  odd or for an odd-electron
system $n1$ was specified as even.  Either way, there is a conflict which the
user  must resolve.
\index{Error message!IMPOSSIBLE NUMBER \ldots} 
\index{IMPOSSIBLE NUMBER OF \ldots} 

\item[\comp{IMPOSSIBLE OPTION REQUESTED} (FATAL)]~\\
A  general  catch-all.   This  message  will  be  printed  if   two
incompatible  options  are  used,  such  as  both   \comp{MINDO/3} and \comp{AM1}
being specified.  Check the keywords, and resolve the conflict.
\index{Error message!IMPOSSIBLE OPTION\ldots}
\index{IMPOSSIBLE OPTION REQ\ldots}

\item[\comp{IMPOSSIBLE VALUE OF DELTA S} (FATAL)]~\\
The keywords used imply that either the number of $\alpha$ or the number
of $\beta$ electrons is negative!  Correct error and re-run.
\index{Error message!IMPOSSIBLE VALUE OF \ldots}
\index{IMPOSSIBLE VALUE \ldots}

\item[\comp{INPUT FILE MISSING OR EMPTY} (FATAL)]~\\
The data set is either empty or does not exist, or MOPAC has not been
installed correctly.  Correct error and re-run.
\index{INPUT FILE MISSING \ldots}
\index{Error message!INPUT FILE MISSING OR \ldots}

\item[\comp{INSUFFICIENT DATA ON DISK FILES FOR A FORCE} (FATAL)]~\\
A \comp{FORCE} calculation has been attempted using \comp{RESTART}, however,
the $<$filename$>$.res file is either corrupt or does not exist.
The best course of action would be to start over from the beginning---that
is, remove \comp{RESTART}, and re-run the job.
\index{INSUFFICIENT DATA ON \ldots}
\index{Error message!INSUFFICIENT DATA ON \ldots}

\item[\comp{JOB STOPPED BY OPERATOR}]~\\
Any MOPAC calculation, for which the \comp{shut} command works, can be stopped 
by  a  user  who issues the command \comp{shut $<$filename$>$},  from the
directory which contains \comp{ $<$filename$>$.dat}.
\index{JOB STOPPED BY OPERATOR}
\index{Error message!JOB STOPPED BY\ldots}

MOPAC will then stop the calculation at the first convenient point, usually 
after  the  current cycle has finished.  A restart file will be written and the
job ended.  The message will be printed as soon as it is detected, which would
be the next time the timer routine is accessed.
                    
\item[\comp{KEYWORD AIDER SPECIFIED, BUT NOT PRESENT AFTER Z-MATRIX.  JOB STOPPED} (FATAL)]~\\
When \comp{AIDER} is used, the {\em ab initio} derivatives must be supplied
after the Z-matrix.  Correct fault and re-run.
\index{KEYWORD AIDER \ldots}
\index{Error message!KEYWORD AIDER SPECIF\ldots} 

\item[\comp{KEYWORD ANALYT CANNOT BE USED HERE} (FATAL)]~\\
Analytical derivatives (which use STO$n$ Gaussian orbitals) cannot be
used with RHF open shell derivatives calculated using Liotard's method.
To correct this, either add \comp{NOANCI} or remove \comp{ANALYT}.
\index{KEYWORD ANALYT \ldots}
\index{Error message!KEYWORD ANALYT \ldots}

\item[\comp{LINE OF KEYWORDS DOES NOT HAVE ENOUGH  SPACES FOR PARSING. PLEASE CORRECT LINE.} (FATAL)]~\\
Every keyword must be preceded by a space.  This applies to the first keyword.
As supplied, the keywords line does not have a space before the first keyword,
and there is no space to move the keywords around in order to put a space there.
\index{LINE OF KEYWORDS \ldots}
\index{Error message!LINE OF KEYWORDS \ldots}

Delete or abbreviated keywords so that there are unused spaces.

\item[\comp{MAX.\ NUMBER OF ATOMS ALLOWED: \ldots} (FATAL)]~\\
To correct this, edit the file \comp{sizes.h} to increase the value of 
\comp{NUMATM}, re-run ``make'' and re-run the job.
\index{MAX. NUMBER OF ATOMS \ldots}
\index{Error message!MAX. NUMBER OF AT\ldots}
                      
\item[\comp{MAXIMUM ALLOWED NUMBER OF ANIONS EXCEEDED} (FATAL)]~\\
In a routine MOZYME calculation, the maximum number of anions  allowed is 200.
If there are more than 200, add \comp{LET}, however, first check that the data 
set is correct.
\index{MAXIMUM ALLOWED \ldots}
\index{Error message!MAXIMUM ALLOWED \ldots}

\item[\comp{MAXIMUM ALLOWED NUMBER OF CATIONS EXCEEDED} (FATAL)]~\\
In a routine MOZYME calculation, the maximum number of cations allowed is 200.
If there are more than 200, add \comp{LET}, however, first check that the data set
is correct.
\index{MAXIMUM ALLOWED \ldots}
\index{Error message!MAXIMUM ALLOWED \ldots}

\item[\comp{MICROSTATES SPECIFIED BY KEYWORDS  BUT MISSING FROM DATA} (FATAL)]~\\
If \comp{MICROS=$n$} is present, then after the geometry and symmetry
data, if any, there should be a line with the word
\htmlref{\comp{MICRO}}{rs} followed by $n$ microstates.

\begin{latexonly}
See p.~\pageref{rs} for examples
\end{latexonly}.
\index{MICROSTATES SPECIFIED \ldots}
\index{Error message!MICROSTATES SPEC\ldots}

\item[\comp{MISSING VAN DER WAALS RADIUS  $Chemical-symbol$} (FATAL)]~\\
In the \comp{COSMO} method and in generating Lewis structures, van der Waals
radii are used.  If a VDW radius is missing, it can be  supplied by use of
\hyperref[pageref]{\comp{VDW($text$)}}{, see p.~}{}{key_vdw}.
\index{MISSING VAN DER WAALS \ldots}
\index{Error message!MISSING VAN DER \ldots}

\item[\comp{MIXED PARAMETER SETS.  USE ''PARASOK'' TO CONTINUE} (FATAL)]~\\
A calculation using mixed parameter sets has been attempted.  By default,
MNDO parameters are used if the element has not been parameterized for
the specified method.  Use of such mixed parameter sets is not usually
recommended.  To allow such sets to be used, add \comp{PARASOK}.
\index{MIXED PARAMETER SETS.  \ldots}
\index{Error message!MIXED PARAMETER \ldots}

\item[\comp{MODIFY SUBROUTINE NAMES} (FATAL)]~\\
The MOZYME function is designed to work with proteins of up to 2,000 residues,
about 28,000 atoms.  Larger systems can be run, provided no attempt is made
to work out the residue sequence.  This means that \comp{RESIDUES}, \comp{PDBOUT}, and
\comp{RESEQ} cannot be used.  If larger systems need to be run, modify \comp{MAXRES}
in ``atomrs.F'' and in ``names.F''.
\index{MODIFY SUBROUTINE NAMES}
\index{Error message!MODIFY SUBROUTINE \ldots}

\item[\comp{MORE THAN 3N-6 COORDINATES OPTIMIZED!} (FATAL)]~\\
In an \comp{EF} calculation, more than 3$N$-6 coordinates are flagged for
optimization.  By implication, at least one root of the Hessian matrix is
exactly zero.  Since \comp{EF} involves using the inverse of an approximate
Hessian, the method is intrinsically unstable.  However, for most systems,
the geometry optimizes completely before the Hessian is accurate enough for
the instability to cause problems.  
\index{MORE THAN 3N-6 COORD\ldots}
\index{Error message!MORE THAN 3N-6 COORD\ldots}

To correct this fault, either reduce the number of coordinates being optimized,
or add \comp{GEO-OK}.

\item[\comp{MORE THAN ONE GEOMETRY OPTION HAS BEEN SPECIFIED} (FATAL)]~\\
The keywords indicate that two or more geometric operations have been
requested for one system.  Only one operation (e.g.\ geometry optimization by
\comp{TS} or by \comp{NLLSQ}) is allowed for any given system.
\index{MORE THAN ONE GEO\ldots}
\index{Error message!MORE THAN ONE GEO\ldots}

\item[\comp{MULLIKEN POPULATION NOT AVAILABLE WITH UHF} (FATAL)]~\\
The requested operation, a Mulliken population analysis using a UHF wavefunction,
has not been written in the program.  This type of calculation will not run.
\index{MULLIKEN POPULATION \ldots}
\index{Error message!MULLIKEN POPULATION \ldots}

\item[\comp{NAME NOT FOUND} (FATAL)]~\\
Various atomic parameters can  be  modified  in  MOPAC  by  use  of 
\comp{EXTERNAL=}.  These comprise the symbols given in Table~\ref{expar}.
\index{NAME NOT FOUND}
\index{Error message!NAME NOT FOUND}

\begin{table}
\begin{center}
\caption{\label{expar} Names of Parameters for use with \comp{EXTERNAL=$<$file$>$}}
\begin{tabular}{llll}\\
          Uss    &    Betas    &    Gp2    &     GSD   \\
          Upp    &    Betap    &    Hsp    &     GPD   \\
          Udd    &    Betad    &    AM1    &     GDD   \\
          Zs     &    Gss      &    Expc   &     FN1   \\
          Zp     &    Gsp      &    Gaus   &     FN2   \\
          Zd     &    Gpp      &    Alp    &     FN3   \\
\end{tabular}
\end{center}
\end{table}
         
Thus to change the Uss of hydrogen to $-13.6$ the line \comp{USS    H    -13.6}
could be used.  If an attempt is made to modify  any  other  parameters, then
an error message is printed, and the calculation terminated.

\item[\comp{NEGATIVE SYMBOLICS MUST BE PRECEDED BY THE POSITIVE EQUIVALENT} (FATAL)]~\\
When specifying GAUSSIAN Z-matrix geometries, a negative symbolic must be
related to an already defined positive symbolic.  Correct the data set and
re-run.
\index{NEGATIVE SYMBOLICS \ldots}
\index{Error message!NEGATIVE SYMBOLICS \ldots}

\item[\comp{NEWGEO CAN ONLY BE RUN IF THERE ARE NO DUMMY ATOMS, OR IF `RESEQ' IS USED} (FATAL)]~\\
In a protein, \comp{NEWGEO} converts backbone atoms to Cartesian, and 
side-chain atoms  to internal coordinates.  For this function to work
correctly, there must  not be any dummy  atoms.  Dummy atoms can be removed by
\comp{RESEQ}, or by a  \comp{0SCF} + \comp{INT} run.
\index{NEWGEO CAN ONLY \ldots}
\index{Error message!NEWGEO CAN ONLY \ldots}

\item[\comp{NLLSQ USED WITH REACTION PATH;} (FATAL)]~\\
The capability of using \comp{NLLSQ} with a reaction path is not available
within MOPAC. \ As an alternative, use \comp{TS}.
\index{NLLSQ USED WITH \ldots}
\index{Error message!NLLSQ USED WITH \ldots}

\item[\comp{NO ATOMS IN SYSTEM} (FATAL)]~\\
The system provided does not contain any atoms!  Check the data-set.  A common
error is to have a blank line before the keyword line.  There should be
exactly three lines before the Z-matrix, unless `+' is used.
\index{NO ATOMS IN SYSTEM}
\index{Error message!NO ATOMS IN SYSTEM}

\item[\comp{NO DUMMY ATOMS ALLOWED BEFORE} (FATAL)]~\\
In a \comp{FORCE} calculation on a polymer or solid, no dummy atoms are
allowed.  This is a program limitation.  Modify the data-set (use \comp{0SCF}
and \comp{INT} or \comp{XYZ} to get rid of the dummy atoms), and re-run.
\index{NO DUMMY ATOMS \ldots}
\index{Error message!NO DUMMY ATOMS \ldots}

\item[\comp{NO PEPTIDE LINKAGES FOUND.  CHECK DATA SET} (FATAL)]~\\
At least one of the following keywords: \comp{RESIDUES}, \comp{PDBOUT}, and
\comp{RESEQ} was used on a system that had no peptide linkages.  Modify the
data set and re-run.
\index{NO PEPTIDE LINKAGES \ldots}
\index{Error message!NO PEPTIDE LINK\ldots}


\item[\comp{NO POINTS SUPPLIED FOR REACTION PATH} (FATAL)]~\\
A reaction path calculation is indicated by a `--1' in the optimization
flags.  If \comp{STEP=$n$} and \comp{POINT=$m$} are present, then the
reaction path is defined by $n$ and $m$.  If these keywords are not
present, \htmlref{the reaction path must be specified by numbers after
the Z-matrix and symmetry data}{sn2} (if any). 
\begin{latexonly}
See p.~\pageref{sn2} for an example.
\end{latexonly}
\index{NO POINTS SUPPLIED \ldots}
\index{Error message!NO POINTS SUPPLIED \ldots}

\item[\comp{NO RESTART FILE EXISTS!} (FATAL)]~\\
An attempt has been made to restart a job, but the \comp{$<$filename$>$.res}
file does not exist.  The easiest correction is to remove \comp{RESTART}
and re-run.
\index{NO RESTART FILE EXISTS!}
\index{Error message!NO RESTART FILE EXISTS!}


\item[\comp{NOANCI MUST BE USED FOR RHF OPEN-SHELL SYSTEMS THAT INVOLVE TRANSLATION VECTORS} (FATAL)]~\\
Liotard's analytical RHF open shell derivatives have not been extended to allow
polymers or solids to be calculated.  An alternative is to use \comp{UHF}.
\index{NOANCI MUST BE USED \ldots}
\index{Error message!NOANCI MUST BE USED \ldots}

\item[\comp{NONET SPECIFIED WITH ODD NUMBER OF ELECTRONS, CORRECT FAULT} (FATAL)]~\\
When \comp{NONET} is specified, the system must have an even number of  
electrons.  Check (a) the system, and (b) the charge (if any).  Correct 
data set and re-run. 
\index{NONET SPECIFIED \ldots}
\index{Error message!NONET SPECIFIED \ldots}
 
\item[\comp{NUMBER OF DOUBLY FILLED PLUS PARTLY FILLED LEVELS GREATER THAN TOTAL NUMBER OF ORBITALS} (FATAL)]~\\
The keywords used here imply a system that is larger than that used.
Correct data set (probably by changing the keywords) and re-run.
\index{NUMBER OF DOUBLY \ldots}
\index{Error message!NUMBER OF DOUBLY \ldots}

\item[\comp{NUMBER OF ELECTRONS IN M.O.s BELOW ACTIVE SPACE IS LESS THAN ZERO} (FATAL)]~\\
In a C.I.\ calculation, the active space extends below the lowest energy level.
Correct data set (probably by changing the keywords) and re-run. 
\index{NUMBER OF ELECTRONS \ldots}
\index{Error message!NUMBER OF ELEC\ldots}

\item[\comp{NUMBER OF M.O.s IN ACTIVE SPACE EXCEEDS MAXIMUM ALLOWED SIZE OF ACTIVE SPACE} (FATAL)]~\\
In a C.I.\ calculation, the number of M.O.s in the active space is greater
than that allowed by the program.
Modify the data set (probably by changing the keywords) and re-run. 
\index{NUMBER OF M.O.s IN  \ldots}
\index{Error message!NUMBER OF M.O.s in \ldots}

\item[\comp{NUMBER OF M.O.s REQUESTED IN C.I.\ IS GREATER THAN THE NUMBER OF ORBITALS} (FATAL)]~\\
In a C.I.\ calculation, the active space requested is greater than the
number of orbitals in the system.
Correct data set (probably by changing the keywords) and re-run. 
\index{NUMBER OF M.O.s REQ\ldots}
\index{Error message!NUMBER OF M.O.s REQ\ldots}

\item[\comp{NUMBER OF OCCUPIED ORBITALS IS INCORRECT IN MAKVEC} (FATAL)]~\\
During the construction of the starting LMOs in a MOZYME calculation, more occupied
LMOs were found than expected.  This is a programming problem.
Please inform Dr~Stewart as soon as possible.
\index{NUMBER OF OCCUPIED \ldots}
\index{Error message!NUMBER OF OCCUPIED \ldots}


\item[\index{NUMBER OF OPEN-SHELLS \ldots} (FATAL)]~\\
\comp{NUMBER OF OPEN-SHELLS ALLOWED IN C.I.\ IS LESS THAN 
THAT SPECIFIED BY OTHER KEYWORDS}
The size of the active space in a C.I.\ calculation implied by \comp{C.I.=($n$,$m$)}
is less than that implied by other keywords, e.g., \comp{SEXTET}.
Correct data set by changing the keywords and re-run.
\index{Error message!NUMBER OF OPEN-S\ldots}

\item[\comp{NUMBER OF PARTICLES, $nn$ GREATER THAN \ldots}]~\\
When user-defined microstates are not used, the MECI will calculate all 
possible  microstates  that  satisfy the space and spin constraints imposed. 
This is done in PERM, which permutes $N$ electrons in $M$  levels. If  $N$ is
greater than $M$, then no possible permutation is valid.  This is not a fatal
error---the program will continue to run, but  no  C.I.\ will be done.
\index{NUMBER OF PARTICLES \ldots}
\index{Error message!NUMBER OF PART\ldots}
                  
\item[\comp{NUMBER OF PERMUTATIONS TOO GREAT, LIMIT }{\em nnnn} (FATAL)]~\\
Unless the file \comp{meci.h} is changed, the number of permutations of  alpha
or beta microstates is  limited to 4$\times$\comp{MAXCI} or 4800.  Thus  if 3
alpha electrons are permuted among 5 M.O.s, that will generate $10 = 5!/(3!2!)$
alpha microstates, which is an allowed  number. However  if 7 alpha electrons
are permuted among 15 M.O.s, then 6435 alpha microstates result and the arrays
defined will  be  insufficient.    To correct this error, increase \comp{MAXCI}
in \comp{meci.h} and recompile.
\index{NUMBER OF PERM\ldots}
\index{Error message!NUMBER OF PERM\ldots}


\item[\comp{NUMERICAL PROBLEMS IN BRACKETING LAMBDA}]~\\
Although this is not a deadly error, it does indicate that there are potential
problems in optimizing geometries.  If the run finishes correctly, don't worry.
If the geometry is not optimized sufficiently, try one or more of the following
strategies:
\index{NUMERICAL PROBLEMS \ldots}
\index{Error message!NUMERICAL PROBLEMS\ldots}
\begin{itemize}
\item Use \comp{LET}.  This allows more of the potential energy surface to
be sampled, thus giving more information to the Hessian.
\item Tighten up the SCF criterion.  Try \comp{RELSCF=0.1} or \comp{RELSCF=0.01}.
\item If the calculation involves an open shell RHF, consider running it with 
\comp{UHF}.
\item Carefully examine the data set---is there any possibility that it is
faulty?
\item Go to a different coordinate system.  If Cartesian, go to internal
coordinates, and {\em vice versa}.
\end{itemize}

\item[\comp{OCCUPIED C VECTOR DAMAGED IN SELMOS} (FATAL)]~\\
In a MOZYME calculation, when \comp{RAPID} is used (in a partial geometry 
optimization), the occupied M.O.\ coefficients were found to be corrupt.
Please inform Dr~Stewart as soon as possible.
\index{OCCUPIED C VECTOR \ldots}
\index{Error message!OCCUPIED C VECTOR \ldots}

\item[\comp{OCCUPIED IC VECTOR DAMAGED IN SELMOS} (FATAL)]~\\
In a MOZYME calculation, when \comp{RAPID} is used (in a partial geometry 
optimization), the list of atoms in the occupied M.O.\ was found to be corrupt.
Please inform Dr~Stewart as soon as possible.
\index{OCCUPIED IC VECTOR \ldots}
\index{Error message!OCCUPIED IC VECTOR \ldots}

\item[\comp{OCTET SPECIFIED WITH EVEN NUMBER OF ELECTRONS, CORRECT FAULT} (FATAL)]~\\
When \comp{OCTET} is specified, the system must have an odd number of
electrons.  Check (a) the system, and (b) the charge (if any).  Correct
data set and re-run.
\index{OCTET SPECIFIED \ldots}
\index{Error message!OCTET SPECIFIED \ldots}

\item[\comp{OLDENS FILE FOR \emph{file}.den IS CORRUPT} (FATAL)]~\\
Although a file called \emph{file}.den exists, its contents do not match the data
set supplied for the calculation.  The easiest option is to delete \comp{OLDENS}
and re-run.  Note: \comp{OLDENS} cannot be used to switch from a conventional to a
MOZYME run, or {\em vice versa}.
\index{OLDENS FILE FOR $file$.den \ldots}
\index{Error message!OLDENS FILE \ldots}

\item[\comp{OLDGEO used and previous geometry had no atoms.} (FATAL)]~\\
For \comp{OLDGEO} to work, an earlier calculation in the same data set must exist,
and must contain at least one atom.  Correct fault and re-run.
\index{OLDGEO used and \ldots}
\index{Error message!OLDGEO used and \ldots}

\item[\comp{OMIN MUST BE BETWEEN 0 AND 1} (FATAL)]~\\
Keyword \comp{OMIN=$n$} has been used, with an unreasonable value for $n$.
Modify $n$ and re-run.
\index{OMIN MUST BE \ldots}
\index{Error message!OMIN MUST BE \ldots}

\item[\comp{ONLY $N$ POINTS ALLOWED IN REACTION COORDINATE} (FATAL)]~\\
The maximum number of points on a reaction coordinate is three times the maximum
number of atoms allowed.  To increase this number, edit the file \comp{sizes.h} 
to increase the value of \comp{NUMATM}, re-run ``make'' and re-run the job.
\index{ONLY $N$ POINTS \ldots}
\index{Error message!ONLY $N$ POINTS \ldots}

\item[\comp{ONLY DIHEDRAL SYMBOLICS CAN BE PRECEEDED BY A `-' SIGN} (FATAL)]~\\
Symmetry relationships are allowed when a geometry is read in in Gaussian format, 
however the range of relationships is limited to setting bond lengths equal,
setting bond angles equal, setting dihedrals equal, and setting dihedrals
equal to the negative of a reference dihedral.  An attempt has been made to set
a bond length or a bond angle to the negative of a reference.  This is not
allowed.  Correct the data set and re-run.
\index{ONLY DIHEDRAL SYMBOL\ldots}
\index{Error message!ONLY DIHEDRAL SYM\ldots}

\item[\comp{Only H, C, N, O, F, and S allowed} (FATAL)]~\\
The TOMASI model has parameters for H, C, N, O, F, and S only.  This
model will not work for systems that contain other elements.  Try using
COSMO?
\index{Only H, C, N, O, F, and S allowed}
\index{Error message!Only H, C, N, O, F, \ldots}
 
\item[\comp{ONLY ONE REACTION COORDINATE PERMITTED} (FATAL)]~\\
In a reaction coordinate calculation, only one optimization flag can be
set to ``-1'', all others must either be ``0'' or ``1''.  In order for
two optimization flags to be set to ``-1'', \htmlref{a grid calculation
must be run}{step}.  This involves  keywords \comp{STEP1=$n.nn$} and
\comp{STEP2=$m.mm$}.  
\begin{latexonly}
For details on how to do this, see p.~\pageref{step}.
\end{latexonly}
\index{ONLY ONE REACTION \ldots}
\index{Error message!ONLY ONE REACTION\ldots}

\item[\comp{PANIC: Temporary array being destroyed\ldots}]~\\
Although this is not a fatal error, it does indicate a severe programming
error  in the program. Each time a temporary array is destroyed, a check is
made to verify that the correct amount of memory is freed up.  If the check
fails, this message will be printed.  Please inform Dr~Stewart as soon as
possible.
\index{PANIC: Temporary array \ldots}

\item[\comp{PARAMETERS FOR SOME ELEMENTS MISSING} (FATAL)]~\\
An attempt has been made to run a calculation on a system that contains
atoms for which there are no parameters.  Check that the method you are
using \hyperref[pageref]{has been parameterized}{ (p.~}{)}{mndoel}.  If
this bug occurs during a port of MOPAC, it is likely to be caused by
the BLOCKDATA file not being used by the compiler.  One way around this
fault is to paste the file \comp{block.F} at the end of the file
\comp{mopac.F}, and then delete \comp{block.F}.  This usually corrects
this problem.
\index{PARAMETERS FOR SOME \ldots}
\index{Error message!PARAMETERS FOR \ldots}

\item[\comp{PERMANENT INTEGER ARRAY $name$ DAMAGED} (FATAL)]~\\
All permanent arrays are separated by a single datum, that acts as a
sensor.  If this datum is changed, there is a high probability that
part or all of the array has been overwritten, as the result of an
error in programming.  Please inform Dr~Stewart as soon as possible.
\index{PERMANENT INTEGER \ldots}
\index{Error message!PERMANENT INTEGER \ldots}

\item[\comp{PERMANENT REAL ARRAY $name$ DAMAGED} (FATAL)]~\\
All permanent arrays are separated by a single datum, that acts as a sensor.
If this datum is changed, there is a high probability that part or all of the array
has been overwritten, as the result of an error in programming.  Please inform 
Dr~Stewart as soon as possible.
\index{PERMANENT REAL ARRAY \ldots}
\index{Error message!PERMANENT REAL ARRAY}

\item[\comp{POLAR does not work with open-shell RHF} (FATAL)]~\\
The \comp{POLAR} method only works with RHF closed shell systems.  If only the
polarizability is needed, use \comp{STATIC} instead of \comp{POLAR}.
\index{POLAR does not work with \ldots}
\index{Error message!POLAR does not work \ldots}

\item[\comp{POLAR does not work with UHF} (FATAL)]~\\
The \comp{POLAR} calculation only works with Restricted Hartree Fock calculations,
both closed shell and open shell, and both ground and excited states.
If only the polarizability is needed, use \comp{STATIC}; this uses external
fields and works with \comp{UHF}, however it is not as precise as \comp{POLAR}.
\index{POLAR does not work with \ldots}
\index{Error message!POLAR does not work \ldots}

\item[\comp{POLAR has not been implemented with MINDO} (FATAL)]~\\
\comp{POLAR} only works with the NDDO methods, MNDO, AM1, PM3, and MNDO-$d$.
If only the polarizability is needed, use \comp{STATIC}; this uses external
fields and works with \comp{MINDO/3}, however it is not as precise as \comp{POLAR}.

\item[\comp{PROBLEM IN SYMR} (FATAL)]~\\
There is a problem in subroutine \comp{SYMR}.  The system has symmetry, but
small distortions are preventing the symmetry operations from being done
correctly---the subroutine has become confused.  To correct this, add
\hyperref[pageref]{\comp{NOSYM}}{, p.~}{}{nosym}.
\index{PROBLEM IN SYMR}
\index{Error message!PROBLEM IN SYMR}

\item[\comp{QUARTET SPECIFIED WITH EVEN NUMBER} (FATAL)]~\\
When \comp{QUARTET} is specified, the system must have an odd number of 
electrons.  Check (a) the system, and (b) the charge (if any).  Correct
data set and re-run. 
\index{QUARTET SPECIFIED \ldots}
\index{Error message!QUARTET SPECIFIED \ldots}

\item[\comp{QUINTET SPECIFIED WITH ODD NUMBER OF ELECTRONS, CORRECT FAULT} (FATAL)]~\\
When \comp{QUINTET} is specified, the system must have an even number of  
electrons.  Check (a) the system, and (b) the charge (if any).  Correct 
data set and re-run. 
\index{QUINTET SPECIFIED \ldots}
\index{Error message!QUINTET SPECIFIED \ldots}
 
\item[\comp{Ran out of storage for $array name$} (FATAL)]~\\
The amount of dynamic memory reserved for storing arrays is less than that
required by the program.  This is a programming bug.  The strategy to follow
to identify the fault is:
\index{Ran out of storage for $array name$}
\index{Error message!Ran out of storage \ldots}
\begin{itemize}
\item Re-run the calculation, but add \comp{SIZES}.  This will print out the
amount of memory reserved for each array, and the arrays created and destroyed,
and the arrays that exist when the run stopped.
\item Determine whether the error is in the integer or real arrays.
\item  Check the total memory reserved for the permanent arrays.  This should
be slightly greater than the total actually used.  If it is less, then a change
needs to be made to \comp{setupr.F} or \comp{setupi.F}, to include the missing 
array.
\item Check the total memory reserved for the temporary arrays.  This should be
slightly greater than the total actually used.  If it is less, identify which
array(s) exist at the end of the calculation, for which memory was not reserved
at the start.  Make the appropriate correction to \comp{tmpi.F} (if the array 
is integer), \comp{tmpmr.F} (if real, and conventional), or \comp{tmpzr.F} (if
real, and MOZYME).
\end{itemize}
A temporary correction to fix this problem is to add \comp{RMEM=$nnn$} (if the 
shortage is in real memory) or  \comp{IMEM=$nnn$} (is the shortage is in integer
memory), where $nnn$ is the amount of memory (as array elements, not bytes)
needed. Please inform Dr~Stewart as soon as possible.

\item[\comp{RESEQ cannot be used with XYZ} (FATAL)]~\\
The pair of keywords \comp{RESEQ} and \comp{XYZ} is not supported.  However, the
effect of these two keywords can be reproduced in a two stage calculation 
by using \comp{RESEQ} in the first calculation and \comp{OLDGEO}, \comp{XYZ}, 
and \comp{0SCF} in the second calculation.

\item[\comp{RESTART FILE EXISTS, BUT IS CORRUPT} (FATAL)]~\\
In a \comp{BFGS} run involving \comp{RESTART}, the file $<\!$filename$\!>$.res, 
is corrupt.  Remove \comp{RESTART}  and re-run.          
\index{RESTART FILE EXISTS, \ldots}
\index{Error message!RESTART FILE EXISTS, \ldots}
 
\item[\comp{RESTART FILE EXISTS, BUT IS FAULTY} (FATAL)]~\\
In a \comp{FORCE} calculation, involving \comp{RESTART}, the file 
$<\!$filename$\!>$.res, is corrupt.  Remove \comp{RESTART}  and re-run.       
\index{RESTART FILE EXISTS, \ldots}
\index{Error message!RESTART FILE EXISTS, \ldots}
 
\item[\comp{Restart file is corrupt!} (FATAL)]~\\
In a run involving \comp{RESTART}, the file $<\!$filename$\!>$.res, is corrupt.  
Remove \comp{RESTART}  and re-run.       
\index{Restart file is corrupt!}
\index{Error message!Restart file is corrupt!}
 
\item[\comp{ROOT REQUESTED DOES NOT EXIST IN C.I.} (FATAL)]~\\
A specific excited state has been specified, but it does not exist in the 
set of states calculated.  Correct the data set and re-run.
\index{ROOT REQUESTED DOES \ldots}
\index{Error message!ROOT REQUESTED \ldots}
 
\item[\comp{RSOLV MUST NOT BE NEGATIVE} (FATAL)]~\\
In COSMO calculations, the radius of a solvent molecule must be positive.
Correct dat set and re-run.
\index{RSOLV MUST NOT BE \ldots}
\index{Error message!RSOLV MUST NOT BE \ldots}
 
\item[\comp{Run stopped because keyword LEWIS was used}]~\\
\comp{LEWIS} is only used when a description of the Lewis structure is wanted.
This can be done very rapidly, provided no quantum mechanical calculations are
run.  Therefore, when \comp{LEWIS} is used, the calculation is stopped immediately
after the Lewis structure is printed.  If the calculation is to be continued,
remove \comp{LEWIS}.
\index{Run stopped because keyword LEWIS was used}

\item[\comp{SEPTET SPECIFIED WITH ODD NUMBER OF ELECTRONS, CORRECT FAULT} (FATAL)]~\\
When \comp{SEPTET} is specified, the system must have an even number of
electrons.  Check (a) the system, and (b) the charge (if any).  Correct
data set and re-run.
\index{SEPTET SPECIFIED \ldots}
\index{Error message!SEPTET SPECIFIED \ldots}
  
\item[\comp{SETUP FILE MISSING, EMPTY OR CORRUPT} (FATAL)]~\\
In a run involving \comp{SETUP}, the \comp{SETUP} file  is either missing
or is corrupt.  Either create a valid \comp{SETUP} file or remove \comp{SETUP}
from the data set, and re-run.
\index{SETUP FILE MISSING \ldots}
\index{Error message!SETUP FILE MISSING, \ldots}
 
\item[\comp{SEXTET SPECIFIED WITH EVEN NUMBER OF ELECTRONS, CORRECT FAULT} (FATAL)]~\\
When \comp{SEXTET} is specified, the system must have an odd number of  
electrons.  Check (a) the system, and (b) the charge (if any).  Correct 
data set and re-run. 
\index{SEXTET SPECIFIED \ldots}
\index{Error message!SEXTET SPECIFIED \ldots}
 
\item[\comp{SIGMA USED WITH REACTION PATH;} (FATAL)]~\\
The only geometry options allowed with reaction paths are \comp{EF} and
the default BFGS optimizers.  Delete \comp{SIGMA} and re-run.
\index{SIGMA USED WITH \ldots}
\index{Error message!SIGMA USED \ldots}
 
\item[\comp{Size of active space allowed: $N$} (FATAL)]~\\
The largest active space in a C.I.\ calculation is $N$.  If a larger size is
needed, edit \comp{meci.h} to increase the value of \comp{NMECI}, re-run ``make'',
and re-run the job.
\index{Size of active space allowed: $N$}
\index{Error message!Size of active \ldots}

\item[\comp{SOME ELEMENTS HAVE BEEN SPECIFIED FOR WHICH NO PARAMETERS ARE AVAILABLE\ldots} (FATAL)]~\\
Parameters are not available for the element with atomic number $N$. If
new parameters are available, these can be supplied to MOPAC by use of
\hyperref[pageref]{\comp{EXTERNAL=}}{, see p.~}{}{external}.
\index{SOME ELEMENTS HAVE \ldots}
\index{Error message!SOME ELEMENTS HAVE \ldots}

\item[\comp{Something disastrous has happened} (FATAL)]~\\
In a MOZYME calculation, something has gone wrong in an SCF calculation.
The effect is to increase the heat of formation suddenly by over 200 
kcal.mol$^{-1}$.  The job will automatically shut down, and can subsequently
be restarted.  In the unlikely event that the shutdown sequence fails to start,
the geometry will be printed, and the calculation stopped.
\index{Something disastrous \ldots}
\index{Error message!Something disaster\ldots}

Re-start the calculation, either using \comp{RESTART} or the final geometry
printed, but do {\em not} use \comp{OLDENS}.

\item[\comp{SPECIFIED SPIN COMPONENT NOT SPANNED BY ACTIVE SPACE} (FATAL)]~\\
A spin state has been specified, but the active space is too small to
allow that state to exist.  Correct the data set and re-run.
\index{SPECIFIED SPIN COMP\ldots}
\index{Error message!SPECIFIED SPIN \ldots}
 

\item[\comp{Storage needed for next step: $NNN$} (FATAL)]~\\
The amount of dynamic memory reserved for storing arrays is less than that
required by the program.  This is a programming bug.  The strategy to follow
to identify the fault is:
\index{Storage needed for next \ldots}
\index{Error message!Storage needed for \ldots}
\begin{itemize}
\item Re-run the calculation, but add \comp{SIZES}.  This will print out the
amount of memory reserved for each array, and the arrays created and destroyed,
and the arrays that exist when the run stopped.
\item Determine whether the error is in the integer or real arrays.
\item  Check the total memory reserved for the permanent arrays.  This should be
slightly greater than the total actually used.  If it is less, then a change
needs to be made to \comp{setupr.F} or \comp{setupi.F}, to include the missing
array.
\item Check the total memory reserved for the temporary arrays.  This should be
slightly greater than the total actually used.  If it is less, identify which
array(s) exist at the end of the calculation, for which memory was not reserved
at the start.  Make the appropriate correction to \comp{tmpi.F} (if the array
is integer), \comp{tmpmr.F} (if real, and conventional), or \comp{tmpzr.F} (if 
real, and MOZYME).
\end{itemize}
A temporary correction to fix this problem is to add \comp{RMEM=$nnn$} (if the 
shortage is in real memory) or 
\comp{IMEM=$nnn$} (is the shortage is in integer memory), where $nnn$ is the amount
of memory (as array elements, not bytes) needed.
Please inform Dr~Stewart as soon as possible.

\item[\comp{STRUCTURE UNRECOGNIZABLE}]~\\
A problem was encountered while trying to identify the amino acid residues in 
a protein. Although not deadly, this represents a fault in the program.
Please inform Dr~Stewart as soon as possible.
\index{STRUCTURE UNRECOGNIZABLE}

\item[\comp{SYMMETRY SPECIFIED, BUT CANNOT BE USED IN DRC}]~\\
This  is  self  explanatory.   The  DRC  requires   all   geometric
constraints  to  be  lifted.   Any  symmetry  constraints  will first be
applied, to symmetrize the geometry,  and  then  removed  to  allow  the
calculation to proceed.
\index{SYMMETRY SPECIFIED \ldots}
\index{Error message!SYMMETRY SPECIFIED \ldots}

\item[\comp{SYSTEM DOES NOT APPEAR TO BE OPTIMIZABLE}]~\\
This is a gradient norm minimization message.  These routines  will only  
work   if  the  nearest  minimum  to  the  supplied  geometry  in gradient-norm
space is a transition state or a ground  state.   Gradient norm  space  can 
be  visualized  as  the  space  of  the  scalar of the derivative of the energy
space with respect to  geometry.   To  a  first approximation,  there are twice
as many minima in gradient norm space as there are in energy space.
\index{SYSTEM DOES NOT \ldots}
\index{Error message!SYSTEM DOES NOT \ldots}

It is unlikely that  there  exists  any  simple  way  to  refine  a geometry 
that  results in this message.  While it is appreciated that a large amount of
effort has probably already been expended in getting  to this  point,  users 
should  steel  themselves  to writing off the whole geometry.  It is not
recommended that a minor  change  be  made  to  the geometry and the job
re-runted. % re-runted?!

Try using \comp{TS} or \comp{SIGMA} instead of \comp{POWSQ}.
                                                  
\item[\comp{SYSTEM SPECIFIED WITH ODD NUMBER OF ELECTRONS, CORRECT FAULT} (FATAL)]~\\
When \comp{EXCITED} or \comp{BIRADICAL} is used, the system {\em must} be a
singlet---\comp{must} have an even number of electrons.  Correct the data set
and re-run.
\index{SYSTEM SPECIFIED \ldots}
\index{Error message!SYSTEM SPECIFIED \ldots}

\item[\comp{SYSTEMS WITH TV CANNOT BE RUN WITH `INT'} (FATAL)]~\\
The effect of \comp{INT} is to convert the geometry into internal coordinates. 
This is done in two steps: first, the geometry is converted into Cartesian
coordinates,  this removes any dummy atoms; then it is converted into internal
coordinates. An unwanted, but logical, consequence of this is that translation
vectors are made useless.  To achieve the effect of \comp{INT}, remove the
translation vectors, run the calculation with \comp{0SCF}, then add the
translation vectors back in  `by hand'.  Note that translation vectors can be
Cartesian or internal.
\index{SYSTEMS WITH TV \ldots} 
\index{Error message!SYSTEMS WITH TV \ldots}


\item[\comp{TEMPERATURE RANGE STARTS TOO LOW, \ldots}]~\\
The  thermodynamics  calculation  assumes  that   the   statistical summations 
can be replaced by integrals.  This assumption is only valid above 100K, so the
lower temperature  bound  is  set  to  100,  and  the calculation continued.
\index{TEMPERATURE RANGE \ldots}
\index{Error message!TEMPERATURE \ldots}

\item[\comp{The COSMO option cannot be used\ldots} (FATAL)]~\\
Due to a program limitation, sparkles cannot be used in a COSMO calculation.
There is no recovery---try using real atoms instead of sparkles.
\index{The COSMO option cannot be \ldots}
\index{Error message!The COSMO option \ldots}

\item[\comp{The data set contains alternative location indicators. Keyword ALT\_A must be used} (FATAL)]~\\
The PDB format allows for alternate locations.  According to this format,
atoms  are defined by four characters (characters 13-16), after which comes 
the alternate location indicator. Usually,  this (the 17th character) is a
space, however, in atoms that have  positional disorder the 17th character will
be a letter, e.g.\ `A', `B', `C', etc. A valid geometry will contain only one of
the alternative locations, and this can be identified in the data set by use of
\comp{ALT\_A=$letter$}.  For a  full definition of the PDB format, see the WWW
sites: http://pdb.pdb.bnl.gov/pdb-docs/atoms.html  and
http://www.pdb.bnl.gov/pdb-docs/Format.doc/Contents\_Guide\_21.html.
\index{The data set contains alt\ldots}
\index{Error message!The data set contains \ldots}

\item[\comp{The data set contains alternative location indicators. Keyword ALT\_R must be used} (FATAL)]~\\
The PDB format allows for alternate residues.  According to this format, a 
given residue site can be occupied by one of two or more possible residues. 
The indicator is character 27, and by default is a space.  If there are
alternative residues possible, these are indicated by a letter, e.g.\ `A', `B',
`C', etc. A valid geometry will contain only one of the alternative residues,
and this can be identified in the data set by use of \comp{ALT\_R=$letter$} or
by \comp{ALT\_R= }.
\index{The data set contains alt\ldots}
\index{Error message!The data set contains \ldots}
                    
\item[\comp{THE FOLLOWING SYMBOL HAS BEEN DEFINED MORE THAN ONCE: $symbol$} (FATAL)]~\\
A symbolic parameter in a Gaussian Z-matrix geometry has been defined
more than once.  Remove the excess definitions and re-run.
\index{THE FOLLOWING SYMBOL \ldots}
\index{Error message!THE FOLLOWING SYM\ldots}

\item[\comp{THE FOLLOWING SYMBOL WAS NOT USED: $symbol$} (FATAL)]~\\
A symbolic parameter in a Gaussian Z-matrix geometry has been defined,
but was not present in the Z-matrix.  Remove the definition and re-run.
\index{THE FOLLOWING SYMBOL \ldots}
\index{Error message!THE FOLLOWING SYM\ldots}

\item[\comp{THE GEOMETRY DATA-SET CONTAINED ERRORS} (FATAL)]~\\
The geometry was not in MOPAC Z-matrix format, or in Cartesian format,
or in Gaussian format.  The output immediately before this error message
describes the errors detected.  Correct the geometry and re-run.
\index{THE GEOMETRY DATA-SET \ldots}
\index{Error message!THE GEOMETRY DATA-\ldots}
 
\item[\comp{THE GREENS FUNCTION IS LIMITED TO 200 ORBITALS} (FATAL)]~\\
The  Greens function correction to the I.P.'s calculation is limited to
systems of 200 orbitals.  Larger systems cannot be run.  This is a program 
limitation.
\index{THE GREENS FUNCTION IS \ldots}
\index{Error message!THE GREENS FUNCT\ldots}
 
\item[\comp{THE HAMILTONIAN REQUESTED IS NOT AVAILABLE IN THIS PROGRAM} (FATAL)]~\\
This is a bug that occurs only during a port of MOPAC. It is probably caused by
the BLOCKDATA file not being used by the compiler.  One way around this
fault is to paste the file \comp{block.F} at the end of the file \comp{mopac.F},
and then delete \comp{block.F}. \ This usually corrects this problem.
\index{THE HAMILTONIAN REQ\ldots}
\index{Error message!THE HAMILTONIAN \ldots}

\item[\comp{The input data file does not exist} (FATAL)]~\\
This is a bug that occurs only during a port of MOPAC. The first step in
trying to remove this bug is to check that the command:
\index{The input data file does not exist}
\index{Error message!The input data file \ldots}
\begin{verbatim}
% mopac.exe test
\end{verbatim}
will run the data set \comp{test.dat}.  Once this works, that is, once it
does not generate this error message, check out \comp{mopac.csh}.

\item[\comp{THE SCF CALCULATION FAILED.} (FATAL)]~\\
The SCF failed to form.  To correct this, add \comp{PL} and try using
different convergers.  In order, try \comp{SHIFT=$n$}, $n$=50 is a good starting
value, then \comp{PULAY}, then \comp{CAMP}.  Quite often, the fault lies in the
geometry.  Use the information from the output generated by \comp{PL} as a
guide.
\index{THE SCF CALCULATION \ldots}
\index{Error message!THE SCF CALCULATION \ldots}
 
\item[\comp{THE STATE REQUIRED IS NOT PRESENT IN THE SET OF CONFIGURATIONS AVAILABLE} (FATAL)]~\\
A specific excited state has been specified, but it does not exist in the
set of states calculated.  Correct the data set and re-run.
\index{THE STATE REQUIRED IS \ldots}
\index{Error message!THE STATE REQUIRED \ldots}
 
\item[\comp{There are more than 200 atoms in moiety} (FATAL)]~\\
A non-proteinaceous moiety in a protein apparently contains more than $N$
atoms.  If this is true, then edit ``ligand.F'' to change the value of \comp{NATOMR}
in subroutine \comp{MOIETY}, re-run ``make'', and re-run the job.
\index{There are more than $N$ \ldots}
\index{Error message!There are more \ldots}

\item[\comp{There are more than 200 atoms in residue $M$} (FATAL)]~\\
A residue apparently contains more than 200 atoms.  A likely cause is if
the residue contains covalent bonds to another part of the protein (common
cross links, e.g.\ N-S, O-S, and S-S, are automatically broken when 
residues are being analyzed).  Check the atom list printed after the error
message to identify the atoms that are considered part of the residue,
modify the data-set and re-run.
\index{There are more than $N$ \ldots}
\index{Error message!There are more \ldots}

\item[\comp{THERE ARE NO VARIABLES IN THE SADDLE} (FATAL)]~\\
For a \comp{SADDLE} calculation, the two geometries must be optimizable, 
that is, at least one coordinate must have the optimization flag set to ``1''.
\index{THERE ARE NO VARIABLES} \ldots
\index{Error message!THERE ARE NO VAR\ldots}
 
\item[\comp{There are too many keywords} (FATAL)]~\\
Although each line of keywords can have up to 120 characters, 
the maximum number of characters plus spaces for all keywords is limited to 241.
Reduce the number of keywords, or abbreviate those that can be shortened.
\index{There are too many keywords}
\index{Error message!There are too \ldots}

\item[\comp{There is a bug in MOZYME} (FATAL)]~\\
This message should never appear!  When \comp{NEWGEO} was used to convert
a protein geometry to mixed coordinates, an absurd
set of coordinates was detected.  Please inform Dr~Stewart as soon as possible.
\index{There is a bug in MOZYME}
\index{Error message!There is a bug in MOZYME}

\item[\comp{THERE IS A FAULT IN GEOCHK} (FATAL)]~\\
During the check of the Lewis structure, the number of electrons calculated
from the Lewis structure does not equal the number of electrons from the
valence shell of the atoms plus any net charge.  This can only happen if 
there is a program bug.  Please inform Dr~Stewart as soon as possible.
\index{THERE IS A FAULT IN \ldots}
\index{Error message!THERE IS A FAULT \ldots}

\item[\comp{THERE IS A FAULT IN RESEQ} (FATAL)]~\\
During an attempt to re-sequence the atoms in a protein, the number of atoms
in the new system differed from the number in the original system.
This can only happen if there is a program bug.  
Please inform Dr~Stewart as soon as possible.
\index{THERE IS A FAULT IN RESEQ}
\index{Error message!THERE IS A FAULT \ldots}

\item[\comp{There is a mistake in SUTUPI or TMPI} (FATAL)]~\\
The amount of memory needed in annihilating matrix elements in a MOZYME
calculation is estimated.  Sometimes this estimate is low, and more memory
is needed.
Fix this problem by following the instructions printed after the error message.
Please inform Dr~Stewart as soon as possible.
\index{There is a mistake \ldots}
\index{Error message!There is a mistake \ldots}

\item[\comp{There is a mistake in SUTUPR or TMPZR} (FATAL)]~\\
The amount of memory needed in annihilating matrix elements in a MOZYME
calculation is estimated.  Sometimes this estimate is low, and more memory
is needed.
Fix this problem by following the instructions printed after the error message.
Please inform Dr~Stewart as soon as possible.
\index{There is a mistake \ldots}
\index{Error message!There is a mistake \ldots}

\item[\comp{THERE IS A RISK OF INFINITE LOOPING \ldots}]~\\
The SCF criterion has been reset by the user, and the new value  is so  small 
that  the SCF test may never be satisfied.  This is a case of user beware!
\index{THERE IS A RISK OF INF\ldots}
\index{Error message!THERE IS A RISK \ldots}
 
\item[\comp{THERE MUST BE EXACTLY THREE VELOCITY DATA PER LINE} (FATAL)]~\\
The format for the initial velocity in a DRC calculation is three numbers
per line, corresponding to the $x$, $y$, and $z$ speeds in cm.sec$^{-1}$.
Correct the data set and re-run.
\index{THERE MUST BE EXACTLY \ldots}
\index{Error message!THERE MUST BE \ldots}
 
\item[\comp{THIS MESSAGE SHOULD NEVER APPEAR, CONSULT A PROGRAMMER!}]~\\
This message should never appear; a fault has been introduced  into MOPAC, 
most  probably  as  a  result  of  a programming error.  If this message
appears in the basic version of MOPAC  (a  version  ending  in 00),  please 
contact JJPS as I would be most interested in how this was achieved.
\index{THIS MESSAGE SHOULD \ldots}
\index{Error message!THIS MESSAGE \ldots}
 

\item[\comp{THREE ATOMS BEING USED TO DEFINE \ldots} (FATAL)]~\\
If the Cartesian coordinates of an  atom  depend  on  the  dihedral angle  it
makes with three other atoms, and those three atoms fall in an almost straight
line, then a small change in the  Cartesian  coordinates of  one  of  those
three atoms can cause a large change in its position. Normally, the
connectivity will automatically be changed to prevent this  happening, however,
if there is no obvious way to correct the problem, this message will be
printed.  When that happens, the data  should be changed to make the geometric
specification of the atom in  question less ambiguous. Note that neither 
\comp{LET}  nor  \comp{GEO-OK} will allow the calculation to proceed.
\index{THREE ATOMS BEING \ldots}
\index{Error message!THREE ATOMS BEING \ldots}


\item[\comp{TIME UP }]~\\
The time defined on the keywords line or 3,600 seconds, if no  time was 
specified, is likely to be exceeded if another cycle of calculation were to be
performed.  A controlled termination of the run would  follow this  message.  
The  job  may terminate earlier than expected:  this is ordinarily due to one
of the recently completed cycles taking  unusually long,  and  the  safety 
margin  has  been  increased  to  allow for the possibility that the next cycle
might also  run  for  much  longer  than expected.
\index{Error message!TIME UP}
\index{TIME UP}
 
\item[\comp{TOO MANY CONFIGURATIONS} (FATAL)]~\\
The size of the C.I.\ matrix requested is larger than that allowed by
MOPAC.  Either reduce the size requested, or increase the allowed size.
To do that, increase the value of \comp{MAXCI} in the file \comp{meci.h},
and recompile.
\index{TOO MANY CONFIGURATIONS}
\index{Error message!TOO MANY CONFIG\ldots}
 
\item[\comp{TOO MANY ITERATIONS IN LAMDA BISECT} (FATAL)]~\\
During a run involving \comp{EF}, the search procedure failed.
This was most likely due to a faulty geometry.  Check the geometry
carefully, make corrections, and re-run.
\index{TOO MANY ITERATIONS \ldots} 
\index{Error message!TOO MANY ITER\ldots}
 
\item[\comp{TRIPLET SPECIFIED WITH ODD NUMBER OF ELECTRONS, CORRECT FAULT} (FATAL)]~\\
If \comp{TRIPLET} has been specified the number of electrons must be even.
Check  the  charge  on  the  system,  the empirical formula, and whether 
\comp{TRIPLET} was intended.
\index{TRIPLET SPECIFIED \ldots}
\index{Error message!TRIPLET SPECIFIED \ldots}

\item[\comp{TRUST RADIUS NOW LESS THAN \ldots}]~\\
When \comp{EF} is used, the calculated trust radius has become too small.
An easy fix is to add \comp{LET}, another option is to specify \comp{RMIN=-10}.
\index{TRUST RADIUS NOW LESS \ldots}
 
\item[\comp{TS FAILED TO LOCATE TRANSITION STATE} (FATAL)]~\\
The geometry is almost certainly faulty.  Locating transition states
is still more art than science.  Modify geometry, and re-run.
\index{TS FAILED TO LOCATE \ldots}
\index{Error message!TS FAILED \ldots}

\item[\comp{TS SEARCH AND BFGS UPDATE WILL NOT WORK} (FATAL)]~\\
In \comp{EF}, a transition state optimization has been requested.  
The rarely used option \comp{IUPD=2} has been specified.  This option
is not allowed for transition states.  Remove \comp{IUPD=2} and re-run.
\index{TS SEARCH AND BFGS \ldots}
\index{Error message!TS SEARCH AND \ldots}
               
\item[\comp{TS SEARCH REQUIRE BETTER THAN DIAGONAL HESSIAN} (FATAL)]~\\
When using \comp{TS}, do not also use \comp{IGTHES=0}.  Remove \comp{IGTHES=0}
and re-run.
\index{TS SEARCH REQUIRE \ldots}
\index{Error message!TS SEARCH REQUIRE \ldots}
 
\item[\comp{TWO ADJACENT POINTS ARE IDENTICAL:} (FATAL)]~\\
In a reaction path, adjacent points must be different.  A common mistake
is to have the first point (this comes from the Z-matrix) and the second point
(this comes from the extra data after the Z-matrix and symmetry data (if any))
the same.  Correct the data set (the fault is in the reaction coordinate data
after the the Z-matrix and symmetry data (if any)).
\index{TWO ADJACENT POINTS \ldots}
\index{Error message!TWO ADJACENT POINTS \ldots}

\item[\comp{TWO ATOMS ARE COINCIDENT.  A FATAL ERROR.} (FATAL)]~\\
In the input geometry, two atoms have the same coordinates.  Correct the
error and re-run.
\index{TWO ATOMS ARE COIN\ldots}
\index{Error message!TWO ATOMS ARE \ldots}

\item[\comp{UNABLE TO ACHIEVE SELF-CONSISTENCY}]~\\
See the error-message: ``FAILED TO ACHIEVE SCF.''
\index{UNABLE TO ACHIEVE \ldots}
\index{Error message!UNABLE TO ACHIEVE \ldots}

\item[\comp{UNACCEPTABLE VALUE FOR NO. OF ORBITALS ON ATOM} (FATAL)]~\\
Allowed values for the number of orbitals per atom are 1, 4, and 9.
Correct the \comp{EXTERNAL} file, and re-run.
\index{UNACCEPTABLE VALUE \ldots}
\index{Error message!UNACCEPTABLE VALUE \ldots}

\item[\comp{UNDEFINED SYMMETRY FUNCTION USED} (FATAL)]~\\
Symmetry operations are restricted to those defined, i.e.,  in  the
range 1--19.  Any other symmetry operations will trip this fatal message.
\index{UNDEFINED SYMMETRY \ldots}
\index{Error message!UNDEFINED SYM\ldots}

\item[\comp{UNKNOWN RESIDUE: $N$}]~\\
The 20 common amino acid residues are automatically identified. Other residues 
can be made known to MOPAC by use of the keyword \comp{XENO=$text$}. The numbers
printed after this message are the numbers of C, N, O, and S atoms found in the
residue.
\index{UNKNOWN RESIDUE: $N$}
\index{Error message!UNKNOWN RESIDUE: $N$}

Inspect the residue, and identify which of the 20 common residues it is related
to.  In \comp{XENO} specify the number of extra atoms needed to make up the numbers
in the unknown residue.  Re-run the calculation.

\item[\comp{UNRECOGNIZED ELEMENT NAME} (FATAL)]~\\
In the geometric specification a chemical  symbol  which  does  not correspond 
to  any  known element has been used.  The error lies in the first datum on a
line of geometric data.
\index{UNRECOGNIZED ELEMENT \ldots}
\index{Error message!UNRECOGNIZED ELE\ldots}

\item[\comp{UNRECOGNIZED HESS OPTION} (FATAL)]~\\
The allowed values for $n$ in \comp{HESS=$n$} are 0, 1, and 2. Correct the
keyword and re-run.
\index{UNRECOGNIZED HESS OPTION}
\index{Error message!UNRECOGNIZED HESS \ldots}

\item[\comp{UNRECOGNIZED KEY-WORDS.} (FATAL)]~\\
Check these keywords.  Correct any misspellings and re-run, or, if the
keywords are DEBUG keywords, add \comp{DEBUG} and re-run.
\index{UNRECOGNIZED KEY-WORDS.}
\index{Error message!UNRECOGNIZED KEY\ldots}

\item[\comp{UNRECOGNIZED SPECIES: \textit{text}}]~\\
While reading a PDB file, an unrecognized chemical symbol has been used.
Recognized symbols are those for any element that can be handled by MOPAC,
and the symbols `X' (for hydrogen) and `L' (for a lone pair).
\index{UNRECOGNIZED SPECIES: $text$}

The job will not be stopped by this error message, but the
\textit{text} should be examined to see if it is a real element, but
with an unconventional symbol.  If it is, add
\comp{PDB(\textit{code})}, where \textit{code} 
\begin{htmlonly}
\htmlref{is defined elsewhere}{key_pdb}
\end{htmlonly}
\begin{latexonly}
is defined on p.~\pageref{key_pdb}
\end{latexonly}.  Even if $text$ does
not apply to an element, \comp{PDB($code$)} can still be used, but set
the atomic number to zero.  This will prevent the error message from
recurring.

\item[\comp{UPPER BOUND OF ACTIVE SPACE IS GREATER THAN THE NUMBER OF ORBITALS} (FATAL)]~\\
The keywords used here imply a system that is larger than that used.
Correct data set (probably by changing the keywords) and re-run.
\index{UPPER BOUND OF ACTIVE \ldots}
\index{Error message!UPPER BOUND OF ACT\ldots}

\item[\comp{Use AIGIN to allow more geometries to be used} (FATAL)]~\\
Only one GAUSSIAN geometry is allowed in a run, unless each GAUSSIAN
geometry is identified by \comp{AIGIN} on the keyword line.  Add \comp{AIGIN}
to each GAUSSIAN geometry keyword line, and re-run.
\index{Use AIGIN to allow more \ldots}
\index{Error message!Use AIGIN to allow more \ldots}
 
\item[\comp{USE EITHER SAFE OR UNSAFE, BUT NOT BOTH} (FATAL)]~\\
MOPAC can be compiled so as to either minimize memory demand or to
maximize reliability.  Depending on which option was used, the
alternative option can be selected at run time by using
\htmlref{\comp{SAFE}}{safe} or \htmlref{\comp{UNSAFE}}{unsafe}.

\begin{latexonly}
See p.~\pageref{safe} and p.~\pageref{unsafe} for more detail.
\end{latexonly}
\index{USE EITHER SAFE OR \ldots}
\index{Error message!USE EITHER SAFE \ldots}
 
\item[\comp{VALUE OF NPPA NOT ALLOWED: IT MUST BE $NNN$} (FATAL)]~\\
This error is caused by a program bug.  Please make a report to Dr.~Stewart.
\index{VALUE OF NPPA NOT \ldots}
\index{Error message!VALUE OF NPPA NOT \ldots}

\item[\comp{VAN DER WAALS} (FATAL)]~\\
In the \comp{ESP} method, van der Waals radii are used. Only the following
elements are allowed: H, B, C, N, O, F, P, S, Cl, Br, I.  
\index{VAN DER WAALS}
\index{Error message!VAN DER WAALS}

In the \comp{PMEP} method, van der Waals radii are also used. Only the 
elements  up to $Z$=17 and bromine are allowed.

\item[\comp{VECPRT CAN ONLY PRINT ARRAYS OF SIZE .LT. 200} (FATAL)]~\\
To minimize output, the size of arrays that can be printed in a MOZYME
calculation is limited to 200.  Bigger arrays are truncated so that only 
the sub-array of size 200 is printed.  If larger arrays are to be printed, 
edit ``vecprt.F'' to change \comp{MAXARR} to the desired value, re-run ``make'',
and re-run the job.
\index{VECPRT CAN ONLY PRINT ARRAYS OF SIZE .LT. 200}

\item[\comp{VIRTUAL  C VECTOR DAMAGED IN SELMOS} (FATAL)]~\\
In a MOZYME calculation, when \comp{RAPID} is used (in a partial geometry
optimization), the virtual  M.O.\ coefficients were found to be corrupt.
Please inform Dr~Stewart as soon as possible.
\index{VIRTUAL  C VECTOR \ldots}
\index{Error message!VIRTUAL  C VECTOR \ldots}

\item[\comp{VIRTUAL  IC VECTOR DAMAGED IN SELMOS} (FATAL)]~\\
In a MOZYME calculation, when \comp{RAPID} is used (in a partial geometry
optimization), the list of atoms in the virtual M.O.\  was found to be corrupt.
Please inform Dr~Stewart as soon as possible.
\index{VIRTUAL  IC VECTOR \ldots}
\index{Error message!VIRTUAL  IC VECTOR \ldots}

\item[\comp{WARNING -- GEOMETRY IS NOT AT A STATIONARY POINT}]~\\
Under certain circumstances the gradient norm can drop to zero, but the
derivatives of the energy with respect to Cartesian coordinates might be
quite large.  If this happens, this error message will be printed.
\index{WARNING -- GEOMETRY \ldots}
\index{Error message!WARNING -- GEOM\ldots}

To avoid this message, make sure that the geometry can be optimized, given the
optimization flags you have chosen.  In particular, if (3$N$-6) optimization
flags are set, and there are no dummy atoms, then it is unlikely that this
message will be generated. 

\item[\comp{WARNING -- SPARKLES ARE NOT TREATED CORRECTLY}]~\\
When \comp{ENPART} is used on a system involving sparkles, the energy terms
due to the sparkles are not printed correctly.
\index{WARNING -- SPARKLES ARE \ldots}

\item[\comp{WARNING }]~\\
Don't pay too  much  attention  to  this  message.   Thermodynamics
calculations  require  a  higher  precision  than  vibrational frequency
calculations.  In particular, the gradient norm should  be  very  small.
However,  it  is  frequently  not  practical to reduce the gradient norm
further, and to date no-one has determined just how slack  the  gradient
criterion  can be before unacceptable errors appear in the thermodynamic
quantities.  The 0.4 gradient norm is only a suggestion.
\index{Error message!WARNING}
\index{WARNING}

\item[\comp{WARNING! NO BACKBONE ATOMS IDENTIFIED}]~\\
When \comp{NEWGEO} is used to generate a mixed coordinate $Z$-matrix for a
protein, this message will be generated if no backbone atoms (the --NH--CH--CO--
of a peptide) are found.  Check that at least one ``CO--NH'' linkage exists.
\index{WARNING NO BACKBONE ATOMS \ldots}
\index{Error!messages|)}
\end{description}
