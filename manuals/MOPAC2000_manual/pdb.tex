\subsection{Protein Data Bank Format}\index{PDB!format|ff}
Protein geometries are usually defined using the Brookhaven Protein Data Bank
format.  A full definition of this format is given in the web-page: \\
http://www.rcsb.org/pdb/docs/format/pdbguide2.2/guide2.2$\_$frame.html.
The home-page of the PDB is http://www.rcsb.org. This is an excellent
 resource for obtaining biomolecular macromolecules.

There are several ways of converting a PDB file into a form suitable for use by
MOPAC.   For most systems in the PDB the hydrogen atoms are missing.  If they
are present, that is, the data set is complete, then the file can simply be
edited as follows:
\begin{enumerate}
\item Delete the header of the PDB file, down to the first line that starts with
the word \comp{ATOM}.
\item Delete the tail of the file back to the last occurrence of a line starting
with the word \comp{HETATM}.
\item Add the keyword line(s) and comment line(s).
\end{enumerate}

\index{CHEM3D}
An alternative way of generating a MOPAC data set is to read in the geometry
using Cambridge Soft's CHEM3D program.  This is particularly useful if the
hydrogens are {\em not} present in the PDB file.

After reading in the file, hydrogens can be added using the \comp{RECTIFY}
function in CHEM3D.  Once this is done, the data set can be exported for
use by MOPAC using the \comp{SAVE AS} option.

\subsubsection{Preparing the data set}\index{Crambin}
Even when the hydrogens are present in the PDB file, for example in
crambin, file=P1CBN, there
are usually problems that prevent the file being run.  Therefore, before attempting
to calculate the electronic structure, any faults in the data set should be
corrected.  The following procedure has been found to be useful:
\begin{itemize}
\item First, the atoms should be rearranged into the standard PDB sequence.
This is a quick operation, and only one keyword, \comp{RESEQ}, should be used.
Examination of the output will indicate whether any severe problems exist.
Potential problems are:
\begin{itemize}
\item The system consists of many short sections of protein.  This usually
occurs when only a fragment of the system is present, e.g., the active site.
Working with such a system is difficult, because the geometry does not
correspond to anything that can exist naturally.
\item One or more \comp{UNK} residues exist.  Before continuing, identify the
unknown residues.  Ensure that they are not artifacts of a faulty geometry.
\item One or more unknown elements exist.  The author of the PDB file may
have used non-standard labels for atoms.  This fault can be corrected by
use of \comp{PDB(\ldots)}.
\item Structural or positional disorder may exist.  Use \comp{ALT\_R=$n$} and/or
\comp{ALT\_A=$n$} to correct this.
\end{itemize}
\item Edit the output file to create a new data set.
\item If any very small interatomic distances (less than 0.8 \AA ngstroms) were
reported, edit the data set to correct these faults.  A quick way to find
faulty distances is to search the data set for ``\verb+)   0.6+", then
``\verb+)   0.7+", etc.
\item Run the corrected data set using keywords \comp{1SCF}, \comp{RESIDUES},
and \comp{MOZ}. \ Almost certainly, the run will fail.  Usually the charge on the
system, as determined by the \comp{MOZYME} function, will be incorrect.  If it
is in the range +2 to --2, that is, in a chemically meaningful range, then
simply add \comp{CHARGE=$n$} and re-submit.
\item Usually the charge is quite unreasonable (outside the range +2 to --2).
Examination of the charged atoms will usually suggest how to proceed.  If the
charge is large and negative, add hydrogens to the oxygen atoms of
ionized acid residues, to
neutralize them.  If it is large and positive, delete hydrogen atoms from
nitrogen atoms of ionized amine groups.
Be careful to neutralize an even number of ions---if an
odd number is neutralized, then the system will have an odd number of electrons,
and the job will stop as soon as the number of electrons is determined.
\item After correcting the errors, run a single SCF, followed by more
complicated calculations, as needed.
\end{itemize}
