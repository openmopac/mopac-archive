\section{Miertus-Scrocco-Tomasi Solvation Model}
\index{MST Solvation Model}\label{mst}
\index{Orozco@{\bf Orozco, Modesto}}
\index{Luque@{\bf Luque, F. J.}}
\index{Solvation Model!MST}
\begin{center}Based on materials provided by\end{center}

{\bf Modesto Orozco}, Dep.\ Bioqu\'{\i}mica, Facultat de Qu\'{\i}mica,
Universitat de Barcelona, C/ Mart\'{\i} i Franqu\'{e}s 1, Barcelona 08028. 
SPAIN. E-mail: modesto@far1.far.ub.es.

and

{\bf F. J. Luque}, Dep.\ Farmacia, Unitat Fisicoqu\'{\i}mica, Facultat de
Farmacia,  Universitat de Barcelona, Avgda Diagonal sn,  Barcelona 08028,
SPAIN. E-mail: javier@far1.far.ub.es.

\subsection{Outline of the MST Method}
\index{MST solvation model}
\index{Solvation model!MST model}
Solvation is computed using a modified version of Miertus, Scrocco and 
Tomasi~\cite{mst,mt} self-consistent reaction field method (MST/SCRF), which 
has been modified~\cite{lno,glo,nol,lbo,obl} to allow semiempirical
Hamiltonians  to be used. Following this strategy the  solute is placed in a
molecular-shaped cavity, which is built using GEPOL  routines~\cite{pastb}
which use optimized van der Waals' radii~\cite{ojl,blo,rap}.  The solvent is 
represented as a continuum which reacts against the solute charge distribution 
{\em via} a perturbation operator. The perturbation operator is obtained by 
solving the Laplace equation at the solute/solvent interface (the solute
cavity).  The solute electrostatic potential, which is necessary to solve the
Laplace equation, is rigorously computed as the expectation value of the 
$r^{-1}$ operator~\cite{mst,mt,lno,glo,nol,lbo,obl}.

The free energy of hydration is computed as the addition of three
contributions: 

\begin{enumerate}
\item The electrostatic term, which is computed from the linear free energy
response  theory~\cite{mst,mt,lno,glo,nol,lbo,obl}.

\item The cavitation contribution, which is computed from Pierotti's  scaled
particle theory~\cite{ojl,blo}.\index{Pierotti's scaled particle theory}

\item The van der Waals terms, which is computed using a linear relation with 
the solute accessible surface, and optimized ``hardness''
parameters~\cite{rap}. \index{Van der Waals!in MST theory}
\end{enumerate}

In addition to the free energy of hydration a ``solvent-adapted'' wavefunction 
is obtained. Such a wavefunction can be used to determine changes in  solute
properties due to the solvent~\cite{lobg,lo,lo2,lobg2}.

\index{Electrostatic potential!Orozco-Luque model}
\index{MEP}
\index{Self Consistent Reaction Field method}
\index{SCRF Method}

The program allows the use of two different strategies to compute the 
semiempirical Molecular Electrostatic Potential (MEP). The first 
one~\cite{lio,alo,lo3} computes the MEP via the deorthogonalization of the
semiempirical  wavefunction (no keyword), while the second one~\cite{frr,alo2}
maintains the  orthogonality of the wavefunction in the calculation of the MEP
(keyword \comp{ORT}). \label{ort}

The strategy has been tested in different systems. Optimized versions for 
MNDO, AM1 and PM3 are available. The comparison of experimental and 
theoretical free energies of hydration for 23 small neutral molecules yields
to  RMS deviations in the range of 1  kcal/mol. The best results  (RMS 1.0
kcal/mol, r=0.94) are obtained with the AM1 Hamiltonian with  orthogonal MEP's.
The solvent-induced dipoles matches Monte Carlo/QM and {\em ab initio} SCRF
values.

Semiempirical/SCRF methods have been successfully applied to several problems. 
Examples: i) tautomerization of nitroimidazole, and pyridones, ii) solvent 
effects in the isomerization of amides, and iii) H-bond dimerization.

\subsubsection*{Warnings}
\begin{itemize}
\item The solute geometry cannot be properly optimized in solution.
Optimization flags  at the different coordinates should be set to 0

\item The program has been carefully parametrized to work with neutral
molecules.  A reduction of the cavity for charged solutes may be
necessary~\cite{ol}.

\item We cannot recommend the use of the program for solvents other than 
water, chloroform, or carbon tetrachloride at the present moment. A simple 
change in the dielectric constant cannot  be enough to reproduce the
characteristics of other solvents.
\end{itemize}

\subsection{Data Requirements for MST Model}\label{mstdata}
\index{TOM@{\bf TOM}}
For the three solvent for which parameters have been published, special
keywords exist.  These are \comp{H2O}, \comp{CHCL3}, and \comp{CCL4}. If other
solvents are to be modeled, extra data must be provided at  the end of the data
set.  This extra data is described by example, using water as the solvent.
(This extra data could be avoided by using \comp{H2O}.)

For water, the data needed is shown in Figure~\ref{tomdata}.
\begin{figure}
\begin{makeimage}
\end{makeimage}
\begin{verbatim}
Line
  *   PM3 TOM 
  *
  *   O   .0000000 0     .000000 0     .000000 0    0 0 0
  *   H   .9512971 0     .000000 0     .000000 0    1 0 0
  *   H   .9512971 0  107.457468 0     .000000 0    1 2 0
  *
  1   78.5  1.0  1  1  0  3
  2   298.15  18.07  2.77  0.00026  71.690  0.657  1.277
  3   40.0  1.5  0.2  0.7  0.02  5  3  1
  4   1  1.75000  8
  5   2  1.00000  99
  6   3  1.00000  99
  7   1.00
\end{verbatim}
\caption{\label{tomdata}Data set for Calculation of Water using the Miertus-Scrocco-Tomasi Model}
\end{figure}

The extra data are indicated by the lines numbered 1 to 7 in the Figure.  These
data are described as follows:
\begin{description}
\item[Line 1. Parameters indicating the level of computation]~\\
(see Miertus, Scrocco, Tomasi, Chem.\ Phys., 55 (1981) 117 and Miertus, Tomasi,
Chem.\ Phys., 65 (1982) 239 for further details)

Layout of data:  A B C D E F

\begin{center}
\begin{tabular}{llll}  \hline
 Datum & Value & Name   &  Description  \\ \hline
    A  & 78.5  & EPS    & Solvent dielectric constant          \\
    B  & 1.0   & DMP    & Self-polarization acceleration factor\\
    C  & 1     & MC     & Self-polarization cycles             \\
    D  & 1     & ICOMP  & Charge compensation                  \\
    E  & 0     & IFIELD & Reaction field calculation on nuclei \\
    F  & 3     & ICAV   & Cavitation energy computation        \\  \hline 
\end{tabular}
\end{center}

\item[Line 2. Parameters related to the cavitation energy]~\\
(This line is needed only if \comp{ICAV $\ne$ 0})

Layout of data: A B C D E F G                                                      

\begin{center}
\begin{tabular}{llll}  \hline
 Datum & Value & Name &  Description  \\  \hline
  A   & 298.15 & TABS &  Absolute temperature (in kelvin) \\
  B   &  18.07 & VMOL &  Solvent molar volume (\aa ngstroms$^3$) \\  
  C   &   2.77 & DMOL &  Solvent molar diameter (\aa ngstroms) \\
  D   & 0.00026 & TCE &  Thermal expansion coefficient (kelvin$^{-1}$) \\ 
  E   & 71.69  & STEN &  Surface tension (dyne$\times$cm$^{-1}$) \\
  F   & 0.657  & DSTEN &  Surface tension derivative \\
  G   & 1.227  & CMF  &  Cavity microscopic coefficient \\ \hline
\end{tabular}
\end{center}

If \comp{ICAV=1} parameters A-D are needed, and Pierotti's cavitation energy
is  computed. 

If \comp{ICAV=2} parameters E-G are needed, and Sinanoglu's cavitation energy
is  computed.  

If \comp{ICAV=3} parameters A-G are needed, and Pierotti's and Sinanoglu's
cavitation energies are computed.

\item[Line 3. Parameters related to the cavity surface]~\\
(see Pacual-Ahuir, Silla, Tomasi, Bonaccorsi, J.\ Comput.\ Chem.\ 77 (1982)
3654 for further details)   

Layout of data: A B C D E F G H

\begin{center}
\begin{tabular}{llll} 
\hline
 Datum & Value & Name   &  Description  \\ \hline
A & 40.0 & OMEGA  &  Sphere overlapping parameter \\
B & 1.5  & RD     &  Solvent molecular radius \\
C & 0.2  & RET    &  Minimal radius to define new spheres \\
D & 0.7  & FRO    &  Numerical factor to define new spheres \\
E & 0.02 & DR     &  Step to compute electric field on cavity surface \\
F &  5   & NDIV   &  Parameter of cavity surface partition \\
G &  3   & NESF   &  Number of spheres \\
H &  1   & ICENT  &  Definition of spheres \\
\hline
\end{tabular}
\end{center}

\item[Lines 4--6. Coordinates and radii of spheres used to build up cavity surface]~\\
\comp{ICLASS} specifies the type of atom.  This information is used  in the
calculation of the van der Walls' energy, determines as the sum of products
between the surface of each atom and a parameter which is  characteristic for
each atom.

\comp{ICLASS} is, in fact, the atomic number of each atom, with the only
exception of hydrogen, for which we distinguish between  hydrogen atoms bound
to a heteroatom (\comp{ICLASS=99}) and those bound to a carbon
(\comp{ICLASS=1}).

If \comp{ICENT=0}, then five parameters are needed for each sphere.

Layout of data: A B C D E                                                            

\begin{center}
\begin{tabular}{llll}
\hline
 Datum & Value & Name   &  Description \\ \hline
    A  & n/a   & XE     &  X-coordinate \\
    B  & n/a   & YE     &  X-coordinate \\
    C  & n/a   & ZE     &  X-coordinate \\
    D  & n/a   & RE     &  Radius of sphere \\
    E  & n/a   & ICLASS &  Type of atom \\ \hline
\end{tabular}
\end{center}

If \comp{ICENT$\ne$0}, then three parameters are needed for each sphere.  The
spheres are defined as being centered on each atom.

Layout of data: A B C
 
\begin{center}
\begin{tabular}{lcccll} 
\hline 
 Datum & \multicolumn{3}{c}{Values} & Name   &  Description \\ \hline
   A  & 1 & 2 & 3            & NC1    &  Atom Number \\
   B  & 1.75 & 1.0 & 1.0     & RE     &  Radius of sphere \\
   C  & 8 & 99 & 99          &ICLASS  &  Type of atom \\ \hline
\end{tabular}
\end{center}
                                                                      
\item[Line 7. Factor used to scale the MEP in the deorthogonal procedure]~\\
Layout of data:    A 

\begin{center}
\begin{tabular}{llll}  
\hline 
 Datum & Value & Name   &  Description  \\ \hline
    A  & 1.0 & FACTOR  &  Factor used to scale the MEP \\ \hline
\end{tabular} 
\end{center}
\end{description}
