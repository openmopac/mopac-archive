\section{Polarizability and Hyperpolarizability Calculation}\label{t_polar}
\index{Polarizability|ff} \index{Non-Linear Optics}
\index{Hyperpolarizability}
\index{Dipole!induced}
\index{Applied electric fields}\index{Electric fields!applied}
\index{Kurtz@{\bf Kurtz, Henry, A.}}
%
%Finite Field Procedure:
%
%By default, an electric field gradient of 0.001 is used. This can
%be modified by specifying POLAR=n.nnnnn, where n.nnnnn is the new field.
%
%POLAR calculates the polarizabilities from the heat of formation and
%from the dipole. The degree to which they agree is a measure of the
%precision (not the accuracy) of the calculation. The results from the
%heat of formation calculation are more trustworthy than those from the
%dipole.
%
%Users should note that the hyperpolarizabilities obtained have to be
%divided by 2.0 for beta and 6.0 for gamma to conform with experimental
%convention.
%
%Two sets of results are printed: a set (labeled E4) derived from
%the effect of the applied electric field on the heat of formation, and a
%set (labeled DIP) derived from the value of the dipole in various
%electric fields.
%
This section was written based on material provided by:
\begin{center} Henry Kurtz and Prakashan Korambath\\
Department of Chemistry\\Memphis State University\\
Memphis TN 38152.
\end{center}

\subsection{Time-Dependent Hartree-Fock}
This procedure is based on the detailed description given by
M.\ Dupuis and S.\ Karna (J.\ Comp.\ Chem.\ 12, 487 (1991)). The
program is capable of calculating the quantities shown in Table~\ref{nlo}.
\begin{table}
\index{Pockels effect}\index{Electro-optic Pockels effect}
\index{Optical!rectification}
\index{Third Harmonic Generation}
\index{Harmonic, third, generation}
\index{DC-EFISH}
\index{Kerr effect}
\index{Optical!rectification}
\index{Refractive index}
\index{Frequency-dependent NLO}
\caption{\label{nlo} Quantities Calculable using \comp{POLAR}}
\begin{center}
\begin{tabular}{ll}\hline
\multicolumn{1}{c}{Type of Phenomenon} & \multicolumn{1}{c}{Symbol} \\ \hline
Frequency Dependent Polarizability   &   $\alpha(-\omega;\omega)$ \\
Second Harmonic Generation           &   $\beta(-2\omega;\omega,\omega)$ \\
Electrooptic Pockels Effect          &   $\beta(-\omega;0,\omega)$ \\
Optical Rectification                &   $\beta(0;-\omega,\omega)$ \\
Third Harmonic Generation            &   $\gamma(-3\omega;\omega,\omega,\omega)$  \\
DC-EFISH                             &   $\gamma(-2\omega;0,\omega,\omega)$  \\
Optical Kerr Effect                  &  $\gamma(-\omega;0,0,\omega)$  \\
Intensity Dependent Index of Refraction & $\gamma(-\omega;\omega,-\omega,\omega)$ \\
\hline
\end{tabular}
\end{center}
\end{table}

Keywords for the \comp{POLAR} calculation are given inside the \comp{POLAR}
keyword.  Quantities under user-control are:
\begin{description}
\item[\comp{IWFLB=$n$}] The type of $\beta$ calculation to be performed.  
This variable is only important if iterative beta calculations are chosen.
\begin{description}
\item[\comp{IWFLB=0}] static (This is the default)
\item[\comp{IWFLB=1}] second harmonic generation
\item[\comp{IWFLB=2}] electrooptic Pockels effect
\item[\comp{IWFLB=3}] optical rectification
\end{description}
\item[\comp{E=($n_1, n_2, n_3, \ldots$)}] The energies, in eV, of 
the radiation to be used. Up to 10 energies can be specified.
If this option is not used, the default energies of 0.0, 0.25, and 0.50 eV will
be used.

\item[\comp{BETA=$n$}] Type of beta calculation.
\begin{description}   
\item[\comp{BETA=0}] $\beta$(0;0) static (This is the default)
\item[\comp{BETA=1}] iterative calculation with type of $\beta$ chosen by \comp{IWFLB}
\item[\comp{BETA=-1}] Noniterative calculation of second harmonic generation
\item[\comp{BETA=-2}] Noniterative calculation of electrooptic Pockels effect
\item[\comp{BETA=-3}] Noniterative calculation of optical rectification
\end{description}
\item[\comp{GAMMA=$n$}] Type of gamma calculation:
\begin{description}
\item[\comp{GAMMA=0}] No gamma calculation
\item[\comp{GAMMA=1}] third harmonic generation (This is the default)
\item[\comp{GAMMA=2}] DC-EFISH
\item[\comp{GAMMA=3}] intensity dependent index of refraction
\item[\comp{GAMMA=4}] optical Kerr effect
\end{description}
\item[\comp{TOL=$n.nn$}] Cutoff tolerance for $\alpha$ calculations, 
default=0.001.
\item[\comp{MAXITU=$nnn$}] Maximum number of interactions for beta, default:
500.
\item[\comp{MAXITA=$nnn$}] Maximum number of iterations for $\alpha$
calculations, default: 150.
\item[\comp{BTOL=$n.nn$}]  Cutoff tolerance for $\beta$ calculations
The default is 0.001.
\end{description}

\subsubsection*{Examples of \comp{POLAR} keyword}

To calculate the NLO quantities $\alpha, \beta,$ and $\gamma$ at 1.0eV:\\
  \comp{POLAR(E=(1.0)) }\\
This same calculation can be set up by setting all the variables to their
default value:
\begin{verbatim}
POLAR(IWFLB=0,E=(1.),BETA=0,GAMMA=1,TOL=1.D-3,MAXITU=500,MAXITA=150,BTOL=1.D-3)
\end{verbatim}

This takes up the entire keyword line.  If more than one line is needed to hold
the keyword, use the \comp{+} option, as in:
\begin{verbatim}
+ symmetry 1scf uhf POLAR(IWFLB=0,E=(1.),BETA=0,GAMMA=1,TOL=1.D-3,MAXITU=501,
 MAXITA=151,BTOL=1.D-3)
\end{verbatim}

Note: This is not a recommended way of writing a keyword.  In order for a
keyword to be recognized, the `join' of the two lines must be perfect.  In
other words, the last character of the first line must be in column 80, unless
character 1 was not blank, in which case the last character must be in column
79. Anyhow, it is unlikely that such long keywords would be used very often.



