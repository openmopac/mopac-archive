\subsection{Correction to the Peptide Linkage}
\index{Peptide linkage!correction to barrier}
The residues in peptides are joined together by  peptide  linkages, --HNCO--.  
These  linkages  are  almost  flat, and normally adopt a trans configuration,
the hydrogen and oxygen atoms being on opposite sides  of the  C--N  bond.  
Experimentally,  the  barrier  to  interconversion  in \index{N-methyl
acetamide|ff} N-methyl acetamide is about 14 kcal/mol, but all  four  methods 
within \index{PM3!fault in peptide linkage} MOPAC  predict  a  significantly 
lower  barrier,  PM3 giving the lowest value.

The low barrier can be traced  to  the  tendency  of  semiempirical methods  
to   give   pyramidal   nitrogens.    The   degree   to  which pyramidalization
of the nitrogen atom is preferred can be  seen  in  the series of compounds
given in Table~\ref{py}.

To correct this, a molecular-mechanics correction has been applied.
\index{MMOK}   This  consists  of  identifying  the --R--HNCO-- unit, and
adding a torsion potential of form: 
\[
k \sin \! ^{2} \theta
\]
where $\theta$ is the X--N--C--O angle, X=R or H, and $k$ varies from method to 
method.   This  has  two effects:  there is a force constraining the nitrogen
to be planar, and the HNCO barrier in N--methyl acetamide is  raised to  14.00 
kcal/mol.   When the MM correction is in place, the nitrogen atom for all
methods for the last three compounds shown above is planar. The correction
should be user-transparent.

\subsubsection{Cautions}\label{mmok}
\begin{enumerate}
\item This correction will lead to errors of 0.5--1.5  kcal/mol  if
the   peptide   linkage   is  made  or  broken  in  a  reaction
calculation.
\item If the correction is applied to formamide the nitrogen will  be
            flat, contrary to experiment.
\item When calculating rotation barriers, take into account the rapid
\index{Rotation barriers}
            rehybridization  which  occurs.   When the dihedral is 0 or 180
            degrees the nitrogen will be planar (sp$^2$), but  at  90  degrees
            the nitrogen should be pyramidal, as the partial double bond is
            broken.  At that geometry the true  transition  state  involves
            motion  of the nitrogen substituent so that the nitrogen in the
            transition state is more nearly sp$^2$.  In other words, a  simple
            rotation  of  the  HNCO  dihedral will not yield the activation
            barrier, however it will be within 2 kcal/mol of  the  correct
            answer.   The  14  kcal barrier mentioned earlier refers to the
            true transition state.
\end{enumerate}     
% 16 lines, including this line
\begin{table}
\caption{\label{py}Comparison of Observed and Calculated Pyramidalization of 
Nitrogen}
\begin{center}
\begin{tabular}{llllll} \hline
        Compound    & MINDO/3  & MNDO  & AM1   &PM3  &  Exp  \\ \hline
  Ammonia           & Py       &  Py   & Py    &Py   &  Py  \\
  Aniline           & Py       &  Py   & Py    &Py   &  Py  \\
  Formamide         & Py       &  Py   & Flat  &Py   &  Py  \\
  Acetamide         & Flat     &  Py   & Flat  &Py   &  Flat  \\
  N-methyl formamide& Flat     &  Py   & Flat  &Py   &  Flat  \\
  N-methyl acetamide& Flat     &  Flat & Flat  &Py   &  Flat \\ \hline 
\end{tabular}
\end{center}
\end{table}

