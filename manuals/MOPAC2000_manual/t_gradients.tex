\section{Gradients}\label{derivs}
\index{Gradients|ff}\index{Derivatives|(}
By ``gradients'' we generally mean ``the derivative of the energy with respect
to coordinates''.  The two most commonly used gradients are with respect to
Cartesian coordinates, in which case the units are kcal/mol/\AA ngstrom, or
with respect to internal coordinates, in which case the units are either 
kcal/mol/\AA ngstrom or kcal/mol/radian, depending on whether the coordinate is
a distance (in which case it would be kcal/mol/\AA ngstrom) or an angle or
dihedral (in which case it would be kcal/mol/radian). The particular gradient
actually being used at any given point should be clear from the context. In all
cases, the gradient can be regarded as the following derivative
$$
g_i = \frac{d(\Delta H_f)}{dx_i}
$$
In discussion ``gradient'' will be reserved for the derivative with respect to
coordinates flagged for optimization (internal or Cartesian), and
``derivative'' will be used for both gradients and terms which are used to
calculate gradients, such as Cartesian derivatives which are used to calculate
internal coordinate gradients.

There are four very different ways to calculate gradients, although all four
result in the same type of derivative.  The four ways are:
\begin{description}
\item[Frozen density matrix finite difference derivatives]~\\
\index{Wavefunctions!variational}
In these procedures, once an SCF has been achieved, the derivatives can be 
calculated using the density matrix from the SCF calculation.  These methods 
can only be used with variationally optimized wavefunctions. 

By default, the derivatives are worked out by calculating the energy of each
pair of atoms, then re-calculating the energy after a small displacement has
been made, and then calculating the derivative from the differences in the
energies and the step.  This is the default, and is the fastest.  If this
method is {\em not} wanted, specify \comp{ANALYT}.

\item[Analytical derivatives, using frozen density matrix approximation]~\\
Not as fast as the first method, but more accurate.  Useful when finite
difference derivatives are suspected to be of insufficient accuracy.  When
analytical derivatives are wanted, specify \comp{ANALYT}.  Analytical
derivatives cannot be used with non-variational finite difference derivatives.

\item[Non-variational finite difference derivatives]~\\ 
For non-variational \index{Wavefunctions!non-variational}wavefunctions (systems
for which the electronic energy is modified after the SCF calculation is done,
e.g.\ C.I.\ calculations), a  sophisticated derivative routine in \comp{DERNVO}
calculates the effect on the  derivative of the post-SCF energy terms.  This
method is used automatically in RHF C.I.\ calculations.  If this method is {\em
not} wanted, specify \comp{NOANCI}.

\item[Brute force gradients]~\\
These should be avoided whenever possible.  To calculate the gradient, a small
change is made in the desired coordinate, then a full SCF is done, and the
gradient calculated from
$$ 
g_i = \frac{\Delta H_f-\Delta H_f'}{x-x^{'}}.
$$ 
These gradients are very slow, and are of poor accuracy, but sometimes they
are the only way to obtain  gradients.  These derivatives cannot be used
with variationally optimized wavefunctions, but can be used with 
non-variational wavefunctions by specifying \comp{NOANCI}.  
\end{description}

Note that \comp{ANALYT} and \comp{NOANCI} apply to two very different things:
\comp{ANALYT} applies to the derivatives using a frozen density matrix
approximation, and uses true analytical methods.  \comp{NOANCI} prevents
Liotard's C.I.\ derivative method being used.  Of course \comp{NOANCI} has no
meaning for variationally optimized wavefunctions.

\subsection{Frozen density matrix finite difference derivatives}

The first step in calculating the gradients is to calculate the derivatives
with respect to Cartesian coordinates.  This is done in subroutine DCART.
\index{DCART}

DCART calculates the energy of each pair of atoms, then moves one atom a small
distance ($10^{-4}$\AA ) in each of the three Cartesian directions.  The
density matrix for the atom-pair is not changed during this calculation, but is
set equal to the SCF density matrix. The derivative for each atom is then
calculated  from:
$$\left (\frac{dE}{dx}\right )_A = \sum_{B\neq A}\frac{E_{AB}-E_{AB}^{'}}{\delta_x}, $$
$$\left (\frac{dE}{dy}\right )_A = \sum_{B\neq A}\frac{E_{AB}-E_{AB}^{'}}{\delta_y}, $$
$$\left (\frac{dE}{dz}\right )_A = \sum_{B\neq A}\frac{E_{AB}-E_{AB}^{'}}{\delta_z}. $$
where $E_{AB}^{'}$ is the energy of the pair of atoms after displacement in the
appropriate direction.  For a stationary point, these derivatives are zero.

To convert from Cartesian coordinate (c.c.) derivatives into gradients (i.c.),
the sum
$$
g_i = \sum_j\frac{dE}{d({\rm c.c.}_j)}\frac{d({\rm c.c.}_i)}{d({\rm i.c.}_j)} 
$$
must be evaluated.  Evaluation of  
$\frac{d({\rm c.c.}_i)}{d({\rm i.c.}_j)}$  
is quite simple, and in done in routine JCARIN.

\subsection{Hessian matrix in \comp{FORCE} calculations}\index{Hessian|(}\label{ssd}
\index{Single-sided derivatives}
The Hessian matrix is the matrix of second derivatives of the energy with
respect to geometry. The most important Hessian is that used in the
\comp{FORCE} calculation.  Normal modes are expressed as Cartesian
displacements, consequently the Hessian is based on Cartesian rather than
internal coordinates.

\index{Derivatives!``single-sided''} 
Although first derivatives are relatively easy to calculate, second derivatives
are not.  The simplest, although not an elegant, way to calculate~\cite{pulayf}
second derivatives is to calculate first derivatives for a given geometry, then
perturb the geometry, do an SCF calculation on the new geometry, and
re-calculate the derivatives. The second derivatives can then be calculated
from the difference of the two first derivatives divided by the step size. 
This method, which is used in the EigenFollowing routine, is called
`single-sided' derivatives.

The Hessian is quite sensitive to geometry, and should only be evaluated at
stationary points.  Because of this sensitivity, ``double-sided'' derivatives
\index{Derivatives!``double-sided''} are used:
$$
H_{i,j} = \frac{g_i^{+\delta_j}-g_i^{-\delta_j}}{2\delta}.
$$
Note the asymmetry in the treatment of the Cartesian coordinates $i$ and $j$.
It can be shown that 
$$
\frac{g_j^{+\delta_i}-g_j^{-\delta_i}}{2\delta} = \frac{g_i^{+\delta_j}-g_i^{-\delta_j}}{2\delta}.
$$
To help improve precision, the Hessian is calculated from
$$
H_{i,j} = \frac{1}{2}\left ( \frac{g_j^{+\delta_i}-g_j^{-\delta_i}}{2\delta} + \frac{g_i^{+\delta_j}-g_i^{-\delta_j}}{2\delta} \right ).
$$
\index{Hessian|)}
\index{Derivatives|)}
