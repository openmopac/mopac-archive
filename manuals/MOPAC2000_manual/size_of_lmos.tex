\subsubsection{Size of LMOs}\label{size_of_lmo}
When a SCF is achieved, the LMOs extend over more atoms than might be
expected.  Each LMO is about 90--99\% on one or two atoms, and if the
surrounding few atoms are included, almost 100\% of the LMO can be accounted
for.   From this, it would appear that the intensity of the LMO would continue
to decrease rapidly with distance from the center.  This is normally not the
case. Instead, LMOs usually have intensity on a large number of atoms,
sometimes several  hundred atoms.  As a result, the calculations take a much
longer time than would otherwise be expected.

Some effort has been expended in trying to find ways of reducing the size of
LMOs. These attempts have not been successful.  The definitive failure was
provided as follows:

Using MOPAC, the LMOs for a large system were generated using \comp{PRECISE}.
The resulting LMOs were as expected, over 90\%\ on two atoms, with most of the
rest of the wavefunction on the nearby atoms.  However, the intensity did not
drop rapidly to zero with increasing distance.  Instead, it held  more or less
constant at about 10$^{-5}$ to 10$^{-8}$ for a large number of atoms before
finally dropping to a negligible value.

This behavior did not change on increasing the precision of the localization.

Because of this result, it was obvious that further localization of the MOZYME 
LMOs would not be useful.
